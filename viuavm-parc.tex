\documentclass[11pt,oneside,a4paper,titlepage,onecolumn]{article}

\usepackage[utf8]{inputenc}
\usepackage{textcomp}
\usepackage[official]{eurosym}
\usepackage[polish]{babel}
\usepackage{amsthm}
\usepackage{graphicx}
\usepackage[T1]{fontenc}
\usepackage{scrextend}
\usepackage{hyperref}
\usepackage{xcolor}
% \usepackage{nameref}
% \usepackage{showlabels}
% \usepackage{titlesec}
\usepackage{geometry}
\geometry{a4paper, portrait, margin=2cm}



\setcounter{secnumdepth}{4}

%% Author and title
% \author{Marek Marecki \and Gr.52c \and Kod: 95 \and 2017\slash2018-2019}
\author{Marek Marecki \and Krzysztof Franek}
\title{%
     Potwierdzenie przydatności Viua VM \\
    \large Implementacja języka wysokiego poziomu dla Viua VM \\
    i realizacja prostej aplikacji \\
    ~\\
    Wstępna analiza kategorii ryzyka}

\begin{document}

\maketitle
{\footnotesize
\begin{center}
  \begin{tabular}{ | l | l | l | }
    \hline
    \parbox[t]{6.5cm}{\textbf{Temat pracy i akronim projektu:}\\ Potwierdzenie przydatności Viua VM. Implementacja języka wysokiego poziomu dla Viua VM i realizacja prostej aplikacji} & \parbox[t]{4.5cm}{\textbf{Zleceniodawca:}\\\colorbox{yellow}{Nieznany}} & \parbox[t]{4.5cm}{\textbf{Konsultant:}\\\colorbox{yellow}{Nieznany}} \\ \hline
    \parbox[t]{6.5cm}{\textbf{Zespół projektowy:}\\Krzysztof Franek, Marek Marecki} & \parbox[t]{4.5cm}{\textbf{Kierownik projektu:}\\Marek Marecki} & \parbox[t]{4.5cm}{\textbf{Opiekun projektu:}\\dr hab. Marek A. Bednarczyk, prof. PJWSTK} \\ \hline
    \parbox[t]{3.5cm}{\textbf{Kierownik projektu:}\\Marek Marecki} & \multicolumn{2}{|l|}{\parbox[t]{9cm}{\textbf{Odpowiedzialny za dokument:}\\Krzysztof Franek}} \\ 
    \hline
  \end{tabular}
\end{center}
}

\section{Otoczenie projektu}

\subsection{Kontektst projektu}

\textbf{Czy wybrano metodologię rozwoju projektu?} Tak, wybrano dwie metodologie: (a) prototypowanie dla
kompilatora oraz (b) Mini-Scrum, tj. uproszczona wersja Scrum dostosowana do jednoosobowej pracy. Są one opisane szczegółowo we Wstępnym Planie Projektu (WPP).

\textbf{Czy istnieje zarys struktury projektu (lub Work Breakdown Structure)?} Tak, projekt został podzielony na dwie części, jedną poświęconą językowi programowania wysokiego poziomu ViuAct wraz z jego kompilatorem, a drugą - przykładowej aplikacji czatu. Zadania są podzielone według tych części. Podział opisano szczegółowo w ramach WPP.

\textbf{Czy produkty projektu zostały zdefiniowane, uzgodnione i zapisane?} Tak, zdefinionwano je w jasny i przejrzysty sposób, w dokumentacji uzgodnionej w zespole i merytorycznie uzgadnianej z opiekunem projektu.

\textbf{Czy ustanowiono środki finansowego nadzoru projektu?} Pytanie to nie dotyczy projektu, z uwagi na jego charakter non-profit.

\textbf{Czy każde z zadań posiada przejrzystą dokumentację?} Tak, jednakże brakuje na obecnym etapie szczegółowej dokumentacji wykonawczej dla niektórych z modułów czatu, pozostając na etapie dokumentacji koncepcyjnej.

\textbf{Czy zidentyfikowano istotnych udziałowców?} Tak, przy czym w przypadku języka programowania i środowiska wytwarzania oprogramowania ViuAct są to realni udziałowcy: programista używający środowiska, użytkownik korzystający z programów wytworzonych w środowisku, zespół deweloperski, opiekun projektu, recenzent. Tymczasem, w przypadku czatu, dochodzą fikcyjni udziałowcy, utworzeni wyłącznie na cele przykładowe, toteż nie będziemy rozpatrywali ich w dalszeym toku.

\textbf{Czy projekt dysponuje utrwalonym, stablinym i powtarzalym środowiskiem deweloperskim?} Tak, przy czym poziom ryzyka niestabiności jest wyższa w projekcie dotyczącym przygotowania aplikacji czatu. Wszakże samo środowisko, w którym jest tworzona, jest przedmiotem pracy dyplomowej i może ulegać dynamicznym zmianom.

\textbf{Czy istnieje proces zarządzania zmianą?} Tak, opiera się on o: (a) bieżące spotkania zespołu i dyskusję nad zmianami, dokumentowane w formie notatek (b) wariantowanie i przygotowywanie różnych alternatywnych dróg realizacji poszczególnych etapów, zanim będzie niezbędne podjęcie samej decyzji w zakresie zmiany.

\subsection{Plan projektu}

\textbf{Czy referencje deweloperów zostały sprawdzone i uznane za godne akceptacji?} Tak, obu z deweloperów miało wcześniej związek z programowaniem w wielu językach programowania, ponadto programista odpowiedzialny za język programowania i kompilator ma szeroką wiedzę na temat budowy kompilatorów i konstrukcji maszyn wirtualnych.

\textbf{Czy doszło to uformowania zespołu deweloperskiego?} Tak, obu czonków zespołu ustaliło kanały komunikacji, zasady współpracy, podział odpowiedzialoności oraz na bieżąco współdziała w zakresie realizacji projektu.

\textbf{Czy członkowie zespołu deweloperskiego posiadają istotne doświadczenia w podobnych projektach?} Tak, członek odpowiedziany za nowy język programowania i kompilator ma doświadczenie w podobnych projektach realizowanych samodzielnie.

\textbf{Czy plan testów, rozmieszczenia, szkolenia i utrzymywania są przejrzyście udokumentowane?} Nie,zaniechano sporządzenia ww. dokumentacji.

\end{document}
