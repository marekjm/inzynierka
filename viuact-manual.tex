\documentclass[11pt,oneside,a4paper,titlepage,onecolumn]{article}

\usepackage[utf8]{inputenc}
\usepackage{textcomp}
\usepackage[official]{eurosym}
\usepackage[polish]{babel}
\usepackage{amsthm}
\usepackage{graphicx}
\usepackage[T1]{fontenc}
\usepackage{scrextend}
\usepackage{hyperref}
\usepackage{xcolor}
% \usepackage{nameref}
% \usepackage{showlabels}
% \usepackage{titlesec}
\usepackage{geometry}
\geometry{a4paper, portrait, margin=2cm}
\graphicspath{ {./fig/} }
\usepackage{listings}

\newenvironment{enumreq}
{ \begin{enumerate}[topsep=0pt,itemsep=-1ex,partopsep=1ex,parsep=1ex] }
{ \end{enumerate}                  } 


\setcounter{secnumdepth}{4}

%% Author and title
% \author{Marek Marecki \and Gr.52c \and Krzysztof Franek}
\author{Marek Marecki i Krzysztof Franek}
\title{%
    Potwierdzenie przydatności Viua VM \\
    \large Implementacja języka wysokiego poziomu dla Viua VM \\
    i realizacja prostej aplikacji \\
    ~\\
    Podręcznik użytkownika dla języka Viuact}

\begin{document}

\lstset{basicstyle=\ttfamily,
columns=fixed,
escapeinside={\%*}{*)},
inputencoding=utf8,
extendedchars=true}

\maketitle
{\footnotesize
\begin{center}
  \begin{tabular}{ | l | l | l | }
    \hline
    \parbox[t]{6.5cm}{\textbf{Temat pracy i akronim projektu:}\\Proving viablity of Viua VM (VVIA)} & \parbox[t]{4.5cm}{\textbf{Zleceniodawca:}\\\colorbox{yellow}{Nieznany}} & \parbox[t]{4.5cm}{\textbf{Konsultant:}\\\colorbox{yellow}{Nieznany}} \\ \hline
    \parbox[t]{6.5cm}{\textbf{Zespół projektowy:}\\Krzysztof Franek, Marek Marecki} & \parbox[t]{4.5cm}{\textbf{Kierownik projektu:}\\Marek Marecki} & \parbox[t]{4.5cm}{\textbf{Opiekun projektu:}\\dr hab. Marek A. Bednarczyk, prof. PJWSTK} \\ \hline
    \parbox[t]{3.5cm}{\textbf{Kierownik projektu:}\\Marek Marecki} &
      \multicolumn{2}{|l|}{\parbox[t]{9cm}{\textbf{Odpowiedzialny za dokument:}\\Marek Marecki}} \\ 
    \hline
  \end{tabular}
\end{center}
}

\tableofcontents
\newpage

\section{Wprowadzenie}

Tradycja nakazuje, aby pierwszym programem jaki pisze się w nowym języku był program, który wypisze na ekran
napis ''\emph{Hello World!}''. W ViuAct ten program wygląda następująco:

\begin{lstlisting}
(let main () {
    (print "Hello World!")
    0
})
\end{lstlisting}

Aby skopilować ten program, należy wykonać w konsoli następujące polecenia:

\begin{lstlisting}
$ viuact-cc --mode exec ./hello_world.lisp
$ viuact-opt ./build/_default/hello_world.asm
\end{lstlisting}

Kod wykonywalny (\emph{bytecode} wykonywalny przez Viua VM) będzie umieszczony w pliku
\texttt{hello\_world.bc} w katalogu \texttt{./build/\_default}.
Aby go uruchomić należy użyć jądra Viua VM:

\begin{lstlisting}
$ viua-vm ./build/_default/hello_world.bc
%*\emph{Hello World!}*)
$
\end{lstlisting}

Nazwy plików pośrednich są wywodzone z nazwy pliku źródłowego:

\begin{description}
    \item[\texttt{\emph{example}.lisp}] plik z kodem źródłowym w języku ViuAct
    \item[\texttt{\emph{example}.asm}] plik wynikowy kompilatora, z kodem źródłowym w języku assemblera Viua
        VM
    \item[\texttt{\emph{example}.bc}] plik wynikowy assemblera Viua VM, zawierający wykonywalny bytecode
\end{description}

Pliki \texttt{.asm} i \texttt{.bc} są umieszczane w katalogu \texttt{./build/\_default}.

\newpage
\section{Opcje kompilatora}

Jedyną opcją kompilatora jest \texttt{--mode}, która przyjmuje dwie możliwe wartości:

\begin{description}
    \item[\texttt{exec}] jeśli plik źródłowy definiuje moduł wykonywalny
    \item[\texttt{module}] jeśli plik źródłowy definiuje moduł biblioteczny
\end{description}

\subsection{Zmienne środowiskowe}

Zachowanie kompilatora można częściowo zmodyfikować ustawiając zmienne środowiskowe.

\subsubsection{\texttt{DEFAULT\_OUTPUT\_DIRECTORY}}

Kontroluje katalog, w którym kompilator składuje pliki wynikowe. Domyślnie pliki wynikowe są składowane w
katalogu \texttt{./build/\_default}.

\subsubsection{\texttt{VIUAC\_LIBRARY\_PATH}}

Jak \texttt{LD\_LIBRARY\_PATH}.

\subsubsection{\texttt{VIUA\_ASM\_PATH}}

Kontroluje ścieżkę do assemblera Viua VM.

\subsubsection{\texttt{VIUAC\_VERBOSE}}

Wartość \texttt{true} powoduje wyświetlenie komunikatów podczas kompilacji.

\subsubsection{\texttt{VIUAC\_DEBUGGING}}

Wartość \texttt{true} włącza komunikaty debugowania.

\subsubsection{\texttt{VIUAC\_INFO}}

Wartość \texttt{true} włącza dodatkowe komunikaty informacyjne.

\subsubsection{\texttt{VIUAC\_DUMP\_INTERMEDIATE}}

Wartość \texttt{tokens} powoduje zrzut strumienia tokenów do pliku \texttt{\emph{example}.tokens}.
Wartość \texttt{exprs} powoduje zrzut drzewa składni do pliku \texttt{\emph{example}.expressions}.
Można podać obie wartości, oddzielone przecinkiem.

\newpage
\section{Opcje programu łączącego}

\end{document}
