\documentclass[11pt,oneside,a4paper,titlepage,onecolumn]{article}

\usepackage[utf8]{inputenc}
\usepackage{textcomp}
\usepackage[official]{eurosym}
\usepackage[polish]{babel}
\usepackage{amsthm}
\usepackage{graphicx}
\usepackage[T1]{fontenc}
\usepackage{scrextend}
\usepackage{hyperref}
\usepackage{xcolor}
\usepackage{listings}
% \usepackage{nameref}
% \usepackage{showlabels}
% \usepackage{titlesec}
\usepackage{geometry}
\geometry{a4paper, portrait, margin=2cm}



\setcounter{secnumdepth}{4}

%% Author and title
% \author{Marek Marecki \and Gr.52c \and Kod: 95 \and 2017\slash2018-2019}
\author{Marek Marecki \and Krzysztof Franek}
\title{%
    Proving viability of Viua VM \\
    \large Implementation of high-level language on Viua VM and\\
    deployment of a simple application \\
    ~\\
    Viuact language specification}

\begin{document}

\lstset{language=Lisp,
basicstyle=\ttfamily,
columns=fixed}

\maketitle
{\footnotesize
\begin{center}
  \begin{tabular}{ | l | l | l | }
    \hline
    \parbox[t]{6.5cm}{\textbf{Temat pracy i akronim projektu:}\\Proving viablity of Viua VM (VVIA)} & \parbox[t]{4.5cm}{\textbf{Zleceniodawca:}\\\colorbox{yellow}{Nieznany}} & \parbox[t]{4.5cm}{\textbf{Konsultant:}\\\colorbox{yellow}{Nieznany}} \\ \hline
    \parbox[t]{6.5cm}{\textbf{Zespół projektowy:}\\Krzysztof Franek, Marek Marecki} & \parbox[t]{4.5cm}{\textbf{Kierownik projektu:}\\Marek Marecki} & \parbox[t]{4.5cm}{\textbf{Opiekun projektu:}\\dr hab. Marek A. Bednarczyk, prof. PJWSTK} \\ \hline
    \parbox[t]{3.5cm}{\textbf{Kierownik projektu:}\\Marek Marecki} & \multicolumn{2}{|l|}{\parbox[t]{9cm}{\textbf{Odpowiedzialny za dokument:}\\Krzysztof Franek}} \\ 
    \hline
  \end{tabular}
\end{center}
}

\tableofcontents
\newpage

\section{Specyfikacja języka Viuact}

\section{Model programowania}

\subsection{Podział programu}

Najmniejszym elementem możliwym do wykonania jest wyrażenie.
Wyrażenia są grupowane w funkcje.
Funkcje są zebrane w modułach.

\subsection{Wykonywanie programu}

Program składa się z jednego lub większej ilości ''aktorów'' (procesów).
Aktory są izolowane od siebie nawzajem i działają równolegle (wszystkie naraz).

Wewnętrznie, każdy z aktorów wykonywany jest sekwencyjnie, bez jakiejkolwiek
współbieżności.

\subsection{Komunikacja między aktorami}

Aktory komunikują się ze sobą za pomocą ''wymiany wiadomości''.
Wymiana wiadomości ma następujące cechy:

\begin{enumerate}
    \item jest asynchroniczna (nadawca nie jest blokowany do momentu aż odbiorca otrzyma
        wiadomość)
    \item nie daje gwarancji dostarczenia (nadawca nie ma gwarancji, że jego wiadomość
        dotrze do odbiorcy)
    \item nie daje gwarancji kolejności dostarczenia (nadawca nie ma gwarancji w jakiej
        kolejności wysłane przez niego wiadomości dotrą do odbiorcy)
\end{enumerate}

Ten model wymiany wiadomości jest warunkowany faktem, że nie jest fizycznie
możliwe zagwarantowanie poprawności komunikacji.

\subsection{Obsługa błędów}

Obsługa błędów jest realizowana za pomocą wyjątków. Viuact oferuje typową
składnię \texttt{try-catch}.

\newpage
\section{Typy danych}

\subsection{Typy proste}

\subsubsection{Liczba całkowita}

\emph{Liczba całkowita ze znakiem, o szerokości 64 bitów, kodowana z dopełnienem do
dwóch}

\subsubsection{Liczba zmiennoprzecinkowa}

\emph{Liczba zmiennoprzecinkowa o szerokości 64 bitów kodowana w standardzie IEEE 754-2008}

\subsubsection{Wartość boolowska}

\emph{Wartość logiczna reprezentująca prawdę lub fałsz o szerokości 8 bitów}

\subsubsection{Tekst}

\emph{Ciąg znaków Unicode, kodowany w formacie UTF-8}

\subsubsection{String}

\emph{Ciąg 8-bitowych bajtów}

\subsection{Typy złożone}

\subsubsection{Wektor}

\emph{Sekwencja wartości dowolnych typów danych, indeksowana od 0}

\subsubsection{Struktura}

\emph{Struktura agregująca wartości dowolnych typów danych, w której jedna
wartość jest mapowana do jednego pola}

\subsection{Typy platformy}

\subsubsection{PID}

\emph{Wartość identyfikująca aktora w programie}

Jest to typ dostarczany przez Viua VM. Dla programów pisanych w Viuact jest to
typ nieprzezroczysty.

\newpage
\section{Konstrukcje języka}

\subsection{Wyrażenia}

\texttt{\emph{literal}
| \emph{name}
| \emph{fn-call}
| \emph{actor-call}
| \emph{operator-call}
| \emph{try-expr}
| \emph{if-expr}}
\\
\texttt{\{ \emph{expression}+ \}}

\begin{lstlisting}
42
true
3.14
"Hello World!"
x
(print x)
(actor server_loop sock)
(+ x 1)

; compound expression
{ (let x 42) (print (+ x 1)) x }
\end{lstlisting}

Wyrażenia są najmniejszą konstrukcją językową jaką udostępnia Viuact.
Dzielą się na wyrażenia proste -- literały, wywołania funkcji w różnej formie, wyrażenia warunkowe, i wyrażenia
obsługi wyjątków; oraz wyrażenia złożone -- kilka wyrażeń prostych zgrupowanych w nawiasach klamrowych.

Wyrażenia zwracają wartość, którą reprezentują. W przypadku literałów jest to wartość, którą bezpośrednio
reprezentują; w przypadku innych wyrażeń wartość ta jest określana na etapie wykonania (''\emph{run-time}'')
zależnie od wcześniejszego stanu programu.

\paragraph{Wyrażenia złożone}

\subsection{Definicje zmiennych}

\texttt{(let \emph{name} \emph{expression})}

\begin{lstlisting}
(let i 42)
(let f -3.14)
(let t "Hello World!")
\end{lstlisting}

\subsection{Definicje funkcji}

\texttt{(let \emph{name} (\emph{formal-parameters}) \emph{expression})}

\begin{lstlisting}
(let f (x) (+ x 1))
(let f_with_print (x) {
    (print x)
    (+ x 1)
})
\end{lstlisting}

\subsubsection{Parametry formalne}
\subsubsection{Ciało funkcji}
\subsubsection{Wartość zwracana}

\subsection{Definicje modułów}

\texttt{(module \emph{module-name} (\emph{function-definition}+))}

\subsection{Wywołanie funkcji}

\texttt{(\emph{name} \emph{expression}*)}

\subsection{Wywołanie ''tailcall''}

\texttt{(tailcall \emph{name} \emph{expression}*)}

\subsection{Wywołanie odroczone}

\texttt{(defer \emph{name} \emph{expression}*)}

\subsection{Wywołanie aktora}

\texttt{(actor \emph{name} \emph{expression}*)}

\subsection{Wywołanie operatora}

\texttt{(\emph{operator} \emph{lhs} \emph{rhs})}

\subsection{Przypisanie do pola struktury}

\texttt{(:= \emph{field} \emph{expression})}

\subsection{Konstrukcja warunkowa}

\texttt{(if \emph{condition-expression} \emph{true-arm} \emph{false-arm})}

\subsection{Obsługa wyjątków}

\texttt{(try \emph{expression} (catch \emph{exception-tag} \emph{name} \emph{expression})+)}

\newpage
\section{Słownik}

\begin{labeling}{model aktorów}
    \item [model aktorów] model przetwarzania współbieżnego, opierający się na
        podstawowych strukturach, nazywanych „aktorami”, posiadających swój
        własny prywatny stan i porozumiewających się pomiędzy sobą za pomocą
        komunikatów
    \item [ViuAct] język wysokiego poziomu, oparty o modelu aktorów, kompilowany
        do języka asemblera Viua VM
    \item [Viua VM] maszyna wirtualna, umożliwiająca uruchamianie programów
        wykorzystujących współbieżność
    \item[runtime] ''\emph{środowisko uruchomieniowe}'' maszyna wirtualna bądź realna, na której
        wykonywany jest program
    \item[\emph{run-time}] ''\emph{czas wykonywania}'' czas, w którym program jest wykonywany przez VM;
        przeciwieństwo \emph{compile-time}
    \item[\emph{compile-time}] ''\emph{czas kompilacji}'' czas, w którym program jest kompilowany
\end{labeling}

\end{document}
