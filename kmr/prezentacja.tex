\documentclass[aspectratio=169]{beamer}

\usepackage[utf8]{inputenc}
\usepackage{textcomp}
\usepackage[official]{eurosym}
\usepackage[polish]{babel}
\usepackage{amsthm}
\usepackage{graphicx}
\usepackage[T1]{fontenc}
\usepackage{scrextend}
\usepackage{hyperref}
\usepackage{xcolor}
\usepackage{listings}
\graphicspath{ {./fig/} }

\usetheme{default}
% \usecolortheme{seahorse}
% \usetheme{Pittsburg}
\usecolortheme{beaver}

\beamertemplatenavigationsymbolsempty

\title{Viua VM}
\subtitle{Stabline środowisko uruchomieniowe dla oprogramowania o\\podwyższonych wymaganiach niezawodności}
% \author{Marek Marecki \and Krzysztof Franek}

\begin{document}
\lstset{basicstyle=\ttfamily\color{black},
columns=fixed,
escapeinside={\%*}{*)},
inputencoding=utf8,
extendedchars=true,
moredelim=**[is][\color{red}]{@}{@}}

\frame{\titlepage}

\begin{frame}
    \frametitle{Problem}

    \begin{enumerate}
        \item niezdefiniowane lub niedoprecyzowane zachowanie istniejących
            środowisk -- w przypadku rakiety Ariane 5 straty \$500 000 000
        \item nieefektywne wykorzystanie zasobów sprzętowych
        \item zawodne, ulegające awariom oprogramowanie
        \item problematyczność skalowania i rozpraszania oprogramowania
    \end{enumerate}
\end{frame}

\begin{frame}
    \frametitle{Rozwiązanie}

    \begin{enumerate}
        \item wykorzystanie maszyny wirtualnej, której ISA ma całkowicie
            zdefiniowane zachowanie
        \item scheduler na poziomie VM, mapujący lekkie, wirtualne procesy
            \emph{MxN} do wątków sprzętowych
        \item kompletna izolacja procesów, komunikacja jedynie poprzez
            asynchroniczną wymianę wiadomości
        \item \emph{actor model} zakładający rozproszenie procesów jako
            podstawowy element sposobu programowania
    \end{enumerate}
\end{frame}

\begin{frame}
    \frametitle{Metryki}

    \begin{enumerate}
        \item wydłużenie średniego czasu pomiędzy awariami
        \item skrócenie całkowitego rocznego czasu awarii (\emph{downtime})
        \item stabline obciążenie zamiast peaków pozwalające na lepsze
            oszacowanie wymagań i redukcję kosztów
        \item redukcja ilości zgłaszanych defektów
    \end{enumerate}
\end{frame}

\begin{frame}
    \frametitle{Rynek i konkurencja -- potencjalni klienci}

    Dostawcy sprzętu telekomunikacyjnego i wojsko.

    \begin{quote}
        \begin{small}
            \emph{Nokia, Ericsson, ADVA, Huawei, NATO, uczelnie...}
        \end{small}
    \end{quote}

    ISP, firmy hostingowe, dostawcy usług ,,w chmurze''.

    \begin{quote}
        \begin{small}
            \emph{Deutsche Telekom, OVH, Amazon...}
        \end{small}
    \end{quote}

    ~

    Grupą docelową są firmy i organizacje, mogące sobie pozwolić na finansowanie
    rozwoju specjalistycznego środowiska uruchomieniowego i potrzebujące
    wysokiej stabliności, dostępności i niezawodności swojego oprogramowania.

    ~

    Wartość rynku zamówień w sektorze wojskowym w strefie EUR-28 w 2018 to około
    \textbf{164 biliony USD}, z czego około 20\% jest przeznaczone na nowe rozwiązania
    (nasz obszar zainteresowań).
\end{frame}

\begin{frame}
    \frametitle{Rynek i konkurencja -- konkurencja}

    Tylko jedno rozwiązanie konkurencyjne o podobnej charakterystyce --
    Erlang/OTP (BEAM).

    ~

    Nasze zalety:

    \begin{enumerate}
        \item zdefiniowane ISA (możliwość implementacji własnego języka)
        \item implementacja pozbawiona niezdefiniowanego zachowania, z obszernym
            zestawem testowym i wysokimi gwarancjami poprawności
    \end{enumerate}
\end{frame}

\begin{frame}
    \frametitle{Model zarabiania}

    \begin{labeling}{20\%}
        \item[20\%] sprzedaż licencji własnościowych (domyślnie GNU GPL)
        \item[4\%] sprzedaż dokumentacji
        \item[20\%] prowadzenie wykładów i szkoleń
        \item[8\%] usługi konsultacyjne
        \item[48\%] wytwarzanie oprogramowania
    \end{labeling}

    \begin{quote}
        \begin{tiny}
            Procentowy udział w zakładanym finalnym przychodzie rocznym na
            poziomie 250~000~EUR.
        \end{tiny}
    \end{quote}

    ~

    Źródłem finansowania w początkowej fazie projektu (2 do 3
    lat\footnote{\begin{tiny}\url{https://www.nato.int/cps/en/natolive/87260.htm}\end{tiny}})
    będzie grant na badania i rozwój z
    NATO\footnote{\begin{tiny}\url{https://www.nato.int/cps/en/natohq/85291.htm} sekcja 1.c.ii\end{tiny}}.
\end{frame}

\begin{frame}
    \frametitle{Model zarabiania -- sprzedaż licencji własnościowych}

    Domyślnie projekt jest udostępniany na licencji GNU GPL. Jeśli dana firma
    wymaga użycia zamkniętego kodu, lub potrzebuje wprowadzić zmiany bez
    udostępniania ich może wykupić licencję własnościową.

    ~

    Minimum 50 000 EUR za licencję (per projekt).

    ~

    Zakłada się, że sprzedaż wyniesie 1-2 licencje rocznie.
\end{frame}

\begin{frame}
    \frametitle{Model zarabiania -- sprzedaż dokumentacji}

    Dokumentacja ogólna (specyfikacja ISA, dokumentacja języka i narzędzi) jest
    dostępna za darmo.

    ~

    Dokumentacja szczegółowa obejmująca m.in.:

    \begin{enumerate}
        \item dostosowywanie schedulerów
        \item instrukcje optymalizacji
        \item \emph{case-studies} implementacji różnego rodzaju oprogramowania
    \end{enumerate}

    będzie dostępna w subskrypcji 1000~EUR na rok, lub 2000~EUR za
    \emph{snapshot} dokumentacji w momencie zakupu.

    ~

    Zakłada się, że przychody z tego punktu będą wynosić około 10~000~EUR rocznie.
\end{frame}

\begin{frame}
    \frametitle{Model zarabiania -- wykłady i szkolenia}

    Ceny wykładów i szkoleń są ustalane \emph{case-by-case}, w zależności od
    poziomu zaawansowania i tematu szkolenia lub wykładu.

    ~

    Zakłada się, że przychody z tego punktu będą wynosić około 50~000~EUR rocznie.
\end{frame}

\begin{frame}
    \frametitle{Model zarabiania -- usługi konsultacyjne}

    Wszelkiego rodzaju pomoc w wytwarzaniu, wdrażaniu, i utrzymywaniu
    oprogramowania działającego na platformie Viua~VM.

    ~

    Zakłada się, że przychody z tego punktu będą wynosić około 20~000~EUR
    rocznie.
\end{frame}

\begin{frame}
    \frametitle{Model zarabiania -- wytwarzanie oprogramowania}

    Wytwarzanie oprogramowania na platformę Viua~VM, lub
    \newline
    wprowadzanie zmian w samej platformie (jądrze, schedulerach, itd.).

    ~

    Zakłada się, że przychody z tego punktu będą wynosić około 120~000~EUR
    rocznie.
\end{frame}

\begin{frame}
    \frametitle{Strategia ekspozycji rynkowej}

    Kampania marketingowa poprzez ,,pracę u podstaw'':

    \begin{enumerate}
        \item prezentacje na konferencjach i publikacje w czasopismach
            branżowych i akademickich -- około 30~000~EUR rocznie
        \item rozwój oprogramowania na licencji GNU~GPL jako Free~Software --
            koszt związany z wynagrodzeniem programistów
    \end{enumerate}
\end{frame}

\begin{frame}
    \frametitle{Wykres krzywej finansowej}

    Podjęcie decyzji o przyznaniu finansowania dla projektów NATO to około
    9~miesięcy. Do tego momentu projekt jest finansowany z kapitału i pracy
    zawodowej członków zespołu i działa na ,,zwolnionych obrotach''.

    ~

    Po zakończeniu okresu finansowania projektu badawczego (2 do 3 lat),
    spodziewamy się mieć portfolio, które zapewni stałe konktrakty i przychody.
\end{frame}

\begin{frame}
    \frametitle{W przypadku niepowodzenia}

    Ryzyko porażki projektu jest bardzo duże z uwagi na trudność przebicia się
    przez mainstreamowe języki i brak zaufania do nowych, niesprawdzonych
    technologii.

    ~

    W przypadku projektu \emph{Viua VM} możliwości pivotowania są bardzo
    ograniczone. Możliwe jest:

    \begin{enumerate}
        \item przejście na model crowd-fundingowy (jak Elixir bądź Zig)
        \item próba uzyskania grantu od uczelni lub instytucji naukowych (jak
            Haskell lub OCaml)
        \item wykorzystanie zalet platformy do implementacji oprogramowania i
            jego sprzedaż
    \end{enumerate}
    Jednak najbardziej prawdopodobnym wynikiem porażki jest porzucenie projektu.
\end{frame}

\begin{frame}
    \frametitle{Źródła}

    \begin{small}
    \begin{enumerate}
        \item \url{https://www.europarl.europa.eu/factsheets/en/sheet/65/defence-industry}
        \item \url{https://tradingeconomics.com/euro-area/gdp}
        \item \url{https://www.dw.com/en/eu-to-support-common-defense-market-boost-joint-spending/a-36593660}
        \item
            \url{https://ec.europa.eu/eurostat/statistics-explained/index.php/Government_expenditure_on_defence}
    \end{enumerate}
    \end{small}
\end{frame}

\end{document}
