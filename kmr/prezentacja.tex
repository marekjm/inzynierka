\documentclass{beamer}

\usepackage[utf8]{inputenc}
\usepackage{textcomp}
\usepackage[official]{eurosym}
\usepackage[polish]{babel}
\usepackage{amsthm}
\usepackage{graphicx}
\usepackage[T1]{fontenc}
\usepackage{scrextend}
\usepackage{hyperref}
\usepackage{xcolor}
\usepackage{listings}
\graphicspath{ {./fig/} }

\usetheme{default}
% \usecolortheme{seahorse}
% \usetheme{Pittsburg}
\usecolortheme{beaver}

\beamertemplatenavigationsymbolsempty

\title{Viua VM}
\subtitle{Stabline środowisko uruchomieniowe dla oprogramowania o\\podwyższonych wymaganiach niezawodności}
% \author{Marek Marecki \and Krzysztof Franek}

\begin{document}
\lstset{basicstyle=\ttfamily\color{black},
columns=fixed,
escapeinside={\%*}{*)},
inputencoding=utf8,
extendedchars=true,
moredelim=**[is][\color{red}]{@}{@}}

\frame{\titlepage}

\begin{frame}
    \frametitle{Problem}

    \begin{enumerate}
        \item niezdefiniowane lub niedoprecyzowane zachowanie istniejących
            środowisk
        \item nieefektywne wykorzystanie zasobów sprzętowych
        \item zawodne, ulegające awariom oprogramowanie
        \item problematyczność skalowania i rozpraszania oprogramowania
    \end{enumerate}
\end{frame}

\begin{frame}
    \frametitle{Rozwiązanie}

    \begin{enumerate}
        \item wykorzystanie maszyny wirtualnej, której ISA ma całkowicie
            zdefiniowane zachowanie
        \item scheduler na poziomie VM, mapujący lekkie, wirtualne procesy
            \emph{MxN} do wątków sprzętowych
        \item kompletna izolacja procesów, komunikacja jedynie poprzez
            asynchroniczną wymianę wiadomości
        \item \emph{actor model} zakładający rozproszenie procesów jako
            podstawowy element sposobu programowania
    \end{enumerate}
\end{frame}

\begin{frame}
    \frametitle{Metryki}

    \begin{enumerate}
        \item średni czas pomiędzy awariami zwiększony o co najmniej 40\%
        \item całkowity roczny downtime poniżej 5 minut
        \item stabline obciążenie zamiast peaków pozwalające na lepsze
            oszacowanie wymagań i redukcję kosztów o ok. 20\%
        \item ilość zgłaszanych defektów zredukowana o ok. 60\%
    \end{enumerate}
\end{frame}

\begin{frame}
    \frametitle{Rynek i konkurencja -- potencjalni klienci}

    Niewielu dużych klientów -- firmy telekomunikacyjne i wojsko.

    \begin{quote}
        \begin{small}
            \emph{Nokia, Ericsson, Huawei, NATO, uczelnie...}
        \end{small}
    \end{quote}

    Sporo mniejszych klientów -- ISP, firmy hostingowe (oprogramowanie do
    monitorowania stanu infrastruktury).
\end{frame}

\begin{frame}
    \frametitle{Rynek i konkurencja -- konkurencja}

    Tylko jedno rozwiązanie konkurencyjne -- Erlang/OTP (BEAM) -- ale z
    całkowitą penetracją rynku.

    ~

    Dajemy zdefiniowane podstawy (specyfikacja~ISA) i zwiększenie konkurencji.
\end{frame}

\begin{frame}
    \frametitle{Model zarabiania}

    W początkowej fazie głównie dofinansowanie projektów badawczych z
    NATO\footnote{\begin{tiny}\url{https://www.nato.int/cps/en/natohq/85291.htm} sekcja 1.c.ii\end{tiny}}
    i UE. Po ugruntowaniu pozycji:

    \begin{enumerate}
        \item sprzedaż licencji własnościowych
        \item sprzedaż dokumentacji (domyślnie GNU GPL)
        \item prowadzenie wykładów
        \item usługi konsultacyjne
    \end{enumerate}
\end{frame}

\begin{frame}
    \frametitle{Strategia ekspozycji rynkowej}

    Kampania marketingowa poprzez ,,pracę u podstaw'':

    \begin{enumerate}
        \item prezentacje na konferencjach i publikacje w czasopismach
            branżowych i akademickich
        \item rozwój oprogramowania na licencji GNU~GPL jako Free~Software
        \item darmowe szkolenia podczas konferencji
    \end{enumerate}
\end{frame}

\begin{frame}
    \frametitle{Wykres krzywej finansowej}

    Podjęcie decyzji o przyznaniu finansowania dla projektów NATO to około
    9~miesięcy. Do tego momentu projekt jest finansowany z kapitału i pracy
    zawodowej członków zespołu i działa na ,,zwolnionych obrotach''.

    ~

    Po zakończeniu okresu finansowania projektu badawczego (2 do 3 lat),
    spodziewamy się mieć portfolio, które zapewni stałe konktrakty i przychody.
\end{frame}

\begin{frame}
    \frametitle{W przypadku niepowodzenia}

    Ryzyko porażki projektu jest bardzo duże z uwagi na trudność przebicia się
    przez mainstreamowe języki i brak zaufania do nowych, niesprawdzonych
    technologii.

    ~

    W przypadku projektu \emph{Viua VM} możliwości pivotowania są bardzo
    ograniczone. Możliwe jest przejście na model crowd-fundingowy (jak
    Elixir) lub próba uzyskania grantu od uczelni lub instytucji naukowych (jak
    Haskell lub OCaml). Jednak najbardziej prawdopodobnym wynikiem porażki jest
    porzucenie projektu i wpisanie go jako silnej pozycji w CV.
\end{frame}

\end{document}
