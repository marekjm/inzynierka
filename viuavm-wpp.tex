\documentclass[11pt,oneside,a4paper,titlepage,onecolumn]{article}

\usepackage[utf8]{inputenc}
\usepackage{textcomp}
\usepackage[official]{eurosym}
\usepackage[polish]{babel}
\usepackage{amsthm}
\usepackage{graphicx}
\usepackage[T1]{fontenc}
\usepackage{scrextend}
\usepackage{hyperref}
\usepackage{xcolor}
% \usepackage{nameref}
% \usepackage{showlabels}
% \usepackage{titlesec}
\usepackage{geometry}
\geometry{a4paper, portrait, margin=2cm}



\setcounter{secnumdepth}{4}

%% Author and title
% \author{Marek Marecki \and Gr.52c \and Kod: 95 \and 2017\slash2018-2019}
\author{Marek Marecki \and Krzysztof Franek}
\title{%
    Proving viability of Viua VM \\
    \large Implementation of high-level language on Viua VM and\\
    deployment of simple a application \\
    ~\\
    Wstępny Plan Projektu}

\begin{document}

\maketitle
{\footnotesize
\begin{center}
  \begin{tabular}{ | l | l | l | }
    \hline
    \parbox[t]{6.5cm}{\textbf{Temat pracy i akronim projektu:}\\Proving viablity of Viua VM (VVIA)} & \parbox[t]{4.5cm}{\textbf{Zleceniodawca:}\\\colorbox{yellow}{Nieznany}} & \parbox[t]{4.5cm}{\textbf{Konsultant:}\\\colorbox{yellow}{Nieznany}} \\ \hline
    \parbox[t]{6.5cm}{\textbf{Zespół projektowy:}\\Krzysztof Franek, Marek Marecki} & \parbox[t]{4.5cm}{\textbf{Kierownik projektu:}\\Marek Marecki} & \parbox[t]{4.5cm}{\textbf{Opiekun projektu:}\\dr hab. Marek A. Bednarczyk, prof. PJWSTK} \\ \hline
    \parbox[t]{3.5cm}{\textbf{Kierownik projektu:}\\Marek Marecki} & \multicolumn{2}{|l|}{\parbox[t]{9cm}{\textbf{Odpowiedzialny za dokument:}\\Krzysztof Franek}} \\ 
    \hline
  \end{tabular}
\end{center}
}

Celem dokumentu jest określenie założeń projektu (cele, zakres, ograniczenia, priorytety), przedstawienie wizji docelowego rozwiązania – kształtu systemu i strategii procesu wytwórczego oraz opisanie poszczególnych etapów, przedstawienie harmonogramu prac projektowych i konstrukcyjnych.

Praca wygenerowana w systemie \LaTeX.

\section{Założenia projektu}

\subsection{Cel projektu}
        
Celem projektu jest umożliwienie programistom łatwego wytwarzania niezawodnych, współbieżnych aplikacji. W drodze do jego osiągnięcia, zespół projektowy musi zrealizować cele pośrednie, jakimi są:

\begin{enumerate}
	\item Dostarczenie udokumentowanego środowiska wytwórczego z dedykowanym językiem wysokiego poziomu, którego semantyka wymusi tworzenie aplikacji o należytym poziomie izolacji procesów dzięki zastosowaniu modelu aktorów;
	\item Dostarczenie kompilatora i środowiska uruchomieniowego, pozwalającego na zachowanie przenośności utworzonych aplikacji pomiędzy różnymi architekturami
\end{enumerate}
    
\subsection{Zakres projektu}

W zakresie opisywanego projektu zostały zawarte:

\begin{enumerate}
	\item Sporządzenie pełnej dokumentacji projektowej;
    \item Przygotowanie składni języka wysokiego poziomu ViuAct, stosującego model aktorów;
    \item Przygotowanie kompilatora ViuAct do języka asemblera Viua VM;
    \item Przetestowanie ww. kompilatora, w tym poprawności skompilowanego kodu i wydajności wynikowych programów;
    \item Zaplanowanie i zaprojektowanie aplikacji przykładowej – serwera czatu;
    \item Uzupełnienie maszyny wirtualnej Viua VM o biblioteki zewnętrzne, ułatwiające przygotowanie serwera czatu
    \item Wytworzenie aplikacji serwera czatu z użyciem języka ViuAct;
    \item Testy i poprawki do aplikacji serwera czatu.
\end{enumerate}

\subsection{Spodziewane produkty}

Zespół oczekuje, że w trakcie realizacji projektu dojdzie do wytworzenia następujących produktów:

\begin{enumerate}
    \item Kompletny, prawidłowy opis składni języka ViuAct;
    \item Programy testowe, umożliwiające przetestowanie kompilatora języka ViuAct;
    \item Kompilator języka ViuAct do języka asemblera maszyny Viua VM;
    \item Kod źródłowy aplikacji serwera czatu, sporządzony w języku ViuAct;
    \item Przypadki testowe dla aplikacji serwera czatu;
    \item Biblioteki zewnętrzne maszyny wirtualnej Viua VM, przydatne dla utworzenia aplikacji serwera czatu;
    \item Dokumentacja projektowa;
    \item Praca dyplomowa.
\end{enumerate}

\subsection{Kontekst biznesowy i udziałowcy}

Zidentyfikowano następujących udziałowców projektu:

\begin{enumerate}
	\item Uczestników;
    \item Opiekuna;
    \item Recenzentów;
    \item Technologię maszyny Viua VM, dostępną w chwili rozpoczęcia projektu.
\end{enumerate}

Wyznaczono również użytkowników, którzy mają docelowo korzystać z produktów projektu:

\begin{enumerate}
	\item \textbf{Programista} – jest to osoba, która będzie wytwarzać aplikacje przy użyciu narzędzi, jakie dostarczy projekt. Oczekuje, że składnia języka będzie czytelna i prosta, a kompilator oraz środowisko uruchomieniowe przejmą od niego część obowiązków związanych z wykrywaniem i obsługą błędów. Chce, aby programy były stabilne.
    \item \textbf{Odbiorca} – jest to osoba, która uruchamia program przygotowany przez programistę. Chce, aby aplikacje były wydajne, stabilne i niezawodne, niezależnie od środowiska, w którym będzie je uruchamiał
\end{enumerate}

\subsection{Uwarunkowania i ograniczenia}

W pierwszej kolejności, należy wymienić ograniczenia, przed jakimi stoi projekt. Są to przede wszystkim okrojony skład osobowy (jedynie 2 członków) oraz krótkie terminy realizacji, szczególnie w porównaniu do zakresu projektu.

Projekt, w przeciwieństwie do innych, inżynierskich prac dyplomowych, nie proponuje konkretnego rozwiązania problemu w postaci gotowej aplikacji, ale próbuje dostarczyć narzędzi do odnajdywania takich rozwiązań, poprzez zastosowanie stabilnych mechanizmów współbieżności. Stąd przeniesienie środka ciężkości pracy z typowych procesów wytwórczych na rozważania o charakterze badawczym. Serwer czatu jest jedynie pretekstem do tych rozważań, a także dowodem ich słuszności.

\subsection{Założenia strategii}

\subsection{Priorytety}

Zespół ustalił następujące priorytety projektowe:

\begin{itemize}
	\item \textbf{Użytkowe} - możliwość kompilowania programów napisanych w języku ViuAct, dzięki zaimplementowaniu modelu aktorów, do programów nadających się do wykonania w maszynie wirtualnej Viua VM;
	\item \textbf{Jakościowe} - skompilowane programy powinny zachowywać się w sposob stabilny i współbieżny, tj. być wolne od wycieków zasobów, nie kontynuować pracy po wykryciu błedu bez jego uprzedniego obsłużenia a także zawłaszczenia maszyny;
	\item \textbf{Czasowe} - prototyp kompilatora ViuAct powinien funkcjonować pod koniec pierwszego semestru zajęć, zaś aplikacja czatu powinna być gotowa do oddania najpóźniej na przedostatnim zjeździe drugiego semestru.
\end{itemize}

\subsection{Projekty powiązane i partnerzy zewnętrzni}

Do zrealizowania zadań, jakie postawił przed sobą zespół projektowy, niezbędne jest wykorzystanie maszyny wirtulanej Viua VM, autorstwa jednego z uczestników przedsięwzięcia, Marka Mareckiego. Dzięki innowacyjnemu podejściu do zarządzania pamięcią oraz implementacji wielowątkowości na poziomie kodu bajtowego, maszyna ta została jednogłośnie wybrana jako podstawa do dalszych prac.

Co istotne, sama konstrukcja kodu źródłowego maszyny Viua VM przewiduje rozbudowę o dodatkowe, zewnętrzne biblioteki, pisane nie tylko w natywnym kodzie maszynowym, ale również w języku C++, co też może zostać wykorzystane do uzupełnienia środowiska uruchomieniowego o biblioteki przydatne przy realizacji przykładowego serwera czatu i ograniczyć pracę związaną z implementacją najbardziej fundamentalnych funkcjonalności.


\subsection{Analiza ryzyka}
 
{\footnotesize
\begin{tabular}{ l l l l l l }
	\hline
	Lp & Czynnik ryzyka & Zagrożenia & Prawdopodob. & \parbox[t]{3.0cm}{Skutki i wpływ na projekt} & Plan przeciwdziałania \\ \hline
	1 & \parbox[t]{3.0cm}{Konieczność dzielenia prywatnego czasu z pracą dyplomową} & \parbox[t]{3.0cm}{Nieukończenie poszczeglnych faz projektu w ustalonych terminach} & Wysokie & \parbox[t]{3.0cm}{Nieosiągnięcie zamierzonych celów, opóźnienie terminu obrony pracy} & \parbox[t]{3.0cm}{Cotygodniowe, wspólne podsumowania postępów prac} \\	\hline
	2 & \parbox[t]{3.0cm}{Bliskość terminu realizacji projektu} & \parbox[t]{3.0cm}{Niedopasowanie złożoności poszczególnych faz do czasowych możliwości} & Średnie & \parbox[t]{3.0cm}{Nieosiągnięcie zamierzonych celów, opóźnienie terminu obrony pracy} & \parbox[t]{3.0cm}{Stopniowanie wymagań i wyznaczenie priorytetów, na wypadek mniejszej ilości czasu} \\	\hline
	3 & \parbox[t]{3.0cm}{Brak stabilnej wersji Viua VM w dniu rozpoczęcia prac} & \parbox[t]{3.0cm}{Nieoczekiwane problemy podczas kompilacji i wykonywania skompilowanych programów} & Średnie & \parbox[t]{3.0cm}{Opóźnienia i niemożność zrealizowania podstawowych celów projektu} & \parbox[t]{3.0cm}{Udział Viua VM Na wszystich etapach opracowywania kompilatora ViuAct;\\Programy testowe stosowane na wszystkich etapach tworzenia kompilatora} \\	\hline
	4 & \parbox[t]{3.0cm}{Tworzenie oprogramowania w nowym, nietestowanym uprzednio języku oprogramowania} & \parbox[t]{3.0cm}{Nieoczekiwane błędy wynkające z nietestowanych wcześniej złożeń konstrukcji językowych} & Średnie & \parbox[t]{3.0cm}{Opóźnienia i niemożność zrealizowania demonstracyjnego serwera czatu} & \parbox[t]{3.0cm}{Utworzenie kompletnej i wewnętrznie spójnej specyfikacji języka przed podjęciem prac programistycznych} \\	\hline
\end{tabular}
}

\section{Wizja rozwiązania}

\subsection{Zakres systemu}

Język ViuAct powinien oferować, między innymi, następujące funkcjonalności:

\begin{itemize}
	\item Definicje zmiennych
    \item Definicje funkcji
    \item Definicje modułów
    \item Wywołania funkcji (w tym wywołania rekurencji ogonowej, tzw. \emph{tail-calls})
    \item Mechanizm tworzenia nowych procesów (aktorów)
	\item Instrukcja warunkowa (\emph{if})
	\item Instrukcje do prowadzenia komunikacji pomiędzy procesami (\emph{send}, \emph{receive});
    \item Podstawowe typy danych: liczba całkowita, liczba zmiennoprzecinkowa, napis, wartość boolowska
    \item Złożony typ danych: wektor
	\item Mechanizm tworzenia struktur danych zdefiniowanych przez programistę
\end{itemize}

Kompilator powinien przeprowadzać pełny proces kompilacji, obejmujący:

\begin{itemize}
	\item Analizę leksykalną i składniową
    \item Przetworzenie kodu źródłowego na formę pośrednią, wygodną w obróbce
    \item Przetworzenie formy pośredniej na kod w języku assemblera Viua VM
	\item Rejestrowanie i zwracanie błędów na poszczególnych etapach
\end{itemize}

Powyższy proces nie zakłada etapu optymalizacji programów.
Sprawdzenie programu pod kątem poprawności (spójności typów, obecności wymaganych funkcji, poprawnej ilości
parametrów, itd.) jest w większości delegowane do assemblera Viua VM, który jest w stanie te operacje
przeprowadzić.

Z uwagi na krótki czas na realizację projektu przy jednoczesnym dużym zakresie wymaganych prac, zespół nie
jest w stanie duplikować funkcjonalności na poziomie kompilatora i assemblera.

Wymienione wyżej punkty są jedynie wstępnym zarysem języka, a prezentowane końcowo funkcjonalności mogą
odbiegać od tego co tutaj przedstawiono. Szczegółowy, i wiążący, opis języka (tj. składnia, dozwolone
konstrukcje, oferowane typy danych, dostęp do mechanizów IO, itd.) będzie zawarty w osobnym dokumencie -
specyfikacji języka ViuAct.

\subsection{Rozwiązania techniczne}

\emph{systemu, koncepcja, standardy, technologia}

\subsection{Technologia i zamierzone środowisko}

\emph{Docelowe, wytwórcze}

\section{Proces wytwarzania}

\emph{dobór strategii wytwarzania (patrz poniższe podpunkty)}

\subsection{Strategia prowadzenia prac}

\emph{wskazanie wg jakiej strategii będzie prowadzony projekt; dyskusja alternatywnych rozwiązań;; punkt powinien być zakończony uzasadnionym wyborem strategii prowadzenia prac}

\subsection{Proces wytwórczy}

\emph{szczegółowe opisanie wybranej strategii procesu, etapy procesu, cele, zadania, powiązania, produkty, miary}

\subsubsection{Nazwa etapu/zadania}

\emph{należy dla każdego etapu określić: cele, produkty etapu, główne zadania, miary oceny i kryteria akceptacji; wskazanie potencjalnych zagrożeń realizacji etapu; zasady zapewnienia jakości i zarządzania.Jeśli projekt realizowany bez podziału na kilka etapów powyższe musi być wyspecyfikowane tylko jeden raz dla całego projektu.}

\subsection{Zapewnianie jakości}

\emph{Zasady przyjęte w celu zapewnienia wysokiej jakości oprogramowania}

\section{Infrastruktura}

\emph{technologia, narzędzia, środowisko i ich dostępność; dopasowanie infrastuktury do przyjętych założeń i procesu wytwórczego}

\subsection{Zakładane zasoby}

\emph{Czym dysponujemy? Co jest nieodzowne? Co musimy pozyskać?; praca ludzka; środki materialne (trwałe, nietrwałe tj.: zasoby sprzętowe, specyficzny sprzęt, zasoby programowe, narzędzia, infrastuktura; środki niematerialne (wiedza, doświadczenie); usługi; pieniądze; czas}

\subsection{Organizacja zespołu, odpowiedzialności}

\emph{struktura organizacyjną projektu: kierownik, zespół, odpowiedzialności; zależność od specyfiki projektu (np. wdrożeniowego)}

\subsection{Infrastruktura techniczna}

\emph{Środowiska wytwarzania i integracji oprogramowania, także - zarządzania projektem}

\subsection{Infrastruktura komunikacyjna}

\emph{Przyjęte zasady komunikacji w zespole, z klientem i in. udziałowcami}

\subsection{Infrastruktura dokumentacyjna}

\emph{Zasady raportowania i dokumentacji}

\section{Ramowy harmonogram}

\subsection{Ograniczenia i uwarunkowania}

\subsection{Zadania i produkty}

\emph{Zadania i proodukty projektu w zgodzie z p. 3.}

\subsection{Oszacowanie czasu i czasu realizacji etapów}

\emph{Ramowy plan projektu}

\subsection{Harmonogram projektu (wykres Gantt’a)}

\emph{Powiązania zadań, oszacowania czasowe; etapy realizacji i produkty ‘w czasie’}

\section{Słownik}

\begin{labeling}{model aktorów}
    \item [model aktorów] model przetwarzania współbieżnego, opierający się na podstawowych strukturach, nazywanych „aktorami”, posiadających swój własny prywatny stan i porozumiewających się pomiędzy sobą za pomocą komunikatów
    \item [ViuAct] język wysokiego poziomu, oparty o modelu aktorów, kompilowany do języka asemblera Viua VM
    \item [Viua VM] maszyna wirtualna, umożliwiająca uruchamianie programów wykorzystujących współbieżność
\end{labeling}

\end{document}
