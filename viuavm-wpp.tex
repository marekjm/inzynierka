\documentclass[11pt,oneside,a4paper,titlepage,onecolumn]{article}

\usepackage[utf8]{inputenc}
\usepackage{textcomp}
\usepackage[official]{eurosym}
\usepackage[polish]{babel}
\usepackage{amsthm}
\usepackage{graphicx}
\usepackage[T1]{fontenc}
\usepackage{scrextend}
\usepackage{hyperref}
\usepackage{xcolor}
% \usepackage{nameref}
% \usepackage{showlabels}
% \usepackage{titlesec}
\usepackage{geometry}
\geometry{a4paper, portrait, margin=2cm}



\setcounter{secnumdepth}{4}

%% Author and title
% \author{Marek Marecki \and Gr.52c \and Kod: 95 \and 2017\slash2018-2019}
\author{Marek Marecki \and Krzysztof Franek}
\title{%
    Proving viability of Viua VM \\
    \large Implementation of high-level language on Viua VM and\\
    deployment of simple application \\
    ~\\
    Wstępny plan projektu}

\begin{document}

\maketitle
{\footnotesize
\begin{center}
  \begin{tabular}{ | l | l | l | }
    \hline
    \parbox[t]{6.5cm}{\textbf{Temat pracy i akronim projektu:}\\Proving viablity of Viua VM (VVIA)} & \parbox[t]{4.5cm}{\textbf{Zleceniodawca:}\\\colorbox{yellow}{Nieznany}} & \parbox[t]{4.5cm}{\textbf{Konsultant:}\\\colorbox{yellow}{Nieznany}} \\ \hline
    \parbox[t]{6.5cm}{\textbf{Zespół projektowy:}\\Krzysztof Franek, Marek Marecki} & \parbox[t]{4.5cm}{\textbf{Kierownik projektu:}\\Marek Marecki} & \parbox[t]{4.5cm}{\textbf{Opiekun projektu:}\\dr hab. Marek A. Bednarczyk, prof. PJWSTK} \\ \hline
    \parbox[t]{3.5cm}{\textbf{Kierownik projektu:}\\Marek Marecki} & \multicolumn{2}{|l|}{\parbox[t]{9cm}{\textbf{Odpowiedzialny za dokument:}\\Krzysztof Franek}} \\ 
    \hline
  \end{tabular}
\end{center}
}

Celem dokumentu jest określenie założeń projektu (cele, zakres, ograniczenia, priorytety), przedstawienie wizji docelowego rozwiązania – kształtu systemu i strategii procesu wytwórczego oraz opisanie poszczególnych etapów, przedstawienie harmonogramu prac projektowych i konstrukcyjnych.

Praca wygenerowana w systemie \LaTeX.

\section{Założenia projektu}

\subsection{Cel projektu}
        
Celem projektu jest umożliwienie programistom łatwego wytwarzania niezawodnych, współbieżnych aplikacji. W drodze do jego osiągnięcia, zespół projektowy musi zrealizować cele pośrednie, jakimi są:

\begin{enumerate}
	\item Dostarczenie udokumentowanego środowiska wytwórczego z dedykowanym językiem wysokiego poziomu, którego semantyka wymusi tworzenie aplikacji o należytym poziomie izolacji procesów dzięki zastosowaniu modelu aktorów;
	\item Dostarczenie kompilatora i środowiska uruchomieniowego, pozwalającego na zachowanie przenośności utworzonych aplikacji pomiędzy różnymi architekturami
\end{enumerate}
    
\subsection{Zakres projektu}

W zakresie opisywanego projektu zostały zawarte:

\begin{enumerate}
	\item Sporządzenie pełnej dokumentacji projektowej;
    \item Przygotowanie składni języka wysokiego poziomu ViuAct, stosującego model aktorów;
    \item Przygotowanie kompilatora ViuAct do języka asemblera Viua VM;
    \item Przetestowanie ww. kompilatora, w tym poprawności skompilowanego kodu i wydajności wynikowych programów;
    \item Zaplanowanie i zaprojektowanie aplikacji przykładowej – serwera czatu;
    \item Uzupełnienie maszyny wirtualnej Viua VM o biblioteki zewnętrzne, ułatwiające przygotowanie serwera czatu
    \item Wytworzenie aplikacji serwera czatu z użyciem języka ViuAct;
    \item Testy i poprawki do aplikacji serwera czatu.
\end{enumerate}

\subsection{Spodziewane produkty}

Zespół oczekuje, że w trakcie realizacji projektu dojdzie do wytworzenia następujących produktów:

\begin{enumerate}
    \item Kompletny, prawidłowy opis składni języka ViuAct;
    \item Programy testowe, umożliwiające przetestowanie kompilatora języka ViuAct;
    \item Kompilator języka ViuAct do języka asemblera maszyny Viua VM;
    \item Kod źródłowy aplikacji serwera czatu, sporządzony w języku ViuAct;
    \item Przypadki testowe dla aplikacji serwera czatu;
    \item Biblioteki zewnętrzne maszyny wirtualnej Viua VM, przydatne dla utworzenia aplikacji serwera czatu;
    \item Dokumentacja projektowa;
    \item Praca dyplomowa.
\end{enumerate}

\subsection{Kontekst biznesowy i udziałowcy}

Zidentyfikowano następujących udziałowców projektu:

\begin{enumerate}
	\item Uczestników;
    \item Opiekuna;
    \item Recenzentów;
    \item Technologię maszyny Viua VM, dostępną w chwili rozpoczęcia projektu.
\end{enumerate}

Wyznaczono również użytkowników, którzy mają docelowo korzystać z produktów projektu:

\begin{enumerate}
	\item \textbf{Programista} – jest to osoba, która będzie wytwarzać aplikacje przy użyciu narzędzi, jakie dostarczy projekt. Oczekuje, że składnia języka będzie czytelna i prosta, a kompilator oraz środowisko uruchomieniowe przejmą od niego część obowiązków związanych z wykrywaniem i obsługą błędów. Chce, aby programy były stabilne.
    \item \textbf{Odbiorca} – jest to osoba, która uruchamia program przygotowany przez programistę. Chce, aby aplikacje były wydajne, stabilne i niezawodne, niezależnie od środowiska, w którym będzie je uruchamiał
\end{enumerate}

\subsection{Uwarunkowania i ograniczenia}

W pierwszej kolejności, należy wymienić ograniczenia, przed jakimi stoi projekt. Są to przede wszystkim okrojony skład osobowy (jedynie 2 członków) oraz krótkie terminy realizacji, szczególnie w porównaniu do zakresu projektu.

Projekt, w przeciwieństwie do innych, inżynierskich prac dyplomowych, nie proponuje konkretnego rozwiązania problemu w postaci gotowej aplikacji, ale próbuje dostarczyć narzędzi do odnajdywania takich rozwiązań, poprzez zastosowanie stabilnych mechanizmów współbieżności. Stąd przeniesienie środka ciężkości pracy z typowych procesów wytwórczych na rozważania o charakterze badawczym. Serwer czatu jest jedynie pretekstem do tych rozważań, a także dowodem ich słuszności.

\subsection{Założenia strategii}

\subsection{Priorytety}

Zespół ustalił następujące priorytety projektowe:

\begin{itemize}
	\item \textbf{Użytkowe} - możliwość kompilowania programów napisanych w języku ViuAct, dzięki zaimplementowaniu modelu aktorów, do programów nadających się do wykonania w maszynie wirtualnej Viua VM;
	\item \textbf{Jakościowe} - skompilowane programy powinny zachowywać się w sposob stabilny i współbieżny, tj. być wolne od wycieków zasobów, nie kontynuować pracy po wykryciu błedu bez jego uprzedniego obsłużenia a także zawłaszczenia maszyny;
	\item \textbf{Czasowe} - prototyp kompilatora ViuAct powinien funkcjonować pod koniec pierwszego semestru zajęć, zaś aplikacja czatu powinna być gotowa do oddania najpóźniej na przedostatnim zjeździe drugiego semestru.
\end{itemize}

\subsection{Projekty powiązane i partnerzy zewnętrzni}

Do zrealizowania zadań, jakie postawił przed sobą zespół projektowy, niezbędne jest wykorzystanie maszyny wirtulanej Viua VM, autorstwa jednego z uczestników przedsięwzięcia, Marka Mareckiego. Dzięki innowacyjnemu podejściu do zarządzania pamięcią oraz implementacji wielowątkowości na poziomie kodu bajtowego, maszyna ta została jednogłośnie wybrana jako podstawa do dalszych prac.

Co istotne, sama konstrukcja kodu źródłowego maszyny Viua VM przewiduje rozbudowę o dodatkowe, zewnętrzne biblioteki, pisane nie tylko w natywnym kodzie maszynowym, ale również w języku C++, co też może zostać wykorzystane do uzupełnienia środowiska uruchomieniowego o biblioteki przydatne przy realizacji przykładowego serwera czatu i ograniczyć pracę związaną z implementacją najbardziej fundamentalnych funkcjonalności.


\subsection{Analiza ryzyka}
 
{\footnotesize
\begin{tabular}{ l l l l l l }
	\hline
	Lp & Czynnik ryzyka & Zagrożenia & Prawdopodob. & \parbox[t]{3.0cm}{Skutki i wpływ na projekt} & Plan przeciwdziałania \\ \hline
	1 & \parbox[t]{3.0cm}{Konieczność dzielenia prywatnego czasu z pracą dyplomową} & \parbox[t]{3.0cm}{Nieukończenie poszczeglnych faz projektu w ustalonych terminach} & Wysokie & \parbox[t]{3.0cm}{Nieosiągnięcie zamierzonych celów, opóźnienie terminu obrony pracy} & \parbox[t]{3.0cm}{Cotygodniowe, wspólne podsumowania postępów prac} \\	\hline
	2 & \parbox[t]{3.0cm}{Bliskość terminu realizacji projektu} & \parbox[t]{3.0cm}{Niedopasowanie złożoności poszczególnych faz do czasowych możliwości} & Średnie & \parbox[t]{3.0cm}{Nieosiągnięcie zamierzonych celów, opóźnienie terminu obrony pracy} & \parbox[t]{3.0cm}{Stopniowanie wymagań i wyznaczenie priorytetów, na wypadek mniejszej ilości czasu} \\	\hline
	3 & \parbox[t]{3.0cm}{Brak stabilnej wersji Viua VM w dniu rozpoczęcia prac} & \parbox[t]{3.0cm}{Nieoczekiwane problemy podczas kompilacji i wykonywania skompilowanych programów} & Średnie & \parbox[t]{3.0cm}{Opóźnienia i niemożność zrealizowania podstawowych celów projektu} & \parbox[t]{3.0cm}{Udział Viua VM Na wszystich etapach opracowywania kompilatora ViuAct;\\Programy testowe stosowane na wszystkich etapach tworzenia kompilatora} \\	\hline
	4 & \parbox[t]{3.0cm}{Tworzenie oprogramowania w nowym, nietestowanym uprzednio języku oprogramowania} & \parbox[t]{3.0cm}{Nieoczekiwane błędy wynikające z nietestowanych wcześniej złożeń konstrukcji językowych} & Średnie & \parbox[t]{3.0cm}{Opóźnienia i niemożność zrealizowania demonstracyjnego serwera czatu} & \parbox[t]{3.0cm}{Utworzenie kompletnej i wewnętrznie spójnej specyfikacji języka przed podjęciem prac programistycznych} \\	\hline
\end{tabular}
}

\section{Wizja rozwiązania}

\subsection{Zakres systemu}

Język ViuAct powinien oferować, między innymi, następujące funkcjonalności:

\begin{itemize}
	\item Definicje zmiennych
    \item Definicje funkcji
    \item Definicje modułów
    \item Wywołania funkcji (w tym wywołania rekurencji ogonowej, tzw. \emph{tail-calls})
    \item Mechanizm tworzenia nowych procesów (aktorów)
	\item Instrukcja warunkowa (\emph{if})
	\item Instrukcje do prowadzenia komunikacji pomiędzy procesami (\emph{send}, \emph{receive});
    \item Podstawowe typy danych: liczba całkowita, liczba zmiennoprzecinkowa, napis, wartość boolowska
    \item Złożony typ danych: wektor
	\item Mechanizm tworzenia struktur danych zdefiniowanych przez programistę
\end{itemize}

Kompilator powinien przeprowadzać pełny proces kompilacji, obejmujący:

\begin{itemize}
	\item Analizę leksykalną i składniową
    \item Przetworzenie kodu źródłowego na formę pośrednią, wygodną w obróbce
    \item Przetworzenie formy pośredniej na kod w języku assemblera Viua VM
	\item Rejestrowanie i zwracanie błędów na poszczególnych etapach
\end{itemize}

Powyższy proces nie zakłada etapu optymalizacji programów.
Sprawdzenie programu pod kątem poprawności (spójności typów, obecności wymaganych funkcji, poprawnej ilości
parametrów, itd.) jest w większości delegowane do assemblera Viua VM, który jest w stanie te operacje
przeprowadzić.

Z uwagi na krótki czas na realizację projektu przy jednoczesnym dużym zakresie wymaganych prac, zespół nie
jest w stanie duplikować funkcjonalności na poziomie kompilatora i assemblera.

Wymienione wyżej punkty są jedynie wstępnym zarysem języka, a prezentowane końcowe funkcjonalności mogą
odbiegać od tego co tutaj przedstawiono. Szczegółowy, i wiążący, opis języka (tj. składnia, dozwolone
konstrukcje, oferowane typy danych, dostęp do mechanizmów IO, itd.) będzie zawarty w osobnym dokumencie -
specyfikacji języka ViuAct.

\subsection{Technologia i zamierzone środowisko}
\label{sec:finalenvironment}

Środowisko wytwórcze i docelowe to dowolna dystrybucja systemu Linux wydana w ciągu ostatnich 4 lat (tzn.
późniejsza niż rok 2014). Jest to uwarunkowane tym, że maszyna wirtualna Viua VM będąca podstawą projektu
wykorzystuje w części swojej implementacji mechanizmy standardu POSIX (np. dostęp do zmiennych środowiskowych,
pobieranie danych losowych z systemu). Teoretycznie, możliwe byłoby więc założenie, że środowisko wytwórcze i
docelowe to dowolny system zgodny z POSIX (czyli np. Mac OS, FreeBSD, OpenBSD, itd.) jednak z uwagi na
ograniczone zasoby przeznaczone na projekt nie byłoby możliwe przetestowanie pracy na wielu platformach.

Dla środowiska wytwórczego musi istnieć port kompilatora GCC (w wersji 6 lub wyższej) lub Clang (w wersji 6
lub wyższej). Jest to wymagane, gdyż kompilatory w starszych wersjach nie implementują pełnego standardu
C++17, który jest wymagany do kompilacji Viua VM.

Środowisko wytwórcze i docelowe musi posiadać bibliotekę glibc (w wersji 2.28 lub nowszej).

Dla środowiska wytwórczego musi istnieć port GNU Make (w wersji 4.2.1 lub wyższej). Zakładane jest również
istnienie standardowych narzędzi dostępnych w nowoczesnej dystrybucji systemu unixopodobnego (np. grep, sed,
xargs).

Obsługiwane architektury procesora fizycznego to x86 i ARM, przy czym zakłada się wariant 64-bitowy tych
architektur, oraz model procesora wyprodukowany po 2012 roku. Obsługa architektury ARM nie jest testowana tak
dokładnie jak architektury x86 i może zostać usunięta, jeśli okaże się, że koszt takiego wsparcia będzie zbyt
wysoki.

Środowisko wytwórcze musi być wyposażone w co najmniej 16GB pamięci RAM (w przeciwnym przypadku kompilacja
Viua VM może się nie powieść jeśli zostanie wybrany najwyższy poziom optymalizacji).

Środowisko wytwórcze i docelowe musi posiadać port interpretera CPython (w wersji 3.6 lub wyższej). To
wymaganie może ulec zmianie. Prototyp kompilatora jest napisany w języku Python 3.6. Jeśli zasoby czasowe
okażą się niewystarczające na przepisanie prototypu, zostanie on awansowany do wersji produkcyjnej, a to
wymaganie będzie wiążące.

\section{Proces wytwarzania}

\subsection{Strategia prowadzenia prac}
Przed przystąpieniem do prac, zespół przeanalizował kilka różnych strategii realizacji oprogramowania, wśród których znalazły się m.in.:
\begin{itemize}
	\item \textbf{Modele fazowe}, takie jak przyrostowo-kaskadowy oraz ,,V'', dobre przy określonych z góry wymaganiach 
	oraz możliwości zużycia większych nakładów, nadające się do utrzymywania oprogramowania w odpowiedniej jakości, w praktyce - najlepszy przy niewielkich projektach;
	\item \textbf{Model prototypowy}, dopasowany do przedsięwzięć, przy których klient nie jest w stanie wyraźić w sposób jednoznaczny swoich wymagań i należy je sformułować dopiero na etapie realizacji, nadaje się również doskonale w sytuacjach, w których należy zademonstrować działanie systemu, jego wewnętrzną spójność i prawidłowość;
	\item \textbf{Modele lekkie} inaczej zwinne (z ang. \textit{agile}), powszechne we współczesnych przedsiębiorstwach IT, kładące nacisk na szybkie dostarczanie oprogramowania w niewielkich odstępach czasu, szybką adaptację do zmian oraz brak ścisłej specyfikacji czy sztywnego projektu architektury.
\end{itemize}

Aby uniknąć chaosu oraz skupić się na efektywnej pracy, należało możliwie szybko zawężyć zakres rozważanych strategii. W podjęciu decyzji pomogły zespołowi liczne przesłanki. 

Projekt dyplomowy można uznać niejako za złożenie dwóch \textit{podprojektów}: pierwszego związanego z ViuAct oraz drugiego, obejmującego chat ViuaChat. Charakter obu z nich jest zgoła odmienny, podobnie jak doświadczenie członków zespołu, desygnowanych do ich urzeczywistnienia. W przypadku ViuAct, składnia języka jest z góry przewidziana i ukształtowana przez twórcę maszyny ViuaVM, doświadczonego w konstruowaniu kompilatorów. Zakres funkcjonalny nowego języka wysokiego poziomu, jak i środowiska wytwórczego, które z nim powiązano, są dość klarownie sprecyzowane i nie ulegną zasadniczym zmianom. Tymczasem, w przypadku czatu ViuaChat, osoba odpowiedzialna za jego utworzenie, skupi się na licznych aspektach funkcjonalnych, znanych z bardziej konwencjolanych zagadnień z inżynierii oprogramowania. O ile wyjściowy katalog funkcjonalności przewidzianych dla czatu jest znany, o tyle może on ulegać z czasem stopniowej ewolucji.

Kolejnym ważnym aspektem jest zdeterminowanie harmonogramu projektu przez dwa kamienie milowe, mianowicie prezentację prototypów przypadającą na ukończenie semestru zimowego oraz obrona pracy dyplomowej u schyłku semestru letniego. Należało zatem tak zsynchronizować pracę obu członków zespołu oraz rozplanować tempo rozwoju obu \textit{podprojektów}, aby oba z nich mogły być realizowane w sposób maksymalnie równoległy, przy uwzględnieniu charakteru każdego z nich.

\subsection{Proces wytwórczy}

Po burzliwej dyskusji, członkowie zespołu projektowego zadecydowali kolegialnie o poprowadzenie projektu w dwóch
różnych, ale prowadzonych równolegle, strategiach wytwórczych:

\begin{enumerate}
	\item Kompilator języka ViuAct będzie prowadzony samodzielnie przez Marka Mareckiego strategią prototypowania, wspieranego autorską techniką \textit{issue tracking}
	\item Czat ViuaChat bedzie rozwijany przez Krzysztofa Franka strategią mini-Scrumu, zorganizowanego przy użyciu techniki Kanban;
\end{enumerate}

\subsubsection{Prototypowanie}
Główna strategia, zastosowania podczas wytwarzania kompilatora języka ViuAct - prototypowanie - została 
wybrana intuicyjnie. Ponieważ p. Marecki dysponował już wcześniej 
doświadczeniem i wiedzą na temat budowy kompilatorów, zdecydował się na napisanie modelu programowego w
języku wysokiego poziomu - Pythonie. Uznał bowiem, że tworzenie kompilatora nowego języka jest wyzwaniem 
ryzykownym przede wszystkim z perspektywy spójności nowatorskiej składni i zastosowanych w niej rozwiązań, a nie 
implementacji powtarzalnych elementów, wspólnych dla większości kompilatorów, takich preprocesor czy parser.

\subsubsection{\textit{Issue tracking}}
\textit{\textbf{Issue tracking}} jest to technika pochodna do Feature Driven Development, opracowana przez 
kierownika projektu, Marka Mareckiego, podczas
realizacji zadania w ramach przedmiotu Budowa i Integracja Systemów (BYT). Decyzja była podyktowana niewielkim
rozmiarem zespołu oraz uznaniem klasycznych metod zwinnych za zbyt obciążające pod względem tworzonej 
dokumentacji w proporcji do kodu. Za inspirację do jej opracowania posłużyły praktyki popularne wśród twórców 
wolnego oprogramowania z otwartym kodem źródłowym, aktywnych na serwisach takich jak GitHub czy GitLab. 

Zasadniczymi artefaktami tej metodologii są \textbf{zadania} (ang. \textit{issue}), ustawiane na 
\textbf{liście zadań}. Każde z zadań posiada opis, jeden ze stanów (otwarty lub zamknięty), a także
notuje automatycznie czas swojego otwarcia i zamknięcia. Zadania mogą być w razie potrzeby dzielone na podzadania,
przy czym nie jest możliwe zamknięcie zadania bez uprzedniego zamknięcia jego wszystkich podzadań.

Projekt rozpoczyna się od wytworzenia zadań i umieszczenia ich na liście, a następnie 
polega na cyklicznej realizacji poniższych kroków:

\begin{enumerate}
	\item Wybierz pierwsze zadanie na liście, możliwe do zrealizowania na obecnym etapie projektu;
	\item Zrealizuj zadanie;
	\item Utwórz i dodaj do listy nowe zadania, które ujawniły się w trakcie realizacji zadania;
	\item Zamknij zadanie;
\end{enumerate}

P. Marecki opracował dla strategii \textit{issue trackingu} osobne narzędzie o nazwie Issue, ułatwiające
śledzenie i zarządzanie zadaniami. Było ono wykorzystywane przy jego innych projektach, a także w
pracach nad komercyjnymi rozwiązaniami informatycznymi.

\subsubsection{mini-Scrum}
Dla czatu ViuaChat, Krzysztof Franek, odpowiedzialny za tę część projektu, wybrał strategię mini-Scrum. Jest to
wariant techniki Scrum, korzystający z tych podobnych wydarzeń (Sprint, Planowanie Sprintu, Przegląd Sprintu, 
Retrospektywa Sprintu), tych samych artefaktów (Backlog Produktu, Backlog Sprintu, Przyrost) i niektórych ról 
(Właściciel Produktu, Zespół Deweloperski). 

Podstawową zaletą tej metodyki działania był podział zadań na mniejsze fragmenty, a także możliwość łatwego 
planowania i nadzorowania tempa prac poprzez obserwowanie wykresu wypalania produktu. Umówiono się, że rolę 
właściciela produktu (ang. \textit{product owner}) będzie pełnił p. Marecki, zaś w rolę zespołu deweloperskiego 
wcieli się p. Franek. Z powodu mało licznego zespołu, w metodologi mini-Scrum dokonano pewnych uproszczeń w 
stosunku do oryginalnego modelu Scruma, zaproponowanego przez Kena Schwabera i Jeffa Sutherlanda. Zrezygnowano 
z codziennych scrumów (ang. \textit{daily scrums}), z uwagi na nieregularny charakter prac (uczestnicy projektu
studiują w trybie zaocznym i realizują zadania w czasie wolnym od pracy), a także z odrębnej roli \textit{scrum 
mastera}.

\subsubsection{Kanban}
Dodatkowo, jako technikę prowadzenia Sprintów mini-Scruma wybrano Kanban, w ramach którego każda z historyjek 
w Sprincie będzie umieszczana w postaci ,,karteczki'' na wirtualnej tablicy i przemieszczana pomiędzy jej kolumnami, 
wraz z ich kolejnymi etapami realizacji. Na tablicy przewidziano następujące kolumny: ,,Do zrobienia'', ,,W 
realizacji'', ,,W testowaniu'', ,,Zrealizowano''. Ponadto, każda karteczka ma przypisaną etykietę z priorytetem. 

Dla każdego sprintu zostanie utworzona osobna tablica. Karteczki reprezentujące historyjki, których nie udało się 
zrealizować w sprincie, który był dla niego pierwotnie zaplanowany, będą kopiowane do tablic kolejnych sprintów 
i wyróżniane dodatkową, ostrzegawczą etykietą.

Oprogramowanie zaproponowane do utrzymywania mini-Scruma dla ViuaChat to Trello, umożliwiające łatwe
zaimplementowanie techniki Kanban, rejestrowanie czasu pracy oraz integrację z najpopularniejszymi
dostarczycielami repozytoriów Git.

\subsection{Zapewnianie jakości}

Jakość  będzie zapewniona poprzez testowanie automatyczne, sporządzanie specyfikacji i dokumentacji 
oraz dzięki wsparciu narzędzi, np. instrumentacji programów przez kompilator i sanitizery, użycie walidatorów (np.
Valgrind).

\section{Infrastruktura}

\subsection{Zakładane zasoby}

Specyficzny sprzęt nie jest wymagany do przeprowadzenia projektu do końca, więc zespół nie musi ponosić
kosztów zakupu nowego hardware'u.

Całość infrastruktury jest oparta o narzędzia dostępne na licencjach zgodnych z GNU GPL. Zespół nie musi
przeznaczać środków finansowych na zakup oprogramowania.

Zasób, który jest dla zespołu najbardziej deficytowy to czas. Z uwagi
na zakres projektu oraz fakt, że podczas prac mogą wystąpić nieprzewidziane problemy (awaria sprzętu, błąd
założeń, trudności implementacyjne, choroba członka zespołu) jest to zasób, z którym wiąże się największe
ryzyko.

\subsection{Organizacja zespołu, odpowiedzialności}

Jak ustalono, kierownikiem projektu jest Marek Marecki, a jego drugim członkiem jest Krzysztof Franek.
Podział jednak odbywa się przede wszystkim według wcześniej wspomnianych \textit{podprojektów}, tj. Marek
Marecki odpowiada za stronę związaną z kształtem i kompilatorem języka ViuAct, zaś Krzysztof Franek
ma za zadanie przygotować program czatu, demonstrujący użyteczność zarówno języka, jak i jego kompilatora.

\subsection{Infrastruktura techniczna}

\subsubsection{Infrastruktura dla wytworzenia kompilatora}
Zespół w momencie rozpoczęcia projektu dysponuje zasobami sprzętowymi wymaganymi do przeprowadzenia prac
rozwojowych i badawczych:

\begin{itemize}
    \item 1 PC z procesorami x86
    \item 1 laptop z procesorem 64 bit
    \item SoC z procesorem ARM
    \item VPS na procesorze x86; do hostowania repozytorium kodu źródłowego, dokumentacji, infrastruktury CI, oraz
        z działającym kanałem IRC
\end{itemize}

\subsubsection{Infrastruktura czatu}
Jako środowisko wytwórcze dla aplikacji czatu wybrano dwa przenośne komputery (laptopy).

\textbf{Komputer A} to komputer przenośny z procesorem Intel Core i5 oraz systemem operacyjnym Windows 10,
wyposażony w pakiet oprogramowania JetBrains WebStorm oraz PhpStorm. Pakiet wyselekcjonowano z uwagi
na bogaty pakiet narzędzi, wspomagających praktyczną realizację \textit{frontendu} aplikacji ViuaChat.
Był on również objęty licencją studencką, co pozwoliło wyeliminować koszty zakupu. Jako framework dla
\textit{frontendu} został wybrany Vue.js, wybrany z uwagi na to, że jest łatwy do zastosowania
przez osoby początkujące, a także lekkie od strony składni.

Ponadto, na komputerze A zainstalowano zestaw XAMPP 7.2.7, obejmujący serwer Apache 2.4 oraz interpreter 
języka PHP w wersji 7.2.7. Wybór został podjęty przede wszystkim z uwagi na prostotę instalacji i
uruchomienia - docelowo, zestaw ma być wykorzystywany do uruchamiania prostych skryptów w języku
PHP, pozwalających na zasymulowanie komunikacji \textit{frontendu} czatu z podstawionym \textit{backendem},
dzięki zestawowi prostych skryptów PHP, a tym samym - przetestowanie i rozwój \textit{frontendu}, 
niezależnie od tempa prac nad kompilatorem języka ViuAct oraz nad docelową, \textit{backendową} częścią 
serwera, stworzoną w tym języku.

\textbf{Komputer B} to komputer przenośny, na którym zainstalowano system operacyjny Linux Mint 19 
,,Tara'', GNU Compiler Collection 8.2 (ostatnia stabilna wersja w dniu instalacji), a ponadto maszynę
wirtualną Viua VM w wersji 0.9.0. Komputer ten został przeznaczony do przyszłych prac nad 
\textit{backendem} ViuaChat w jezyku ViuAct. Ma również być drugą maszyną, która pozwoli praktyczne
działanie czatu z wieloma użytkownikami, łączacymi się z różnych urządzeń.

Ponadto, na obu komputerach zainstalowano edytor Texmaker oraz kompilator MiKTeX, aby prowadzić
dokumentację projektową oraz sporządzać pracę dyplomową. W celu koordynacji i śledzenia postępu
system wersjonowania Git.

\emph{Środowiska wytwarzania i integracji oprogramowania, także - zarządzania projektem}

\subsection{Infrastruktura komunikacyjna}

Komunikacja w zespole odbywa się poprzez kanał IRC, pocztę elektroniczną, kanał w aplikacji Discord oraz na spotkaniach osobistych.

\subsection{Infrastruktura dokumentacyjna}

Zespół zdecydował, aby dokumentacja mająca postać ,,klasycznych" dokumentów miała postać plików tekstowych, 
sporządzanych i kompilowanych z użyciem środowiska \LaTeX. Kod źródłowy tych dokumentów będzie rozwijany i utrzymywany 
w systemie wersjonowania Git. Do tych dokumentów zaliczone zostały:
\begin{enumerate}
	\item Karta projektu;
	\item Dokument założeń wstępnych;
	\item Wstępny plan projektu;
	\item Specyfikacja EBNF języka ViuAct;
	\item Specyfikacja wymagań biznesowych i historyjek (\textit{user stories}) czatu ViuaChat;
	\item Wyjściowy rejestr zaległości czatu ViuaChat;
	\item Główny tekst pracy dyplomowej (tzw. ,,książka")
\end{enumerate}

Dokumentacja nad kompilatorem języka ViuAct powstaje, zgodnie z metodyką \textit{issue trackingu}, równolegle z
powstającym kodem i kolejnymi zmianami (ang. \textit{commits}), w postaci \textit{issues}. Można wygodnie przeglądać
ich stan, opis, daty otwarcia i zamknięcia, a także filtrować według wybranych kryteriów, a w razie potrzeby (np.
umieszczenia w głównym tekście pracy dyplomowej) również wyeksportować w formie zrzutu z wiersza poleceń.

Ostatnia część dokumentacji stanowi zbiór zadań, list i tablic, utrzymywanych na platformie Trello i zorganizowanych
według wcześniej opisanego schematu. On również umożliwia wygodne przeglądanie, utrzymanie i eksport poszczególnych
artefaktów.

Każdy z członków zespołu ma równoczesny dostęp do wszystkich składników dokumentacji.

\section{Ramowy harmonogram}

\subsection{Ograniczenia i uwarunkowania}
Harmonogram prac został uwarunkowany przede wszystkim przez:
\begin{itemize}
	\item \textbf{Podzielenie toku przedmiotu Projekt Zespołowy na dwa semestry} - wymusza to konieczność rozbicia prac w czasie na co najmniej dwa główne etapy, rozdzielone kamieniem milowym, jakim jest weryfikacja wyników zespołu pod koniec pierwszego semestru zajęć;
	\item \textbf{Złożenie projektu z części poświęconej kompilatorowi ViuAct oraz części dedykowanej aplikacji czatu ViuaChat} - podział ten powinien zostać odzwierciedlony w harmonogramie, tak, by uwzględniał niezależność prac oraz maksymalizował ich współbieżność, aby projekt mógł być kontynuowany nawet wówcza, gdy jeden z \textit{podprojektów} zostanie opóźniony;
	\item \textbf{Zespół składający się jedynie z dwóch osób} - ten aspekt wymusza rozsądny podział obowiązków, dostosowany do doświadczeń i indywidualnych predyspozycji każdego z członków.
\end{itemize}

\subsection{Oszacowanie czasu realizacji etapów}

\subsubsection{Pierwszy etap - kamień milowy}
Celem tego etapu projektu jest przygotowanie gruntu pod efektywne przygotowanie kompilatora wraz z programem
czatu, oraz zaprezentowanie postępu prac nad projektem inżynierskim jako całością. Termin realizacji
tego kamienia milowego przypada na 8 zjazd 7 semestru, czyli 9 luty 2019 roku.

Aby osiągnąć ten cel, z obu \textit{podprojektów} wyekstrahowano części nadające się na zaprezentowanie
w charakterze prototypów. Uznano, że produktami pierwszego etapu prac powinny być:
\begin{enumerate}
	\item Dla języka ViuAct i jego kompilatora:
		\begin{enumerate}
			\item kompletna definicja języka ViuAct w notacji EBNF,
			\item w pełni funkcjonalny model programowy (prototyp) kompilatora, napisany w języku Python,
			\item zestawem testowych, demonstracyjnych fragmentów kodu w języku ViuAct (w tym co najmniej 
			jeden, eksponujący zastosowanie języka w programowaniu współbieżnym);
		\end{enumerate}
	\item Dla czatu ViuaChat:
		\begin{enumerate}
			\item działający \textit{frontend} aplikacji czatu, działającej w przeglądarce,
			\item zestaw skryptów opracowanych w języku PHP, stanowiących makietę przyszłego, właściwego serwera czatu.
		\end{enumerate}
\end{enumerate}

Jak wcześniej opisano, za rozwój języka i kompilatora ViuAct będzie odpowiadał p. Marecki, którego zadaniem 
będzie opracowanie produktów na nim wymaganych. Z kolei, w przypadku czatu ViuaChat, kod źródłowy (zarówno 
\textit{frontendu}, jak i makiety serwera czatu) zostanie sporządzony przez p. Franka pod nadzorem p. Mareckiego.

Techniki zapewniania jakości prototypu kompilatora opisano w dalszej części oprogramowania. Z kolei w przypadku
oprogramowania ViuaChat, dla każdej z historyjek został przewidziany dodatkowy etap testu, zanim zostanie
oznaczona jako ,,ukończona'' (jest to element zindywidualizowanej definicji ukończenia, o której mowa w metodyce 
Scrum).

Dodatkowym zadaniem, spadającym na Krzysztofa Franka, jest przygotowanie listy bibliotek rozszerzeń, o które
powinna zostać uzupełniona maszyna wirtualna ViuaVM, aby przyspieszyć prace nad serwerem czatu w języku
ViuAct w kolejnym etapie prac.

Każdy z członków zespołu realizuje swoje zadania w dużej mierze autonomicznie, przy czym przewidywane są co
najmniej raz w tygodniu spotkania (online lub osobiście), podczas których będą podsumowywane postępy prac.

\subsubsection{Drugi etap - finalizacja projektu}

Celem tego etapu projektu jest osiągnięcie głównych celów projektu, tj.:
\begin{enumerate}
	\item Dla języka ViuAct i jego kompilatora:
		\begin{enumerate}
			\item w pełni funkcjonalny kompilator, napisany w języku OCaml;
		\end{enumerate}
	\item Dla czatu ViuaChat:
		\begin{enumerate}
			\item w pełni funkcjonalna aplikacja serwera czatu.
		\end{enumerate}
\end{enumerate}
Termin realizacji tej części projektu (a tym samym jego całości) przypada na 7 zjazd 8 semestru, czyli 25 
maja 2019 roku. Metody zarządzania jakością oraz zasady współpracy nie ulegną zmianie.

W razie, gdyby realizacja tego etapu była zagrożona z uwagi na opóźnienia w pierwszym kamieniu milowym,
przewidziano następujące działania zastępcze:

\begin{enumerate}
	\item Dla języka ViuAct i jego kompilatora:
		\begin{enumerate}
			\item rezygnacja z kompilatora w języku OCaml,
			\item skupienie się na dokończeniu lub dopracowaniu modelu programowego w języku Python;
		\end{enumerate}
	\item Dla czatu ViuaChat:
		\begin{enumerate}
			\item ograniczenie zakresu funkcjonalności czatu i priorytetyzacja historyjek,
			\item zmiana sposobu dzielenia Sprintów,
			\item zmiana strategii z mini-Scrum na jeszcze lżejszą.
		\end{enumerate}
\end{enumerate}

\section{Słownik}

\begin{labeling}{model aktorów}
    \item [model aktorów] model przetwarzania współbieżnego, opierający się na podstawowych strukturach, nazywanych „aktorami”, posiadających swój własny prywatny stan i porozumiewających się pomiędzy sobą za pomocą komunikatów
    \item [ViuAct] język wysokiego poziomu, oparty o modelu aktorów, kompilowany do języka asemblera Viua VM
    \item [Viua VM] maszyna wirtualna, umożliwiająca uruchamianie programów wykorzystujących współbieżność
\end{labeling}

\end{document}
