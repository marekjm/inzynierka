\chapter{Strategia prowadzenia prac}
\label{strategia_prac}

W tym rozdziale opisujemy metodologię prac nad poszczególnymi częściami projektu. Każda z nich była rozwijana
innym sposobem z dużych rozbieżności specyfiki konkretnych problemu -- zupełnie inaczej pracuje się nad
semi-formalną specyfikacją języka programowania niż nad aplikacją użytkową, więc podjęliśmy decyzję o
rozdzieleniu tych części i poprowadzeniu ich jako osobne, mniejsze projekty wewnątrz pracy inżynierskiej.

\section{Język ViuAct}

% Specyfikacja. Analiza wymagań. Formalizm. Kaskada.

Tworzenie specyfikacji języka odbywa się w modelu kaskadowo-prototypowym i przeplata się z implementacją
prototypu kompilatora.

\subsection{Określenie wymagań}

Przed rozpoczęciem specyfikowania poszczególnych konstrukcji językowych należało podjąć decyzję co w języku
powinno się znaleźć, a co nie powinno zostać włączone do specyfikacji.

\subsection{Specyfikacja konkretnej konstrukcji języka}

Każda konstrukcja językowa jest najpierw projektowana, następnie dokumentowana i omawiana (żeby wychwycić
pewne oczywiste błędy ,,grube'' na wczesnym etapie prac). Potem proces specyfikacji zostaje zawieszony do
momentu wytworzenia w kompilatorze prototypu danej konstrukcji, lub odrzucenia jej jako niewykonalnej w
zakładanym czasie lub zakresie funkcjonalności.

Wynik prac prototypowych (sukces bądź porażka) decyduje o tym czy prace specyfikacyjne są prowadzone dalej.
Jeśli konstrukcja jest możliwa do zaimplementowania jej specyfikacja jest formalizowana oraz przygotowana dla
niej zostaje dokumentacja (obejmująca m.in. przykłady użycia).

\section{Kompilator języka ViuAct}

Issue tracking, open-source. Prototypowanie. Iteracja.
Prace przeplatają się z pracami nad specyfikacją celem walidacji co jest możliwe.

\subsection{Testowanie}

Zestaw prostych programów testowych. Test sprawdza czy program się kompiluje i czy po uruchomieniu daje
spodziewane wyniki.

\section{Program ViuaChat}

Mini-SCRUM.

\subsection{Testowanie}

Jak wygląda testowanie czatu.
