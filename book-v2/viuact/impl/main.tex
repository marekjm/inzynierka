\chapter{Język ViuAct i jego kompilator -- implementacja}
\label{viuact_impl}
\label{jezyk_viuact_i_jego_kompilator}

Pierwszą częścią naszej pracy jest zaprojektowanie wysokopoziomowego języka programowania i opracowanie jego
implementacji. Z uwagi na to, że platforma uruchomieniowa, którą wykorzystujemy (czyli Viua VM) jest maszyną
wirtualną wykonującą programy w postaci \emph{bytecode} wybranym sposobem implementacji języka jest
kompilator - program tłumaczący kod źródłowy w jednym języku na kod źródłowy w innym języku przy jednoczesnym
zachowaniu zachowania programu. W przypadku naszej pracy językiem źródłowym jest język Viuact (dokładniej
opisany w rozdziale \ref{specyfikacja_jezyka_viuact}.~\nameref{specyfikacja_jezyka_viuact} na stronie
\pageref{specyfikacja_jezyka_viuact}), a językiem docelowym język assemblera Viua VM.

\section{Architektura}

\subsection{Użyte wzorce projektowe -- Sposób konstrukcji kompilatora}

\begin{figure}[!htp]
    \centering
    \includegraphics[width=5cm]{basic-compiler-flow}
    \caption{Podstawowy schemat budowy kompilatora}
    \label{basic_compiler_flow}
\end{figure}

Na rysunku \ref{basic_compiler_flow} przedstawiony jest uproszczony schemat budowy kompilatora.
W kompilatorach ''produkcyjnych'' (np. GCC, Clang, czy ICC) tych faz jest więcej -- przede wszystkim etap
emisji kodu jest dużo bardziej rozbudowany, oraz dochodzą etapy analizy semantycznej (czy program ma sens) czy
optymalizacji (prób takiego przekształcenia kodu programu żeby przy zachowaniu znaczenia działał wydajniej).

Kompilator języka ViuAct dostarczany jako element tej pracy inżynierskiej jest pozbawiony etapów
analizy semantycznej oraz optymalizacji. Analiza semantyczna (oraz weryfikacja typów i wykrywanie błędów na
etapie kompilacji) jest oddelegowana do assemblera dostarczanego przez platformę. Optymalizacja jest
całkowicie pominięta gdyż jest to temat niezwykle rozległy; implementacja i doszlifowanie algorytmów
optymalizujących kod jest sama w sobie materiałem wystarczającym na napisanie osobnej pracy inżynierskiej.

Architektura kompilatora języka ViuAct jest dokładniej opisana w rozdziale
\ref{architektura_kompilatora_viuact} (\nameref{architektura_kompilatora_viuact}) na
stronie \pageref{architektura_kompilatora_viuact}.
Sposób działania kompilatora jest opisany w rozdziale \ref{opis_etapow_kompilacji}
(\nameref{opis_etapow_kompilacji}) na stronie \pageref{opis_etapow_kompilacji}.
Omówienie interakcji kompilatora języka ViuAct z narzędziami dostarczanymi przez platformę Viua VM znajduje
się w rozdziale \ref{lang_architektura_systemu} (\nameref{lang_architektura_systemu}) na stronie
\pageref{lang_architektura_systemu}.

Oprócz kompilatora (rozdział \ref{opis_kompilatora} na stronie \pageref{opis_kompilatora}) dostarczany jest
również ''program łączący'' (rozdział \ref{opis_linkera} na stronie \pageref{opis_linkera}) dokonujący
automatycznego połączenia wymaganych modułów w gotowy plik wykonywalny.

\subsection{Architektura systemu}
\label{lang_architektura_systemu}

Rysunek \ref{schemat_interakcji_viuact_z_viuavm} (''\nameref{schemat_interakcji_viuact_z_viuavm}'') prezentuje
schemat interakcji jakie zachodzą w całym systemie od momentu wczytania pliku źródłowego przez kompilator do
uruchomienia programu przez jądro Viua VM.

Ostatnią fazą jaką zajmuje się kompilator języka ViuAct dostarczany jako element tej pracy inżynierskiej jest
emisja kodu (''Assembly code emission''), której wynikiem jest plik z kodem źródłowym w języku assemblera Viua
VM (''\texttt{hello\_world.asm}'' na rysunku \ref{schemat_interakcji_viuact_z_viuavm}).
Rozdział \ref{architektura_kompilatora_viuact} (\nameref{architektura_kompilatora_viuact}) dokładniej opisuje
działanie samego kompilatora.

\begin{figure}[!htp]
    \centering
    \includegraphics[width=9cm]{viuact-pipeline}
    \caption{Interakcje: od pliku źródłowego do działającego programu}
    \label{schemat_interakcji_viuact_z_viuavm}
\end{figure}

Zakres pracy inżynierskiej obejmuje wygenerowanie pliku zawierającego poprawny kod w języku
assemblera Viua VM oraz plików pomocniczych (zadanie kompilatora), oraz takim pokierowaniu
narzędziami dostarczanymi przez platformę, żeby wyemitowały one plik wykonywalny bądź bibliotekę (zadanie
''programu łączącego''). Zakładamy, że narzędzia dostarczane przez platformę działają poprawnie.

Pliki pomocnicze są wymagane przez ''program łączący'' (opisany w rozdziale \ref{opis_linkera} na stronie
\pageref{opis_linkera}). Ich dokładniejsze opisy znajdują się w rozdziałach
''\nameref{pliki_interfejsow_modulow}'' na stronie \pageref{pliki_interfejsow_modulow} i
''\nameref{pliki_zaleznosci_modulow}'' na stronie \pageref{pliki_zaleznosci_modulow}

\subsection{Dekompozycja systemu na podsystemy}
\label{architektura_kompilatora_viuact}

Język ViuAct jest implementowany przez dwa programy:

\begin{enumerate}
    \item \textbf{kompilator} - który przetwarza kod źródłowy w języku ViuAct na kod źródłowy w języku
        assemblera Viua VM
    \item \textbf{linker} - który na podstawie wyników pracy kompilatora tworzy pliki wykonywalne, które
        mogą być uruchomione na jądrze Viua VM
\end{enumerate}

Większość pracy w tym układzie wykonuje kompilator, opisany w rozdziale \ref{opis_kompilatora} na stronie
\pageref{opis_kompilatora}. Generuje on pliki zawierające kod źródłowy w języku assemblera gotowe do
przetworzenia przez assembler Viua VM na formę binarną, oraz pliki pomocnicze.

Z plików pomocniczych korzysta zarówno sam kompilator (do określenia interfejsów modułów importowanych przez
aktualnie kompilowany moduł), ale też linker -- do określenia jakie moduły powinny być dołączone do aktualnie
emitowanego pliku wykonywalnego, aby zapewnić dostępność wszystkich wymaganych funkcji. Linker jest dokładniej
opisany w rozdziale \ref{opis_linkera} na stronie \pageref{opis_linkera}.

\subsubsection{Kompilator -- \texttt{viuact-cc}}
\label{opis_kompilatora}

Kompilator składa się z kilku podsystemów, zgodnie z
przedstawieniem na rysunku \ref{ogolny_schemat_kompilatora_viuact}.

\begin{figure}[!htp]
    \centering
    \includegraphics[width=10cm]{viuact-ogolny-schemat-kompilatora}
    \caption{Podział kompilatora na podsystemy}
    \label{ogolny_schemat_kompilatora_viuact}
\end{figure}

Każdy podsystem implementuje jedną z faz kompilacji:

\begin{enumerate}
    \item \textbf{lexer} dokonuje analizy leksykalnej wczytanego pliku źródłowego, dzieląc go na listę tokenów
    \item \textbf{parser} dokonuje analizy składniowej łącząc tokeny w grupy reprezentujące większe
        konstrukcje językowe
    \item \textbf{lowerer} mapuje grupy wyprodukowane przez \emph{parser} do odpowiednich funkcji
        \emph{emittera}; jest to dość banalny etap, ale upraszcza budowę kompilatora
    \item \textbf{emitter} tłumaczy konstrukcje językowe ViuAct na równoznaczne konstrukcje w języku
        assemblera Viua VM
\end{enumerate}

Różnica między podsystemami \emph{lowerer} i \emph{emitter} może być niejasna. Oba biorą udział w emisji
kodu wynikowego, ale \emph{lowerer} bezpośrednio zajmuje się tylko modułami i funkcjami, natomiast
\emph{emitter} implementuje emisję pojedynczych wyrażeń języka ViuAct -- przypisań \texttt{let}, konstrukcji
warunkowych \texttt{if}, wywołań funkcji, itd.

Proces kompilacji dokłaniej opisany jest w rozdziale \ref{opis_etapow_kompilacji}
(\nameref{opis_etapow_kompilacji}) na stronie \pageref{opis_etapow_kompilacji}.

\subsubsection{Program łączący -- \texttt{viuact-opt}}
\label{opis_linkera}

Program łączący (tzw. ''\emph{linker}'') zajmuje się finalną fazę ''kompilacji''.
Jest to stwierdzenie o tyle trafne, co niepoprawne. Zazwyczaj jednak nie ma to znaczenia, ponieważ zarówno
linker jak i kompilator jest ukrywany przed programistą. Popularne ''kompilatory'' jak np \emph{\texttt{g++}}
z GCC to tak naprawdę ''drivery''; wywołanie polecenia \emph{\texttt{g++}} powoduje wywołanie zarówno
kompilatora (\emph{\texttt{cc1}}), assemblera (\emph{\texttt{as}}), jak i linkera (\emph{\texttt{ld}}) w taki
sposób aby na wyjściu uzyskać oczekiwany wynik, czyli na przykład plik wykonywalny w formacie
ELF \emph{\texttt{a.out}}.

W przypadku kompilatora ViuAct proces ten wygląda podobnie, ale nie jest aż tak zautomatyzowany.
Rolę ''drivera'' pełni programista, który jest odpowiedzialny za wywołanie zarówno kompilatora jak i linkera.
Przykładowo:

\begin{lstlisting}
viuact-cc --mode exec ./hello_world.lisp
viuact-opt ./build/_default/hello_world.asm
\end{lstlisting}

Program łączący przeprowadzi proces asemblacji pliku podanego na wejściu, oraz dołączy do niego wszelkie
wymagane moduły. Zarówno asemblacja jak i łączenie będa przeprowadzone przez narzędzie dostarczane przez
platformę Viua VM -- \texttt{viuact-opt} zajmuje się jedynie wygenerowaniem odpowiednich poleceń dla tego
narzędzia.

Informacja o tym jakie moduły muszą zostać dołączone jest tworzona w oparciu o pliki zależności (opisane w
rozdziale \nameref{pliki_zaleznosci_modulow} na stronie \pageref{pliki_zaleznosci_modulow}).
Dla uproszczenia projektu program łączący nie zbiera informacji o zależnościach rekurencyjnie.

Po zebraniu informacji o zależnościach program łączący dokonuje asemblacji wszystkich modułów, od których
zależy kompilowany moduł główny. Następnie asembluje moduł główny i łączy wszystkie wyemitowane modułu
bytecode'u w gotowy plik wykonywalny.

\subsection{Przebieg procesu kompilacji}
\label{opis_etapow_kompilacji}

Ten rozdział zawiera dokładny opis procesu kompilacji, od momentu wczytania pliku z kodem źródłowym w języku
ViuAct do momentu wyemitowania kodu wynikowego w języku assemblera Viua VM. Kompilator jest wywoływany
poleceniem \texttt{viuact-cc} z opcją \texttt{-}\texttt{-mode} określającą czy kompilowany jest moduł wykonywalny
(\texttt{exec}) czy moduł biblioteki (\texttt{module}):

\begin{lstlisting}
viuact-cc --mode ( 'exec' | 'module' ) file.lisp
\end{lstlisting}

\subsubsection{Wczytanie pliku źródłowego}

Kompilator wczytuje do pamięci cały plik źródłowy jako pojedynczy string.

\subsubsection{Analiza leksykalna}

Lexer patrzy na wczytany kod źródłowy i do pierwszego nieprzeanalizowanego znaku (czyli na początku analizy do
znaku na indeksie 0) próbuje przypasować wzorzec określający jaki token znajduje się na tej pozycji. Po udanym
przypasowaniu pozycja, która będzie rozpatrywana przez lexer jest przesuwana o tyle znaków ile wynosi długość
wygenerowanego tokenu i lexer rozpoczyna pracę od nowa. Ten proces trwa do momentu aż cały string wejściowy
nie zostanie przeanalizowany, albo odrzucony jako nieprawidłowy.

Algorytm przypasowania jest banalny. Lexer dysponuje listą wzorców (określonych przez wyrażenia regularne),
które określają jak wygląda każdy możliwy token w języku. Lexer po kolei próbuje przypasować każdy wzorzec z
listy i kończy na pierwszym trafieniu. Jeśli żaden wzorzec nie może zostać przypisany lexer odrzuca kod
źródłowy jako nieprawidłowy.

Wzorce są uszeregowane w taki sposób żeby nie była możliwa pomyłka i
na przykład przypasowanie początku nazwy zmiennej \texttt{letter} jako słowa kluczowego \texttt{let}.

\subsubsection{Analiza składniowa}
\label{opis_etapow_kompilacji_analiza_skladniowa}

Składnia języka została zaprojektowana w taki sposób aby analiza składniowa mogła być uproszczona do maksimum
i prosta w implementaji.

\paragraph{Grupowanie nawiasów}

W pierwszej fazie analizy składniowej tokeny grupowane są wegług nawiasów okrągłych, przy czym grupowanie jest
rekurencyjne (jeśli jakaś grupa zawiera podgrupę w nawiasach to zagnieżdżona grupa będzie widoczna jako
pojedynczy element w grupie zewnętrznej).
Dla przykładu:

\begin{lstlisting}
(let x (frobnicate 42))
\end{lstlisting}

będzie zgrupowane w następujący sposób:

\begin{lstlisting}
[ "let"; "x"; [ "frobnicate"; "42" ] ]
\end{lstlisting}

\paragraph{Grupowanie id}

Kolejnym etapem jest grupowanie id. Id jest nazwą składającą się z kilku członów, na przykład
\texttt{Std.Posix.Network.socket} składa się z 7 tokenów: trzech \emph{nazw modułów} (\texttt{Str},
\texttt{Posix}, i \texttt{Network}), trzech \emph{operatorów dostępu} (kropek), i jednej \emph{nazwy}
(\texttt{socket}).
Taka grupa zostanie na tym etapie zredukowana do pojedynczego elementu.

\paragraph{Oznczanie wyrażeń złożonych}

Wyrażenia złożone składają się z kilku wyrażeń (prostych bądź złożonych). Z uwagi na fakt, że formą pośrednią
wykorzystywaną na etapie grupowania są listy tokenów takie wyrażenie byłoby nieodróżnialne od listy
reprezentującej wywołanie funkcji. Dlatego na etapie analizy składniowej do list reprezentujących wyrażenia
złożone dodawany jest specjalny token-fantom. Dzięki temu zostaje zachowana właściwość umożliwiająca szybkie
klasyfikowanie grup. Dla przykładu:

\begin{lstlisting}
(let x { ... })
\end{lstlisting}

będzie zgrupowane w następujący sposób:

\begin{lstlisting}
[ "let"; "x"; [ Compound_expression_marker; ... ] ]
\end{lstlisting}

\paragraph{Klasyfikacja grup}

Ostatnim etapem analizy składniowej jest klasyfikacja grup. W większości przypadków do klasyfikacji listy
tokenów do grupy reprezentującej konkretną konstrukcję językową wystarczy spojrzeć na pierwszy token na
liście. W niektórych przypadkach algorytm musi się posiłkować długością listy.

Dla przykładu:

\begin{lstlisting}
[ "let"; "x";          ... ]        -> let-binding
[ "let"; "x"; [ ... ]; ... ]        -> function-definition
[ ... ]                             -> function-call
[ "actor"; ... ]                    -> actor-call
[ Compound_expression_marker; ... ] -> compound-call
\end{lstlisting}

Różnicą między definicją zmiennej (\texttt{let-binding}), a definicją funkcji (\texttt{function-definition})
jest długość listy - definicja zmiennej zawiera trzy elementy (słowo kluczowe \texttt{let}, nazwę, i
wyrażenie), a definicja funkcji cztery elementy (słowo kluczowe \texttt{let}, nazwę, listę parametrów
formalnych, i wyrażenie).

\subsubsection{Emisja modułów}

W następnej kolejności emitowane są wszystkie moduły zagnieżdżone w aktualnie kompilowanym module, przy czym
ten etap postępuje rekurencyjnie. Moduły zagnieżdżone musżą być wyemitowane przed modułem głównym, aby
kompilator miał dostęp do ich plików interfejsów i umożliwić ich importowanie.

\subsubsection{Analiza importu modułów}

Kolejnym krokiem jest analiza modułów importowanych przez aktualnie kompilowany moduł i wczytanie ich
interfejsów. Kompilator ładuje listy sygnatur funkcji i wyliczenia z każdego zaimportowanego modułu.

Jeśli kompilator nie może znaleźć pliku interfejsu danego modułu to kończy kompilację informując o błędzie.
Kompilator szuka plików interfejsów i modułów w ścieżkach podanych w zmiennej środowiskowej
\texttt{VIUAC\_LIBRARY\_PATH} (opisanej na stronie \pageref{viuact_manual_env_viuac_library_path}).

\subsubsection{Emisja kodu wynikowego}
\label{opis_etapow_kompilacji_emisja_kodu_wynikowego}

Na końcu następuje emisja kodu wynikowego w języku assemblera Viua VM. Ten etap jest wykonywany osobno dla
każdej funkcji zdefiniowanej w kompilowanym module.

\paragraph{Redukcja poziomu wyrażeń}

Najpierw następuje redukcja poziomu wyrażeń. Tym etapem zajmuje się \emph{lowerer}. Jest to mechaniczny proces
mapujący sklasyfikowane grupy reprezentujące konretne konstrukcje językowe do funkcji udostępnianych przez
\emph{emitter}, opakowanie wyniku w sposób jakiego wymagają zasady języka assemblera Viua VM, oraz
serializacja wyników do stringów.

\paragraph{Emisja instrukcji języka assemblera}

Emisja instrukcji języka assemblera jest wykonywana per-wyrażenie. Ten etap przeplata się z redukcją poziomu
wyrażeń i jest implementowany przez \emph{emitter}. \emph{Emitter} emituje sekwencje instrukcji języka
assemblera Viua VM odpowiadające zadanym konstrukcjom językowym ViuAct.

Dla przykładu \texttt{(let x 42)} zostanie wyemitowane jako pojedyncza instrukcja: \texttt{integer \%x local
42}.  Natomiast \texttt{(Some\_module.frobnicate 42)} zostanie wyemitowane jako sekwencja instrukcji:

\begin{lstlisting}
integer %3 local 42
frame %1 arguments
copy %0 arguments %3 local
call void Some_module::frobnicate/1
\end{lstlisting}

\subsubsection{Zapis pliku \texttt{.asm}}

Dla każdego wyemitowanego modułu kompilator zapisze plik \texttt{\emph{nazwa\_modulu}.asm} zawierający kod
wynikowy w języku assemblera Viua VM.

\subsubsection{Zapis pliku \texttt{.i}}

Dla każdego wyemitowanego modułu biblioteki kompilator zapisze plik \texttt{\emph{nazwa\_modulu}.i}
zawierający definicję interfejsu tego modułu. Pliki interfejsów są opisane w rozdziale
\nameref{pliki_interfejsow_modulow} na stronie \pageref{pliki_interfejsow_modulow}.

\subsubsection{Zapis pliku \texttt{.d}}

Dla każdego wyemitowanego modułu kompilator zapisze plik \texttt{\emph{nazwa\_modulu}.d}
zawierający definicję zależności tego modułu. Pliki zależności są opisane w rozdziale
\nameref{pliki_zaleznosci_modulow} na stronie \pageref{pliki_zaleznosci_modulow}.


\section{Architektura systemu}
\label{chat_architektura_systemu}

Architektura systemu opiera się na klasycznym modelu klient-serwer.

Klienci będą uzyskiwali dostęp do usługi czatu za pośrednictwem przeglądarki
internetowej. Łącząc się z podanym adresem (IP lub domeny), na którym połączeń
nasłuchuje serwer, w pierwszej kolejności przeglądarka będzie próbować połączyć
się z nim przy użyciu protokołu http i standardowego portu 80, wysyłając do
niego żądanie metodą GET. Wówczas, serwer będzie zawsze odpowiadał statycznym
plikiem HTML, zawierającym odwołania do skryptów w języku JavaScript (JS) oraz
pozostałej, statycznej treści (np. grafiki czy arkusze stylów CSS). Serwer
będzie odsyłał te pliki do przeglądarki w odpowiedzi na kolejne żądania HTTP
GET, wysyłane w miarę dalszego renderowania pliku HTML. W ten sposób, po stronie
klienta zostanie pobrana i uruchomiona aplikacja internetowa typu Single Page
Application, której interfejs będzie reagował z użytkownikiem oraz ulegał
zmianom wskutek działania skryptów JS, załadowanych na pierwszym etapie
uruchomienia. Po stronie serwera, dostarczaniem treści statycznych będzie
zajmował się daemon HTTP - Nginx.

Gdy tylko skrypty JS wykryją pobranie wszystkich plików składowych aplikacji
z serwera, podjęta zostanie próba nawiązania połączenia z tym serwerem przy
użyciu protokołu WebSocket. Będzie on od tego momentu podstawowym kanałem
komunikacji pomiędzy klientem a serwerem.

Zgodnie ze standardem WebSocketu, zanim zostanie nawiązane właściwe połączenie,
powinno dojść do „uścisku dłoni” (ang. \textit{handshake}) pomiędzy klientem a
serwerem. W związku z tym, pierwsza próba połączenia również zostanie podjęta
przy użyciu protokołu http, jednakże tym razem pod innym, dedykowanym portem
(w naszym przypadku będzie to port 8000), a także zawierać nagłówki wskazujące
na żądanie zmiany używanego protokołu na WebSocket, jego wersję oraz klucz
(„Sec-WebSocketKey”). Serwer udzieli wówczas odpowiedzi ze swoim własnym
kluczem, informując o zmianie stosowanego protokołu na WebSocket.

W chwili prawidłowego rozpoczęcia połączenia WebSocket, aplikacja po stronie
klienta wyświetli użytkownikowi okno autoryzacyjne. Wpisane tam dane zostaną
następnie przesłane do serwera. Po jego stronie, komunikat zostanie zdekodowany
przez aktora \texttt{WSConnector} i przekazany powiązanemu aktorowi
\texttt{Authorizer}. Jego zadaniem będzie weryfikacja przedstawionych informacji
oraz podjęcie decyzji o autoryzacji lub jej odmowie. Decyzja ta jest odsyłana do
\texttt{WSConnectora} i następnie przekazywana do aplikacji po stronie klienta.

Jezeli autentykacja przebiegnie pomyślnie, aktor \texttt{Authorizer} uruchamia
aktora \texttt{UserSession}, spina go z aktorem \texttt{WSConnector} używanym
wcześniej do komunikacji z frontendem, oraz ulega autodestrukcji.

\newpage

\subsection{Dekompozycja systemu na podsystemy}
\label{architektura_chatu}

\subsubsection{Strona serwera (,,backend'')}
Na serwer czatu składa się grupa współdziałających, ale zupełnie odrębnych od
siebie aktorów, ukazanych na rysunku \ref{diag-komp}

\nameref{diag-komp}.
\begin{figure}[!htp]
	\centering
	\includegraphics[width=\textwidth]{chat/fig/pck-diag}
	\caption{Diagram komponentów serwera ViuaChat}
	\label{diag-komp}
\end{figure}

\begin{labeling}{UsernameLessor}

  \item[\texttt{Architect}] Uruchamiany jako pierwszy wraz z całym
  serwerem, a następnie inicjalizuje i nadzoruje aktorów \texttt{WSListener},
  \texttt{UsernameLessor} oraz \texttt{Landlord}. W razie nieprawidłwego
  działania lub wyłączenia któregokolwiek z tych trzech głównych aktorów,
  \texttt{Architect} automatycznie zainicjuje przeładowanie całego serwera.
  Ponadto, \texttt{Architect} potrafi w momencie uruchamiania serwera odczytać
  jego pliki konfiguracyjne i na tej podstawie należycie skonfigurować pokoje
  oraz administratorów. Zawsze występuje w jednym egzemplarzu.

  \item[\texttt{WSListener}] Odpowiada za nasłuchiwanie na porcie 8000 po
  stronie serwera, a przy każdej próbie połączenia będzie tworzyć kolejnego,
  niezależnego aktora \texttt{WSInitializer}. Występuje zawsze w jednym
  egzemplarzu.

  \item[\texttt{WSInitializer}] Odpowiada za  realizację ,,uścisku dłoni'' i
  formułowanie odpowiedzi zwrotnej po stronie serwera w odniesieniu do
  połączenia na konkretnym gnieździe. W razie prawidłowego nawiązania
  połączenia, aktor ten utworzy kolejną parę aktorów, \texttt{WSConnector} oraz
  \texttt{Authorizer}, zaś WSInitializer ulegnie samozniszczeniu.

  \item[\texttt{WSConnector}] Odpowiada za dalszą, bezpośrednią obsługę
  przydzielonego gniazda. Występuje ich tylu, ile jest otwartych połącznień.
  Jego rolą będzie również kodowanie i dekodowanie wiadomości (ang.
  ,,messages''), czyli podstawowych logicznych jednostek informacji, które są
  używane przy połączeniach z użyciem protokołu WebSocket. Pilnuje również, czy
  połączenie nie zostało zerwane oraz inicjuje zamykanie sesji użytkownika.

  \item[\texttt{Authorizer}] Wymienia wiadomości od \texttt{WSConnectora}, który
  został uruchomiony wraz z nim i odpowiada za należytą autentykację i/lub
  autoryzację użytkownika w usłudze czatu. Egzemplarz aktora tego typu jest
  powoływany dla każdego otwartego połączenia bez nawiązanej sesji. Aby dokonać
  autoryzacji, aktor \texttt{Authorizer} kontaktuje się \texttt{UsernameLessor}.
  Jeżeli autentykacja przebiegnie pomyślnie, aktor \texttt{Authorizer} uruchamia
  aktora \texttt{UserSession}, spina go z aktorem \texttt{WSConnector} używanym
  wcześniej do komunikacji z frontendem, oraz ulega autodestrukcji.

  \item[\texttt{UsernameLessor}] Jego zadaniem jest zarządzanie informacjami na
  temat tymczasowych nazw użytkowników, należących do	użytkowników bez stałych
  kont (ich gromadzenie, udzielanie, weryfikacja, dbanie o unikalność), a także
  weryfikacja tożsamości kont administratorów z dodatkowym użyciem hasła.
  Występuje w jednym egzemplarzu, przez cały czas istnienia serwera. Ponadto,
  nadzoruje działanie aktorów \texttt{Authorizer} oraz \texttt{UserSession}.

  \item[\texttt{UserSession}] Przejmuje komunikację z użyciem
  \texttt{WSConnector}, pozwalając na zwyczajne użytkowanie czatu. Aktor
  ,,UserSession'' gromadzi informacje na temat nazwy oraz poziomu uprawnień
  użytkownika, a także tego, z jakim pokojem jest obecnie spięty.

  \item[\texttt{Landlord}] Jego zadaniem jest współudział w podpinaniu
  użytkowników do pokoju, tworzeniem nowych i usuwaniem istniejących pokojów, a
  także utrzymywanie i udostępnianie kompletnej listy aktywnych pokojów.

  \item[\texttt{Room}] Działa jak router wiadomości i przechowuje listę użytkowników którzy są do niego wpięci. Istnieje w tylu egzemplarzach, ile jest aktywnych pokojów.

  \item[\texttt{MessageCache}] Przechowuje i odtwarza 10 najnowszych wiadomości wysłanych do pokoju. Występuje po jednym egzemplarzu dla każdego aktywnego aktora \texttt{Room}.

\end{labeling}

\newpage

\subsubsection{Warstwa interfejsu użytkownika (,,frontend'')}
Podczas pracy nad wartwą frontendu, zastosowano framework webowy Vue.js. Jedną
z przyczyn dla tej decyzji jest możliwość zdekomponowania projektowanej aplikacji na mniejsze części, nazywane modułami (ang. \textit{modules}). Są one zorganizowane hierarchiczne. Każdy z modułów zawiera własny skrypt JavaScript, a także kod HTML i arkusz CSS. Powoduje to, że każdy z modułów jest niezależny od pozostałych i może realizować swoje zadania w pełni autonomicznie.

Każdy z modułów udostępnia swojemu rodzicowi pewne określone parametry. Ich zmiana
jest podstawowym sposobem na interakcję pomiędzy nimi, co jest zgodne z
paradygmatem \textit{data driven application} (z ang. ,,aplikacja sterowana
poprzez dane''). W podobny sposób następuje zmiana kodu HTML modułów. Zamiast
zmieniać węzły DOM w sposób jawny poprzez skrypt, programista wskazuje w
szablonie HTML te miejsca, które ulegają określonym przemianom wraz ze
zmian wewnętrznych parametrów modułu.

Podstawowa struktura aplikacji frontendowej przewiduje podział na
pięć części:
\begin{itemize}
	\item Część autoryzacyjna -- służy do nawiązania połączenia z
	serwerem i rozpoczęcie sesji użytkownika.

	\item Pokoje (publiczne) -- dostarcza listę ogólnodostępnych pokojów,
	umożliwia podpięcie się wybranych z nich oraz prowadzenie rozmów z
	innymi użytkownikami, którzy się do nich podpięli.

	\item Wiadomości prywatne -- umożliwia nawiązywanie prywatnych
	rozmów pomiędzy użytkownikami oraz wymianę wiadomości prywatnych.

	\item Narzędzia administracyjne -- pozwalają administratorom
	na dodawanie i usuwanie pokojów, wyrzucanie użytkowników z pokojów
	i z serwera.

	\item Profil użytkownika -- umożliwia podejrzenie informacji na
	temat własnego konta, zmianę przez administratora swojego hasła
	użytkownika oraz rozłączenie się z serwerem.

\end{itemize}

W projekcie przewidziano zastosowanie modułów, przedstawionych na rysunku \ref{diag-komp-front}.

\nameref{diag-komp-front}.
\begin{figure}[!htp]
	\centering
	\includegraphics[width=\textwidth]{chat/fig/pck-diag-front}
	\caption{Diagram komponentów aplikacji webowej ViuaChat}
	\label{diag-komp-front}
\end{figure}

\begin{labeling}{\texttt{UserProfile}}

  \item[\texttt{App}] Nadrzędny moduł, istniejący przez cały czas użycia
	instancji aplikacji. Obejmuje najbardziej zewnętrzne struktury HTML,
	przechowuje podstawowy stan aplikacji (stan autoryzacji, nazwę
	użytkownika, poziom uprawnień, gniazdo WebSocket dla połączeń
	z wartwą backendu), a także zapewnia routing do głównych części
	aplikacji - czyli modułu logowania, modułu pokojów (publicznych),
	pokojów prywatnych wiadomości, profilu użytkownika i narzędzi
	administracyjnych.

	\item[\texttt{SignIn}] Moduł odpowiadający za początkową autoryzację
	do serwera (potocznie znane jako ,,logowanie'').

	\item[\texttt{Rooms}] Moduł reprezentuje część aplikacji poświęconą
	ogólnodostępnym pokojom dla wspólnych konwersacji. Podlegają mu:

	\begin{labeling}{\texttt{RoomsChat}}
		\item[\texttt{RoomsChat}] Moduł odpowiedzialny \textit{stricte} za
		okno czatu - wyświetlanie konwersacji oraz wysyłanie wiadomości
		do pokoju.

		\item[\texttt{RoomsList}] Moduł wyświetlający listę pokojów, a także
		umożliwiający podpięcie się do wybranego z nich.

	\end{labeling}

	\item[\texttt{PM}] Moduł odwzorowujący całą część aplikacji poświęconą
	wymianie pomiędzy użytkownikami wiadomości prywatnych. W jego skład
	wchodzą:

	\begin{labeling}{\texttt{PMUserList}}
		\item[\texttt{PMFindUser}] Moduł, którego zadaniem jest odnalezienie
		użytkowika, do którego ma zostać wysłana wiadomość prywatna

		\item[\texttt{PMUserList}] Moduł obsługujący listę użytkowników,
		którzy wcześniej otrzymali lub wysłali wiadomości prywatne.

		\item[\texttt{PMChat}] Moduł odpowiedzialny za obsługę zasadniczego
		okna czatu wiadomości prywatnych, pozwalającego wymieniać wiadomości
		prywatne z jednym, wybranym użytkownikem.
	\end{labeling}

\item[\texttt{UserProfile}] Moduł pozwalający na podstawową kontrolę
własnego profilu, w tym:

\begin{labeling}{\texttt{ChangePassword}}
	\item[\texttt{SignOut}] Moduł odpowiedzialny za zamknięcie sesji
	użytkownika, rozłączenie z serwerem oraz powrót aplikacji do
	stanu sprzed autoryzacji

	\item[\texttt{ChangePassword}] Moduł mający umożliwiać administratorom
	zmianę swoich własnych haseł

\end{labeling}

\item[\texttt{Admin}] Moduł obejmujący pod sobą wszystkie narzędzia
administracyjne. W jego skład wchodzą:

\begin{labeling}{\texttt{BanFromServer}}
	\item[\texttt{AdminRooms}] Moduł do zarządzania ogólnodostępnymi
	pokojami czatu, w tym:

	\begin{labeling}{\texttt{AdminRoomsList}}
		\item[\texttt{AdminRoomsList}] Moduł listy wszystkich ogólnodostępnych
		pokojów z narzędziami do ich edycji

		\item[\texttt{CreateRoom}] Moduł pozwalający na dodawanie pokojów
		do serweru czatu

		\item[\texttt{RemoveRoom}] Moduł pozwalający na usuwanie istniejących
		pokojów z serwera
	\end{labeling}

	\item[\texttt{BanFromRoom}] Moduł służący do wyrzucania użytkowników
	z pokojów, w których obecnie się znajdują.

	\item[\texttt{BanFromServer}] Moduł pozwalający na wyrzucenie
	użytkowników z serwera.
\end{labeling}

\end{labeling}
\section{Decyzje projektowe}

\subsection{Środowisko docelowe}

\subsection{Środowisko implementacji}

\subsection{Priorytety implementacyjne}

\section{Projekt algorytmów i przyjętych protokołów}

\subsection{Protokół frontend-backend}
Komunikacja pomiędzy frontendem a backendem...

\section{Projekt rozwiązań sprzętowych}

\section{Projekt interfejsu}

\subsection{Interfejs użytkownika}

\subsubsection{Założenia konstrukcji interfejsu}

\section{Projekt bazy danych}

\section{Opis implementacji}


\section{Testowanie}

Testy kompilatora.

\subsection{Zestaw przypadków testowych}

Jeden testowy program na każdą funkcjonalność języka.
Kilka większych testowych programów sprawdzających integrację języka, np. wielomodułowych, wykorzystujących
moduły obce.

\subsection{Wykonanie testów}

Opis tego jak wygląda uruchomienie testów, w jaki sposób został zbudowany framework, itp.

\subsection{Trudności w testowaniu}

Niedeterminizm wynikający z równoległego działania aktorów stwarza problemy w testach. Trzeba uciekać się do
"sztuczek", np. sortowanie wyników programu testowego.

\section{Instrukcja użytkownika kompilatora języka Viuact}
\label{viuact_manual}

Tradycja nakazuje, aby pierwszym programem jaki pisze się w nowym języku był program, który wypisze na ekran
napis ,,\emph{Hello World!}''. W ViuAct ten program wygląda następująco:

\begin{lstlisting}
(let main () {
    (print "Hello World!")
    0
})
\end{lstlisting}

Aby skopilować ten program, należy wykonać w konsoli następujące polecenia:

\begin{lstlisting}
$ viuact-cc --mode exec ./hello_world.lisp
$ viuact-opt ./build/_default/hello_world.asm
\end{lstlisting}

Kod wykonywalny (\emph{bytecode} wykonywalny przez Viua VM) będzie umieszczony w pliku
\texttt{hello\_world.bc} w katalogu \texttt{./build/\_default}.
Aby go uruchomić należy użyć jądra Viua VM:

\begin{lstlisting}
$ viua-vm ./build/_default/hello_world.bc
%*\emph{Hello World!}*)
$
\end{lstlisting}

Nazwy plików pośrednich są wywodzone z nazwy pliku źródłowego:

\begin{description}
    \item[\texttt{\emph{example}.lisp}] plik z kodem źródłowym w języku ViuAct
    \item[\texttt{\emph{example}.asm}] plik wynikowy kompilatora, z kodem źródłowym w języku assemblera Viua
        VM
    \item[\texttt{\emph{example}.bc}] plik wynikowy assemblera Viua VM, zawierający wykonywalny bytecode
\end{description}

Pliki \texttt{.asm} i \texttt{.bc} są umieszczane w katalogu \texttt{./build/\_default}.

\subsection{Opcje kompilatora}

Jedyną opcją kompilatora jest \texttt{--mode}, która przyjmuje dwie możliwe wartości:

\begin{description}
    \item[\texttt{exec}] jeśli plik źródłowy definiuje moduł wykonywalny
    \item[\texttt{module}] jeśli plik źródłowy definiuje moduł biblioteczny
\end{description}

\subsection{Zmienne środowiskowe}

Zachowanie kompilatora można częściowo zmodyfikować ustawiając zmienne środowiskowe.

\subsubsection{\texttt{DEFAULT\_OUTPUT\_DIRECTORY}}

Kontroluje katalog, w którym kompilator składuje pliki wynikowe. Domyślnie pliki wynikowe są składowane w
katalogu \texttt{./build/\_default}.

\subsubsection{\texttt{VIUAC\_LIBRARY\_PATH}}

Jak \texttt{LD\_LIBRARY\_PATH}.

\subsubsection{\texttt{VIUA\_ASM\_PATH}}

Kontroluje ścieżkę do assemblera Viua VM.

\subsubsection{\texttt{VIUAC\_VERBOSE}}

Wartość \texttt{true} powoduje wyświetlenie komunikatów podczas kompilacji.

\subsubsection{\texttt{VIUAC\_DEBUGGING}}

Wartość \texttt{true} włącza komunikaty debugowania.

\subsubsection{\texttt{VIUAC\_INFO}}

Wartość \texttt{true} włącza dodatkowe komunikaty informacyjne.

\subsubsection{\texttt{VIUAC\_DUMP\_INTERMEDIATE}}

Wartość \texttt{tokens} powoduje zrzut strumienia tokenów do pliku \texttt{\emph{example}.tokens}.
Wartość \texttt{exprs} powoduje zrzut drzewa składni do pliku \texttt{\emph{example}.expressions}.
Można podać obie wartości, oddzielone przecinkiem.

\subsection{Opcje programu łączącego}

