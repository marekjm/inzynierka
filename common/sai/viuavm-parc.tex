\documentclass[11pt,oneside,a4paper,titlepage,onecolumn]{article}

\usepackage[utf8]{inputenc}
\usepackage{textcomp}
\usepackage[official]{eurosym}
\usepackage[polish]{babel}
\usepackage{amsthm}
\usepackage{graphicx}
\usepackage[T1]{fontenc}
\usepackage{scrextend}
\usepackage{hyperref}
\usepackage{xcolor}
% \usepackage{nameref}
% \usepackage{showlabels}
% \usepackage{titlesec}
\usepackage{geometry}
\usepackage{tabularx}
\geometry{a4paper, portrait, margin=2cm}

\setcounter{secnumdepth}{4}

%% Author and title
% \author{Marek Marecki \and Gr.52c \and Kod: 95 \and 2017\slash2018-2019}
\author{Marek Marecki \and Krzysztof Franek}
\title{
     Viua VM w akcji \\
    \large Implementacja wysokopoziomowego języka programowania i
    realizacja prostej aplikacji \\
    ~\\
    Wstępna analiza kategorii ryzyka wg SODiS SPA}

\begin{document}

\maketitle

\section{Otoczenie projektu}

\subsection{Kontekst projektu}

\begin{tabularx}{\textwidth}{|l|X|l|X|}
    \hline
    Lp. & Pytanie & Odpowiedź & Uwagi \\ \hline
    
    1
    & \textbf{Czy wybrano metodologię rozwoju projektu?} 
    & Tak 
    & Wybrano dwie metodologie: (a) prototypowanie dla kompilatora oraz (b) Mini-Scrum, tj. uproszczona wersja 
    Scrum dostosowana do jednoosobowej pracy. Są one opisane szczegółowo we Wstępnym Planie Projektu (WPP). 
    \\\hline
    
    2
    & \textbf{Czy istnieje zarys struktury projektu (lub Work Breakdown Structure)?}
    & Tak
    & Projekt został podzielony na dwie części, jedną poświęconą językowi programowania wysokiego poziomu ViuAct 
    wraz z jego kompilatorem, a drugą - przykładowej aplikacji czatu. Zadania są podzielone według tych części. 
    Podział opisano szczegółowo w ramach WPP.
    \\\hline
    
    3
    & \textbf{Czy produkty projektu zostały zdefiniowane, uzgodnione i zapisane?} 
    & Tak
    & Zdefiniowano je w jasny i przejrzysty sposób, w dokumentacji uzgodnionej w zespole i merytorycznie 
    uzgadnianej z opiekunem projektu.
    \\\hline
    
    4
    & \textbf{Czy ustanowiono środki finansowego nadzoru projektu?} 
    & --
    & Pytanie to nie dotyczy projektu, z uwagi na jego charakter non-profit.
    \\\hline
    
    5
    & \textbf{Czy każde z zadań posiada przejrzystą dokumentację?}
    & Tak
    & Brakuje na obecnym etapie szczegółowej dokumentacji wykonawczej dla niektórych z modułów czatu, pozostając 
    na etapie dokumentacji koncepcyjnej.
    \\\hline
    
    6
    & \textbf{Czy zidentyfikowano istotnych udziałowców?}
    &  Tak
    &  W przypadku języka programowania i środowiska wytwarzania oprogramowania ViuAct są to realni udziałowcy: 
    programista używający środowiska, użytkownik korzystający z programów wytworzonych w środowisku, zespół 
    deweloperski, opiekun projektu, recenzent. Tymczasem, w przypadku czatu, dochodzą fikcyjni udziałowcy, utworzeni 
    wyłącznie na cele przykładowe.
    \\\hline
    
    7
    & \textbf{Czy projekt dysponuje utrwalonym, stabilnym i powtarzalnym środowiskiem deweloperskim?} 
    &  Tak
    & Poziom ryzyka niestabilności jest wyższy w projekcie dotyczącym przygotowania aplikacji czatu. Wszakże samo 
    środowisko, w którym jest tworzona, jest przedmiotem pracy dyplomowej i może ulegać dynamicznym zmianom.
    \\\hline
    
    8
    & \textbf{Czy istnieje proces zarządzania zmianą?} 
    & Tak
    & Opiera się on o: (a) bieżące spotkania zespołu i dyskusję nad zmianami, dokumentowane w formie notatek 
    (b) wariantowanie i przygotowywanie różnych alternatywnych dróg realizacji poszczególnych etapów, zanim będzie 
    niezbędne podjęcie samej decyzji w zakresie zmiany.
    \\\hline

\end{tabularx}

\subsection{Plan projektu}

\begin{tabularx}{\textwidth}{|l|X|l|X|}
    \hline
    Lp. & Pytanie & Odpowiedź & Uwagi \\ \hline
    
    1
    & \textbf{Czy referencje deweloperów zostały sprawdzone i uznane za godne akceptacji?} 
    & Tak
    & Obu z deweloperów miało wcześniej związek z programowaniem w wielu językach programowania, ponadto programista 
    odpowiedzialny za język programowania i kompilator ma szeroką wiedzę na temat budowy kompilatorów i konstrukcji
    maszyn wirtualnych.
    \\\hline

	2
	& \textbf{Czy doszło to uformowania zespołu deweloperskiego?}
	& Tak
	& Obu członków zespołu ustaliło kanały komunikacji, zasady współpracy, podział odpowiedzialności oraz na bieżąco 
	współdziała w zakresie realizacji projektu.
	\\\hline

	3
	& \textbf{Czy członkowie zespołu deweloperskiego posiadają istotne doświadczenia w podobnych projektach?}
	& Tak
	& Członek odpowiedziany za nowy język programowania i kompilator ma doświadczenie w podobnych projektach 
	realizowanych samodzielnie.
	\\\hline
	
	4
	& \textbf{Czy plan testów, rozmieszczenia, szkolenia i utrzymywania są przejrzyście udokumentowane?} 
	& Nie
	& Zaniechano sporządzenia ww. dokumentacji.
    
    \\\hline
\end{tabularx}

\section{Analiza zadań związanych z projektem}

\subsection{Zadanie \textit{Utworzenie specyfikacji języka}}

\begin{tabularx}{\textwidth}{|l|X|l|X|}
    \hline
    Lp. & Pytanie & Odpowiedź & Uwagi \\ \hline
    
    1 
    & Czy tworzenie języka programowania ma jasno określone 
    rezultaty?
    & Tak
    & 
    \\\hline
    
    2
	& Czy tworzenie języka programowania zostało uregulowane
	przez wiążące umowy pomiędzy udziałowcami?
	& Nie 
	& Nie ma żadnych wiążących umów
	\\\hline
	
	3
	& Czy tworzenie języka programowania poprzedzono wstępną analizą
	ryzyka lub planem unikania ryzyka?
	& Nie
	& Przeprowadzono wstępną analizę ryzyka dla całego projektu, a
	nie dla konkretnego zadania
	\\\hline
	
	4
	& Czy w tworzeniu języka programowania bierze udział 
	zidentyfikowana, kompletna lista uczestników?
	& Tak
	& Jedynym autorem jest Marek Marecki
	\\\hline
	
	5
	& Czy w tworzenie języka programowania zaangażowano deweloperów 
	z wymaganą wiedzą techniczną?
	& Tak
	& Marek Marecki dysponuje wymaganą wiedzą techniczną
	\\\hline
	
	6
	& Czy w tworzenie języka programowania zaangażowano deweloperów 
	z doświadczeniem w zagadnieniach, których dotyczy to zadanie?
	& Tak
	& Marek Marecki już wcześniej implementował języki programowania
	\\\hline
	
	7
	& Czy tworzenie języka programowania ma adekwatny plan testów?
	& Tak
	& Pozostałe części projektu -- kompilator i aplikacja czatu --
	mają za zadanie przetestować prawidłowość parsera
	\\\hline	
	
	8
	& Czy tworzenie języka programowania objęto udokumentowaną
	specyfikacją?
	& Tak
	& Opracowano \textit{Specyfikację języka ViuAct} 
	\\\hline

\end{tabularx}

\subsection{Zadanie \textit{Implementacja podsystemu emisji kodu}}

\begin{tabularx}{\textwidth}{|l|X|l|X|}
	\hline
	
	\hline
    Lp. & Pytanie & Odpowiedź & Uwagi \\ \hline
    
    1 
    & Czy implementacja podsystemu emisji kodu ma jasno określone 
    rezultaty?
    & Tak
    & Produkt jest jasno określony
    \\\hline

    2
	& Czy implementacja podsystemu emisji kodu została uregulowana
	przez wiążące umowy pomiędzy udziałowcami?
	& Tak
	& Wytyczono przejrzystą specyfikację języka ViuAct oraz języka
	asemblera Viua VM
	\\\hline	
	
	3
	& Czy implementację podsystemu emisji kodu poprzedzono wstępna
	analizą	ryzyka lub planem unikania ryzyka?
	& Nie
	& Przeprowadzono wstępną analizę ryzyka dla całego projektu, a
	nie dla konkretnego zadania
	\\\hline
	
	4
	& Czy w implementacji podsystemu emisji kodu bierze udział 
	zidentyfikowana, kompletna lista uczestników?
	& Tak
	& Jedynym autorem jest Marek Marecki
	\\\hline

	5
	& Czy w implementację podsystemu emisji kodu zaangażowano
	deweloperów z wymaganą wiedzą techniczną?
	& Tak
	& Marek Marecki dysponuje wymaganą wiedzą techniczną
	\\\hline
	
	6
	& Czy w implementację podsystemu emisji kodu zaangażowano
	deweloperów z doświadczeniem w zagadnieniach, których dotyczy 
	to zadanie?
	& Tak
	& Marek Marecki już wcześniej implementował języki programowania
	\\\hline

	7
	& Czy implementacja podsystemu emisji kodu ma adekwatny plan
	testów?
	& Tak
	& Asembler i jądro Viua VM przetestują, czy kod jest emitowany
	prawidłowo, aplikacja czatu przetestuje czy wymagane konstrukcje
	języka mogą być prawidłowo wyemitowane
	\\\hline	
	
	8
	& Czy implementacja podsystemu emisji kodu objęto udokumentowaną
	specyfikacją?
	& Tak
	& Opracowano \textit{Specyfikację języka ViuAct} oraz 
	\textit{Viua VM ISA} (specyfikację języka asemblera Viua VM)
	\\\hline
	
\end{tabularx}
	
\subsection{Zadanie \textit{Implementacja infrastruktury dla systemu modułów}}

\begin{tabularx}{\textwidth}{|l|X|l|X|}
	\hline
	
	\hline
    Lp. & Pytanie & Odpowiedź & Uwagi \\ \hline
    
    1 
    & Czy implementacja infrastruktury dla systemu modułów ma jasno 
    określone rezultaty?
    & Tak
    & Produkt jest jasno określony
    \\\hline

    2
	& Czy implementacja infrastruktury dla systemu modułów została
	uregulowana	przez wiążące umowy pomiędzy udziałowcami?
	& Tak
	& Wytyczono przejrzystą specyfikację języka ViuAct
	\\\hline	
	
	3
	& Czy implementację infrastruktury dla systemu modułów
	poprzedzono wstępna	analizą	ryzyka lub planem unikania ryzyka?
	& Nie
	& Przeprowadzono wstępną analizę ryzyka dla całego projektu, a
	nie dla konkretnego zadania
	\\\hline
	
	4
	& Czy w implementacji infrastruktury dla systemu modułów bierze
	udział zidentyfikowana, kompletna lista uczestników?
	& Tak
	& Jedynym autorem jest Marek Marecki
	\\\hline

	5
	& Czy w implementację infrastruktury dla systemu modułów
	zaangażowano deweloperów z wymaganą wiedzą techniczną?
	& Tak
	& Marek Marecki dysponuje wymaganą wiedzą techniczną
	\\\hline
	
	6
	& Czy w implementację infrastruktury dla systemu modułów
	zaangażowano deweloperów z doświadczeniem w zagadnieniach,
	których dotyczy to zadanie?
	& Tak
	& Marek Marecki już wcześniej implementował języki programowania
	\\\hline
   
	7
	& Czy implementacja infrastruktury dla systemu modułów ma
	adekwatny plan testów?
	& Tak
	& Aplikacja czatu przetestuje, czy system modułów jest właściwy
	a Viua VM przetestuje czy da się go skompilować w dopuszczalny
	sposób.
	\\\hline	
	
	8
	& Czy implementacja infrastruktury dla systemu modułów objęto
	udokumentowaną specyfikacją?
	& Tak
	& Opracowano \textit{Specyfikację języka ViuAct} oraz 
	\textit{Viua VM ISA} (specyfikację języka asemblera Viua VM)
	\\\hline
	
\end{tabularx}

\subsection{Implementacja sterownika kompilatora}

\begin{tabularx}{\textwidth}{|l|X|l|X|}
	\hline
	
	\hline
    Lp. & Pytanie & Odpowiedź & Uwagi \\ \hline
    
    1 
    & Czy implementacja sterownika kompilatora ma jasno 
    określone rezultaty?
    & Tak
    & Produkt jest jasno określony
    \\\hline

    2
	& Czy implementacja sterownika kompilatora została
	uregulowana	przez wiążące umowy pomiędzy udziałowcami?
	& Nie
	& Nie ustalono w szczegółowy sposób, jak ma wyglądać kompilator.
	\\\hline	
	
	3
	& Czy implementację sterownika kompilatora
	poprzedzono wstępna	analizą	ryzyka lub planem unikania ryzyka?
	& Nie
	& Przeprowadzono wstępną analizę ryzyka dla całego projektu, a
	nie dla konkretnego zadania
	\\\hline
	
	4
	& Czy w implementacji sterownika kompilatora bierze
	udział zidentyfikowana, kompletna lista uczestników?
	& Tak
	& Jedynym autorem jest Marek Marecki
	\\\hline

	5
	& Czy w implementację sterownika kompilatora
	zaangażowano deweloperów z wymaganą wiedzą techniczną?
	& Tak
	& Marek Marecki dysponuje wymaganą wiedzą techniczną
	\\\hline
	
	6
	& Czy w implementację sterownika kompilatora
	zaangażowano deweloperów z doświadczeniem w zagadnieniach,
	których dotyczy to zadanie?
	& Tak
	& Marek Marecki już wcześniej implementował języki programowania
	\\\hline
   
	7
	& Czy implementacja sterownika kompilatora ma
	adekwatny plan testów?
	& Tak
	& Przewidziano liczne przykładowe fragmenty kodu, zaś aplikacja
	czatu przetestuje, czy dostarczone narzędzia kompilacji są
	właściwe. 
	\\\hline	

	& Czy implementacja sterownika kompilatora objęto
	udokumentowaną specyfikacją?
	& Tak
	& Opracowano dokumentację i manual użytkownika.
	\\\hline
\end{tabularx}
    
\section{Udziałowcy. Analiza udziałowców (wstępna analiza)}

\subsection{Udziałowiec \textit{Członkowie zespołu projektowego}}

\begin{tabularx}{\textwidth}{|l|X|l|X|}
	\hline
	
	\hline
    Lp. & Pytanie & Odpowiedź & Uwagi \\ \hline
    
    1
    & Czy cykl życia projektu i produktu jest zrozumiały i
    akceptowalny przez członków zespołu projektowego?
    & Tak
    & Członkowie wspólnie uzgodnili cykl życia produktu/projektu
    \\\hline
    
    2
    & Czy plan projektu jest akceptowalny dla członków zespołu
    projektu?
    & Tak
    & Krzysztof Franek opracował Wstępny Plan Projektu (WPP), 
    uzgodniony i zaakceptowany przez obu członków zespołu
    \\\hline
    
    3
    & Czy norma ukończenia projektu jest zrozumiała dla członków
    zespołu projektowego?
    & Tak
    & Norma jest w równym stopniu zrozumiała dla obu członków zespołu
    \\\hline
    
    4
    & Czy potrzeby i wymagania zostały przejrzyście wyrażone
    przez członków zespołu projektowego
    & Tak
    & Potrzeby i wymagania projektu są niejako ,,zapisane'' w
    specyfikacji języka
    \\\hline
    
    5
    & Czy plan Testu Akceptacji Użytkowników (z ang. UAT) będzie
    działał na rzecz członków zespołu projektowego?
    & Tak
    & Cały program testów został opracowany przez członków zespołu
    i działa na ich rzecz
    \\\hline
    
    6
    & Czy członkowie zespołu projektowego rozumieją konsekwencje i
    skutki projektu?
    & Tak
    & Członkowie zespołu projektowego doskonale znają wpływ projektu
    na siebie i na innych udziałowców
    \\\hline
    
    7
    & Czy członkowie zespołu projektowego potrzebują dodatkowego
    przeszkolenia?
    & Nie
    & Nie ma takiej potrzeby.
    \\\hline
    
    8
    & Czy istnieje plan dalszego utrzymywania po tym, gdy
    potrzeby członków zespołu projektowego zostaną zaspokojone?
    & Tak
    & Zaplanowano rezygnację z oprogramowania lub przepisanie go 
    w taki sposób, aby odciąć powiązania z PJATK
    \\\hline
    
\end{tabularx}

\subsection{Udziałowiec \textit{Promotor}}

\begin{tabularx}{\textwidth}{|l|X|l|X|}
	\hline
	
	\hline
    Lp. & Pytanie & Odpowiedź & Uwagi \\ \hline
    
    1
    & Czy cykl życia projektu i produktu jest zrozumiały i
    akceptowalny przez promotora?
    & Tak
    & Promotor jest na bieżąco angażowany w prace nad
    produktem/projektem.
    \\\hline
    
    2
    & Czy plan projektu jest akceptowalny dla promotora?
    & Tak
    & Promotor zapoznał się z WPP i zaakceptował jego brzmienie.
    \\\hline
    
    3
    & Czy norma ukończenia projektu jest zrozumiała dla członków
    zespołu projektowego?
    & Tak
    & Promotor sam zaproponował rys normy ukończenia projektu i
    aktywnie brał udział w jej dopracowaniu.
    \\\hline
    
    4
    & Czy potrzeby i wymagania zostały przejrzyście wyrażone
    przez promotora?
    & Tak
    & Promotor od początku istnienia projektu proponował wymagania
    i zadania do zrealizowania w ramach projektu. Zespół projektowy
    zna również oczekiwania promotora w stosunku do nich oraz
    projektu
    \\\hline
    
    5
    & Czy plan Testu Akceptacji Użytkowników (z ang. UAT) będzie
    działał na rzecz promotora?
    & Tak
    & Bez akceptacji promotora, projekt nie zakończy się sukcesem.
    \\\hline
    
    6
    & Czy promotor rozumie konsekwencje i skutki projektu?
    & Tak
    & Promotor zaproponował rys projektu w oczekiwaniu na 
    zaspokojenie luk i problemów, które miałby on rozwiązać i
    w taki sposób postrzega on potencjalne skutki wynikające
    z projektu.
    \\\hline
    
    7
    & Czy promotor potrzebują dodatkowego przeszkolenia?
    & Tak
    & Bieżąca prezentacja dokumentacja projektu oraz powstającego
    kodu może zastąpić regularne szkolenie.
    \\\hline
    
    8
    & Czy istnieje plan dalszego utrzymywania po tym, gdy
    potrzeby promotora zostaną zaspokojone?
    & Nie
    & Promotor nie wyartykułował, w jaki sposób chciałby sam
    stosować wytworzone oprogramowanie po zaliczeniu projektu.
    \\\hline
    
\end{tabularx}

%TASKS/STAKEHOLDERS ANALYSIS (PRELIMINARY ANALYSIS)
%
%1. Yes. (task budget is acceptable to stakeholder)
%2. Yes. (task time line is acceptable to stakeholder)
%3. Yes. (task will integrate; we have everything coded from scratch to suit the
%   project)
%4. Yes. (stakeholder accpts task implementation, maintenance, and training plan)


\end{document}
