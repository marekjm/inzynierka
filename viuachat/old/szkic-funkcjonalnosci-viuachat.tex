\documentclass[11pt,oneside,a4paper,onecolumn]{article}

\usepackage[utf8]{inputenc}
\usepackage{textcomp}
\usepackage[official]{eurosym}
\usepackage[polish]{babel}
\usepackage{amsthm}
\usepackage{graphicx}
\usepackage[T1]{fontenc}
\usepackage{scrextend}
\usepackage{hyperref}
\usepackage{xcolor}
% \usepackage{nameref}
% \usepackage{showlabels}
% \usepackage{titlesec}
\usepackage{geometry}
\geometry{a4paper, portrait, margin=2cm}
\graphicspath{ {./fig/} }
\usepackage{listings}

\setcounter{secnumdepth}{4}
\title{Szkic funkcjonalności Viua Chat}

\begin{document}


\maketitle

System użytkowników
\begin{enumerate}
\item Podczas wejścia na czat, użytkownikowi pokazuje się monit z polami do wpisania nazwy użytkownika i hasła
\item Użytkowników dzielimy na
  \begin{enumerate}
  \item Użytkowników z kontem – podczas logowania podają nazwę i odpowiadające mu hasło
  \item Użytkowników bez konta – podczas logowania podają tylko nazwę użytkownika, pole hasła pozostaje puste
  \end{enumerate}
\item Konta użytkowników są gromadzone na serwerze w postaci trójek: login, hasło (md5), admin/nieadmin (bool)
\item Login to ciąg od 3 do 32 alfanumerycznych znaków a-zA-Z0-9
\item Nie można rozpocząć sesji użytkownika o nazwie, która pasuje do istniejącego konta, jeżeli nie zostanie podane prawidłowe hasło (nie można podszywać się pod nazwy użytkowników z istniejącym kontem)
\item Można rozpocząć sesję jako użytkownik bez konta, pod warunkiem, że zadeklarowana nazwa nie będzie powtarzać się z nazwami kont użytkowników ani z już zalogowanymi użytkownikami bez konta
\item W okienkach czatu, loginy użytkowników z kontem są pogrubione i pokolorowane: nieadmini na zielono, admini na czerwono
\item Admini mają prawo
\begin{enumerate}
\item Tworzyć i usuwać pokoje
\item Ustanawiać, zmieniać i usuwać hasła do pokojów
\item Wyrzucać użytkowników z pokoi i z serwera
\item Tworzyć, usuwać, zmieniać hasła, zmieniać uprawnienia kont
\end{enumerate}
\item Użytkownicy z kontami mają prawo zmieniać swoje hasło
\end{enumerate}

Pokoje
\begin{enumerate}
\item Pokoje to właściwe czaty – tam użytkownicy mogą wejść i pisać do siebie nazwajem
\item Każdy pokój ma unikalną nazwę będącą ciągiem alfanumerycznym a-zA-Z0-9\_ od 3 do 32 znaków, oraz (opcjonalnie) hasło dostępu
\item Lista pokojów jest widoczna po zalogowaniu się do serwera czatu
\item Użytkownik może być równocześnie wpięty do jednego pokoju
\item Pokój może mieć ustanowione hasło, które użytkownik musi wpisać przed podpięciem się do niego
\item Wiadomość wysłana w pokoju jest widoczna w oknie pokoju dla wszystkich użytkowników podpiętych do tego pokoju
\item Użytkownik może się samodzielnie wypiąć z pokoju, do którego jest wpięty, lub może to zrobić admin
\end{enumerate}

Prywatne wiadomości
\begin{enumerate}
\item Oprócz okien czatu dla każdego z wpiętych pokojów, użytkownik dysponuje dodatkowym oknem, na którym widzi wiadomości prywatne.
\item Aby wysłać wiadomość prywatną, należy poprzedzić ją znakiem \# i nazwą użytkownika, do którego jest kierowana wiadomość, oddzielona spacją od komunikatu, np.: „\#user Tajna wiadomość”. Trafia ona wówczas do okna wiadomości prywatnych.
\item Wiadomość prywatna może być wysłana z okna czatu pokoju –wiadomości prywatne nie są pokazywane wszystkim uczestnikom czatu, a jedynie użytkownikowi, do którego jest adresowana.
\item W oknie czatu pokoju dopuszczalne jest wysyłanie wiadomości prywatnych do dowolnych użytkowników, nawet tych, którzy nie są w danym momencie podpięci do tego pokoju.
\item Z okna wiadomości prywatnych można wysyłać wyłącznie wiadomości prywatne.
\item Próba wysłania wiadomości prywatnej powinna zostać odrzucona przez serwer i zwrócić błąd w przypadku gdy
\begin{enumerate}
\item Po znaku „\#” pojawi się od razu znak spacji lub nie będzie żadnych znaków (nie zostanie podana nazwa użytkownika który jest adresatem)
\item Nazwa adresata jest dłuższa niż 32 znaki
\item Na serwerze nie istnieje użytkownik z nazwą użytą jako nazwa adresata
\item Adresat wiadomości ma swoje konto na serwerze, ale nie jest zalogowany
\end{enumerate}
\end{enumerate}

\end{document}