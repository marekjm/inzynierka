\documentclass[11pt,oneside,a4paper,twocolumn]{article}

\usepackage[utf8]{inputenc}
\usepackage{textcomp}
\usepackage[official]{eurosym}
\usepackage[polish]{babel}
\usepackage{amsthm}
\usepackage{graphicx}
\usepackage[T1]{fontenc}
\usepackage{scrextend}
\usepackage{hyperref}
\usepackage{xcolor}
% \usepackage{nameref}
% \usepackage{showlabels}
% \usepackage{titlesec}
\usepackage{geometry}
\geometry{a4paper, portrait, margin=2cm}
\graphicspath{ {./fig/} }
\usepackage{listings}

\setcounter{secnumdepth}{4}
\title{Abstrakt prezentacji inżynierskiej ViuaVM}

\begin{document}

\maketitle

\begin{enumerate}
	\item Wprowadzenie
	\begin{enumerate}
		\item Akademicki kontekst projektu
		\item Skład zespołu
	\end{enumerate}
	
	\item Przedstawienie problemu
	\begin{enumerate}
		\item Problem niezawodności
		\item Wykrywanie błędów
		\item Obsługa błędów
		\item Izolacja części systemu
		\item Problem równoległości
		\item Wielowątkowość we współczesnym sprzęcie
		\item Podsumowanie problemu
	\end{enumerate}
	
	\item Rozwiązania
	\begin{enumerate}
		\item Communicating Sequential Processes (CSP)
		\item Calculus of Communicating Systems (CCS)
		\item $\pi$-caluculus
		\item Actor model
		\item Podsumowanie propozycji
		\item Actor model - uzasadnienie decyzji
	\end{enumerate}		
	
	\item ViuaVM
	\begin{enumerate}
		\item Czym jest ViuaVM
		\item Czym się wyróżnia ViuaVM
		\item Rejestry vs Klasyczna pamięć
		\item Model programowania -- wielowątkowość ,,by design''
		\item Język asemblera ViuaVM
		\item Trudność języka asemblera z perspektywy programisty
	\end{enumerate}
	
	\item Produkt pracy - Środowisko ViuaVM
	\begin{enumerate}
		\item Przedstawienie środowiska
		\item Cele
		\item Kontekst
		\item Minimalny zakres funkcjonalności
		\item Wymagania jakościowe
		\item Wymagania wydajnościowe
		\item Wymagania narzędziowe
	\end{enumerate}
	
	\item Język ViuAct
	\begin{enumerate}
		\item Zakres pracy na tle modelu kompilacji
		\item ViuAct jako język wysokiego poziomu
		\item Założenia do składni języka
		\item Składnia ,,lispopodobna''
		\item Idea wyrażeń
		\item Typy danych
		\begin{enumerate}
			\item Typy proste
			\item Typy złożone i typy platformy
			\item Konstrukcje języka
			\item Dowiązania \texttt{let}
		\end{enumerate}				
		\item Operatory
		\item Zmienne
		\item Funkcje
		\begin{enumerate}
			\item Funkcje
			\item Wywołania funkcji
			\item Gdzie są pętle - czyli wywołania ogonowe (\texttt{tailcall})
			\item Wywołania odroczone (\texttt{defer})
		\end{enumerate}
		\item Aktorzy
		\begin{enumerate}
			\item Pojęcie aktora w ViuAct
			\item Tworzenie aktora
			\item Włączanie aktorów (\texttt{fork-join})
			\item Wolni aktorzy
		\end{enumerate}
		\item Konstrukcje warunkowe
		\item Wyjątki
		\item Moduły
		\item Biblioteki
		\begin{enumerate}
			\item Rozszerzanie zakresu funkcjonalności
			\item Biblioteka standardowa \texttt{Std}
			\item Moduł \texttt{Std.Actor}
			\item Moduł \texttt{Std.Posix}
			\item Moduł \texttt{Std.Random}
		\end{enumerate}
		
		\item Hello World!
		\item Ciąg Fibonacciego
		\item Prosty czat
		\begin{enumerate}
			\item Serwer
			\item Klient
			\item Działający przykład
		\end{enumerate}
	\end{enumerate}
	
	\item Kompilator ViuAct
	\begin{enumerate}
		\item Umiejscowienie kompilatora w środowisku
		
		\item Ogólny schemat procesu kompilacji
		\begin{enumerate}
			\item Analiza leksykalna
			\item Analiza składniowa
			\item Emisja kodu
		\end{enumerate}
		
		\item Architektura kompilatora ViuAct
		\begin{enumerate}
			\item Schemat procesu kompilacji
			\item Analiza leksykalna (\texttt{Lexer})
			\item Grupowanie nawiasów
			\item Kategoryzacja grup
			\item Emisja modułów
			\item Redukcja poziomu wyrażeń (\texttt{Lowerer})
			\item Emisja instrukcji języka asemblera (\texttt{Emitter})
		\end{enumerate}
		
		\item Pliki wynikowe
		\begin{enumerate}
			\item Pliki \texttt{.asm}
			\item Pliki \texttt{.i}
			\item Pliki \texttt{.d}
\end{enumerate}				
		
		\item Cechy charakterystyczne kompilatora ViuAct [na tle innych kompilatorów]
		\item Interakcja z platformą
		\begin{enumerate}
			\item Platforma kompilacji
			\item Platforma asemblacji i linkowania
			\item Platforma uruchomieniowa
		\end{enumerate}
	\end{enumerate}

\item ViuaChat jako aplikacja demonstracyjna
\begin{enumerate}
	\item Kontekst biznesowy 
	\item Udziałowcy
	\begin{enumerate}
		\item ViuaVM
		\item Opiekun pracy inżynierskiej
		\item Członek zespołu odpowiedzialny za kompilator ViuAct
		\item Członek zespołu odpowiedzialny za ViuaChat
	\end{enumerate}
	\item Charakterystyka użytkowników
	\item Istniejąca infrastruktura
	\item Zarys funkcjonalności
	\begin{enumerate}
		\item Logowanie
		\item System użytkowników
		\item Pokoje
		\item Prywatne wiadomości
	\end{enumerate}
	\item Metodologia wytwarzania -- mini-Scrum
	\begin{enumerate}
		\item Przyczyny wyboru
		\item Różnice w stosunku do ,,klasycznego'' Scrum
		\item Sprint
		\item Zasady biznesowe jako \textit{user stories}
		\item Przebieg sprintu
	\end{enumerate}
	\item Schemat przypadków użycia
	\item Projekt systemu
	\begin{enumerate}
		\item ViuaChat jako aplikacja webowa
		\item Podział frontend -- backend
		\item Frontend jako \textit{Single Page Application}
		\item Backend
		\item WebSocket
	\end{enumerate}

	\item Aktorzy ViuAct
	\begin{enumerate}
		
		\item Podstawowi aktorzy serwera
		\item Aktor inicjujący \texttt{Architect}
		\item Aktor nasłuchujący \texttt{Listener}
		\item Aktorzy pośredniczący z WebSocketem \texttt{WSConnector}
		\item Aktor uwierzytelniający \texttt{Authorizer}
		\item Aktor zarządzający nazwami użytkowników \texttt{UsernameLessor}
		\item Aktorzy sesji użytkowników \texttt{UserSession}
		\item Aktorzy (typowych) pokojów \texttt{Room}
		\item Aktorzy pokojów wiadomości prywatnych \texttt{PMRoom}
		\item Aktorzy bufora najnowszych wiadomości w pokojach \texttt{MessageBuffer}
		\item Aktor zarządzający pokojami \texttt{Landlord}
	\end{enumerate}
	\item Pliki konfiguracyjne
	\item Przykładowe procesy w diagramach przepływu
	\begin{enumerate}
		\item Nawiązywanie połączenia z serwerem
		\item Autoryzacja
		\item Podpinanie użytkownika do pokoju
		\item Wysyłanie i odbiór wiadomości w (typowym) pokoju
		\item Odpinanie użytkownika od pokoju
		\item Wysyłanie wiadomości prywatnej
		\item Opuszczanie serwera
	\end{enumerate}
	
	\item Projekty interfejsu - wybrane przykłady
	\begin{enumerate}
		\item Założenia interfejsu
		\item Okno logowania
		\item Błąd zajętości nazwy użytkownika
		\item Okno z listą pokojów
		\item Okno z wiadomościami z (typowego) pokoju
		\item Redagowanie wiadomości w (typowym) pokoju
		\item Okno tworzenia nowego pokoju		
		\item Okno z listą aktywnych użytkowników
		\item Okno wiadomości prywatnych
	\end{enumerate}
\end{enumerate}

\end{enumerate}

Czego zapomniałem:
	
\begin{enumerate}
	\item Kompilator: Infrastrutktura, metodologia wytwarzania,  prototypowanie, issue,  zasady, jezyk w którym zostanie/został napisany (Python, OCaml), testy, sposób użycia (komendy CLI)
	\item Cały projekt: harmonogram, ocena ryzyka, zasoby sprzętowe
	\item ViuaVM: wpływ projektu na kształt maszyny, biblioteki		
	\item Pogrubienie nagłówków sekcji prezentacji
\end{enumerate}


\end{document}


