\documentclass[11pt,oneside,a4paper,titlepage,onecolumn]{article}

\usepackage[utf8]{inputenc}
\usepackage{textcomp}
\usepackage[official]{eurosym}
\usepackage[polish]{babel}
\usepackage{amsthm}
\usepackage{graphicx}
\usepackage[T1]{fontenc}
\usepackage{scrextend}
\usepackage{hyperref}
\usepackage{xcolor}
% \usepackage{nameref}
% \usepackage{showlabels}
% \usepackage{titlesec}
\usepackage{geometry}
\geometry{a4paper, portrait, margin=2cm}
\graphicspath{ {./fig/} }

\newenvironment{enumreq}
{ \begin{enumerate}[topsep=0pt,itemsep=-1ex,partopsep=1ex,parsep=1ex] }
{ \end{enumerate}                  } 


\setcounter{secnumdepth}{4}

%% Author and title
\author{Marek Marecki \and Krzysztof Franek}
\title{%
    Proving viability of Viua VM \\
    \large Implementation of high-level language on Viua VM\\
    and deployment of simple application \\
    ~\\
    Projekt Systemu\\
    dla języka Viuact}

\begin{document}

\maketitle
{\footnotesize
\begin{center}
  \begin{tabular}{ | l | l | l | }
    \hline
    \parbox[t]{6.5cm}{\textbf{Temat pracy i akronim projektu:}\\Proving viablity of Viua VM (VVIA)} & \parbox[t]{4.5cm}{\textbf{Zleceniodawca:}\\\colorbox{yellow}{Nieznany}} & \parbox[t]{4.5cm}{\textbf{Konsultant:}\\\colorbox{yellow}{Nieznany}} \\ \hline
    \parbox[t]{6.5cm}{\textbf{Zespół projektowy:}\\Krzysztof Franek, Marek Marecki} & \parbox[t]{4.5cm}{\textbf{Kierownik projektu:}\\Marek Marecki} & \parbox[t]{4.5cm}{\textbf{Opiekun projektu:}\\dr hab. Marek A. Bednarczyk, prof. PJWSTK} \\ \hline
    \parbox[t]{3.5cm}{\textbf{Kierownik projektu:}\\Marek Marecki} &
      \multicolumn{2}{|l|}{\parbox[t]{9cm}{\textbf{Odpowiedzialny za dokument:}\\Marek Marecki}} \\ 
    \hline
  \end{tabular}
\end{center}
}

\tableofcontents
\newpage

\section{Architektura}

\subsection{Użyte wzorce projektowe -- Sposób konstrukcji kompilatora}

\emph{Ogólny sposób konstrukcji kompilatora. Prosty diagram.}

\subsection{Architektura systemu}

Kompilator -> assembler -> VM

\subsection{Dekompozycja systemu na podsystemy}

lexer -> parser -> lowerer -> emitter

\subsection{Opis interakcji pomiędzy systemami}

Co po kolei robi kompilator.
Od wczytania pliku z kodem źródłowym do wyemitowania kodu wynikowego w języku assemblera.

\section{Projekt struktury}

\subsection{Diagram klas}

Brak diagramu klas, ponieważ jest zbędny (program nie jest pisany w stylu obiektowym).

\subsection{Opis zmian i uszczegółowień w stosunku do diagramu analitycznego}

\section{Wykorzystane elementy}

\emph{Wykorzystane komponenty, programy, itp.}

\section{Decyzje projektowe}

\subsection{Środowisko docelowe}

\subsection{Środowisko implementacji}

\subsection{Priorytety implementacyjne}

Maksymalizacja prostoty budowy kompilatora i języka.
Marginalizacja obsługi błędów w kompilatorze z uwagi na brak czasu.
Marginalizacja optymalizacji z uwagi na brak czasu.

\section{Projekt algorytmów i przyjętych protokołów}

\section{Projekt rozwiązań sprzętowych}

Brak w tym projekcie. Jest on wyłącznie softwareowy.

\section{Projekt interfejsu}

\subsection{Interfejs użytkownika}

\subsubsection{Założenia konstrukcji interfejsu}

\subsection{Interfejs kompilatora}

Kompilator składa się z dwóch programów: \texttt{viuact-cc} (kompilatora właściwego) i \texttt{viuact-opt}
(programu łączącego).

\subsection{Interfejs języka}

Interfejsem języka jest jego składnia.
Jest ona opisana w specyfikacji języka.

\subsection{Inne interfejsy}

\subsubsection{Pliki interfejsów modułów (\texttt{.i})}

\subsubsection{Pliki zależności modułów (\texttt{.d})}

\section{Projekt bazy danych}

Brak bazy danych w projekcie.

\section{Opis implementacji}

Krótka dyskusja jak przyjęte rozwiązania projektowe spełniają wymagania, szczególnie jakościowe, np.
bezpieczeństwo.

\section{Załączniki}

Specyfikacja języka ViuAct -- ''\emph{viuact-specification.pdf}''.

\end{document}
