\documentclass[11pt,oneside,a4paper,titlepage,onecolumn]{article}

\usepackage[utf8]{inputenc}
\usepackage{textcomp}
\usepackage[official]{eurosym}
\usepackage[polish]{babel}
\usepackage{amsthm}
\usepackage{graphicx}
\usepackage[T1]{fontenc}
\usepackage{scrextend}
\usepackage{hyperref}
\usepackage{xcolor}
% \usepackage{nameref}
% \usepackage{showlabels}
% \usepackage{titlesec}
\usepackage{geometry}
\geometry{a4paper, portrait, margin=2cm}
\graphicspath{ {./fig/} }

\newenvironment{enumreq}
{ \begin{enumerate}[topsep=0pt,itemsep=-1ex,partopsep=1ex,parsep=1ex] }
{ \end{enumerate}                  } 


\setcounter{secnumdepth}{4}

%% Author and title
\author{Marek Marecki \and Krzysztof Franek}
\title{%
    Proving viability of Viua VM \\
    \large Implementation of high-level language on Viua VM\\
    and deployment of simple application \\
    ~\\
    Specyfikacja Wymagań Systemowych\\
    dla języka Viuact}

\begin{document}

\maketitle
{\footnotesize
\begin{center}
  \begin{tabular}{ | l | l | l | }
    \hline
    \parbox[t]{6.5cm}{\textbf{Temat pracy i akronim projektu:}\\Proving viablity of Viua VM (VVIA)} & \parbox[t]{4.5cm}{\textbf{Zleceniodawca:}\\\colorbox{yellow}{Nieznany}} & \parbox[t]{4.5cm}{\textbf{Konsultant:}\\\colorbox{yellow}{Nieznany}} \\ \hline
    \parbox[t]{6.5cm}{\textbf{Zespół projektowy:}\\Krzysztof Franek, Marek Marecki} & \parbox[t]{4.5cm}{\textbf{Kierownik projektu:}\\Marek Marecki} & \parbox[t]{4.5cm}{\textbf{Opiekun projektu:}\\dr hab. Marek A. Bednarczyk, prof. PJWSTK} \\ \hline
    \parbox[t]{3.5cm}{\textbf{Kierownik projektu:}\\Marek Marecki} &
      \multicolumn{2}{|l|}{\parbox[t]{9cm}{\textbf{Odpowiedzialny za dokument:}\\Marek Marecki}} \\ 
    \hline
  \end{tabular}
\end{center}
}

\section{Wprowadzenie}

\subsection{Cel dokumentu}
Celem dokumentu jest zdefiniowanie wymagań dla języka ViuAct i kompilatora tego języka na podstawie analizy projektu.

\subsection{Zakres dokumentu}
Niniejszy dokument jest produktem pierwszego etapu procesu wytwórczego czatu ViuaChat, na który składają się:
\begin{itemize}
    \item analiza otoczenia, wraz z z klientami;
    \item wskazanie kontekstu biznesowego systemu;
    \item określenie udziałowców;
\end{itemize}

Praca wygenerowana w systemie \LaTeX.

\subsection{Dokumenty powiązane}
\begin{itemize}
	\item Specyfikacja języka ViuAct
\end{itemize}

\subsection{Odbiorcy}

Odbiorcami dokumentu są członkowie zespołu.
Jest on wyznacznikiem tego co jest wymagane od języka Viuact, a zatem tego co można wykorzystać przy tworzeniu
programu ViuaChat.

Kolejną grupą adresatów niniejszego dokumentu są pracownicy uczelni, odpowiedzialni za nadzór nad prawidłowym ukształtowaniem i przebiegiem projektu.
Wśród nich, szczególną rolę odgrywa JE Dziekan ZWI, prof. Marek Bednarczyk, będący opiekunem projektu.

\subsection{Słownik pojęć}
\begin{description}
  \item[ViuAct] Język programowania wysokiego poziomu, oparty o model aktorów, powstały na potrzeby niniejszego projektu inżynierskiego
\end{description}

\section{Język ViuAct i jego kompilator w kontekście}

\subsection{Kontekst biznesowy}

\subsection{Udziałowcy}

\subsection{Klienci}

\subsection{Charakterystyka użytkowników}

\subsection{Istniejąca infrastruktura}

\section{Wymagania}

\subsection{Wymagania ogólne i dziedzinowe}

Doprecyzowanie celów projektu, prezentacja przedsięwzięć zdefiniowanych przez Kartę Projektu i
Dokument Założeń Wstępnych.

\vspace{1em}

\begin{tabular}{ | l | l | }
    \hline
    \textbf{Identyfikator} & \parbox[t]{11cm}{WD-01} \\
    \hline
    \textbf{Priorytet} & S \\
    \hline
    \textbf{Nazwa} & Udowodnienie przydatności (ang. \emph{viability}) Viua VM \\
    \hline
    \textbf{Opis} & \parbox[t]{11cm}{
        Udowodnienie, że maszyna wirtualna Viua może być celem kompilacji dla
        języków wyższego poziomu oraz jest możliwe uruchomienie na niej nietrywialnego
        oprogramowania (w przypadku tego projektu będzie to ViuaChat).} \\
    \hline
    \textbf{Udziałowiec} & \phantom{} \\
    \hline
    \textbf{Wymagania powiązane} & \phantom{} \\
    \hline
\end{tabular}

\vspace{1em}

\begin{tabular}{ | l | l | }
    \hline
    \textbf{Identyfikator} & \parbox[t]{11cm}{WD-02} \\
    \hline
    \textbf{Priorytet} & M \\
    \hline
    \textbf{Nazwa} & Projekt języka ViuAct \\
    \hline
    \textbf{Opis} & \parbox[t]{11cm}{
        Wymagane jest zaprojektowanie języka programowania wyższego poziomu oraz przygotowanie
        jego specyfikacji.
    } \\
    \hline
    \textbf{Udziałowiec} & \phantom{} \\
    \hline
    \textbf{Wymagania powiązane} & WD-01 \\
    \hline
\end{tabular}

\vspace{1em}

\begin{tabular}{ | l | l | }
    \hline
    \textbf{Identyfikator} & \parbox[t]{11cm}{WD-03} \\
    \hline
    \textbf{Priorytet} & M \\
    \hline
    \textbf{Nazwa} & Implementacja komilatora \\
    \hline
    \textbf{Opis} & \parbox[t]{11cm}{
        Wymagana jest implementacja kompilatora przetwarzającego kod źródłowy napisany w języku
        wyższego poziomu (zaprojektowanym w punkcie WD-02) na kod w języku assemblera Viua VM.
    } \\
    \hline
    \textbf{Udziałowiec} & \phantom{} \\
    \hline
    \textbf{Wymagania powiązane} & WD-02 \\
    \hline
\end{tabular}

\vspace{1em}

\begin{tabular}{ | l | l | }
    \hline
    \textbf{Identyfikator} & \parbox[t]{11cm}{WD-04} \\
    \hline
    \textbf{Priorytet} & M \\
    \hline
    \textbf{Nazwa} & Chat \\
    \hline
    \textbf{Opis} & \parbox[t]{11cm}{
        Musi powstać implementacja czatu napisana w języku zaprojektowanym w punkcie WD-02.
    } \\
    \hline
    \textbf{Udziałowiec} & \phantom{} \\
    \hline
    \textbf{Wymagania powiązane} & \parbox[t]{11cm}{
        WD-03 (\emph{do wykonania tego wymagania niezbędny jest kompilator})
    } \\
    \hline
\end{tabular}

\subsection{Wymagania funkcjonalne}

Część wymagań funkcjonalnych języka ViuAct jest opisana w osobnym dokumencie: ''Specyfikacja języka ViuAct''.
Poniżej opisane są wymagania, które nie mogły być zawarte bezpośrednio w specyfikacji języka (np. sposób
obsługi i raportowania błędów przez kompilator).

\subsubsection{Funkcjonalności}

% \begin{tabular}{ | l | l | }
% 	\hline
% 		\textbf{Identyfikator} &
% 		WF-07
% 		\\

% 	\hline
% 		\textbf{Treść} & \parbox[t]{11cm}{
% 			Jako użytkownik serwera czatu, chcę odfiltrować
% 			wiadomości prywatne od jednego użytkownika, aby
% 			prowadzić z nim ciągłą konwersację.
% 		}\\

% 	\hline
% 		\parbox[t]{4cm}{\textbf{Powiązane zasady biznesowe}} & \parbox[t]{11cm}{
% 			ZP-09 Serwer czatu automatycznie wysyła do pokoju
% 			wiadomości, zawierające powiadomienia o wydarzeniach
% 			związanych z pokojem, tzw. wiadomości systemowe \\
% 			ZP-12 Wiadomość systemowa zostaje wysłana podczas
% 			wpięcia wypięcia użytkownika z pokoju \\
% 			ZP-13 Wiadomość systemowa zostaje wysłana, gdy użytkownik
% 			wpięty do pokoju traci połączenie z serwerem \\
% 			ZP-14 Wiadomość systemowa zostaje wysłana, gdy użytkownik
% 			zostaje wyrzucony z pokoju

% 		}\\

% 	\hline
% 		\parbox[t]{4cm}{\textbf{Kryteria akceptacji}} & \parbox[t]{11cm}{
% 			\begin{enumreq}
% 				\item Niezwłocznie po wypięciu się użytkownika z
% 				pokoju, serwer wyśle wiadomość systemową, widoczną
% 				dla wszystkich użytkowników wpiętych do tego pokoju,
% 				o treści:
% 				\begin{enumerate}
% 					\item ,,Użytkownik ... opuścił pokój'', gdy
% 					użytkownik samodzielnie wypiął się z pokoju
% 					\item ,,Użytkownik ... stracił połączenie'',
% 					gdy użytkownik został wypięty z pokoju na skutek
% 					przerwania sesji z uwagi na zerwanie połączenia
% 					\item ,,Użytkownik ... został wyrzucony'', gdy
% 					użytkownik został wypięty wskutek interwencji
% 					administratora
% 				\end{enumerate}
% 			\end{enumreq}
% 			}
% 		\\

% 	\hline
% \end{tabular}

\subsubsection{Interfejs z otoczeniem}

\subsection{Wymagania niefunkcjonalne}

\subsection{Wymagania na środowisko docelowe}

\subsection{Wymagania dotyczące procesu wytwarzania}

\section{Kryteria akceptacji rozwiązania}

\section{Odwołania do literatury}

\section{Załączniki}

% \section{Język ViuAct w kontekście}

% \subsection{Kontekst biznesowy}

% Niniejszy czat stanowi część szerszego kontekstu, jakim jest potrzeba zademonstrowania działania języka ViuAct oraz całego środowiska wytwórczego powiązanego z maszyną wirtualną ViuaVM.

% Cel demonstracyjny jest pierwszym i najważniejszym, jaki przyświeca konstrukcji czatu. Ponadto, sam proces wytwórczy pozwoli
% przetestować wydajność całego środowiska w jego praktycznym wymiarze. Tym samym, możliwe będzie poprawienie konstrukcji kompilatora 
% lub zastosowanych konstrukcji językowych ViuAct, podnoszących jego użyteczność.

% Wszelcy odbiorcy dla aplikacji czatu zostaną, podobnie jak sama aplikacja, skonstruowani na cele demonstracyjne. Nie powinni oni
% odbiegać od modelowych odbiorców podobnych komunikatorów, tak, aby potencjalny, poczatkujący użytkownik środowiska ViuaVM mógł
% zrozumieć intencje stojące za rozwiązaniami zastosowanymi w ViuaChat oraz przenieść je do swoich pierwszych programów, opracowanych
% w tym środowisku.

% \subsection{Udziałowcy}

% Poniżej wyszczególniono udziałowców, mających wpływ na rozwój czatu.

% \begin{tabular}{ | l | l | }
  
% 	\hline
% 	\multicolumn{2}{ | l | }{\textbf{Karta udziałowca}}  \\
  
% 	\hline
%     \parbox[t]{3cm}{
%     	\textbf{Identyfikator}
%     } & UN-01 \\  
    
%     \hline
%     \parbox[t]{3cm}{
%     	\textbf{Nazwa}
%     } & ViuaVM \\  
    
%     \hline
%     \parbox[t]{3cm}{
%     	\textbf{Opis}
%     } & \parbox[t]{12cm}{
%     	Maszyna wirtualna, oparta o przechowywanie danych w rejestrach zamiast \textit{płaskich} tablic pamięci. Stanowi ona 
%     	platformę, na której musi zostać uruchomiony serwer czatu. Ponieważ jej największym atutem jest zorientowanie na kod
%     	wykonywany współbieżnie, sam serwer czatu powinien tę cechę wykorzystywać w maksymalnym stopniu. 
%     	} \\ 
    
%     \hline
%     \parbox[t]{3cm}{
%     	\textbf{Typ}
%     } & Nieożywiony, bezpośredni \\  
    
%     \hline
%     \parbox[t]{3cm}{
%     	\textbf{Punkt widzenia}
%     } & \parbox[t]{12cm}{
%     	ViuaVM jest absolutnie nieodzownym elementem projektu, a serwer czatu stanowi przede wszystkim dowód jej użyteczności.
%     	O ile jądro maszyny nie ma być poddawane już żadnym zmianom i być wykorzystane takie, jakie było na inicjalnym etapie
%     	pracy inżynierskiej, o tyle dopuszcza się poszerzanie jej funkcjonalności o dodatkowe biblioteki zewnętrzne.
%     	} \\ 
    
%     \hline
%     \parbox[t]{3cm}{
%     	\textbf{Ograniczenia}
%     } & \parbox[t]{12cm}{
%     	Maszyna wirtualna, jakkolwiek stanowi istotny czynnik dla decyzji w zakresie architektury czy konstrukcji oprogramowania,
%     	nie powinna mieć wpływu na wymagania stricte biznesowe, jest powiem jedynie środowiskiem do uruchamiania współbieżnych
%     	programów, \textit{przezroczystym} dla końcowego użytkownika czy zleceniodawcy zrealizowanego oprogramowania.
%     	} \\ 
    
%     \hline
%     \parbox[t]{3cm}{
%     	\textbf{Wymagania}
%     } & \colorbox{yellow}{...} \\ 
  
%     \hline
% \end{tabular}

% \vspace{2em} 

% \begin{tabular}{ | l | l | }

% 	\hline
% 	\multicolumn{2}{ | l | }{\textbf{Karta udziałowca}}  \\
  
% 	\hline
%     \parbox[t]{3cm}{
%     	\textbf{Identyfikator}
%     } & UO-01 \\  
    
%     \hline
%     \parbox[t]{3cm}{
%     	\textbf{Nazwa}
%     } & Opiekun pracy inżynierskiej \\ 
    
%     \hline
%     \parbox[t]{3cm}{
%     	\textbf{Opis}
%     } & \parbox[t]{12cm}{
%     	Pracownik uczelni, wyznaczony do opieki nad całym projektem inżynierskim - nadzorowania jego postępów, wskazywania problemów
%     	oraz sugerowania decyzji podwyższających walor pracy oraz szanse na jej skuteczne obronienie. Ma również zasadniczy wpływ na 
%     	decyzję o dopuszczeniu pracy do recenzji.
%     } \\ 
    
%     \hline
%     \parbox[t]{3cm}{
%     	\textbf{Typ}
%     } & Ożywiony, bezpośredni \\  
    
%     \hline
%     \parbox[t]{3cm}{
%     	\textbf{Punkt widzenia}
%     } & \parbox[t]{12cm}{
%     	Opiekun pracy patrzy na czat przede wszystkim przez pryzmat jego użyteczności jako efektownego przykładu implementacji modelu
%     	aktora w praktycznym, programistycznym ujęciu. Stąd, jego uwaga skupia się przede wszystkim na konstrukcjach językowych,
%     	strukturach oraz rozwiązaniach od strony kodu źródłowego. Czat stanowi jedynie pretekst do przeniesienia teoretycznych, akademickich
%     	rozważań na praktyczny grunt. 
%     	} \\ 
    
%     \hline
%     \parbox[t]{3cm}{
%     	\textbf{Ograniczenia}
%     } & \parbox[t]{12cm}{
%     	Opiekun pracy, pomimo bycia jej nadzorcą i posiadania istotnych uprawnień decyzyjnych w stosunku do jej dalszego rozwoju, 
%     	nie ma możliwości bieżącego śledzenia prac oraz
%     	podejmowania decyzji w przypadku konkretnych problemów. Powinien zachować dystans, pozwalający na samodzielną realizację projektu
%     	przez zespół. Stąd, jego faktyczny udział ogranicza się do udzielania porad w przypadku strategicznych kierunków, w jakich
%     	będzie podążała grupa, a także doraźnego recenzowania ograniczonej puli zagadnień, wyłapanych w trakcie wspólnych spotkań.
%     	} \\ 
    
%     \hline
%     \parbox[t]{3cm}{
%     	\textbf{Wymagania}
%     } & \colorbox{yellow}{...} \\ 
  
%     \hline
% \end{tabular}

% \vspace{2em} 

% \begin{tabular}{ | l | l | }

% 	\hline
% 	\multicolumn{2}{ | l | }{\textbf{Karta udziałowca}}  \\
  
% 	\hline
%     \parbox[t]{3cm}{
%     	\textbf{Identyfikator}
%     } & UO-02 \\  
    
%     \hline
%     \parbox[t]{3cm}{
%     	\textbf{Nazwa}
%     } & \parbox[t]{12cm}{
%     Członek zespołu ds. ViuAct
%     } \\ 
    
%     \hline
%     \parbox[t]{3cm}{
%     	\textbf{Opis}
%     } & \parbox[t]{12cm}{
%     	Student i członek zespołu, skupiający się w pierwszej kolejności nad rozwojem języka programowania ViuAct, jego kompilatora oraz
%     	ewentualnego rozbudowania maszyny ViuaVM o kolejne, zewnętrzne biblioteki.
%     } \\ 
    
%     \hline
%     \parbox[t]{3cm}{
%     	\textbf{Typ}
%     } & Ożywiony, bezpośredni \\  
    
%     \hline
%     \parbox[t]{3cm}{
%     	\textbf{Punkt widzenia}
%     } & \parbox[t]{12cm}{
%     	Przede wszystkim, postrzega czat jako produkt, realizowany na końcowej platformie. Stąd, musi brać udział w formułowaniu
%     	wymagań związanych z ViuaVM oraz językiem ViuAct. Jego zadaniem jest doprowadzenia do zaprojektowania czatu w sposób, 
%     	który ukaże możliwości ViuAct jako solidnego, kompletnego rozwiązania. Przy tym, musi trzymać rękę na pulsie i reagować,
%     	gdyby pojawiały się przeszkody w zaprogramowaniu czatu, wynikające z niedoskonałości środowiska wytwórczego.
    	
%     	Podczas współudziału w definiowaniu wymagań, istotny jest dla niego zakres pracy, wiążący się z 
%     	urzeczywistnianiem poszczególnych, proponowanych wymagań. Zbyt rozbudowany czat może opóźnić prace nad całym projektem,
%     	a w efekcie - zniweczyć trud włożony w rozwój języka programowania i dedykowanego mu kompilatora.
%     	} \\ 
    
%     \hline
%     \parbox[t]{3cm}{
%     	\textbf{Ograniczenia}
%     } & \parbox[t]{12cm}{
%     	Jego udział w pracach nad czatem jest z gruntu nieograniczony. Jednakże, decydując się na podział odpowiedzialności 
%     	podyktowany zespołowym charakterem projektu oraz własnymi ograniczeniami czasowymi, zrezygnował z decydowania o biznesowej 
%     	części wymagań, faktycznie pozostając w roli konsultanta.
    	
%     	} \\ 
    
%     \hline
%     \parbox[t]{3cm}{
%     	\textbf{Wymagania}
%     } & \colorbox{yellow}{...} \\ 
  
%     \hline
% \end{tabular}

% \vspace{2em} 

% \begin{tabular}{ | l | l | }

% 	\hline
% 	\multicolumn{2}{ | l | }{\textbf{Karta udziałowca}}  \\
  
% 	\hline
%     \parbox[t]{3cm}{
%     	\textbf{Identyfikator}
%     } & UO-03 \\  
    
%     \hline
%     \parbox[t]{3cm}{
%     	\textbf{Nazwa}
%     } & \parbox[t]{12cm}{
%     Członek zespołu ds. Czatu
%     } \\ 
    
%     \hline
%     \parbox[t]{3cm}{
%     	\textbf{Opis}
%     } & \parbox[t]{12cm}{
%     	Student i członek zespołu, odpowiedzialny za prace nad czatem
%     } \\ 
    
%     \hline
%     \parbox[t]{3cm}{
%     	\textbf{Typ}
%     } & Ożywiony, bezpośredni \\  
    
%     \hline
%     \parbox[t]{3cm}{
%     	\textbf{Punkt widzenia}
%     } & \parbox[t]{12cm}{
%     	Czat stanowi dla niego, obok dokumentacji, najistotniejszą część przedsięwzięcia. Musi z jednej strony nauczyć się poruszać
%     	w nowym, dynamicznie zmieniającym się środowisku programistycznym, a z drugiej strony - zrealizować przy jego użyciu serwer
%     	czatu, który pokaże jego możliwości i zastosowania innym nowicjuszom.
    	
%     	Podczas współudziału w definiowaniu wymagań, istotny jest dla niego zakres końcowych funkcjonalności czatu. Nie może być zbyt 
%     	wąski. Z drugiej strony, konstrukcja programu powinna pozostać prosta i przejrzysta. Przykładowy kod nie powinien odstraszać
%     	potencjalnego programisty, dla którego cała koncepcja ViuaVM oraz modelu aktorów może wydawać się na pierwszy rzut oka nieco egzotyczna.
%     	} \\ 
    
%     \hline
%     \parbox[t]{3cm}{
%     	\textbf{Ograniczenia}
%     } & \parbox[t]{12cm}{
%     	Nie ma w zasadzie organizacyjnych czy kompetencyjnych ograniczeń dla formułowania wymagań. Nie oznacza to jednak, że może
%     	definiować wymagań w oderwaniu od pozostałych udziałowców (ich role i punkty widzenia opisano wcześniej).
%     	} \\ 
    
%     \hline
%     \parbox[t]{3cm}{
%     	\textbf{Wymagania}
%     } & \colorbox{yellow}{...} \\ 
  
%     \hline
% \end{tabular}

% \subsection{Charakterystyka użytkowników}

% Na etapie analizy kontekstu, w którym ma zostać zaprojektowany i zrealizowany czat, zadecydowano o zaprojektowaniu następujących,
% modelowych użytkowników docelowego oprogramowania:

% \begin{enumerate}

% 	\item \textbf{Użytkownik tymczasowy.} Typ użytkownika, którego konto jest tworzone podczas połączenia z serwerem czatu oraz 
% 		niszczone po jego zakończeniu. Podczas łączenia z czatem, nie będzie musiał się autoryzować przy użyciu hasła, a deklarować
% 		tylko unikalną nazwę, nie powtarzającą się z nazwą innego użytkownika, posiadającego konto na danym serwerze czatu. Ten typ
% 		konta jest przeznaczony dla osób, zainteresowanych dołączeniem do dyskusji na czacie bez dodatkowych zobowiązań.
		
% 	\item \textbf{Użytkownik stały.} Typ użytkownika, którego konto jest utrzymywane przez serwer pomiędzy połączeniami do czatu. W
% 		zamierzeniu, adresatami takiego rozwiązania mają być stali bywalcy serwera, którzy chcą mieć zarezerwowaną określoną nazwę
% 		dla siebie i uniknąć ewentualnego podszywania się. Stąd każdorazowo, przed rozpoczęciem sesji połączenia z serwerem, muszą się
% 		dodatkowo autoryzować przy użyciu hasła. Równocześnie, ich nazwa jest zarezerwowana wyłącznie do jego użytku oraz niedostępna
% 		dla użytkowników tymczasowych.
		
% 	\item \textbf{Administrator.} To użytkownik stały, który jest dodatkowo wyróżniony i posiada uprawienia do szeroko pojętego 
% 		zarządzania serwerem (w tym - pozostałymi użytkownikami). Nie wyróżnia się wśród administratorów żadnych dodatkowych, szczególnych
% 		ról (np. superadministrator, właściciel).
		
% \end{enumerate}

% Poza wspomnianymi różnicami, wszyscy użytkownicy po rozpoczęciu sesji połączenia mają prawo do dołączania do pokojów oraz wysyłania sobie
% nawzajem wiadomości prywatnych. Łącznie, pula użytkowników przebywających na serwerze czatu w jednym momencie nie powinna przekraczać 320,
% zaś w jednym pokoju - nie więcej niż 32. W związku z tym można przyjąć, że czat jest przeznaczony dla niewielkich społeczności, np.
% szkolnych, uczelnianych czy hobbystycznych.

% \subsection{Istniejąca infrastruktura}

% \begin{itemize}
% 	\item \textbf{Komputer A}
% 	\begin{itemize}
% 		\item komputer przenośny z procesorem Intel Core i5 oraz systemem operacyjnym Windows 10
% 		\item XAMPP 7.2.7, obejmujący serwer Apache 2.4 oraz interpreter języka PHP w wersji 7.2.7.
% 	\end{itemize}
	
% 	\item \textbf{Komputer B}
% 	\begin{itemize}
% 		\item komputer przenośny, na którym zainstalowano system operacyjny Linux Mint 19 ,,Tara''
% 		\item GNU Compiler Collection 8.2
% 		\item wirtualna maszyna Viua VM w wersji 0.9.0
% 		\item \textit{należy doinstalować serwer Nginx, odpowiedzialny za wysłanie frontendu do
% 		użytkownika łączącego się z czatem oraz za handshake Websocketu}
% 	\end{itemize}
	
% 	\item \textit{\textbf{Do uzupełnienia}
% 	\begin{itemize}
% 		\item Kolejne urządzenie końcowe (trzecie), dzięki któremu będzie można symulować połączenie kolejnej
% 		osoby do usługi czatu
% 	\end{itemize}}

% \end{itemize}

% \section{Zasady biznesowe}

% Zidentyfikowane zasady pogrupowano w 3 kategorie, biorąc pod uwagę podstawowe bloki funkcjonalności. Przydzielenie
% do kategorii jest sygnalizowanie literą alfabetu, będącą prefiksem identyfikatora danej zasady. Dokonano
% również priorytetyzacji zasad biznesowych według klasycznej skali ,,MoSCoW":

% \begin{itemize}
% 	\item \textbf{,,M''} (z ang. \textit{must}) - zasady, których spełnienie jest niezbędne dla realizacji systemu
% 	\item \textbf{,,S''} (z ang. \textit{should}) - są to zasady o wysokim priorytecie, które powinny;
% 	zostać spełnione, o ile tylko jest to możliwe;
% 	\item \textbf{,,C''} (z ang. \textit{could}) - dobrze byłoby zrealizować takie wymagania, ale zależy to od czasu
% 	i zasobów, jakie pozostaną do dyspozycji po ukończeniu zadań ,,M" i ,,C";
% 	\item \textbf{,,W''} (z ang. \textit{won't}) - takie wymagania, po dyskusji, zostały wycofane dalszej realizacji.
% \end{itemize}

% \subsection{System użytkowników [ZU]}
%   \begin{tabular}{ | l | l | l | }
% 	\hline
%     \textbf{ID} & \parbox[t]{14cm}{
%     	\textbf{Zasada biznesowa}
%     } & \textbf{Priorytet} \\  
  
%     \hline
%     ZU-01 & \parbox[t]{14cm}{
%       Podczas wejścia na czat, użytkownikowi pokazuje się monit z polem do wpisania nazwy użytkownika. 
%     } & M \\

%     \hline
%     ZU-02 & \parbox[t]{14cm}{
%       Użytkownicy bez stałego konta podczas logowania podają tylko nazwę użytkownika, pole hasła pozostaje puste .
%     } & M \\

%     \hline
%     ZU-03 & \parbox[t]{14cm}{
%       Nazwa użytkownika to ciąg od 3 do 32 alfanumerycznych znaków.
%     } & M \\

%     \hline
%     ZU-04 & \parbox[t]{14cm}{
%       Można rozpocząć sesję jako użytkownik, pod warunkiem, że zadeklarowana nazwa nie będzie powtarzać się z nazwami już zalogowanych użytkowników. 
%     } & M \\

%     \hline
%     ZU-05 & \parbox[t]{14cm}{
%       Monit podczas wejścia na czat jest wyposażony w pole do wpisania hasła (nieobowiązkowe). 
%     } & S \\

%     \hline
%     ZU-06 & \parbox[t]{14cm}{
%       Użytkownicy czatu ze stałym kontem, podczas logowania podają nazwę i odpowiadające mu hasło.
%     } & S \\

%     \hline
%     ZU-07 & \parbox[t]{14cm}{
%       Stałe konta użytkowników są utrzymywane na serwerze w postaci trójek wartości: nazwa użytkownika, hasło (md5), czy jest administratorem. 
%     } & S \\

%     \hline
%     ZU-08 & \parbox[t]{14cm}{
%       Nie można rozpocząć sesji użytkownika o nazwie, która pasuje do istniejącego konta, jeżeli nie zostanie podane prawidłowe hasło (nie można podszywać się pod nazwy użytkowników ze stałym kontem).
%     } & S \\

%     \hline
%     ZU-09 & \parbox[t]{14cm}{
%       Można rozpocząć sesję jako użytkownik bez podawania hasła, pod warunkiem, że zadeklarowana nazwa nie będzie powtarzać się z nazwami stałych kont użytkowników.
%     } & S \\

%     \hline
%     ZU-10 & \parbox[t]{14cm}{
%       W okienkach czatu, loginy użytkowników ze stałym kontem są pogrubione i pokolorowane: Administratorzy na czerwono,
%       Pozostali na zielono. 
%       } & S \\

%     \hline
%     ZU-11 & \parbox[t]{14cm}{
%       Administratorzy mają prawo przeglądać nazwy pokojów na serwerze. 
%     } & M \\

%     \hline
%     ZU-12 & \parbox[t]{14cm}{
%       Administratorzy mają prawo tworzyć i usuwać pokoje. 
%     } & S \\

%     \hline
%     ZU-13 & \parbox[t]{14cm}{
%       Administratorzy mają prawo ustanawiać, zmieniać i usuwać hasła do pokojów. 
%     } & C \\

%     \hline
%     ZU-14 & \parbox[t]{14cm}{
%       Administratorzy mają prawo wyrzucać użytkowników z pokojów. 
%     } & C \\

%     \hline
%     ZU-15 & \parbox[t]{14cm}{
%       Administratorzy mają prawo wyrzucać użytkowników z serwera. 
%     } & C \\

%     \hline
%     ZU-16 & \parbox[t]{14cm}{
%       Administratorzy mają prawo przeglądać nazwy i poziomy uprawnień kont stałych użytkowników. 
%     } & M \\

%     \hline
%     ZU-17 & \parbox[t]{14cm}{
%       Administratorzy mają prawo tworzyć i usuwać użytkowników. 
%     } & S \\

%     \hline
%     ZU-18 & \parbox[t]{14cm}{
%       Administratorzy mają prawo zmieniać hasła użytkowników. 
%     } & C \\

%     \hline
%     ZU-19 & \parbox[t]{14cm}{
%       Administratorzy mają prawo zmieniać uprawnienia stałych kont użytkowników. 
%     } & C \\

%     \hline
%     ZU-20 & \parbox[t]{14cm}{
%       Użytkownicy ze stałymi kontami mogą zmieniać swoje hasło.
%     } & W \\

    
%     \hline
%   \end{tabular}

% \subsection{Pokoje [ZP]}
%   \begin{tabular}{ | l | l | l | }
% 	\hline
%     \textbf{ID} & \parbox[t]{14cm}{
%     	\textbf{Zasada biznesowa}
%     } & \textbf{Priorytet} \\  
    
%     \hline
%     ZP-01 & \parbox[t]{14cm}{
%       Pokoje to właściwe czaty – tam użytkownicy mogą wejść i pisać do siebie nazwajem
%     } & M \\

%     \hline
%     ZP-02 & \parbox[t]{14cm}{
%       Każdy pokój ma unikalną nazwę będącą ciągiem alfanumerycznym od 3 do 32 znaków
%     } & M \\

%     \hline
%     ZP-03 & \parbox[t]{14cm}{
%       Lista pokojów jest widoczna dla każdego użytkownika po zalogowaniu się do serwera czatu
%     } & M \\

%     \hline
%     ZP-04 & \parbox[t]{14cm}{
%       Użytkownik może być równocześnie wpięty do jednego pokoju
%     } & M \\

%     \hline
%     ZP-05 & \parbox[t]{14cm}{
%       Wiadomość wysłana w pokoju jest widoczna w oknie pokoju dla wszystkich użytkowników podpiętych do tego pokoju
%     } & M \\

%     \hline
%     ZP-06 & \parbox[t]{14cm}{
%       Użytkownik może się samodzielnie wypiąć z pokoju, do którego jest wpięty
%     } & S \\

%     \hline
%     ZP-07 & \parbox[t]{14cm}{
%       Pokój może mieć ustanowione hasło, które użytkownik musi wpisać przed podpięciem się do niego
%     } & C \\
    
%     \hline
%     ZP-08 & \parbox[t]{14cm}{
%       Nowo wpięty użytkownik widzi 10 najnowszych wiadomości, 
%       które zostały wysłane do pokoju tuż przed wpięciem
%     } & S \\
    
%     \hline
%     ZP-09 & \parbox[t]{14cm}{
%       Serwer czatu automatycznie wysyła do pokoju wiadomości, 
%       zawierające powiadomienia o wydarzeniach związanych z
%       pokojem, tzw. wiadomości systemowe
%     } & S \\
    
%     \hline
%     ZP-10 & \parbox[t]{14cm}{
%       Wiadomości systemowe są niepodpisane przez
%       żadnego użytkownika i zapisane kursywą
%    	} & C \\
   	
%    	\hline
%     ZP-11 & \parbox[t]{14cm}{
%       Wiadomość systemowa zostaje wysłana podczas wpięcia się
%       nowego użytkownika do pokoju
%    	} & S \\
   	
%    	\hline
%     ZP-12 & \parbox[t]{14cm}{
%       Wiadomość systemowa zostaje wysłana podczas wypięcia
%       użytkownika z pokoju
%    	} & S \\
   	
%    	\hline
%     ZP-13 & \parbox[t]{14cm}{
%       Wiadomość systemowa zostaje wysłana, gdy użytkownik
%       wpięty do pokoju traci połączenie z serwerem czatu
%    	} & S \\
   	
%    	\hline
%     ZP-14 & \parbox[t]{14cm}{
%       Wiadomość systemowa zostaje wysłana, gdy użytkownik
%       zostaje wyrzucony z pokoju
%    	} & S \\	
    
%     \hline
    
%   \end{tabular}

% \subsection{Prywatne wiadomości [ZW]}
%   \begin{tabular}{ | l | l | l | }
%   	\hline
%     \textbf{ID} & \parbox[t]{14cm}{
%     	\textbf{Zasada biznesowa}
%     } & \textbf{Priorytet} \\  
    
    
%     \hline
%     ZW-01 & \parbox[t]{14cm}{
%       Oprócz okien czatu dla każdego z wpiętych pokojów, użytkownik dysponuje dodatkowym oknem, na którym widzi wiadomości prywatne. 
%     } & M \\

%     \hline
%     ZW-02 & \parbox[t]{14cm}{
%       Aby wysłać wiadomość prywatną w oknie pokoju, należy poprzedzić ją znakiem \# i nazwą użytkownika, do którego jest kierowana wiadomość, oddzielona spacją od komunikatu, np.: „\#user Tajna wiadomość”. Trafia ona wówczas do okna wiadomości prywatnych. 
%     } & M \\

%     \hline
%     ZW-03 & \parbox[t]{14cm}{
%       Wiadomość prywatna może być wysłana z okna czatu pokoju –wiadomości prywatne nie są pokazywane wszystkim uczestnikom czatu, a jedynie użytkownikowi, do którego jest adresowana. 
%     } & C \\

%     \hline
%     ZW-04 & \parbox[t]{14cm}{
%       W oknie czatu pokoju dopuszczalne jest wysyłanie wiadomości prywatnych do dowolnych użytkowników, nawet tych, którzy nie są w danym momencie podpięci do tego pokoju. 
%     } & C \\

%     \hline
%     ZW-05 & \parbox[t]{14cm}{
%       Z okna wiadomości prywatnych można odbierać i wysyłać wyłącznie
%       wiadomości prywatne. 
%     } & M \\

%     \hline
%     ZW-06 & \parbox[t]{14cm}{
%       Próba wysłania wiadomości prywatnej powinna zostać odrzucona przez serwer i zwrócić błąd w przypadku gdy:
%       \begin{itemize}
%       	\item Po znaku „\#” pojawi się od razu znak spacji lub nie będzie żadnych znaków (nie zostanie podana 
% 		nazwa użytkownika który jest adresatem)
% 		\item Nazwa adresata jest dłuższa niż 32 znaki
% 		\item Na serwerze nie istnieje użytkownik z nazwą użytą jako nazwa adresata
% 		\item Adresat wiadomości ma swoje konto na serwerze, ale nie jest zalogowany 
%       \end{itemize}
		
%     } & M \\
%     \hline
    
%     \hline
%     ZW-07 & \parbox[t]{14cm}{
%       W oknie wiadomości prywatnych można podglądać równocześnie
%       wiadomości, których nadawcą lub odbiorcą jest jeden, wskazany
%       użytkownik
      
%     } & S \\
    
%     \hline
%     ZW-08 & \parbox[t]{14cm}{
%       W oknie wiadomości prywatnych można wysyłać wiadomości
%       wyłącznie do nadawcy, którego wiadomości są w danym momencie
%       pokazywane.
%     } & S \\
    
%     \hline
%     ZW-10 & \parbox[t]{14cm}{
%       Czas istnienia wiadomości prywatnych zależy od typu użytkownika
%       który jest jej nadawcą i odbiorcą:
%       \begin{itemize}
%       	\item Jeżeli obie strony komunikacji są użytkownikami
%       	tymczasowymi, wiadomość jest utrzymywana dopóki obie
%       	strony konwersacji nie zakończą sesji połączenia z serwerem
%       	\item Jeżeli jedna strona komunikacji jest użytkownikiem
%       	tymczasowym, a druga stałym, to wiadomość jest utrzymywana
%       	dopóki użytkownik tymczasowy skończy sesję połączenia z
%       	serwerem
%       	\item Jeżeli obie strony są użytkownikami stałymi, to
%       	wiadomość jest trzymana bezterminowo
%       \end{itemize}
%     } & S \\
    
%     \hline
%     ZW-11 & \parbox[t]{14cm}{
%       Dla każdej pary użytkowników, na serwerze jest gromadzone co
%       najwyżej 100 wiadomości prywatnych.
%     } & S \\
%     \hline
%   \end{tabular}

% \section{Wymagania}

% \subsection{Wymagania funkcjonalne}

% Ponieważ obraną metodologią wytwarzania aplikacji jest \textit{mini-Scrum}, należący do kategorii metodyk zwinnych, wymagania funkcjonalne ujęto w formie historyjek (\textit{user stories}).

% \vspace{2em} 

% \begin{tabular}{ | l | l | }
% 	\hline
% 		\textbf{Identyfikator} & 
% 		WF-01
% 		\\
		
% 	\hline
% 		\textbf{Treść} & \parbox[t]{11cm}{
% 			Jako użytkownik serwera czatu, chcę się do niego zalogować, aby zobaczyć listę pokojów dyskusyjnych.
% 		}\\
		 
% 	\hline
% 		\parbox[t]{4cm}{\textbf{Powiązane zasady biznesowe}} & \parbox[t]{11cm}{
% 			ZU-01 Podczas wejścia na czat, użytkownikowi pokazuje się monit z polem do wpisania nazwy użytkownika. \\
% 			ZP-03 Lista pokojów jest widoczna dla każdego użytkownika
% 			po zalogowaniu się do serwera czatu
% 		}\\
		
% 	\hline
% 		\parbox[t]{4cm}{\textbf{Kryteria akceptacji}} & \parbox[t]{11cm}{
% 			\begin{enumreq}
% 				\item Po wejściu na czat bez rozpoczętej sesji, pokazuje się monit o podanie nazwy użytkownika.
% 				\item Po wpisaniu nazwy użytkownika i zatwierdzeniu, użytkownik rozpocznie sesję na serwerze czatu.
% 				\item Tuż po rozpoczęciu sesji czatu, użytkownik zobaczy listę pokojów.
% 			\end{enumreq}
% 			}
% 		\\

% 	\hline
% \end{tabular}

% \vspace{2em} 

% \begin{tabular}{ | l | l | }
% 	\hline
% 		\textbf{Identyfikator} & 
% 		WF-02
% 		\\
		
% 	\hline
% 		\textbf{Treść} & \parbox[t]{11cm}{
% 			Jako użytkownik serwera czatu, chcę wpiąć się do pokoju,
% 			aby wziąć udział w dyskusji.
% 		}\\
		 
% 	\hline
% 		\parbox[t]{4cm}{\textbf{Powiązane zasady biznesowe}} & \parbox[t]{11cm}{
% 			ZP-01 Pokoje to właściwe czaty - tam użytkownicy mogą
% 			wejść i pisać do siebie nawzajem
% 		}\\
		
% 	\hline
% 		\parbox[t]{4cm}{\textbf{Kryteria akceptacji}} & \parbox[t]{11cm}{
% 			\begin{enumreq}
% 				\item Użytkownik, który ma otwartą sesję 
% 				połączenia z serwerem czatu i nie jest wpięty
% 				do żadnego pokoju, zobaczy listę pokojów.
% 				\item Użytkownik, po kilknięciu w liście pokojów
% 				na nazwę pokoju, zostanie do niego podpięty
% 				\item Użytkownik po wpięciu się do pokoju zobaczy
% 				okno pokoju
% 				\item Użytkownik, który ma otwartą sesję
% 				połączenia z serwerem i jest wpięty do pokoju,
% 				po odświeżeniu przeglądarki zobaczy okno pokoju, 
% 				do którego jest wpięty
% 			\end{enumreq}
% 			}
% 		\\

% 	\hline
% \end{tabular}

% \vspace{2em} 

% \begin{tabular}{ | l | l | }
% 	\hline
% 		\textbf{Identyfikator} & 
% 		WF-03
% 		\\
		
% 	\hline
% 		\textbf{Treść} & \parbox[t]{11cm}{
% 			Jako użytkownik serwera czatu, chcę po wpięciu
% 			do pokoju zobaczyć ostatnie wiadomości wysłane
% 			przed moim dołączeniem, aby dowiedzieć się, co
% 			tam się obecnie dzieje.
% 		}\\
		 
% 	\hline
% 		\parbox[t]{4cm}{\textbf{Powiązane zasady biznesowe}} & \parbox[t]{11cm}{
% 			ZP-08 Nowo wpięty użytkownik widzi 10 najnowszych
% 			wiadomości, które zostały wysłane do pokoju tuż
% 			przed wpięciem
% 		}\\
		
% 	\hline
% 		\parbox[t]{4cm}{\textbf{Kryteria akceptacji}} & \parbox[t]{11cm}{
% 			\begin{enumreq}
% 				\item Użytkownik po wpięciu się do pokoju zobaczy
% 				10 najnowszych wiadomości wysłanych do pokoju
% 				przed jego dołączeniem (lub mniej, jeżeli
% 				dotychczas nie wysłano do pokoju co najmniej
% 				10 wiadomości)
% 			\end{enumreq}
% 			}
% 		\\

% 	\hline
% \end{tabular}

% \vspace{2em} 

% \begin{tabular}{ | l | l | }
% 	\hline
% 		\textbf{Identyfikator} & 
% 		WF-04
% 		\\
		
% 	\hline
% 		\textbf{Treść} & \parbox[t]{11cm}{
% 			Jako użytkownik serwera czatu, chcę chcę wysłać
% 			wiadomość do pokoju w który jestem wpięty, aby
% 			zobaczyli ją inni uczestnicy dyskusji.
% 		}\\
		 
% 	\hline
% 		\parbox[t]{4cm}{\textbf{Powiązane zasady biznesowe}} & \parbox[t]{11cm}{
% 			ZP-01 Pokoje to właściwe czaty - tam użytkownicy mogą
% 			wejść i pisać do siebie nawzajem
% 		}\\
		
% 	\hline
% 		\parbox[t]{4cm}{\textbf{Kryteria akceptacji}} & \parbox[t]{11cm}{
% 			\begin{enumreq}
% 				\item Użytkownik wpisze tekst wiadomości w polu
% 				tekstowym u dołu czatu
% 				\item Wiadomość wpisana w polu tekstowym zostanie
% 				wysłana po wciśnięciu klawisza ,,Enter'', gdy aktywne
% 				będzie pole tekstowe
% 				\item Wiadomość wpisana w polu tekstowym zostanie
% 				wysłana po naciśnięciu przycisku ,,Wyślij'',
% 				widocznego obok pola tekstowego
% 				\item Po wysłaniu wiadomości, pole tekstowe zostanie
% 				wyczyszczone (niezależnie od tego czy wiadomość
% 				zostanie doręczona)
% 				\item Wiadomość wysłana do pokoju jest pokazywana
% 				wszystkim użytkownikom podpiętym do czatu u dołu
% 				strony
% 				\item Nowa wiadomość jest pokazywana wraz z nazwą
% 				użytkownika wysyłającego u dołu konwersacji
% 			\end{enumreq}
% 			}
% 		\\

% 	\hline
% \end{tabular}

% \vspace{2em} 

% \begin{tabular}{ | l | l | }
% 	\hline
% 		\textbf{Identyfikator} & 
% 		WF-05
% 		\\
		
% 	\hline
% 		\textbf{Treść} & \parbox[t]{11cm}{
% 			Jako użytkownik serwera czatu, chcę wiedzieć czy
% 			moja wiadomość została doręczona innym dyskutantom,
% 			aby spróbować wysłać ją w razie potrzeby jeszcze raz.
% 		}\\
		 
% 	\hline
% 		\parbox[t]{4cm}{\textbf{Powiązane zasady biznesowe}} & \parbox[t]{11cm}{

% 		}\\
		
% 	\hline
% 		\parbox[t]{4cm}{\textbf{Kryteria akceptacji}} & \parbox[t]{11cm}{
% 			\begin{enumreq}
% 				\item Po wysłaniu wiadomości, pole tekstowe wraz 
% 				z przyciskiem ,,Wyślij'' zostaje zablokowane (ale
% 				nie wyczyszczone)
% 				\item Dopiero po otrzymaniu od serwera potwierdzenia
% 				odebrania wiadomości, pole tekstowe jest czyszczone
% 				i odblokowywane
% 				\item Jeżeli serwer nie potwierdzi odebrania
% 				wiadomości po upływie 5 sekund lub zwróci błąd,
% 				pole tekstowe i przycisk ,,Wyślij'' zostanie 
% 				odblokowane, ale nie wyczyszczone
% 			\end{enumreq}
% 			}
% 		\\

% 	\hline
% \end{tabular}

% \vspace{2em} 

% \begin{tabular}{ | l | l | }
% 	\hline
% 		\textbf{Identyfikator} & 
% 		WF-06
% 		\\
		
% 	\hline
% 		\textbf{Treść} & \parbox[t]{11cm}{
% 			Jako użytkownik serwera czatu, chcę chcę zobaczyć
% 			powiadomienie o wpięciu się nowego użytkownika do
% 			pokoju w którym sam jestem obecnie wpięty, aby powitać
% 			nowego dyskutanta
% 		}\\
		 
% 	\hline
% 		\parbox[t]{4cm}{\textbf{Powiązane zasady biznesowe}} & \parbox[t]{11cm}{
% 			ZP-09 Serwer czatu automatycznie wysyła do pokoju
% 			wiadomości, zawierające powiadomienia o wydarzeniach
% 			związanych z pokojem, tzw. wiadomości systemowe \\
% 			ZP-11 Wiadomość systemowa zostaje wysłana podczas
% 			wpięcia się nowego użytkownika do pokoju
% 		}\\
		
% 	\hline
% 		\parbox[t]{4cm}{\textbf{Kryteria akceptacji}} & \parbox[t]{11cm}{
% 			\begin{enumreq}
% 				\item Niezwłocznie po wpięciu się użytkownika do
% 				pokoju, serwer wyśle wiadomość systemową o treści
% 				,,Użytkownik ... dołączył do pokoju'', widoczną
% 				dla wszystkich użytkowników wpiętych do tego pokoju
% 			\end{enumreq}
% 			}
% 		\\

% 	\hline
% \end{tabular}

% \vspace{2em} 

% \begin{tabular}{ | l | l | }
% 	\hline
% 		\textbf{Identyfikator} & 
% 		WF-07
% 		\\
		
% 	\hline
% 		\textbf{Treść} & \parbox[t]{11cm}{
% 			Jako użytkownik serwera czatu, chcę zobaczyć
% 			powiadomienie o opuszczeniu pokoju przez użytkownika,
% 			aby łatwo zorientować się, że nie bierze już udziału 
% 			w dyskusji.
% 		}\\
		 
% 	\hline
% 		\parbox[t]{4cm}{\textbf{Powiązane zasady biznesowe}} & \parbox[t]{11cm}{
% 			ZP-09 Serwer czatu automatycznie wysyła do pokoju
% 			wiadomości, zawierające powiadomienia o wydarzeniach
% 			związanych z pokojem, tzw. wiadomości systemowe \\
% 			ZP-12 Wiadomość systemowa zostaje wysłana podczas
% 			wpięcia wypięcia użytkownika z pokoju \\
% 			ZP-13 Wiadomość systemowa zostaje wysłana, gdy użytkownik
% 			wpięty do pokoju traci połączenie z serwerem \\
% 			ZP-14 Wiadomość systemowa zostaje wysłana, gdy użytkownik
% 			zostaje wyrzucony z pokoju
			
% 		}\\
		
% 	\hline
% 		\parbox[t]{4cm}{\textbf{Kryteria akceptacji}} & \parbox[t]{11cm}{
% 			\begin{enumreq}
% 				\item Niezwłocznie po wypięciu się użytkownika z
% 				pokoju, serwer wyśle wiadomość systemową, widoczną
% 				dla wszystkich użytkowników wpiętych do tego pokoju,
% 				o treści:
% 				\begin{enumerate}
% 					\item ,,Użytkownik ... opuścił pokój'', gdy
% 					użytkownik samodzielnie wypiął się z pokoju
% 					\item ,,Użytkownik ... stracił połączenie'',
% 					gdy użytkownik został wypięty z pokoju na skutek
% 					przerwania sesji z uwagi na zerwanie połączenia
% 					\item ,,Użytkownik ... został wyrzucony'', gdy
% 					użytkownik został wypięty wskutek interwencji
% 					administratora
% 				\end{enumerate}
% 			\end{enumreq}
% 			}
% 		\\

% 	\hline
% \end{tabular}

% \vspace{2em} 

% \begin{tabular}{ | l | l | }
% 	\hline
% 		\textbf{Identyfikator} & 
% 		WF-08
% 		\\
		
% 	\hline
% 		\textbf{Treść} & \parbox[t]{11cm}{
% 			Jako użytkownik serwera czatu, chcę odpiąć się od pokoju,
% 			aby wpiąć się do innego pokoju.
% 		}\\
		 
% 	\hline
% 		\parbox[t]{4cm}{\textbf{Powiązane zasady biznesowe}} & \parbox[t]{11cm}{
% 			ZP-06 Użytkownik może się samodzielnie wypiąć z pokoju,
% 			do którego jest wpięty
			
% 		}\\
		
% 	\hline
% 		\parbox[t]{4cm}{\textbf{Kryteria akceptacji}} & \parbox[t]{11cm}{
% 			\begin{enumreq}
% 				\item W oknie pokoju użytkownik zobaczy przycisk
% 				lub link ,,Opuść pokój''.
% 				\item Po kliknięciu w ,,Opuść pokój'', użytkownik
% 				zobaczy listę pokojów.
% 			\end{enumreq}
% 			}
% 		\\

% 	\hline
% \end{tabular}

% \vspace{2em} 

% \begin{tabular}{ | l | l | }
% 	\hline
% 		\textbf{Identyfikator} & 
% 		WF-09
% 		\\
		
% 	\hline
% 		\textbf{Treść} & \parbox[t]{11cm}{
% 			Jako użytkownik serwera czatu, chcę zobaczyć
% 			okno wiadomości prywatnych, aby odczytać wiadomości,
% 			które wysłano specjalnie do mnie.
% 		}\\
		 
% 	\hline
% 		\parbox[t]{4cm}{\textbf{Powiązane zasady biznesowe}} & \parbox[t]{11cm}{
% 			ZW-01 Oprócz okien czatu dla każdego z wpiętych pokojów,
% 			użytkownik dysponuje dodatkowym oknem, w którym widzi
% 			wiadomości prywatne
% 		}\\
		
% 	\hline
% 		\parbox[t]{4cm}{\textbf{Kryteria akceptacji}} & \parbox[t]{11cm}{
% 			\begin{enumreq}
% 				\item Nad listą pokojów, użytkownik zobaczy link
% 				,,Wiadomości prywatne''.
% 				\item Po kliknięciu w link ,,Wiadomości prywatne'',
% 				użytkownik zobaczy okno wiadomości prywatnych.
% 			\end{enumreq}
% 			}
% 		\\

% 	\hline
% \end{tabular}

% \vspace{2em} 

% \begin{tabular}{ | l | l | }
% 	\hline
% 		\textbf{Identyfikator} & 
% 		WF-07
% 		\\
		
% 	\hline
% 		\textbf{Treść} & \parbox[t]{11cm}{
% 			Jako użytkownik serwera czatu, chcę odfiltrować
% 			wiadomości prywatne od jednego użytkownika, aby
% 			prowadzić z nim ciągłą konwersację.
% 		}\\
		 
% 	\hline
% 		\parbox[t]{4cm}{\textbf{Powiązane zasady biznesowe}} & \parbox[t]{11cm}{
% 			ZP-09 Serwer czatu automatycznie wysyła do pokoju
% 			wiadomości, zawierające powiadomienia o wydarzeniach
% 			związanych z pokojem, tzw. wiadomości systemowe \\
% 			ZP-12 Wiadomość systemowa zostaje wysłana podczas
% 			wpięcia wypięcia użytkownika z pokoju \\
% 			ZP-13 Wiadomość systemowa zostaje wysłana, gdy użytkownik
% 			wpięty do pokoju traci połączenie z serwerem \\
% 			ZP-14 Wiadomość systemowa zostaje wysłana, gdy użytkownik
% 			zostaje wyrzucony z pokoju
			
% 		}\\
		
% 	\hline
% 		\parbox[t]{4cm}{\textbf{Kryteria akceptacji}} & \parbox[t]{11cm}{
% 			\begin{enumreq}
% 				\item Niezwłocznie po wypięciu się użytkownika z
% 				pokoju, serwer wyśle wiadomość systemową, widoczną
% 				dla wszystkich użytkowników wpiętych do tego pokoju,
% 				o treści:
% 				\begin{enumerate}
% 					\item ,,Użytkownik ... opuścił pokój'', gdy
% 					użytkownik samodzielnie wypiął się z pokoju
% 					\item ,,Użytkownik ... stracił połączenie'',
% 					gdy użytkownik został wypięty z pokoju na skutek
% 					przerwania sesji z uwagi na zerwanie połączenia
% 					\item ,,Użytkownik ... został wyrzucony'', gdy
% 					użytkownik został wypięty wskutek interwencji
% 					administratora
% 				\end{enumerate}
% 			\end{enumreq}
% 			}
% 		\\

% 	\hline
% \end{tabular}

% \subsection{Wymagania niefunkcjonalne}

% \begin{tabular}{ | l | l | }
% 	\hline
% 		\textbf{Identyfikator} & 
% 		HN-1
% 		\\
		
% 	\hline
% 		\textbf{Treść} & \parbox[t]{13cm}{
% 			Długość nazwy użytkownika jest ograniczona od 2 do 32 znaków alfanumerycznych, w celu uniknięcia problemów z identyfikacją użytkownika na serwerze.
% 		}\\
		 
% 	\hline
% 		\parbox[t]{4cm}{\textbf{Powiązane zasady biznesowe}} & \parbox[t]{13cm}{
% 			U-3 Nazwa użytkownika to ciąg od 3 do 32 alfanumerycznych znaków.
% 		}\\
		
% 	\hline
% 		\parbox[t]{4cm}{\textbf{Kryteria akceptacji}} & \parbox[t]{13cm}{
% 			\begin{enumreq}
% 				\item Po wpisaniu do pola użytkownika nazwy krótszej niż 2 znaki, dłużej niż 32 znaki lub zawierającej inne znaki niż alfanumeryczne, zwracany jest błąd.
% 			\end{enumreq}
% 			}
% 		\\

% 	\hline
% \end{tabular}




% \subsection{Wymagania na środowisko docelowe}

% \textit{W jakim środowisku będzie pracować system – o ile jest istotne, np. system operacyjny, rodzaje i wersje przeglądarek internetowych, itp. Może się zdarzyć, że na tym etapie użytkownicy i inni udziałowcy nie wyspecyfikują środowiska docelowego.}

% \subsection{Wymagania dotyczące procesu wytwarzania}

% \textit{Jaki ma być proces powstawania rozwiązania projektowego np. ile czasu ma trwać, czy ma być wykonany zgodnie z jakąś metodologią lub też jakie mają być cechy tego procesu np. tajny}

% \section{Kryteria akceptacja rozwiązania}

% \textit{Specyfikacja kryteriów akceptacji gotowego produktu przez klienta – mogą to być wybrane kluczowe wymagania funkcjonalne lub niefunkcjonalne. Chodzi o zdefiniowanie priorytetów: czas, budżet, wydajność, bezpieczeństwo, itp. Pod koniec procesu realizacji wraca się do kryteriów akceptacji zdefiniowanych na początku projektu i weryfikuje poprawność rozwiązania. }

% \section{Odwołania do literatury}

% \textit{Lista przywoływanych pozycji literowych, ponumerowanych lub z przydzielonymi identyfikatorami; w treści właściwej dokumentu posługujemy się wyłącznie numerami/ identyfikatorami do wskazania źródła treści. Usunąć jeśli nie dotyczy.}

\end{document}
