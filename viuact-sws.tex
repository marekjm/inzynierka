\documentclass[11pt,oneside,a4paper,titlepage,onecolumn]{article}

\usepackage[utf8]{inputenc}
\usepackage{textcomp}
\usepackage[official]{eurosym}
\usepackage[polish]{babel}
\usepackage{amsthm}
\usepackage{graphicx}
\usepackage[T1]{fontenc}
\usepackage{scrextend}
\usepackage{hyperref}
\usepackage{xcolor}
% \usepackage{nameref}
% \usepackage{showlabels}
% \usepackage{titlesec}
\usepackage{geometry}
\geometry{a4paper, portrait, margin=2cm}
\graphicspath{ {./fig/} }

\newenvironment{enumreq}
{ \begin{enumerate}[topsep=0pt,itemsep=-1ex,partopsep=1ex,parsep=1ex] }
{ \end{enumerate}                  } 


\setcounter{secnumdepth}{4}

%% Author and title
% \author{Marek Marecki \and Gr.52c \and Krzysztof Franek}
\author{Marek Marecki i Krzysztof Franek}
\title{%
    Dowód na żywotność Viua VM \\
    \large Implementacja języka wysokiego poziomu dla Viua VM \\
    i realizacja prostej aplikacji \\
    ~\\
    Specyfikacja Wymagań Systemowych\\
    dla języka Viuact}

\begin{document}

\maketitle
{\footnotesize
\begin{center}
  \begin{tabular}{ | l | l | l | }
    \hline
    \parbox[t]{6.5cm}{\textbf{Temat pracy i akronim projektu:}\\Proving viablity of Viua VM (VVIA)} & \parbox[t]{4.5cm}{\textbf{Zleceniodawca:}\\\colorbox{yellow}{Nieznany}} & \parbox[t]{4.5cm}{\textbf{Konsultant:}\\\colorbox{yellow}{Nieznany}} \\ \hline
    \parbox[t]{6.5cm}{\textbf{Zespół projektowy:}\\Krzysztof Franek, Marek Marecki} & \parbox[t]{4.5cm}{\textbf{Kierownik projektu:}\\Marek Marecki} & \parbox[t]{4.5cm}{\textbf{Opiekun projektu:}\\dr hab. Marek A. Bednarczyk, prof. PJWSTK} \\ \hline
    \parbox[t]{3.5cm}{\textbf{Kierownik projektu:}\\Marek Marecki} &
      \multicolumn{2}{|l|}{\parbox[t]{9cm}{\textbf{Odpowiedzialny za dokument:}\\Marek Marecki}} \\ 
    \hline
  \end{tabular}
\end{center}
}

\section{Wprowadzenie}

\subsection{Cel dokumentu}
Celem dokumentu jest zdefiniowanie wymagań dla języka ViuAct i kompilatora tego języka na podstawie analizy projektu.

\subsection{Zakres dokumentu}
Niniejszy dokument jest produktem pierwszego etapu procesu wytwórczego czatu ViuaChat, na który składają się:
\begin{itemize}
    \item analiza otoczenia, wraz z z klientami;
    \item wskazanie kontekstu biznesowego systemu;
    \item określenie udziałowców;
\end{itemize}

Praca wygenerowana w systemie \LaTeX.

\subsection{Dokumenty powiązane}
\begin{itemize}
	\item Specyfikacja języka ViuAct
\end{itemize}

\subsection{Odbiorcy}

Odbiorcami dokumentu są członkowie zespołu.
Jest on wyznacznikiem tego co jest wymagane od języka Viuact, a zatem tego co można wykorzystać przy tworzeniu
programu ViuaChat.

Kolejną grupą adresatów niniejszego dokumentu są pracownicy uczelni, odpowiedzialni za nadzór nad prawidłowym ukształtowaniem i przebiegiem projektu.
Wśród nich, szczególną rolę odgrywa JE Dziekan ZWI, prof. Marek Bednarczyk, będący opiekunem projektu.

\subsection{Słownik pojęć}
\begin{description}
  \item[ViuAct] Język programowania wysokiego poziomu, oparty o model aktorów, powstały na potrzeby niniejszego projektu inżynierskiego
\end{description}

\section{Język ViuAct i jego kompilator w kontekście}

\subsection{Kontekst biznesowy}

\subsection{Udziałowcy}

\subsection{Klienci}

\subsection{Charakterystyka użytkowników}

\subsection{Istniejąca infrastruktura}

\section{Wymagania}

\subsection{Wymagania ogólne i dziedzinowe}

Doprecyzowanie celów projektu, prezentacja przedsięwzięć zdefiniowanych przez Kartę Projektu i
Dokument Założeń Wstępnych.

\vspace{1em}

\begin{tabular}{ | l | l | }
    \hline
    \textbf{Identyfikator} & \parbox[t]{11cm}{WD-01} \\
    \hline
    \textbf{Priorytet} & S \\
    \hline
    \textbf{Nazwa} & Udowodnienie przydatności (ang. \emph{viability}) Viua VM \\
    \hline
    \textbf{Opis} & \parbox[t]{11cm}{
        Udowodnienie, że maszyna wirtualna Viua może być celem kompilacji dla
        języków wyższego poziomu oraz jest możliwe uruchomienie na niej nietrywialnego
        oprogramowania (w przypadku tego projektu będzie to ViuaChat).} \\
    \hline
    \textbf{Udziałowiec} & \phantom{} \\
    \hline
    \textbf{Wymagania powiązane} & \phantom{} \\
    \hline
\end{tabular}

\vspace{1em}

\begin{tabular}{ | l | l | }
    \hline
    \textbf{Identyfikator} & \parbox[t]{11cm}{WD-02} \\
    \hline
    \textbf{Priorytet} & M \\
    \hline
    \textbf{Nazwa} & Projekt języka ViuAct \\
    \hline
    \textbf{Opis} & \parbox[t]{11cm}{
        Wymagane jest zaprojektowanie języka programowania wyższego poziomu oraz przygotowanie
        jego specyfikacji.
    } \\
    \hline
    \textbf{Udziałowiec} & \phantom{} \\
    \hline
    \textbf{Wymagania powiązane} & WD-01 \\
    \hline
\end{tabular}

\vspace{1em}

\begin{tabular}{ | l | l | }
    \hline
    \textbf{Identyfikator} & \parbox[t]{11cm}{WD-03} \\
    \hline
    \textbf{Priorytet} & M \\
    \hline
    \textbf{Nazwa} & Implementacja komilatora \\
    \hline
    \textbf{Opis} & \parbox[t]{11cm}{
        Wymagana jest implementacja kompilatora przetwarzającego kod źródłowy napisany w języku
        wyższego poziomu (zaprojektowanym w punkcie WD-02) na kod w języku assemblera Viua VM.
    } \\
    \hline
    \textbf{Udziałowiec} & \phantom{} \\
    \hline
    \textbf{Wymagania powiązane} & WD-02 \\
    \hline
\end{tabular}

\vspace{1em}

\begin{tabular}{ | l | l | }
    \hline
    \textbf{Identyfikator} & \parbox[t]{11cm}{WD-04} \\
    \hline
    \textbf{Priorytet} & M \\
    \hline
    \textbf{Nazwa} & Chat \\
    \hline
    \textbf{Opis} & \parbox[t]{11cm}{
        Musi powstać implementacja czatu napisana w języku zaprojektowanym w punkcie WD-02.
    } \\
    \hline
    \textbf{Udziałowiec} & \phantom{} \\
    \hline
    \textbf{Wymagania powiązane} & \parbox[t]{11cm}{
        WD-03 (\emph{do wykonania tego wymagania niezbędny jest kompilator})
    } \\
    \hline
\end{tabular}

\subsection{Wymagania funkcjonalne}

Część wymagań funkcjonalnych języka ViuAct jest opisana w osobnym dokumencie: ''Specyfikacja języka ViuAct''.
Poniżej opisane są wymagania, które nie mogły być zawarte bezpośrednio w specyfikacji języka (np. sposób
obsługi i raportowania błędów przez kompilator).

\subsubsection{Funkcjonalności kompilatora}

W tej sekcji opisane są wymagania dotyczące kompilatora języka ViuAct.

\vspace{1em}

\begin{tabular}{ | l | l | }
    \hline
    \textbf{Identyfikator} & \parbox[t]{11cm}{WF-01} \\
    \hline
    \textbf{Priorytet} & M \\
    \hline
    \textbf{Nazwa} & Zachowanie znaczenia programu podczas kompilacji \\
    \hline
    \textbf{Opis} & \parbox[t]{11cm}{
        Kompilator musi wygenerować kod wynikowy implementujący dokładnie taki sam program jak oryginalny
        kod źródłowy. Niedopuszczalne jest aby po skompilowaniu program miał inne zachowanie niż to, które
        zostało opisane przez oryginalny kod źródłowy w języku wyższego poziomu.
        \\
        Całkowite zachowanie znaczenia oznacza również, że jeśli oryginalny kod źródłowy zawiera błąd
        logiczny to program wynikowy również będzie go zawierać.
    } \\
    \hline
    \textbf{Udziałowiec} & \phantom{} \\
    \hline
    \textbf{Wymagania powiązane} & \phantom{} \\
    \hline
\end{tabular}

\vspace{1em}

\begin{tabular}{ | l | l | }
    \hline
    \textbf{Identyfikator} & \parbox[t]{11cm}{WF-02} \\
    \hline
    \textbf{Priorytet} & W \\
    \hline
    \textbf{Nazwa} & Obsługa błędów \\
    \hline
    \textbf{Opis} & \parbox[t]{11cm}{
        Kompilator nie będzie w żaden sposób obsługiwał błędów ani ich raportował. \\
        Strategią obsługi błędów będzie rzucenie wyjątku, ewentualnie dodając do niego informacje o tym
        co konkretnie się wydarzyło, i pozwolenie mu na zabicie procesu kompilatora.
        Na podstawie zrzutu stosu i treści wyjątku użytkownik będzie musiał się domyślić co wywołało błąd.
    } \\
    \hline
    \textbf{Udziałowiec} & \phantom{} \\
    \hline
    \textbf{Wymagania powiązane} & \phantom{} \\
    \hline
\end{tabular}

\vspace{1em}

\begin{tabular}{ | l | l | }
    \hline
    \textbf{Identyfikator} & \parbox[t]{11cm}{WF-03} \\
    \hline
    \textbf{Priorytet} & M \\
    \hline
    \textbf{Nazwa} & Emisja plików pomocniczych \\
    \hline
    \textbf{Opis} & \parbox[t]{11cm}{
        Kompilator podczas przetwarzania kodu źródłowego musi zebrać informacje jakie funkcje zawiera dany
        moduł, jakie moduły są w nim zagnieżdżone, i jakie moduły są przez niego importowane.
        Te informacje są potrzebne programowi łączącemu moduły do automatycznego wykonania swojej pracy. \\
        Pliki pomocnicze to:

        \begin{enumerate}
            \item \texttt{.i} pliki opisujące interfejs modułu (eksportowane funkcje)
            \item \texttt{.d} pliki opisujące zależności modułu (importowane moduły)
        \end{enumerate}
    } \\
    \hline
    \textbf{Udziałowiec} & \phantom{} \\
    \hline
    \textbf{Wymagania powiązane} & \phantom{} \\
    \hline
\end{tabular}

\vspace{1em}

\begin{tabular}{ | l | l | }
    \hline
    \textbf{Identyfikator} & \parbox[t]{11cm}{WF-04} \\
    \hline
    \textbf{Priorytet} & C \\
    \hline
    \textbf{Nazwa} & Optymalizacja zużycia rejestrów \\
    \hline
    \textbf{Opis} & \parbox[t]{11cm}{
        Z uwagi na prostą budowę kompilatora generuje on kod, który alokuje duże ilości rejestrów.
        Optymalizacja polegająca na analizie dostępu do rejestrów (odczytach i zapisach) i wyszuaniu
        momentów, po których rejestr \texttt{x} nie jest już więcej używany mogłaby zredukować rozmiar
        alokacji dzięki recyklingowi rejestrów (ponownym użyciu po udowodnieniu, że po danej instrukcji
        żadna inna nie będzie korzystać z wartości w danym rejestrze).
    } \\
    \hline
    \textbf{Udziałowiec} & \phantom{} \\
    \hline
    \textbf{Wymagania powiązane} & \phantom{} \\
    \hline
\end{tabular}

\vspace{1em}

\begin{tabular}{ | l | l | }
    \hline
    \textbf{Identyfikator} & \parbox[t]{11cm}{WF-05} \\
    \hline
    \textbf{Priorytet} & M \\
    \hline
    \textbf{Nazwa} & Kompletność implementacji \\
    \hline
    \textbf{Opis} & \parbox[t]{11cm}{
        Kompilator musi implementować wszystkie konstrukcje językowe opisane w specyfikaji języka
        ViuAct.
    } \\
    \hline
    \textbf{Udziałowiec} & \phantom{} \\
    \hline
    \textbf{Wymagania powiązane} & \phantom{} \\
    \hline
\end{tabular}

\subsubsection{Interfejs z otoczeniem}

Punkty styku projektowanego języka i kompilatora z innymi podsystemami (np. assembler dostarczany jako
część Viua VM), sieciami (np. Internet), i operatorami (np. programista).

\vspace{1em}

\begin{tabular}{ | l | l | }
    \hline
    \textbf{Identyfikator} & \parbox[t]{11cm}{IZO-01} \\
    \hline
    \textbf{Priorytet} & M \\
    \hline
    \textbf{Nazwa} & Poprawność emitowanego kodu \\
    \hline
    \textbf{Opis} & \parbox[t]{11cm}{
        Kompilator musi generować poprawny kod źródłowy w języku assemblera Viua VM.
        Niedopuszczalne jest aby kompilator generował kod, który będzie zawierał błędy składniowe bądź
        nieznane lub niedozwolone instrukcje.
    } \\
    \hline
    \textbf{Udziałowiec} & \phantom{} \\
    \hline
    \textbf{Wymagania powiązane} & \phantom{} \\
    \hline
\end{tabular}

\vspace{1em}

\begin{tabular}{ | l | l | }
    \hline
    \textbf{Identyfikator} & \parbox[t]{11cm}{IZO-02} \\
    \hline
    \textbf{Priorytet} & M \\
    \hline
    \textbf{Nazwa} & Biblioteki implementujące operacji I/O \\
    \hline
    \textbf{Opis} & \parbox[t]{11cm}{
        Muszą być dostarczone biblioteki umożliwiające przeprowadzanie operacji I/O (\emph{wejścia-wyjścia}).
        Takie biblioteki mogą być napisane w języku ViuAct, języku assemblera Viua VM, lub języku C++.
        W przypadku implementacji w języku assemblera Viua VM lub języku C++ muszą zostać dostarczone pliki
        umożliwiające kompilatorowi ViuAct wczytanie listy funkcji oferowanych przez taką bibliotekę.
    } \\
    \hline
    \textbf{Udziałowiec} & \phantom{} \\
    \hline
    \textbf{Wymagania powiązane} & \phantom{} \\
    \hline
\end{tabular}

\subsection{Wymagania niefunkcjonalne}

Brak.

\subsection{Wymagania na środowisko docelowe}

Brak.

\subsection{Wymagania dotyczące procesu wytwarzania}

W tej sekcji opisane są wymagania dotyczące kompilatora języka ViuAct.

\vspace{1em}

\begin{tabular}{ | l | l | }
    \hline
    \textbf{Identyfikator} & \parbox[t]{11cm}{PW-01} \\
    \hline
    \textbf{Priorytet} & M \\
    \hline
    \textbf{Nazwa} & Testy kompilatora \\
    \hline
    \textbf{Opis} & \parbox[t]{11cm}{
        Kompilator musi być dostarczony wraz z zestawem prostych do uruchomienia testów werfikujących
        poprawność jego pracy.
    } \\
    \hline
    \textbf{Udziałowiec} & \phantom{} \\
    \hline
    \textbf{Wymagania powiązane} & \phantom{} \\
    \hline
\end{tabular}

\vspace{1em}

\begin{tabular}{ | l | l | }
    \hline
    \textbf{Identyfikator} & \parbox[t]{11cm}{PW-01} \\
    \hline
    \textbf{Priorytet} & S \\
    \hline
    \textbf{Nazwa} & Kompletność testów kompilatora \\
    \hline
    \textbf{Opis} & \parbox[t]{11cm}{
        Testy kompilatora powinny obejmować jak największe spektrum możliwych konstrukcji językowych i
        ich kombinacji. W uwagi na to, że jest niemożliwym przetestowanie kompilatora na wszystkich
        możliwych do napisania programach ''kompletność'' jest względna, a to wymaganie zakłada jedynie, że
        nie powinna istnieć konstrukcja językowa, która nie będzie przetestowana. \\
        \textbf{Uwaga}: Ostatecznym testem kompilatora pozostaje kompilacja czatu ViuaChat, który jest
        wymagany do poprawnego zaliczenia projektu.
    } \\
    \hline
    \textbf{Udziałowiec} & \phantom{} \\
    \hline
    \textbf{Wymagania powiązane} & \phantom{} \\
    \hline
\end{tabular}

\section{Kryteria akceptacji rozwiązania}

Podstawowym kryterium akceptacji dla języka jest ''\emph{Da się w nim napisać czat}'', a
dla komilatora ''\emph{Da się nim skompilować czat}''.

\section{Odwołania do literatury}

\section{Załączniki}

Specyfikacja języka ViuAct -- ''\emph{viuact-specification.pdf}''.

\end{document}
