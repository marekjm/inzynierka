\chapter{Raport końcowy}
\label{raport_koncowy}

W tym rozdziale prezentujemy raport końcowy z efektów pracy inżynierskiej.
Podsumowujemy odniesione sukcesy (wykonane oprogramowanie, wytworzoną dokumentację), porażki (elementy, na
których wykonanie nie wystarczyło czasu, okazały się zbyt trudne do wykonania w ramach projektu
inżynierskiego, bądź z innych powodów musiały zostać odrzucone), oraz zmiany, które musieliśmy wprowadzać do
planu w trakcie trwania projektu (decyzja o uwzględnieniu raportowania błędów w kompilatorze).

\section{Sukcesy}

Niewątpliwym sukcesem jest doprowadzenie pracy inżynierskiej do końca. Patrząc
na ,,zewnętrzne'' efekty pracy osiągnęliśmy wszystko co było zakładane.

\section{Niepowodzenia}

Najbardziej widoczną porażką projektu jest pozostawienie implementacji języka
\ViuAct\ w fazie prototypowej -- budżet czasowy jakim dysponowaliśmy okazał się
niewystarczający na stworzenie prototypu kompilatora (w języku Python) w celu
przeprowadzenia ,,rozpoznania'' przestrzeni problemu, a następnie korzystając z
nabytej wiedzy napisanie kompilatora ,,produkcyjnego'' (w języku OCaml).

Kolejnym niepowodzeniem jest również fakt, że kompilator, którym dysponujemy nie
jest w stanie przetworzyć stu procent możliwych do napisania konstrukcji
językowych wynikających ze specyfikacji języka. Nie uniemożliwiło to co prawda
napisania programu ViuaChat jednak wprowadziło pewne opóźnienia w pracach nad
nim z powodu rozbieżności między teorią (specyfikacją języka), a praktyką (jego
implementacją).

% Opisać co się nie udało, z czym nie zdążyliśmy, z czego musieliśmy zrezygnować, co okazało się niemożliwe.

\section{Zmiany założeń wprowadzone w trakcie trwania projektu}

\section{Wpływ pracy na platformę Viua VM}

W tym rozdziale omówiony jest wpływ naszej pracy inżynierskiej na platformę Viua VM; zarówno na implementację
samej maszyny wirtualnej i dopracowanie jej mechanizmów (np. doprowadzenie systemu modułów do stanu
używalności), ale też na ,,ekosystem'' dookoła niej -- nowe moduły biblioteki standardowej i zewnętrzne.

\subsection{System modułów}

Viua VM musiała zostać wyposażona w system modułów. Wymagane było przeprojektowanie i ,,ucywilizowanie''
stanu, w którym system modułów Viua VM się znajdował przed rozpoczęciem prac nad projektem inżynierskim.

\subsection{Moduł do obsługi protokołu WebSocket}

Do wykonania czatu potrzebny był moduł do obsługi protokołu WebSocket.
