\chapter{Raport końcowy}
\label{raport_koncowy}

W tym rozdziale prezentujemy raport końcowy z efektów pracy inżynierskiej.
Podsumowujemy odniesione sukcesy (wykonane oprogramowanie, wytworzoną dokumentację), porażki (elementy, na
których wykonanie nie wystarczyło czasu, okazały się zbyt trudne do wykonania w ramach projektu
inżynierskiego, bądź z innych powodów musiały zostać odrzucone), oraz zmiany, które musieliśmy wprowadzać do
planu w trakcie trwania projektu (decyzja o uwzględnieniu raportowania błędów w kompilatorze).

\section{Sukcesy}

Niewątpliwym sukcesem jest doprowadzenie pracy inżynierskiej do końca. Patrząc
na ,,zewnętrzne'' efekty pracy osiągnęliśmy wszystko co było zakładane -- czyli
zarówno specyfikację i implementację (w formie kompilatora) języka wysokiego
poziomu na platformie Viua~VM, oraz nietrywialny program -- ViuaChat -- napisany
w języku \ViuAct. Przygotowano również kompletne, zintegrowane środowisko,
dzięki czemu wyniki prac mogą być użyteczne dla osób spoza zespołu projektowego.

Nieoczywistym sukcesem jest również fakt, że Krzysztof Franek wykorzystał język
\ViuAct\ w trakcie zajęć na Uczelni -- pisząc program zaliczeniowy z
przedmiotu ,,Narzędzia Sztucznej Inteligencji'' w języku \ViuAct.

\subsection{Ocena i poprawa stanu Viua~VM}

Projekt pozwolił krytycznym okiem spojrzeć na stan w jakim znajdowała się
Viua~VM i ocenić ją w obiektywny sposób za pomocą jasnego kryterium: albo da się
oprzeć na niej implementację języka wysokiego poziomu i napisać ,,prawdziwy''
program, albo nie.

W trakcie pracy okazało się, że odpowiedź na to pytanie jest \emph{raczej
pozytywna}. Niektóre aspekty Viua~VM wymagają pracy (który to fakt był znany),
jednak przeprowadzenie projektu inżynierskiego na jej podstawie pozwoliło
zlokalizować największe niedociągnięcia, a nawet część z nich usunąć.

Jednym z obszarów, który został znacząco poprawiony w trakcie prac nad
projektem inżynierskim było łączenie modułów w finalne programy.

\subsection{Opracowanie systemu ViuaChat}
Pomimo licznych przeciwności i problemów wynikających z początkowej niedojrzałości, udało się szczęśliwie ukończyć system ViuaChat. Było to
możliwe dzięki zacieśnionej współpracy obu członków zespołu, a także dzięki
kreatywnemu podejściu do napotykanych utrudnień.

\section{Wyzwania i niepowodzenia}

W trakcie realizacji projektu nie wszystko przebiegło po myśli zespołu.

\subsection{Prędkość kompilacji}

Przede wszystkim, wiele do życzenia pozostawiała prędkość kompilacji, znacznie
spadająca wraz z przyrastającym kodem źródłowym. Co ciekawe, w początkowej fazie
prac, za źródło problemów uważano implementację kompilatora ViuAct w języku
Python ponieważ programy napisane w Pythonie ustępują wydajnością programom
napisanym w językach takich jak C, czy też OCaml.

Ku zaskoczeniu studentów, wraz z kolejnymi linijkami kodu, widocznemu wydłużeniu
ulegała faza asemblacji, za którą odpowiadał assembler maszyny wirtualnej
Viua~VM. W końcowych wersjach serwera ViuaChat, asemblacja była do 5 razy
bardziej czasochłonna od kompilacji. Niestety, poprawianie wydajności asemblera
pozostawało poza zakresem projektu, a wydłużony czas asemblacji musiał zostać
zaakceptowany.

\subsection{Pozostanie przy prototypie kompilatora}

Najbardziej widoczną porażką projektu jest pozostawienie implementacji języka
\ViuAct\ w fazie prototypowej -- budżet czasowy jakim dysponowaliśmy okazał się
niewystarczający na stworzenie prototypu kompilatora (w języku Python) w celu
przeprowadzenia ,,rozpoznania'' przestrzeni problemu, a następnie korzystając z
nabytej wiedzy napisanie kompilatora ,,produkcyjnego'' (w języku OCaml).

\subsection{Niekompletność implementacji}

Niepowodzeniem jest również fakt, że kompilator, którym dysponujemy nie jest w
stanie przetworzyć stu procent możliwych do napisania konstrukcji językowych
wynikających ze specyfikacji języka. Nie uniemożliwiło to co prawda napisania
programu ViuaChat jednak wprowadziło pewne opóźnienia w pracach nad nim z powodu
rozbieżności między teorią (specyfikacją języka), a praktyką (implementacją
języka).

\subsection{Czas kompilacji}

Kolejnym niepowodzeniem jest również szybkość kompilacji. Całość procesu, który
z plików źródłowych w języku \ViuAct\ wyprodukuje pliki z binarną reprezentacją
programu w \emph{bytecode} Viua~VM trwa długo. Przy nietrywialnych programach,
takich jak ViuaChat, etap przetwarzania kodu programu w języku assemblera
Viua~VM na formę binarną (realizowany przez assembler dostarczany przez Viua~VM)
zajmuje dużą ilość czasu.

Wynika to po części z faktu, że w nietrywialne programy zawierają dużą ilość
wyrażeń warunkowych. Analizator statyczny wbudowany w assembler dostarczany
przez Viua~VM duplikuje swój stan dla każdej instrukcji skoku warunkowego
(emitowanej dla każdego wyrażenia warunkowego) i rozważa wykonanie każdej
ścieżki osobno.
Jest to potrzebne ponieważ na podstawie niezależnej analizy każdej ścieżki
wykonania analizator wykrywa między innymi martwe (tj. takie, który program nie
wykorzystuje) wartości i rejestry -- jednak powoduje to, że każda instrukcja
skoku warunkowego w teorii podwaja koszt analizy.

% \subsection{Typowanie dynamiczne}

% Mimo iż nie było to ujęte w wymaganiach, za drobną porażkę uważam również
% dyscyplinę typowania jaką charakteryzuje się języku \ViuAct. Klasyfikując język
% \ViuAct\ na dwóch osiach dyscyplin typowania:
% \begin{enumerate}
% \item dynamiczne---statyczne (\emph{dynamic---static}) -- określająca kiedy
%     weryfikowane są typy danych. W typowaniu dynamicznym na etapie wykonywania
%     (\emph{run-time}), a w typowaniu statycznym na etapie kompilacji
%     (\emph{compile-time}).
% \item słabe---silne (\emph{weak---strong}) -- określająca jaka jest reakcja
%     języka na użycie niekompatybilnych typów danych. Typowanie słabe zakłada
%     automatyczną koercję typów danych, a typowanie silne zakłada zgłoszenie
%     błędu.
% \end{enumerate}
% otrzymujemy typowanie dynamiczne---silne. Lepsze gwarancje poprawności można
% uzyskać kombinacją statycze---silne, jednak zbudowanie poprawnego statycznego
% systemu typów jest samo w sobie czasochłonne i skomplikowane, co
% dyskwalifikowało taką dyscyplinę typowania już na starcie projektu.

\section{Zmiany założeń wprowadzone w trakcie trwania projektu}

Większość założeń, które zostały określone na początku projektu okazała się
słuszna, a ogólna koncepcja oraz kierunek rozwoju projektu nie wymagały dużych
zmian.

\subsection{Obsługa błędów kompilacji}

Drobnej korekcie uległo założenie ,,nie implementujemy obsługi błędów w
kompilatorze''. Okazało się, że potrzebna jest przynajmniej minimalna obróbka
błędów zgłaszanych przez kompilator ponieważ surowe zrzuty stosu nie były
wystarczające do szybkiego odnajdowania błędów.

\subsection{Obsługa protokołu WebSocket}

Większą zmianą założeń była rezygnacja z wykorzystania gotowej biblioteki
implementującej protokół WebSocket i zamiast tego implementacja tego protokołu
,,od zera''. Wynikło to z bardzo prostego powodu -- każda biblioteka, która
była brana pod uwagę okazywała się tak naprawdę \emph{frameworkiem} co
dyskwalifikowało jej wykorzystanie w Viua~VM.

Frameworki narzucają programom swój reżim i wizję tego jak ma wyglądać
współpraca z nimi -- narzucają współpracę o dostarczany przez siebie
\emph{event-loop}. Viua~VM jednak w bardzo ostry sposób wchodzi w konflikt z
takim podejściem ponieważ sama narzuca reżim i model programowania.  Ta
,,niekompatybilność'' wymusiła na nas wykonanie implementacji modułu do obsługi
protokołu WebSocket w formie biblioteki -- czyli modułu pozostawiającego
kontrolę ,,w rękach'' programu.

\subsection{Wskaźniki w ViuAct}
W pierwszych wersjach języka ViuAct, wskaźniki nie istniały w jawnej formie.
Nie przewidziano konstrukcji do tworzenia i rozwiązywania wskażników, ani
pozwalającej na jawne skopiowanie wartości do której odwołuje się wskaźnik.
Nie przeszkadzało to jednak, aby wskaźniki występowały w formie niejawnej.
Kompilator stosował je w wynikowym kodzie asemblera, np. przy odwoływaniu się do zawartości w typach złożonych. Z czasem okazało się, że brak jawnych instrukcji,
które pozwoliłyby na manualne wywołanie operacji na wskaźnikach, istotnie
ogranicza możliwości programisty. Problemy pojawiały się podczas iteracji
wektorów czy też przy wykorzystywaniu wartości pól w strukturach,
które zostały uprzednio ,,wyłuskane'' i zwrócone przez funkcję.

Zgłoszone niedociągnięcia doprowadziły do ubogacenia składni ViuAct o
instrukcję wyłuskania \texttt{(\^ ptr)}, kopiowania wskazywanej wartości
\texttt{(Std.copy ptr)} oraz jawnego utworzenia wskaźnika
 \texttt{(Std.Pointer.take var)}.

 \subsection{}{Zakres funkcjonalności systemu ViuaChat}
 W początkowej fazie projektu, system ViuaChat miał zostać zrealizowany ze
 znacznie większym rozmachem. Z czasem, wskutek ożywionych dyskusji i ciągłych
 korekt harmonogramu, stawało się jasne że utrzymanie pierwotnie założonych
 funkcjonalności nie będzie możliwe. Stąd zrezygnowano z kilku z nich, m.in.:

 \begin{enumerate}
   \item możliwość wysyłania i odbierania wiadomości prywatnych z okien
   pokojów ogólnodostępnych;
   \item możliwość zakładania stałych kont zabezpieczonych hasłem dla użytkowników bez uprawnień administracyjnych;
   \item możliwość zabezpieczania dostępu do wybranych
   pokojów poprzez konieczność podania dodatkowego hasła.
 \end{enumerate}

 Pomimo zarzucenia tych i innych pomysłów, nie naruszono podstawowego rdzenia
 funkcjonalności, ustalonego jeszcze przed przystąpieniem do samego projektu.

\section{Wpływ pracy na platformę Viua VM}

W tym rozdziale omówiony jest wpływ naszej pracy inżynierskiej na platformę
Viua~VM; zarówno na implementację samej maszyny wirtualnej i dopracowanie jej
mechanizmów, ale też na ,,ekosystem'' dookoła niej -- nowe moduły biblioteki
standardowej i zewnętrzne.

\subsection{System modułów}

Umieszczenie w specyfikacji języka \ViuAct\ systemu modułów wymusiło
,,ucywilizowanie'' systemu modułów i linkera Viua~VM. Stan w jakim oba te
komponenty znajdowały się przed rozpoczęciem pracy inżynierskiej powodował, że
nadawały się one do wykorzystania jedynie w najprostrzych przypadkach.

Po zakończeniu pracy inżynierskiej system modułów i linker Viua~VM pozwalają na
wykorzystanie w dużo szerszym wachlarzu zastosowań, są prostsze w użyciu, a
linkowanie dynamiczne ma mniejszy narzut. Ujednolicone zostały również zasady
linkowania modułów ,,własnych'' (tj. modułów \emph{bytecode} Viua~VM) i
,,obcych'' (tj. napisanych w języku C++).

\subsection{Biblioteka standardowa}

Biblioteka standardowa została powiększona o moduły do obsługi I/O na plikach,
moduł dający dostęp do API systemowego \emph{POSIX sockets}, oraz moduł do
obsługi formatu serializacji danych JSON.
Wszystkie te moduły zostały wykorzystane w implementacji programu ViuaChat.

\subsection{Moduł do obsługi protokołu WebSocket}

Do wykonania czatu potrzebny był moduł do obsługi protokołu WebSocket.
Został on dostarczony jako biblioteka obca z plikiem interfejsu umożliwiającym
jej wykorzystanie w języku \ViuAct. Moduł ten jest opisany w załączniku
\ref{plain_websocket_library} na stronie \pageref{plain_websocket_library}.
