\chapter{Raport końcowy}
\label{raport_koncowy}

W tym rozdziale prezentujemy raport końcowy z efektów pracy inżynierskiej.
Podsumowujemy odniesione sukcesy (wykonane oprogramowanie, wytworzoną dokumentację), porażki (elementy, na
których wykonanie nie wystarczyło czasu, okazały się zbyt trudne do wykonania w ramach projektu
inżynierskiego, bądź z innych powodów musiały zostać odrzucone), oraz zmiany, które musieliśmy wprowadzać do
planu w trakcie trwania projektu (decyzja o uwzględnieniu raportowania błędów w kompilatorze).

\section{Sukcesy}

\subsection{Opracowanie języka ViuAct wraz z jego kompilatorem}
Niewątpliwym sukcesem jest ukończenie definicji języka wysokiego poziomu,
jakim jest ViuAct. Przygotowano również kompletne, zintegrowane środowisko,
dzięki czemu wyniki prac mogą być użyteczne dla osób spoza zespołu projektowego.
Pomimo napotkanych przeszkód, kompilator prawidłowo przetwarza podstawowe
instrukcje języka.

\subsection{Opracowanie systemu ViuaChat}
Pomimo licznych przeciwności i problemów wynikających z początkowej niedojrzałości, udało się szczęśliwie ukończyć system ViuaChat. Było to
możliwe dzięki zacieśnionej współpracy obu członków zespołu, a także dzięki
kreatywnemu podejściu do napotykanych utrudnień.

\section{Wyzwania i niepowodzenia}

W trakcie realizacji projektu nie wszystko przebiegło po myśli zespołu.

\paragraph*{Prędkość kompilacji}

Przede wszystkim, wiele do życzenia pozostawiała prędkość kompilacji, znacznie
spadająca wraz z przyrastającym kodem źródłowym. Co ciekawe, w początkowej fazie
prac, za źródło problemów uważano implementację kompilatora ViuAct w języku
Python ponieważ programy napisane w Pythonie ustępują wydajnością programom
napisanym w językach takich jak C, czy też OCaml.

Ku zaskoczeniu studentów, wraz z kolejnymi linijkami kodu, widocznemu wydłużeniu
ulegała faza asemblacji, za którą odpowiadał assembler maszyny wirtualnej
Viua~VM. W końcowych wersjach serwera ViuaChat, asemblacja była do 5 razy
bardziej czasochłonna od kompilacji. Niestety, poprawianie wydajności asemblera
pozostawało poza zakresem projektu, a wydłużony czas asemblacji musiał zostać
zaakceptowany.

\paragraph*{Pozostanie przy prototypie}

Najbardziej widoczną porażką projektu jest pozostawienie implementacji języka
\ViuAct\ w fazie prototypowej -- budżet czasowy jakim dysponowaliśmy okazał się
niewystarczający na stworzenie prototypu kompilatora (w języku Python) w celu
przeprowadzenia ,,rozpoznania'' przestrzeni problemu, a następnie korzystając z
nabytej wiedzy napisanie kompilatora ,,produkcyjnego'' (w języku OCaml).

Dodatkowo, pierwsze tygodnie prac nad systemem ViuaChat ujawniły niezgodności
pomiędzy teoretycznymi możliwościami kompilatora języka \ViuAct\ wynikającymi ze
specyfikacji języka, a praktycznymi możliwościami wynikającymi z implementacji.
Kompilator, którym dysponujemy nie jest w stanie przetworzyć stu procent możliwych do napisania konstrukcji
językowych wynikających ze specyfikacji języka. Nie uniemożliwiło to co prawda
napisania programu ViuaChat jednak wprowadziło pewne opóźnienia w pracach nad
nim z powodu rozbieżności między teorią (specyfikacją języka), a praktyką (jego
implementacją).

Daleko idące uproszczenia implementacji kompilatora języka \ViuAct\ okazały się
zgubne. Wobec tego, nie chcąc ryzykować dalszych problemów, zadecydowano
o kontynuacji rozwoju ,,pythonowego'' prototypu.

% Opisać co się nie udało, z czym nie zdążyliśmy, z czego musieliśmy zrezygnować, co okazało się niemożliwe.

\section{Zmiany założeń wprowadzone w trakcie trwania projektu}

\paragraph*{Podejście do wersji kompilatora opracowanej w języku Python.}
Jak już ujęto w sekcji poświęconej wyzwaniom i porażkom, w trakcie trwania
projektu zmieniły się założenia co do początkowej wersji kompilatora ViuAct,
opracowanego w języku Python. W grudniu 2018 roku podjęto ostatecznie decyzję,
że wersja ta, dotychczas ujmowana jako prototyp, nie będzie zarzucana na rzecz
innego, docelowego rozwiązania w języku OCaml. Dccyzja ta zaważyła m.in. na
tempie prac nad systemem ViuaChat oraz na założeniach projektowych kompilatora.
Śladem tych założeń jest ,,ocamlowska'' notacja, pozostawiona w niektórych sekcjach niniejszej pracy.

\paragraph*{Zakres funkcjonalności systemu ViuaChat.}
W początkowej fazie projektu, system ViuaChat miał zostać zrealizowany ze
znacznie większym rozmachem. Z czasem, wskutek ożywionych dyskusji i ciągłych
korekt harmonogramu, stawało się jasne że utrzymanie pierwotnie założonych
funkcjonalności nie będzie możliwe. Stąd zrezygnowano z kilku z nich, m.in.:

\begin{enumerate}
  \item możliwość wysyłania i odbierania wiadomości prywatnych z okien
  pokojów ogólnodostępnych;
  \item możliwość zakładania stałych kont zabezpieczonych hasłem dla użytkowników bez uprawnień administracyjnych;
  \item możliwość zabezpieczania dostępu do wybranych
  pokojów poprzez konieczność podania dodatkowego hasła.
\end{enumerate}

Pomimo zarzucenia tych i innych pomysłów, nie naruszono podstawowego rdzenia
funkcjonalności, ustalonego jeszcze przed przystąpieniem do samego projektu.

\paragraph*{Poziom rozbudowania systemu modułów ViuAct.}
Pierwsze szkice języka ViuAct przewidywały pojawienie się konstrukcji, które
pozwoliłyby na prymitywną modularyzację kodu źródłowego. Rozwiązania te okazały się jednak niedostateczne, co uwypukliło się w miarę rozrostu serwera
ViuaChat. Przede wszystkim, brakowało mechanizmu pozwalającego na złożone
zależności pomiędzy poszczególnymi modułami. Dzięki poprawieniu i poszerzeniu
konstrukcji modułów, stało się możliwe odwoływanie przez kilka modułów do jednej
biblioteki. Pozwoliło to na odejście od jednego, ,,dużego'' pliku z kodem
źródłowym na rzecz kilkunastu mniejszych.

\paragraph*{Wskaźniki w ViuAct.}
W pierwszych wersjach języka ViuAct, wskaźniki nie istniały w jawnej formie.
Nie przewidziano konstrukcji do tworzenia i rozwiązywania wskażników, ani
pozwalającej na jawne skopiowanie wartości do której odwołuje się wskaźnik.
Nie przeszkadzało to jednak, aby wskaźniki występowały w formie niejawnej.
Kompilator stosował je w wynikowym kodzie asemblera, np. przy odwoływaniu się do zawartości w typach złożonych. Z czasem okazało się, że brak jawnych instrukcji,
które pozwoliłyby na manualne wywołanie operacji na wskaźnikach, istotnie
ogranicza możliwości programisty. Problemy pojawiały się podczas iteracji
wektorów czy też przy wykorzystywaniu wartości pól w strukturach,
które zostały uprzednio ,,wyłuskane'' i zwrócone przez funkcję.

Zgłoszone niedociągnięcia doprowadziły do ubogacenia składni ViuAct o
instrukcję wyłuskania \texttt{(\^ ptr)}, kopiowania wskazywanej wartości
\texttt{(Std.copy ptr)} oraz jawnego utworzenia wskaźnika
 \texttt{(Std.Pointer.take var)}.

\section{Wpływ pracy na platformę Viua VM}

W tym rozdziale omówiony jest wpływ naszej pracy inżynierskiej na platformę Viua VM; zarówno na implementację
samej maszyny wirtualnej i dopracowanie jej mechanizmów (np. doprowadzenie systemu modułów do stanu
używalności), ale też na ,,ekosystem'' dookoła niej -- nowe moduły biblioteki standardowej i zewnętrzne.

\subsection{Biblioteka do obsługi protokołu WebSocket}

Do wykonania czatu potrzebny był moduł do obsługi protokołu WebSocket.
