\chapter{Raport końcowy}
\label{raport_koncowy}

W tym rozdziale prezentujemy raport końcowy z efektów pracy inżynierskiej.
Podsumowujemy odniesione sukcesy (wykonane oprogramowanie, wytworzoną dokumentację), porażki (elementy, na
których wykonanie nie wystarczyło czasu, okazały się zbyt trudne do wykonania w ramach projektu
inżynierskiego, bądź z innych powodów musiały zostać odrzucone), oraz zmiany, które musieliśmy wprowadzać do
planu w trakcie trwania projektu (decyzja o uwzględnieniu raportowania błędów w kompilatorze).

\section{Sukcesy}

Podsumowanie prac.
Pochwalić się co się udało zrobić, jakie efekty osiągnęliśmy, co się sprawdziło.

\section{Niepowodzenia} %% Jakąś lepszą nazwę dla tego rozdziału trzeba wymyślić.

Opisać co się nie udało, z czym nie zdążyliśmy, z czego musieliśmy zrezygnować, co okazało się niemożliwe.

\section{Zmiany założeń wprowadzone w trakcie trwania projektu}

\section{Wpływ pracy na platformę Viua VM}

W tym rozdziale omówiony jest wpływ naszej pracy inżynierskiej na platformę Viua VM; zarówno na implementację
samej maszyny wirtualnej i dopracowanie jej mechanizmów (np. doprowadzenie systemu modułów do stanu
używalności), ale też na ,,ekosystem'' dookoła niej -- nowe moduły biblioteki standardowej i zewnętrzne.

\subsection{System modułów}

Viua VM musiała zostać wyposażona w system modułów. Wymagane było przeprojektowanie i ,,ucywilizowanie''
stanu, w którym system modułów Viua VM się znajdował przed rozpoczęciem prac nad projektem inżynierskim.

\subsection{Moduł do obsługi protokołu WebSocket}

Do wykonania czatu potrzebny był moduł do obsługi protokołu WebSocket.
