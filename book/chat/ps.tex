\section{Architektura}

\subsection{Użyte wzorce projektowe}

\subsection{Architektura systemu}
\label{chat_architektura_systemu}

Architektura systemu opiera się na klasycznym modelu klient-serwer.
 
Klienci będą uzyskiwali dostęp do usługi czatu za pośrednictwem przeglądarki internetowej. Łącząc się z podanym adresem (IP lub domeny), 
na którym połączeń nasłuchuje serwer, w pierwszej kolejności przeglądarka będzie próbować połączyć się z nim przy użyciu protokołu http 
i standardowego portu 80, wysyłając do niego żądanie metodą GET. Wówczas, serwer będzie zawsze odpowiadał statycznym plikiem HTML, zawierającym 
odwołania do skryptów w języku JavaScript (JS) oraz pozostałej, statycznej treści (np. grafiki czy arkusze stylów CSS). Serwer będzie 
odsyłał te pliki do przeglądarki w odpowiedzi na kolejne żądania HTTP GET, wysyłane w miarę dalszego renderowania pliku HTML. W ten sposób, 
po stronie klienta zostanie pobrana i uruchomiona aplikacja internetowa typu Single Page Application, której interfejs będzie reagował z 
użytkownikiem oraz ulegał zmianom wskutek działania skryptów JS, załadowanych na pierwszym etapie uruchomienia. Po stronie serwera, 
dostarczaniem treści statycznych będzie zajmował się daemon HTTP - Nginx.

Gdy tylko skrypty JS wykryją pobranie wszystkich plików składowych aplikacji z serwera, podjęta zostanie próba nawiązania połączenia 
z tym serwerem przy użyciu protokołu WebSocket. Będzie on od tego momentu podstawowym kanałem komunikacji pomiędzy klientem a serwerem. 

Zgodnie ze standardem WebSocketu, zanim zostanie nawiązane właściwe połączenie, powinno dojść do „uścisku dłoni” (ang. \textit{handshake}) pomiędzy 
klientem a serwerem. W związku z tym, pierwsza próba połączenia również zostanie podjęta przy użyciu protokołu http, jednakże tym razem pod
innym, dedykowanym portem (w naszym przypadku będzie to port 8000), a także zawierać nagłówki wskazujące na żądanie zmiany używanego protokołu 
na WebSocket, jego wersję oraz klucz („Sec-WebSocketKey”). Serwer udzieli wówczas odpowiedzi ze swoim własnym kluczem, informując o zmianie 
stosowanego protokołu na WebSocket.

Za nasłuchiwanie na porcie 8000 po stronie serwera będzie odpowiadał aktor ,,ChatListener'', który przy pierwszej próbie połączenia 
utworzy kolejnego, niezależnego aktora ,,WebSocketConnector'', odpowiadającego za realizację ,,uścisku dłoni'' i formułowanie odpowiedzi zwrotnej 
po stronie serwera w odniesieniu do połącznia na konkretnym gnieździe. W razie prawidłowego nawiązania połączenia, aktor ten będzie odpowiadał 
za dalszą, bezpośrednią obsługę przydzielonego gniazda. Jego rolą będzie również kodowanie i dekodowanie wiadomości (ang.
,,messages''), czyli podstawowych logicznych jednostek informacji, które są używane przy połączeniach z użyciem protokołu WebSocket.

W momencie, gdy ,,uścisk dłoni'' zostanie pozytywnie ukończony, aktor ,,WebSocketConnector'' utworzy kolejnego, niezależnego aktora o nazwie
\,,AuthConnector''. Będzie on od tego momentu odbierał wiadomości od ,,WebSocketConnector'', który go uruchomił, i odpowiadał za należytą
autentykację i/lub autoryzację użytkownika w usłudze czatu.

W chwili prawidłowego rozpoczęcia połączenia WebSocket, aplikacja po stronie klienta wyświetli użytkownikowi formularz autoryzacyjny. Wpisane
tam dane zostaną następnie przesłane do serwera. Po jego stronie, komunikat zostanie zdekodowany przez aktora ,,WebSocketConnector'' i
przekazany powiązanemu aktorowi ,,AuthConnector''. Jego zadaniem będzie weryfikacja przedstawionych informacji oraz podjęcie decyzji o autoryzacji
lub jej odmowie. Decyzja ta jest odsyłana do ,,WebSocketConnector'' i następnie przekazywana do aplikacji po stonie klienta.

Za gromadzenie informacji związanych z autentykacją odpowiada dwóch, na stałe uruchomionych aktorów:
\begin{itemize}
	\item \textbf{TemporaryUsernameLessor} - jego zadaniem jest zarządzanie informacjami na temat tymczasowych nazw użytkowników, należących do
	użytkowników bez stałych kont (ich gromadzenie, udzielanie, weryfikacja)
	\item \textbf{UserAccountKeeper} - służy do zarządzania informacjami na temat stałych kont użytkowników (tworzenia, usuwania, gromadzenia,
	utrwalania, weryfikacji)
\end{itemize}
W zależności od tego, czy w trakcie autoryzacji podano w formularzu hasło użytkownika, czy też nie, zostanie odpytany odpowiedni aktor. Warto tu
zwrócić uwagę, że ,,TemporaryUsernameLessor'' oraz ,,UserAccountKeeper'' utrzymują komunikację, dzięki czemu np. nie jest możliwe podszywanie się
użytkownika tymczasowego pod nazwę użytkownika ze stałym kontem.

Jezeli autentykacja przebiegnie pomyślnie, aktor ,,AuthConnector'' przestaje istnieć wraz z utworzeniem aktora ,,UserSession''. Od tego momentu
przejmuje on komunikację z użyciem ,,WebSocketConnector'', pozwalając na zwyczajne użytkowanie czatu. Aktor ,,UserSession'' gromadzi informacje
na temat nazwy, rodzaju konta oraz poziomu uprawnień użytkownika. 

\subsection{Dekompozycja systemu na podsystemy}
\label{architektura_kompilatora}

\section{Projekt struktury}

\subsection{Diagram klas}

\section{Decyzje projektowe}

\subsection{Środowisko docelowe}

\subsection{Środowisko implementacji}

\subsection{Priorytety implementacyjne}

\section{Projekt algorytmów i przyjętych protokołów}

\section{Projekt rozwiązań sprzętowych}

\section{Projekt interfejsu}

\subsection{Interfejs użytkownika}

\subsubsection{Założenia konstrukcji interfejsu}

\section{Projekt bazy danych}

\section{Opis implementacji}
