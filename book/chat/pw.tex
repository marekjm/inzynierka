\section{Przebieg wdrożenia}
W niniejszej sekcji skupimy się wyłącznie na części powiązanej z realizacją
systemu ViuaChat -- jej przebiegu oraz zmianach, jakie wprowadzono do jego
pierwotnego projektu.

\section{Zmiany w projekcie}

\section{Przebieg realizacji}
 Poniżej przedstawiono zadania, jakie realizowano w ramach kolejnych
 sprintów. Część z nich odnosi się do pierwotnych wymagań (w tym
 \textit{user stories}, zaś niektóre zostały sformułowane w trakcie samej
 realizacji, pod wpływem pierwotnych potrzeb.

 \subsection{Sprint 1}

\textbf{Termin realizacji:} 2-9 marca 2019 r.

\subsubsection{Cel sprintu}
Celem sprintu było utworzenie podstawowej funkcjonalności związanej z
komunikacją frontend-backend -- ustanawiania połączenia WebSocket,
dekodowania komunikatów bo obu stronach aplikacji, a także podstawowa
walidacja funkcjonalności.

\subsubsection{Zadania oparte o pierwotne wymagania}
\textit{Nie dotyczy.}

\subsubsection{Zadania wykraczające poza pierwotne wymagania}

\begin{tabular}{ | l | l | }
	\hline
		\textbf{Identyfikator} &
		ZZ-01
		\\

	\hline
		\textbf{Treść} & \parbox[t]{11.5cm}{\strut
			Frontend aplikacji jest w stanie nawiązać połączenie WebSocket z backendem.
		\strut}\\

	\hline
		\parbox[t]{4cm}{\textbf{Kryteria akceptacji}} & \parbox[t]{11.5cm}{\strut
			\begin{enumreq}
				\item Frontend nawiązał połączenie z serwerem (tj. doszło do
        zmiany protokołu z HTTP na WS).
        \item Frontend wysłał wiadomość do backendu, a wiadomość została
        odebrana.
        \item Backend wysłał wiadomość do frontendu, a wiadomość została
        odebrana.
			\end{enumreq}
			\strut}
		\\

    \hline
  		\parbox[t]{4cm}{\textbf{Nakład godzinowy (planowany / włożony)}} & \parbox[t]{11.5cm}{\strut
  			5h / 5.5h
  		\strut}\\

    \hline
      \parbox[t]{4cm}{\textbf{Ukończono?}} &
      \parbox[t]{11.5cm}{\strut
        Tak.
      \strut}\\

    \hline
  \end{tabular}

  \vspace{1em}

\begin{tabular}{ | l | l | }
	\hline
		\textbf{Identyfikator} &
		ZZ-02
		\\

	\hline
		\textbf{Treść} & \parbox[t]{11.5cm}{\strut
			Obie strony (tj. frontend i backend) potrafią zakodować wiadomości w
      ustalonym formacie JSON przed ich wysyłką i odebrać je po ich odebraniu.
		\strut}\\

	\hline
		\parbox[t]{4cm}{\textbf{Kryteria akceptacji}} & \parbox[t]{11.5cm}{\strut
			\begin{enumreq}
				\item Frontend potrafi zakodować wiadomość w formacie JSON -- wiadomość
        synchroniczną oraz wiadomość asynchroniczną.
        \item Frontend potrafi odkodować wiadomość w formacie JSON.
        \item Frontend potrafi ustalić, czy odebrana wiadomość jest synchroniczna,
        czy asynchroniczna, a w przypadku synchronicznej -- dopasować ją do
        wysłanej wcześniej wiadomości.
        \item Frontend potrafi właściwie obsługiwać wyjątki wynikające z błędów
        w składni JSON odbieranych wiadomości oraz z błędów w strukturze obiektów
        JSON: braku obowiązkowych własności oraz nieprawidłowych typów tych
        wiadomości. Błędy te nie powodują krytycznego przerwania pracy działania
        programu.
        \item Backend potrafi odkodować otrzymywane wiadomości w  formacie JSON.
        \item Backend potrafi ustalić, czy otrzymana wiadomość jest synchroniczna,
        czy asynchroniczna.
        \item Backend potrafi odesłać odpowiedź pasującą do otrzymanej wiadomości
        synchronicznej.
        \item Backend potrafi właściwie obsługiwać wyjątki z tytułu niewłaściwie
        skonstruowanych wiadomości. W razie otrzymania takiej wiadomości, nie
        dochodzi do przerwania pracy serwera (tj. aktora \texttt{WsConnector}).
			\end{enumreq}
			\strut}
		\\

    \hline
  		\parbox[t]{4cm}{\textbf{Nakład godzinowy (planowany / włożony)}} & \parbox[t]{11.5cm}{\strut
  			10h / 12h
  		\strut}\\

        \hline
          \parbox[t]{4cm}{\textbf{Ukończono?}} &
          \parbox[t]{11.5cm}{\strut
            Tak.
          \strut}\\

          \hline
      \end{tabular}

\subsection{Sprint 2}

\textbf{Termin realizacji:} 10-23 marca 2019 r.

\subsection{Cel sprintu}
W ramach tego sprintu zaplanowano opracowanie mechanizm autoryzacji do czatu i
możliwość przejrzenia listy pokojów.

\subsubsection{Zadania oparte o pierwotne wymagania}

\begin{tabular}{ | l | l | }
	\hline
		\textbf{Identyfikator} &
		WF-01
		\\

	\hline
		\textbf{Treść} & \parbox[t]{11.5cm}{\strut
			Jako użytkownik serwera czatu, chcę się do niego zalogować, aby zobaczyć listę pokojów dyskusyjnych.
		\strut}\\

	\hline
		\parbox[t]{4cm}{\textbf{Kryteria akceptacji}} & \parbox[t]{11.5cm}{\strut
			\begin{enumreq}
				\item Po wejściu na czat bez rozpoczętej sesji, pokazuje się monit o podanie nazwy użytkownika.
				\item Po wpisaniu nazwy użytkownika i zatwierdzeniu, użytkownik rozpocznie sesję na serwerze czatu.
				\item Tuż po rozpoczęciu sesji czatu, użytkownik zobaczy listę pokojów.
			\end{enumreq}
			\strut}
		\\

    \hline
      \parbox[t]{4cm}{\textbf{Nakład godzinowy (planowany / włożony)}} & \parbox[t]{11.5cm}{\strut
        13h / 15h
      \strut}\\
	\hline

    \hline
      \parbox[t]{4cm}{\textbf{Ukończono?}} &
      \parbox[t]{11.5cm}{\strut
        Tak.
      \strut}\\

      \hline
  \end{tabular}


\subsubsection{Zadania wykraczające poza pierwotne wymagania}

\begin{tabular}{ | l | l | }
	\hline
		\textbf{Identyfikator} &
		ZZ-03
		\\

	\hline
		\textbf{Treść} & \parbox[t]{11.5cm}{\strut
			Połączenie po autoryzacji powinno zostać zabezpieczone losowym łańcuchem
      znaków -- kluczem sesji. Klucz sesji powinien być generowany
      automatycznie po stronie backendu i wysyłany frontendowi tuż po
      autoryzacji użytkownika. Od tej pory backend nie zaakceptuje żadnej
      wiadomości od frontendu, dopóki nie
      będzie zawierała dodatkowego parametru \texttt{session\_key} z treścią
      klucza sesji. Celem tego środka jest uniknięcie sytuacji zwielokrotnienia
      klientów na tym samym kanale w razie zresetowania lub wielokrotnego
      nawiązywania połączenia z tego samego urządzenia.
		\strut}\\

	\hline
		\parbox[t]{4cm}{\textbf{Kryteria akceptacji}} & \parbox[t]{11.5cm}{\strut
			\begin{enumreq}
				\item W momencie autoryzacji, backend wygeneruje 32-znakowy,
        alfanumeryczny, losowy klucz.
        \item Backend odeśle klucz z informacją o autoryzacji do frontendu.
        \item Frontent zapisze i utrzyma otrzymany klucz sesji
        \item Frontent będzie automatycznie dodawać parametr \texttt{session\_key},
        otrzymany podczas autoryzacji, do każdej kolejnej, dodawanej wiadomości.
			\end{enumreq}
			\strut}
		\\

    \hline
  		\parbox[t]{4cm}{\textbf{Nakład godzinowy (planowany / włożony)}} &
      \parbox[t]{11.5cm}{\strut
  			2h / 1h
  		\strut}\\

        \hline
          \parbox[t]{4cm}{\textbf{Ukończono?}} &
          \parbox[t]{11.5cm}{\strut
            Tak.
          \strut}\\

          \hline
      \end{tabular}

      \vspace{1em}


\textbf{Termin realizacji:} 24 marca -- 6 kwietnia 2019 r.

\subsection{Cel sprintu}
Po zakończeniu sprintu, powinno być możliwe podpięcie się do pokoju i rozmowa z
innymi użytkownikami, którzy są podpięci do tego samego pokoju.

\subsubsection{Zadania oparte o pierwotne wymagania}

\begin{tabular}{ | l | l | }
	\hline
		\textbf{Identyfikator} &
		WF-02
		\\

	\hline
		\textbf{Treść} & \parbox[t]{11.5cm}{\strut
			Jako użytkownik serwera czatu, chcę wpiąć się do pokoju,
			aby wziąć udział w dyskusji.
		\strut}\\

	\hline
		\parbox[t]{4cm}{\textbf{Kryteria akceptacji}} & \parbox[t]{11.5cm}{\strut
			\begin{enumreq}
				\item Użytkownik, który ma otwartą sesję
				połączenia z serwerem czatu i nie jest wpięty
				do żadnego pokoju, zobaczy listę pokojów.
				\item Użytkownik, po kilknięciu w liście pokojów
				na nazwę pokoju, zostanie do niego podpięty
				\item Użytkownik po wpięciu się do pokoju zobaczy
				okno pokoju
				\item Użytkownik, który ma otwartą sesję
				połączenia z serwerem i jest wpięty do pokoju,
				po odświeżeniu przeglądarki zobaczy okno pokoju,
				do którego jest wpięty
			\end{enumreq}
			\strut}
		\\

    \hline
      \parbox[t]{4cm}{\textbf{Nakład godzinowy (planowany / włożony)}} &
      \parbox[t]{11.5cm}{\strut
        2h / 1h
      \strut}\\

        \hline
          \parbox[t]{4cm}{\textbf{Ukończono?}} &
          \parbox[t]{11.5cm}{\strut
            Tak.
          \strut}\\

          \hline
      \end{tabular}

\vspace{1em}

\begin{tabular}{ | l | l | }
	\hline
		\textbf{Identyfikator} &
		WF-03
		\\

	\hline
		\textbf{Treść} & \parbox[t]{11.5cm}{\strut
			Jako użytkownik serwera czatu, chcę po wpięciu
			do pokoju zobaczyć ostatnie wiadomości wysłane
			przed moim dołączeniem, aby dowiedzieć się, co
			tam się obecnie dzieje.
		\strut}\\

	\hline
		\parbox[t]{4cm}{\textbf{Kryteria akceptacji}} & \parbox[t]{11.5cm}{\strut
			\begin{enumreq}
				\item Użytkownik po wpięciu się do pokoju zobaczy
				10 najnowszych wiadomości wysłanych do pokoju
				przed jego dołączeniem (lub mniej, jeżeli
				dotychczas nie wysłano do pokoju co najmniej
				10 wiadomości)
			\end{enumreq}
			\strut}
		\\

    \hline
      \parbox[t]{4cm}{\textbf{Nakład godzinowy (planowany / włożony)}} &
      \parbox[t]{11.5cm}{\strut
        2h / 1h
      \strut}\\

  \hline
    \parbox[t]{4cm}{\textbf{Ukończono?}} &
    \parbox[t]{11.5cm}{\strut
      Tak.
    \strut}\\

    \hline
\end{tabular}

\vspace{1em}

\begin{tabular}{ | l | l | }
	\hline
		\textbf{Identyfikator} &
		WF-04
		\\

	\hline
		\textbf{Treść} & \parbox[t]{11.5cm}{\strut
			Jako użytkownik serwera czatu, chcę wysłać
			wiadomość do pokoju w który jestem wpięty, aby
			zobaczyli ją inni uczestnicy dyskusji.
		\strut}\\

	\hline
		\parbox[t]{4cm}{\textbf{Kryteria akceptacji}} & \parbox[t]{11.5cm}{\strut
			\begin{enumreq}
				\item Użytkownik wpisze tekst wiadomości w polu
				tekstowym u dołu czatu
				\item Wiadomość wpisana w polu tekstowym zostanie
				wysłana po wciśnięciu klawisza ,,Enter'', gdy aktywne
				będzie pole tekstowe
				\item Wiadomość wpisana w polu tekstowym zostanie
				wysłana po naciśnięciu przycisku ,,Wyślij'',
				widocznego obok pola tekstowego
				\item Po wysłaniu wiadomości, pole tekstowe zostanie
				wyczyszczone (niezależnie od tego czy wiadomość
				zostanie doręczona)
				\item Wiadomość wysłana do pokoju jest pokazywana
				wszystkim użytkownikom podpiętym do czatu u dołu
				strony
				\item Nowa wiadomość jest pokazywana wraz z nazwą
				użytkownika wysyłającego u dołu konwersacji
			\end{enumreq}
			\strut}
		\\

    \hline
      \parbox[t]{4cm}{\textbf{Nakład godzinowy (planowany / włożony)}} &
      \parbox[t]{11.5cm}{\strut
        2h / 1h
      \strut}\\
	\hline

    \parbox[t]{4cm}{\textbf{Ukończono?}} &
    \parbox[t]{11.5cm}{\strut
      Tak.
    \strut}\\

    \hline
\end{tabular}

\vspace{1em}

\begin{tabular}{ | l | l | }
	\hline
		\textbf{Identyfikator} &
		WF-07
		\\

	\hline
		\textbf{Treść} & \parbox[t]{11.5cm}{\strut
			Jako użytkownik serwera czatu, chcę odpiąć się od pokoju,
			aby wpiąć się do innego pokoju.
		\strut}\\

	\hline
		\parbox[t]{4cm}{\textbf{Powiązane zasady biznesowe}} & \parbox[t]{11.5cm}{\strut
			ZP-06 Użytkownik może się samodzielnie wypiąć z pokoju,
			do którego jest wpięty

		\strut}\\

	\hline
		\parbox[t]{4cm}{\textbf{Kryteria akceptacji}} & \parbox[t]{11.5cm}{\strut
			\begin{enumreq}
				\item W oknie pokoju użytkownik zobaczy przycisk
				lub link ,,Opuść pokój''.
				\item Po kliknięciu w ,,Opuść pokój'', użytkownik
				zobaczy listę pokojów.
			\end{enumreq}
			\strut}
		\\

    \hline
      \parbox[t]{4cm}{\textbf{Nakład godzinowy (planowany / włożony)}} &
      \parbox[t]{11.5cm}{\strut
        2h / 1h
      \strut}\\

      \hline
        \parbox[t]{4cm}{\textbf{Ukończono?}} &
        \parbox[t]{11.5cm}{\strut
          Tak.
        \strut}\\
	\hline
\end{tabular}

\subsubsection{Zadania wykraczające poza pierwotne wymagania}

\subsection{Sprint 4}

\textbf{Termin realizacji:} 7-27 kwietnia 2019 r.

\subsection{Cel sprintu}
Sprint ma na celu wykonanie mechanizmu prywatnych wiadomości. Wlicza się do
niego lista użytkowników obecnych na czacie, z którymi ma być prowadzona
rozmowa, a także nawiązywanie rozmowy w pokoju PW oraz wysyłanie i odbiór
wiadomości w pokoju PW.

\subsubsection{Zadania oparte o pierwotne wymagania}

\begin{tabular}{ | l | l | }
	\hline
		\textbf{Identyfikator} &
		WF-08
		\\

	\hline
		\textbf{Treść} & \parbox[t]{11.5cm}{\strut
			Jako użytkownik serwera czatu, chcę zobaczyć
			okno wiadomości prywatnych, aby odczytać wiadomości,
			które wysłano specjalnie do mnie.
		\strut}\\


	\hline
		\parbox[t]{4cm}{\textbf{Kryteria akceptacji}} & \parbox[t]{11.5cm}{\strut
			\begin{enumreq}
				\item Po kliknięciu w link ,,PW'', użytkownik
				zobaczy okno prywatnych wiadomości
				\item W oknie wiadomości prywatnych, użytkownik
				zobaczy listę użytkowników, od których otrzymał
				wiadomości prywatne.
				\item Po kliknięciu w link z nazwą użytkownika,
				użytkownik zobaczy prywatne wiadomości, których
				nadawcą i odbiorcą jest wskazana osoba.
			\end{enumreq}
			\strut}
		\\
    \hline
      \parbox[t]{4cm}{\textbf{Nakład godzinowy (planowany / włożony)}} &
      \parbox[t]{11.5cm}{\strut
        ...
      \strut}\\

      \hline
        \parbox[t]{4cm}{\textbf{Ukończono?}} &
        \parbox[t]{11.5cm}{\strut
          Tak.
        \strut}\\
	\hline
\end{tabular}

\vspace{1em}

\begin{tabular}{ | l | l | }
	\hline
		\textbf{Identyfikator} &
		WF-09
		\\

	\hline
		\textbf{Treść} & \parbox[t]{11.5cm}{\strut
			Jako użytkownik serwera czatu, chcę wysłać
			wiadomość prywatną do jednego użytkownika, aby
			prowadzić z nim ciągłą konwersację.
		\strut}\\

	\hline
		\parbox[t]{4cm}{\textbf{Kryteria akceptacji}} & \parbox[t]{11.5cm}{\strut
			\begin{enumreq}
				\item Użytkownik wpisze tekst wiadomości w polu
				tekstowym u dołu okna wiadomości prywatnych
				\item Wiadomość wpisana w polu tekstowym zostanie
				wysłana po wciśnięciu klawisza ,,Enter'', gdy
				aktywne
				będzie pole tekstowe
				\item Wiadomość wpisana w polu tekstowym zostanie
				wysłana po naciśnięciu przycisku ,,Wyślij'',
				widocznego obok pola tekstowego
				\item Po wysłaniu wiadomości, pole tekstowe zostanie
				wyczyszczone (niezależnie od tego czy wiadomość
				zostanie doręczona)
				\item Wiadomość wysłana w oknie zostanie pokazana
				tylko użytkownikowi, z którym trwa otwarta
				konwersacja
				\item Nowa wiadomość jest pokazywana wraz z nazwą
				użytkownika wysyłającego u dołu konwersacji
			\end{enumreq}
			\strut}
		\\

    \hline
      \parbox[t]{4cm}{\textbf{Nakład godzinowy (planowany / włożony)}} &
      \parbox[t]{11.5cm}{\strut
        2h / 1h
      \strut}\\

      \hline
        \parbox[t]{4cm}{\textbf{Ukończono?}} &
        \parbox[t]{11.5cm}{\strut
          Tak.
        \strut}\\
	\hline
\end{tabular}

\vspace{1em}

\begin{tabular}{ | l | l | }
	\hline
		\textbf{Identyfikator} &
		WF-11
		\\

	\hline
		\textbf{Treść} & \parbox[t]{11.5cm}{\strut
			Jako użytkownik serwera czatu, chcę wysłać wiadomość
			prywatną do innego użytkownika, z którym wcześniej nie
			wymieniałem takich wiadomości, aby rozpocząć z nim
			prywatną konwersację.
		\strut}\\

	\hline
		\parbox[t]{4cm}{\textbf{Kryteria akceptacji}} & \parbox[t]{11.5cm}{\strut
			\begin{enumreq}
				\item Użytkownik kliknie w oknie wiadomości
				prywatnych w przyciski ,,Nowy''.
				\item Użytkownik zobaczy monit o podanie nazwy
				użytkownika, z którym chce rozpocząć rozmowę
				\item Jeżeli użytkownik jest aktywny, wówczas
				\item Wiadomość wpisana w polu tekstowym zostanie
				wysłana po wciśnięciu klawisza ,,Enter'', gdy
				aktywne
				będzie pole tekstowe
				\item Wiadomość wpisana w polu tekstowym zostanie
				wysłana po naciśnięciu przycisku ,,Wyślij'',
				widocznego obok pola tekstowego
				\item Po wysłaniu wiadomości, pole tekstowe zostanie
				wyczyszczone (niezależnie od tego czy wiadomość
				zostanie doręczona)
				\item Wiadomość wysłana w oknie zostanie pokazana
				tylko użytkownikowi, z którym trwa otwarta
				konwersacja
				\item Nowa wiadomość jest pokazywana wraz z nazwą
				użytkownika wysyłającego u dołu konwersacji
			\end{enumreq}
			\strut}
		\\
    \hline
      \parbox[t]{4cm}{\textbf{Nakład godzinowy (planowany / włożony)}} &
      \parbox[t]{11.5cm}{\strut
        2h / 1h
      \strut}\\

      \hline
        \parbox[t]{4cm}{\textbf{Ukończono?}} &
        \parbox[t]{11.5cm}{\strut
          Tak.
        \strut}\\

	\hline
\end{tabular}


\subsection{Sprint 5}

\textbf{Termin realizacji:} 28 kwietnia -- 11 maja 2019 r.

\subsection{Cel sprintu}
Celem jest dokończenie mechanizmu prywatnych wiadomości, których nie udało się
dokończyć w poprzednim sprincie -- łączenia z wybranym użytkownikiem oraz
wysyłania i odbierania wiadomości w pokoju PW.

\subsection{Sprint 6}

\textbf{Termin realizacji:} 12-26 maja 2019 r.

\subsection{Cel sprintu}
W tym sprincie zaplanowano wykonanie narzędzi administracyjnych, służących
do zarządzania serwerem.

\subsection{Sprint 7}

\textbf{Termin realizacji:} 27 maja -- 6 czerwca 2019 r.

\subsection{Cel sprintu}
Celem sprintu jest wykonanie testów całego systemu, a także wykrycie i
poprawienie niedoróbek.
