\documentclass[11pt,oneside,a4paper,titlepage,onecolumn]{book}

\usepackage[utf8]{inputenc}
\usepackage{textcomp}
\usepackage[official]{eurosym}
\usepackage[polish]{babel}
\usepackage{amsthm}
\usepackage{graphicx}
\usepackage[T1]{fontenc}
\usepackage{scrextend}
\usepackage{hyperref}
\usepackage{xcolor}
\usepackage{enumitem}
\usepackage{geometry}
\geometry{a4paper, portrait, margin=2cm}
\graphicspath{ {./fig/} }
\usepackage{listings}

\newenvironment{enumreq}
{ \begin{enumerate}[topsep=0pt,itemsep=-1ex,partopsep=1ex,parsep=1ex] }
{ \end{enumerate}                  }

\newcommand{\inzmaintitlePL}{Viua VM w akcji}

%% Below code taken verbatim from:
%% https://tex.stackexchange.com/questions/262977/inserting-external-table-of-content-into-another-document
%%
\makeatletter
\newcommand{\usetocfromothersource}[1]{%
    \begingroup
    \makeatletter
    \IfFileExists{#1}{%
        \chapter*{\inzmaintitlePL~-- spis treści}
        \@input{#1}
        \@nobreakfalse
    }{}%
    \makeatother
    \endgroup
}
\makeatother

\setcounter{secnumdepth}{4}

\author{Zespół: Marek Marecki i Krzysztof Franek\\Promotor: dr hab. Marek A. Bednarczyk, prof. PJWSTK}
\title{%
    \inzmaintitlePL \\
    \large
    Implementacja języka wysokiego poziomu i \\
    prostej aplikacji\\
    ~\\
    Viua VM in action.\\
    Implementation of a high-level programming language and\\ a simple application}

\begin{document}

\maketitle

\usetocfromothersource{book.toc}

\newpage

\chapter*{Konspekt}
\setcounter{chapter}{1}

Praca składa się z 10 rozdziałów, z których każdy zostanie po krótce omówiony i wyjaśnione zostanie jego
znaczenie w całości pracy.

\paragraph*{Numerowanie rozdziałów w konspekcie} Nasza praca inżynierska jest składana w systemie \LaTeX. Jest
on bardzo wygodny i wiele rzeczy robi ,,za nas'' automatycznie, ale nie jest wolny od wad i niedogodności.
Jedną z nich jest fakt, że nie można dołączyć spisu treści z jednego rodzaju dokumentu do dokumentu o innym
typie -- dlatego konspekt jest ,,książką'' (\texttt{\textbackslash{}documentclass\{book\}}). W związku
z tym kiedy w konspekcie pojawia się rozdział ,,1.\textbf{\emph{x}}'' należy o nim myśleć jako o odnoszącym
się do rozdziału ,,\textbf{\emph{x}}'' w pracy inżynierskiej.

\section{Wprowadzenie}

We wprowadzeniu przedstawiamy układ pracy inżynierskiej definiując zadanie każdego z rozdziałów (taki
,,mini-konspekt''). Następnie przedstawiamy problem z którym zmierzymy się w pracy (wykorzystanie maszyny
Viua~VM ,,w akcji'') oraz prezentujemy rozwiązanie, do którego będziemy dążyć (implementacja języka
programowania wysokiego poziomu, oraz wytworzenie aplikacji użytkowej w tym języku).

\section{Strategia prowadzenia prac}

W tym rozdziale prezentujemy strategie, które wybraliśmy do przeprowadzenia każdej z trzech części pracy,
czyli:

\begin{enumerate}
    \item opracowania specyfikacji języka \ViuAct
    \item implementacji języka \ViuAct\ w formie kompilatora
    \item wytworzenia aplikacji użytkowej w języku \ViuAct, korzystając z implementacji języka wytworzonej w
        ramach pracy inżynierskiej
\end{enumerate}

Omawiamy również powody, dla których wybraliśmy taką, a nie inną strategię dla każdego z etapów prac.
Przedstawiamy też strategie testowania poszczególnych elementów pracy.

\section{Dokumentacja prac}

W tym rozdziale prezentujemy historię działań, które podjęliśmy podczas pracy nad projektem inżynierskim.
Omawiamy jakie zadania należało wykonać na poszczególnych etapach prac, i prezentujemy narzędzia jakie
wykorzystaliśmy do śledzenia postępów.

Ten rozdział zawiera również komentarz dotyczący wstępnych założeń harmonogramu i tego jak ich realizacja
wyglądała w rzeczywistości (np. czy wystąpiły opóźnienia i czym były spowodowane).

% \chapter*{Język \ViuAct}

\section{Język \ViuAct\ i jego kompilator -- formalności}

W tym rozdziale prezentowane są założenia wstępne dotyczące języka programowania, który ma zostać wytworzony
podczas projektu inżynierskiego. Omawiamy też jego kontekst biznesowy, charakteryzujemy oczekiwanych
użytkowników oraz definiujemy kryteria akceptacji wytworzonego języka i jego implementacji.

\section{Specyfikacja języka \ViuAct}

W tym rozdziale prezentujemy specyfikację języka \ViuAct. Opisujemy prezentowany model programowania, typy
danych występujące w języku, dostępne konstrukcje językowe (definicje funkcji, konstrukcję warunkową,
itd.). Omawiamy też bibliotekę standardową oraz interakcję języka z platformą Viua VM.

\section{Język \ViuAct\ i jego kompilator}

W tym rozdziale prezentowana jest implementacja języka \ViuAct\ -- kompilator. Prezentujemy użyte ,,wzorce
projektowe'' i dekompozycję kompilatora na podsystemy (w naszym przypadku odpowiadające ,,fazom'' kompilacji).

Omawiamy też zastosowane algorytmy oraz podjęte decyzje projektowe w świetle priorytetów implementacyjnych
(np. rezygnacja z pewnych funkcjonalności dostępnych zwyczajowo w kompilatorach produkcyjnych z uwagi na małą
ilość czasu).

Oprócz opisu implementacji ten rozdział zawiera również dyskusję na temat testowania implementacji języka
programowania, oraz krótką instrukcję użytkownika kompilatora.

% \chapter*{Program ViuaChat}

\section{Program ViuaChat -- formalności}

Ten rozdział opisuje podstawowe założenia dotyczące
czatu ViuaChat. Jest to w istocie zestawienie dokumentacji przedwykonawczej, która powstała przed przystąpieniem do prac programistycznych. W związku z tym, można w niej odnaleźć:
\begin{enumerate}
  \item Założenia metodyki wytwarzania
  \item Projekt systemu
  \item Specyfikację przypadków użycia
\end{enumerate}

\section{Program ViuaChat}

W tym rozdziale opisano sposób, w jaki sposób system był wytwarzany. Pierwsza część rozdziału, zatytułowana jako ,,Wymagania biznesowe'', będzie zawierała zestawienie suchych wymagań biznesowych. Na ich bazie zostaną opracowane historyjki (ang. \textit{user stories}), stanowiące zawartość drugiej części o nazwie ,,Backlog produktu''. Z kolei trzecia część zostanie poświęcona już samemu procesowi wytwarzania oprogramowania. Zostanie ona podzielna na sekcje, z których każda zostanie poświęcona jednemu ze sprintów.

% \chapter*{Praca inżynierska, a ekosystem Viua VM}

\section{Wpływ pracy na platformę Viua VM}

Ten rozdział opisuje wpływ naszej pracy inżynierskiej na platformę Viua VM jako taką, oraz na środowisko
dookoła niej. Omawiamy w nim zmiany, których wprowadzenie w kodzie maszyny wirtualnej wymagane było do
ukończenia projektu inżynierskiego, oraz nowe projekty Free Software opublikowane podczas prac nad projektem
-- biblioteki implementujące mechanizmy (de)serializacji danych oraz komunikację po protokole WebSocket.

% \chapter*{Ostatnie rozdziały}

\section{Wkład własny członków zespołu}

W tym rozdziale każdy z członków zespołu omawia swój indywidualny wkład własny w wykonanie projektu
inżynierskiego.

\section{Słownik pojęć}

W tym rozdziale definiujemy pojęcia pojawiające się w pracy, które mogą być niejednoznaczne bądź nieznane
czytelnikowi.

\section{Bibliografia}

W tym rozdziale wyliczamy artykuły, książki, opracowania itp., z których korzystaliśmy podczas prac nad
projektem inżynierskim.

\end{document}
