\chapter{Wkład własny członków zespołu}
\label{wklad_wlasny}

\section{Wspólnie zrealizowane zadania}

Obaj członkowie zespołu, Marek Marecki i Krzysztof Franek, brali udział w pracach
nad składnią języka \ViuAct. Wspólnym ustaleniom podlegały zastosowane słowa
kluczowe, sposoby budowania wyrażeń, zakres biblioteki standardowej oraz
zbiór struktur jakie miały być dostarczane przez język.

Wspólnie ustalono również zakres funkcjonalności, których obecność była porządana
w systemie ViuaChat. Zakres ten skodyfikowano w postaci zasad biznesowych, będących
przyczynkiem do dalszych samodzielnych prac Krzysztofa Franka. Ożywionej
konsultacji poddawano w szczególności te z nich, z którymi wiązałyby
większe nakłady pracy na etapie szczegółowego projektowania czy implementacji.

\section{Marek Marecki}

Głównymi zadaniami Marka Mareckiego było opracowanie specyfikacji języka
\ViuAct\ oraz implementacja kompilatora i biblioteki standardowej tegoż języka.
Dodatkowymi zadaniami, które wynikły w trakcie projektu były
\begin{enumerate*}[label=(\arabic*)]
\item implementacja bibliotek, na które zapotrzebowanie zgłosił Krzysztof Franek
\item dostosowanie Viua~VM do potrzeb projektu przez drobne poprawki i przeróbki
\item stworzenie skryptu tworzącego środowisko programistyczne, który pozwolił
    obu członkom zespołu pracować w jednolitych warunkach
\end{enumerate*}.

Specyfikacja języka \ViuAct\ obejmowała między innymi określenie ogólnego
charakteru i semantyki języka, oraz skonkretyzowanie tych ogólnych założeń w
formie konstrukcji językowych, typów danych i wbudowanych bezpośrednio w język
mechanizmów (np.~wywołania \emph{watchdog}, komunikacja przez wymianę
wiadomości) w stopniu na tyle dokładnym, aby możliwa była ich implementacja.

Marek Marecki zaprojektował, a następnie zaimplementował kompilator języka
\ViuAct\ -- od podsystemów odpowiedzialnych za analizę leksykalną i składniową,
do podsystemu emitującego kod wynikowy w języku assemblera Viua~VM. Elementem
implementacji kompilatora języka \ViuAct\ było również opracowanie sposobów
alokacji rejestrów (przydzielenia rejestru każdej wartości nazwanej -
tj.~będącej operandem dowiązania \emph{let} - i anonimowej), tłumaczenia
konstrukcji językowych mających swoje bezpośrednie odpowiedniki w języku
assemblera Viua~VM (np.~dowiązania \emph{let} czy wyrażenia warunkowe), oraz
redukcji konstrukcji, które nie posiadały bezpośrednich odpowiedników
(np.~wyliczenia).

W gestii Marka Mareckiego była również implementacja biblioteki standardowej
języka \ViuAct\ oraz pozostałych bibliotek potrzebnych Krzysztofowi Frankowi w
pracach nad czatem. Jedną z tych pozostałych bibliotek jest biblioteka
\texttt{plain-websocket} (w języku \ViuAct\ dostępna jako moduł
\texttt{Websocket}) opisana w dodatku \ref{plain_websocket_library} na stronie
\pageref{plain_websocket_library}.

Jako zadanie poboczne powstał skrypt \texttt{viuact-bootstrap.sh}, którego
zadaniem jest tworzenie jednolitego środowiska do pracy na językiem \ViuAct\ i
programem ViuaChat. Sposób użycia skryptu jest opisany we wstępie do rozdziału
\ref{przebieg_prac}.~\nameref{przebieg_prac} na stronie \pageref{przebieg_prac}.

Dodatkowym narzędziemy wytworzonym na potrzeby utrzymania jednolitego stylu
pisanych programów jest \texttt{viuact-format} czyli program automatycznie
formatujący kod źródłowy programów napisanych w języku \ViuAct.

\section{Krzysztof Franek}

W trakcie prac nad projektem dyplomowym, Krzysztof Franek był przede wszystkim
odpowiedzialny za budowę systemu czatu ViuaChat. W zakresie jego zadań znajdowało
się jego zaprojektowanie -- przygotowanie skonkretyzowanego zakresu wymagań,
\textit{user stories} i przypadków użycia, a także dobór pobocznych technologii
wykorzystanych do przygotowania aplikacji. Odpowiadał również za wybór strategii
i metodologię prac prowadzonych przy systemie ViuaChat.

Krzysztof Franek opracował schematy budowy w makro- i mikroskali
systemu czatu, tak od frontendu (podział
i zadania poszczególnych modułów), jak i od backendu (konstrukcja, podział obowiązków
i zasady współpracy poszczególnych aktorów). Na końcu, Krzysztof wziął na siebie
faktyczną implementację oraz wdrożenie aplikacji ViuaChat. Na jego barki spadła
także konfiguracja sprzętu demonstrującego działanie tej aplikacji w trakcie
prezentacji inżynierskiej. Obejmowało to również przygotowanie obrazu środowiska
wytwórczego (opartego o dystrybucję Linux Mint).

Oprócz tego, Krzysztof w sposób faktyczny testował użyteczność języka
\ViuAct, samodzielnie konstruując backend ViuaChat. Przekazywał na bieżąco
swoje uwagi oraz raportował wykryte błędy Markowi Mareckiemu, znacznie
przyspieszając rozwój kompilatora oraz wnosząc istotny wkład w doskonalenie
składni języka ViuAct. W zaresie jego zadań znajdowało się opracowanie listy
wymaganych bibliotek, które wykraczały poza pierwotny zakres biblioteki
standardowej języka \ViuAct, a które -- jego zdaniem -- były niezbędne
do implementacji ViuaChat.

W zakresie zainteresowań Krzysztofa znajdowała się opieka merytoryczna nad
kształtem i rozwojem dokumentacji w zakresie całego projektu, w tym niniejszej
pracy pisemnej oraz jej półproduktów, zastosowany układ tekstu oraz sposób
prezentacji zgromadzonych informacji.
