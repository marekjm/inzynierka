\chapter{Wkład własny członków zespołu}
\label{wklad_wlasny}

\section{Wspólnie zrealizowane zadania}

Obu członków zespołu, Marek Marecki i Krzysztof Franek, brało udział w pracach
nad składnią języka \ViuAct. Wspólnym ustaleniom podlegały zastosowane słowa
kluczowe, sposoby budowania wyrażeń, zakres biblioteki standardowej oraz
zbiór struktur jakie miały być dostarczane przez język.

Wspólnie ustalono również zakres funkcjonalności, których obecność była porządana
w systemie ViuaChat. Zakres ten skodyfikowano w postaci zasad biznesowych, będących
przyczynkiem do dalszych samodzielnych prac Krzysztofa Franka. Ożywionej
konsultacji poddawano w szczególności te z nich, z którymi wiązałyby
większe nakłady pracy na etapie szczegółowego projektowania czy implementacji.

\section{Marek Marecki}

Opracowanie specyfikacji języka \ViuAct.
Wytworzenie kompilatora języka \ViuAct.
Wytworzenie skrypu \texttt{viuact-bootstrap.sh}.
Wytworzenie biblioteki do obsługi protokołu WebSocket.

\section{Krzysztof Franek}

W trakcie prac nad projektem dyplomowym, Krzysztof Franek był przede wszystkim
odpowiedzialny za budowę systemu czatu ViuaChat. W zakresie jego zadań znajdowało
się jego zaprojektowanie -- przygotowanie skonkretyzowanego zakresu wymagań,
\textit{user stories} i przypadków użycia, a także dobór pobocznych technologii
wykorzystanych do przygotowania aplikacji. Odpowiadał również za wybór strategii
i metodologię prac prowadzonych przy systemie ViuaChat.

Krzysztof Franek opracował schematy budowy w makro- i mikroskali
systemu czatu, tak od frontendu (podział
i zadania poszczególnych modułów), jak i od backendu (konstrukcja, podział obowiązków
i zasady współpracy poszczególnych aktorów). Na końcu, Krzysztof wziął na siebie
faktyczną implementację oraz wdrożenie aplikacji ViuaChat. Na jego barki spadła
także konfiguracja sprzętu demonstrującego działanie tej aplikacji w trakcie
prezentacji inżynierskiej. Obejmowało to również przygotowanie obrazu środowiska
wytwórczego (opartego o dystrybucję Linux Mint).

Oprócz tego, Krzysztof w sposób faktyczny testował użyteczność języka
\ViuAct, samodzielnie konstruując backend ViuaChat. Przekazywał na bieżąco
swoje uwagi oraz raportował wykryte błędy Markowi Mareckiemu, znacznie
przyspieszając rozwój kompilatora oraz wnosząc istotny wkład w doskonalenie
składni języka ViuAct. W zaresie jego zadań znajdowało się opracowanie listy
wymaganych bibliotek, które wykraczały poza pierwotny zakres biblioteki
standardowej języka \ViuAct, a które -- jego zdaniem -- były niezbędne
do implementacji ViuaChat.

W zakresie zainteresowań Krzysztofa znajdowała się opieka merytoryczna nad
kształtem i rozwojem dokumentacji w zakresie całego projektu, w tym niniejszej
pracy pisemnej oraz jej półproduktów, zastosowany układ tekstu oraz sposób
prezentacji zgromadzonych informacji.
