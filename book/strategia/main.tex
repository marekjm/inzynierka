\chapter{Strategia prowadzenia prac}
\label{strategia_prac}

W tym rozdziale opisujemy metodologię prac nad poszczególnymi częściami projektu. Każda z nich była rozwijana
innym sposobem z powodu dużych rozbieżności specyfiki konkretnych problemów -- zupełnie inaczej pracuje się
nad semi-formalną specyfikacją języka programowania niż nad aplikacją użytkową, więc podjęliśmy decyzję o
rozdzieleniu tych części i poprowadzeniu ich jako osobne, mniejsze projekty wewnątrz pracy inżynierskiej.

\section{Język \ViuAct}

% Specyfikacja. Analiza wymagań. Formalizm. Kaskada.

Tworzenie specyfikacji języka odbywa się w modelu kaskadowo-prototypowym i przeplata się z implementacją
prototypu kompilatora.

\subsection{Określenie wymagań}

Przed rozpoczęciem specyfikowania poszczególnych konstrukcji językowych należało podjąć decyzję co w języku
powinno się znaleźć, a co nie powinno zostać włączone do specyfikacji.

\subsection{Specyfikacja konkretnej konstrukcji języka}

Każda konstrukcja językowa jest najpierw projektowana, następnie dokumentowana i omawiana (żeby wychwycić
pewne oczywiste błędy ,,grube'' na wczesnym etapie prac). Potem proces specyfikacji zostaje zawieszony do
momentu wytworzenia w kompilatorze prototypu danej konstrukcji, lub odrzucenia jej jako niewykonalnej w
zakładanym czasie lub zakresie funkcjonalności.

Wynik prac prototypowych (sukces bądź porażka) decyduje o tym czy prace specyfikacyjne są prowadzone dalej.
Jeśli konstrukcja jest możliwa do zaimplementowania jej specyfikacja jest formalizowana oraz przygotowana dla
niej zostaje dokumentacja (obejmująca m.in. przykłady użycia).

\section{Kompilator języka \ViuAct}

Prace nad kompilatorem języka \ViuAct\ były prowadzone w identyczny sposób jak prace nad Viua VM.
Jest to połączenie strategii prototypowania i iteracji, które spawdza się przy projekcie rozwijanym jako
open-source (mowa tu o ,,open-source'' jako modelu wytwarzania, nie jako o ruchu alternatywnym do
Free~Software).

Zadania w tej części projektu były śledzone za pomocą narzędzia
\texttt{issue}\footnote{\url{https://github.com/marekjm/issue}}, które jest również wykorzystywane przez
jednego z członków zespołu do śledzenia zadań w pracy zawodowej i projektach open-source.

Prace implementacyjne przeplatały się z pracami nad specyfikacją w celu weryfikowania na bieżąco co jest
możliwe do wykonania w zamierzonym czasie w całości, co tylko w części, a co nie jest możliwe do wykonania
przy zakładanych zasobach czasowo-ludzkich.

\subsection{Testowanie}

Zestaw prostych programów testowych. Test sprawdza czy program się kompiluje i czy po uruchomieniu daje
spodziewane wyniki. Jest to strategia zaadaptowana z projektu Viua VM, gdzie jest z powodzeniem
wykorzystywana.

\section{Program ViuaChat}

Mini-SCRUM.

\subsection{Testowanie}

Jak wygląda testowanie czatu.
