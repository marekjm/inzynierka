\chapter{Biblioteka \texttt{plain-websocket}}
\label{plain_websocket_library}

Biblioteka \texttt{plain-websocket} jest dostarczana jako zestaw funkcji
opakowujący \emph{POSIX sockets}, wyliczenie kodów operacji w protokole
WebSocket i strukturę opisującą pojedynczą ramkę protokołu. Do użycia biblioteki
nie jest wymagane projektowanie programu dookoła \emph{event loop} -- kontrola
zostaje w całości po stronie programu używającego biblioteki.

\section{Sposób użycia}

Funkcji dostarczanych przez bibliotekę \texttt{plain-websocket} używa się między
wywołaniem \texttt{accept(3)}, a \texttt{shutdown(3)}.

Po uzyskaniu połączenia za pomocą standardowych wywołań z API \emph{POSIX
sockets}, należy wykonać \emph{handshake} protokołu WebSocket za pomocą funkcji
\texttt{Websocket.handshake()}. Po zakończeniu pracy należy zakończyć połączenie
w warstwie protokołu WebSocket za pomocą funkcji \texttt{Websocket.close()}.

\subsection{C++}

Przykład kodu w języku C++:
\begin{small}
\begin{lstlisting}
auto handle_websocket_connection(int sock) -> void {
    auto ws = maelkum::websocket::handshake(sock);

    auto message = maelkum::websocket::read_frame(sock);
    if (message.has_value()) {
        auto const text = maelkum::websocket::get_text(*message);
        if (text.has_value()) {
            std::cerr << *text << "\n";
            maelkum::websocket::write(sock, *text);
        }
    }

    maelkum::close(sock);
}

auto main() -> int {
    auto const host = "127.0.0.1";

    auto server_sock = socket(AF_INET, SOCK_STREAM, 0);
    sockaddr_in addr;
    addr.sin_family = AF_INET;
    addr.sin_port = htons(9090);
    inet_pton(AF_INET, host, &addr.sin_addr.s_addr);

    bind(sock, reinterpret_cast<sockaddr*>(&addr), sizeof(addr));
    listen(sock, 0);

    auto client = accept(sock, nullptr, nullptr);

    handle_websocket_connection(sock);

    shutdown(sock, SHUT_RDWR);
    close(sock);

    shutdown(server_sock, SHUT_RDWR);
    close(server_sock);

    return 0;
}
\end{lstlisting}
\end{small}

\subsection{\ViuAct}

Przykład kodu w języku \ViuAct:
\begin{small}
\begin{lstlisting}
(import Std.Posix.Network)
(import Websocket)

(let handle_websocket_connection (sock) {
    (Websocket.handshake sock)
    (let message (Websocket.read sock))
    (print message)
    (Websocket.write sock message)
    (Websocket.close sock)
    0
})
(let main () {
    (let server_sock (Std.Posix.Network.sock))
    (Std.Posix.Network.bind sock "127.0.0.1" 9090)
    (Std.Posix.Network.listen sock 16)

    (let client (Std.Posix.Network.accept sock))
    (handle_websocket_connection sock)

    (Std.Posix.Network.close sock)
    (Std.Posix.Network.close server_sock)
    0
})
\end{lstlisting}
\end{small}

\section{API}
