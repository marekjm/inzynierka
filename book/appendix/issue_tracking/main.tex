\chapter{Narzędzie śledzenia zadań \texttt{issue}}
\label{issue_tracking_tool}

Narzędzie \texttt{issue} jest to rozproszony system śledzenia zadań.
Zapewnia ono również wysoką ergonomię i prędkość pracy w środowisku tekstowym,
oraz niezbędne minimum funkcjonalności potrzebne do wykonania typowych operacji
związanych ze śledzeniem zadań w projekcie programistycznym. Narzędzie to jest
rozwijane jako Free~Software na licencji GNU~GPL v3 i jest dostępne do pobrania
z serwisu GitHub: \url{https://github.com/marekjm/issue}

\section{Dane w rozproszonym systemie śledzenia zadań}

Informacje o zadaniach, które są zachowywane obejmują:

\begin{enumerate}
\item tytuł i opis zadania
\item datę utworzenia zadania i datę zamknięcia zadania
\item osobę tworzącą i zamykającą zadanie
\item globalnie unikalny identyfikator (GUID) zadania
\item komentarze
\item tagi
\end{enumerate}

Wszystkie modyfikacje są realizowane \emph{offline} i zapisywane w formie plików
,,zmian'' w zadaniu. Takie ,,rozproszenie'' stanu każdego zadania na mnóstwo
mniejszych plików powoduje, że baza danych systemu \texttt{issue} musi być
okresowo indeksowana żeby jej stan prezentowany użytkownikowi nie był
przekłamany, a jednocześnie uniknąć konsolidowania wszystkich zmian dotyczących
zadania za każdym razem kiedy jest ono pokazywane użytkownikowi.

Decyzja o rozbiciu stanu zadania na mnóstwo małych plików jest wymuszona przez
inną decyzję projektową -- rozproszenie systemu i umożliwienie pracy całkowicie
\emph{offline}, co jest jedną z najważniejszych funkcjonalności \texttt{issue}.
Zdecydowanie łatwiej jest pobrać wiele mniejszych plików i utworzyć z nich
lokalny indeks niż projektować algorytm rozproszonej współpracy i negocjowania
zmian w plikach.

,,Baza danych'' programu issue to po prostu zbiór plików umieszczonych w
określonej strukturze katalogów. Umożliwia to bezproblemowe zagnieżdżenie jej
w repozytorium systemu kontroli wersji Git. Narzędzie \texttt{issue} dostarcza
również swój własny sposób dystrybucji plików opisujących zadania dla osob,
które nie chcą zagnieżdżać repozytorium \texttt{issue} w repozytorium kodu
źródłowego. Takie zagnieżdżenie jest jednak zalecaną strategią pracy z
narzędziem \texttt{issue}.

\section{Operacje}

Operacje, które są oferowane przez \texttt{issue} to między innymi:

\begin{enumerate}
\item otwieranie nowych zadań
\item zamykanie zadań
\item łączenie zadań w drzewa (tworzy to zależności między zadaniami, a otwarte
    zadanie ,,liść'' uniemożliwia zamknięcie zadania ,,gałęzi'')
\item automatyczne tworzenie gałęzi w systemie śledzenia wersji Git na podstawie
    tytułów zadań
\item komentowanie zadań
\end{enumerate}
