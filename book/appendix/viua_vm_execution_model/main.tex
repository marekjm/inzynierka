\chapter{Model obliczeniowy Viua VM}
\label{appendix_viua_vm_execution_model}

Model obliczeniowy Viua VM jest oparty na modelu aktorów. Bardzo abstrakcyjnie i
skrótowo może być opisany podanymi poniżej cechami:

\begin{enumerate}
\item programy uruchamiane w Viua VM składają się z jednego lub większej liczby
    równolegle działających, odizolowanych od siebie aktorów (procesów)
\item każdy aktor jest implementowany przez pojedynczą funkcję ,,główną''
\item funkcje mogą wywoływać siebie nawzajem, tworząc stos wywołań
\item funkcje przyjmują wartości w parametrach, informują o wynikach swojej
    pracy zwracając wartości
\item każda funkcja jest reprezentowana przez uszeregowaną sekwencję instrukcji
\item instrukcje wewnątrz funkcji są wykonywane sekwencyjnie, jedna po drugiej
\item instrukcje wykonują pracę modyfikując wartości znajdujące się w rejestrach
\item każdy rejestr może przechowywać dowolną wartość
\item rejestry są grupowane w ,,zestawy'': lokalny (dla wartości lokalnych),
    argumentów (dla przekazanych argumentów), statyczny (dla wartości
    statycznych; ten jest zachowywany pomiędzy wywołaniami funkcji w ramach
    pojedynczego aktora)
\item każdy aktor może tworzyć nowe aktory
\item aktory komunikują się między sobą za pomocą asynchronicznej wymiany
    wiadomości
\item wiadomość jest dowolną wartością
\end{enumerate}

O instrukcjach można myśleć jak o funkcjach, które na wejściu dostają zestawy
rejestrów i adresy rejestrów, które powinny zmodyfikować, a na wyjściu podają
zmodyfikowane zestawy rejestrów. Ich sygnatura w języku OCaml mogłaby wyglądać
następująco:

\begin{lstlisting}
val %*\emph{op}*) : local_register_set
      -> static_register_set
      -> arguments_register_set
      -> register_address list
      = <fun>
\end{lstlisting}

Zasady działania poszczególnych instrukcji opisane są w dodatku \ref{appendix_viua_vm_assembly_language}.
