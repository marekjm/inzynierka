\subsection{Operacje bitowe}
\label{viua_vm_ops_bit_ops}

\subsubsection{\texttt{bits\_of\_integer}}

\begin{lstlisting}
bits_of_integer Rd Rs
\end{lstlisting}

\paragraph*{Opis} Konstruktor bitów z liczby całkowitej.

Dokonuje konwersji liczby całkowitej znajdującej się w rejestrze \texttt{Rs} na
ciąg bitów o długości 64. Wynik konwersji jest zapisywany w rejestrze
\texttt{Rt}.

Wartość w rejestrze \texttt{Rs} pozostaje niezmieniona.

\begin{lstlisting}
integer %1 local 42
bits_of_integer %1 local %1 local
print %1 local
\end{lstlisting}

Powyższy przykład spowoduje wydrukowanie na ekran następującego ciągu bitów:
\newline
\texttt{0000000000000000000000000000000000000000000000000000000000101010}.

\subsubsection{\texttt{integer\_of\_bits}}

\begin{lstlisting}
integer_of_bits Rd Rs
\end{lstlisting}

\paragraph*{Opis} Konstruktor liczby całkowitej z bitów.

Dokonuje konwersji ciągu bitów znajdującej się w rejestrze \texttt{Rs} na liczbę
całkowitą. Wynik konwersji jest zapisywany w rejestrze \texttt{Rt}.

Wartość w rejestrze \texttt{Rs} pozostaje niezmieniona.

\begin{lstlisting}
bits %1 local 0b0101010
integer_of_bits %1 local %1 local
print %1 local
\end{lstlisting}

Powyższy przykład spowoduje wydrukowanie na ekran liczby 42.

\paragraph*{Uwagi}

Konwersja ciągów dłuższych niż 64 bity powinna spowodować błąd. Jednak nie jest
to weryfikowane i obecna wersja maszyny ,,po cichu'' powoduje zawinięcie liczby.

Ciągi krótsze niż 64 bity przed konwersją są rozszerzane do długości 64 bitów.
Na najbardziej znaczące pozycje wpisywane są zera, co może spowodować produkować
,,zaskakujące'' wyniki, gdzie małe liczby ujemne są konwertowane na małe liczby
dodatnie.

Jeśli wymagane są częste konwersje między ciągami bitów, a sprzętowymi liczbami
całkowitymi zaleca się używanie ciągów bitów o długości 64 (odpowiadającej
szerokości bitowej sprzętowej liczby całkowitej). Jeśli konwersje nie są częste,
rozszerzenia ze znakiem (ang. \emph{sign extension}) można dokonać za pomocą
następującej sekwencji instrukcji (dla liczby reprezentowanej przez ciąg o
długości 16 bitów):
\begin{small}
\begin{lstlisting}
.name: 1 se     %*\emph{; ,,sign extended''}*)
.name: 2 input  %*\emph{; wejściowy ciąg bitów, do rozszerzenia}*)
.name: 3 n      %*\emph{; wartość pomocnicza}*)

integer %se local 64
bits    %se local %se local
bitor   %se local %se local %input local

integer %n local 48
shl     void %se local %n local
ashr    void %se local %n local
\end{lstlisting}
\end{small}
Po wykonaniu tej sekwencji instrukcji w rejestrze \texttt{se} znajduje się
rozszerzona wartość wejściowa.

\subsubsection{\texttt{bits}}

\begin{lstlisting}
bits Rd 0x%*\emph{<hex-digits>}*)
bits Rd 0o%*\emph{<octal-digits>}*)
bits Rd 0b%*\emph{<binary-digits>}*)
bits Rd Rs
\end{lstlisting}

\paragraph*{Opis} Konstruktor bitów.

W wariantach 1-3 konstruuje ciąg bitów z literału podanego jako drugi operand.
W wariancie 4 konstruuje ciąg bitów o długości podanej w rejestrze \texttt{Rs}.
We wszystkich wariantach wynikowy ciąg bitów zapisywany jest w rejestrze
\texttt{Rd}.

\begin{lstlisting}
bits %1 local 0xdeadbeef
bits %2 local 0o1337
bits %3 local 0b01

integer %4 local 8
bits %4 local %4 local
\end{lstlisting}

\subsubsection{\texttt{bitand}}

\begin{lstlisting}
bitand Rd Ra Rb
\end{lstlisting}

\paragraph*{Opis} Bitowa operacja \emph{\texttt{and}}.
Długość wynikowego ciągu bitów jest równa długości ciągu, który w znadował się w
rejestrze \texttt{Ra}.

\subsubsection{\texttt{bitor}}

\begin{lstlisting}
bitor Rd Ra Rb
\end{lstlisting}

\paragraph*{Opis} Bitowa operacja \emph{\texttt{or}}.
Długość wynikowego ciągu bitów jest równa długości ciągu, który w znadował się w
rejestrze \texttt{Ra}.

\subsubsection{\texttt{bitnot}}

\begin{lstlisting}
bitxor Rd Rs
\end{lstlisting}

\paragraph*{Opis} Bitowa operacja \emph{\texttt{not}}.
Konstruuje nowy ciąg bitów będący negacją ciągu bitów znajdującego się w
rejestrze \texttt{Rs}.

\subsubsection{\texttt{bitxor}}

\begin{lstlisting}
bitxor Rd Ra Rb
\end{lstlisting}

\paragraph*{Opis} Bitowa operacja \emph{\texttt{xor}}.
Długość wynikowego ciągu bitów jest równa długości ciągu, który w znadował się w
rejestrze \texttt{Ra}.

\subsubsection{\texttt{bitat}}

\begin{lstlisting}
bitat Rd Rs Ri
\end{lstlisting}

\paragraph*{Opis} Sprawdzenie wartości pojedynczego bitu.
Po wykonaniu instrukcji w rejestrze \texttt{Rd} znajduje się wartość boolowska
określająca czy w ciągu bitów znajdującym się w rejestrze \texttt{Rs} bit pod
indeksem określonym przez liczbę całkowitą znajdującą się w rejestrze
\texttt{Ri} jest włączony czy wyłączony.

\begin{lstlisting}
integer %1 local 0
bitat %3 local %2 local %1 local
print %3 local

integer %1 local 1
bitat %3 local %2 local %1 local
print %3 local
\end{lstlisting}
Powyższy przykład wydrukuje na ekran najpierw \texttt{true}, a potem
\texttt{false}.

\subsubsection{\texttt{bitset}}

\begin{lstlisting}
bitset Rt Ri %*\emph{<boolean-literal>}*)
\end{lstlisting}

\paragraph*{Opis} Ustawienie wartości pojedynczego bitu.
Po wykonaniu instrukcji w ciągu bitów znajdującym się w rejestrze \texttt{Rt}
bit znajdujący się pod indeksem określonym przez liczbę całkowitą znajdującą
się w rejestrze \texttt{Ri} jest
\begin{enumerate*}[label=(\arabic*)]
\item włączony jeśli \texttt{\emph{<boolean-literal>}} był wartością
    \texttt{true}
\item wyłączony jeśli \texttt{\emph{<boolean-literal>}} był wartością
    \texttt{fale}
\end{enumerate*}

\subsubsection{\texttt{shl}}

\begin{lstlisting}
shl Ro Rt Rn
\end{lstlisting}

\paragraph*{Opis} Przesunięcie bitowe w lewo.

Przesuwa ciąg bitów w rejestrze \texttt{Rt} o ilość bitów określoną przez liczbę
całkowitą znajdującą się w rejestrze \texttt{Rn}. Wartość w rejestrze
\texttt{Rt} jest modyfikowana \emph{w miejscu}, a w rejestrze \texttt{Ro}
umieszczany jest ,,wysunięty'' ciąg bitów.
\begin{lstlisting}
bits %3 local 0x4a
print %3 local
integer %4 local 4
shl %4 local %3 local %4 local
print %3 local
print %4 local
\end{lstlisting}

Powyższy przykład spowoduje wydrukowanie na ekranie następujących wartości:
\begin{lstlisting}
01001010
10100000
0100
\end{lstlisting}

\subsubsection{\texttt{shr}}

\begin{lstlisting}
shr Rd Rs Rn
\end{lstlisting}

\paragraph*{Opis} Przesunięcie bitowe w prawo.

Przesuwa ciąg bitów w rejestrze \texttt{Rt} o ilość bitów określoną przez liczbę
całkowitą znajdującą się w rejestrze \texttt{Rn}. Wartość w rejestrze
\texttt{Rt} jest modyfikowana \emph{w miejscu}, a w rejestrze \texttt{Ro}
umieszczany jest ,,wysunięty'' ciąg bitów.
\begin{lstlisting}
bits %3 local 0x4a
print %3 local
integer %4 local 4
shr %4 local %3 local %4 local
print %3 local
print %4 local
\end{lstlisting}

Powyższy przykład spowoduje wydrukowanie na ekranie następujących wartości:
\begin{lstlisting}
01001010
00000100
1010
\end{lstlisting}

\subsubsection{\texttt{ashl}}

\begin{lstlisting}
ashl Rd Rs Rn
\end{lstlisting}

\paragraph*{Opis} Arytmetyczne (z zachowaniem znaku) przesunięcie bitowe w lewo.

Przesuwa ciąg bitów w rejestrze \texttt{Rt} o ilość bitów określoną przez liczbę
całkowitą znajdującą się w rejestrze \texttt{Rn}. Wartość w rejestrze
\texttt{Rt} jest modyfikowana \emph{w miejscu}, a w rejestrze \texttt{Ro}
umieszczany jest ,,wysunięty'' ciąg bitów.
\begin{lstlisting}
bits %3 local 0xa4
print %3 local
integer %4 local 4
ashl %4 local %3 local %4 local
print %3 local
print %4 local
\end{lstlisting}

Powyższy przykład spowoduje wydrukowanie na ekranie następujących wartości:
\begin{lstlisting}
10100100
11000000
1010
\end{lstlisting}

\subsubsection{\texttt{ashr}}

\begin{lstlisting}
ashr Rd Rs Rn
\end{lstlisting}

\paragraph*{Opis} Arytmetyczne przesunięcie bitowe w prawo.

Przesuwa ciąg bitów w rejestrze \texttt{Rt} o ilość bitów określoną przez liczbę
całkowitą znajdującą się w rejestrze \texttt{Rn}. Wartość w rejestrze
\texttt{Rt} jest modyfikowana \emph{w miejscu}, a w rejestrze \texttt{Ro}
umieszczany jest ,,wysunięty'' ciąg bitów.
\begin{lstlisting}
bits %3 local 0xa4
print %3 local
integer %4 local 4
ashr %4 local %3 local %4 local
print %3 local
print %4 local
\end{lstlisting}

Powyższy przykład spowoduje wydrukowanie na ekranie następujących wartości:
\begin{lstlisting}
10100100
11111010
0100
\end{lstlisting}

\subsubsection{\texttt{rol}}

\begin{lstlisting}
rol Rt Rn
\end{lstlisting}

\paragraph*{Opis} Rotacja bitowa w lewo.

,,Obraca'' ciąg bitów w rejestrze \texttt{Rt} o ilość bitów określoną przez
liczbę całkowitą znajdującą się w rejestrze \texttt{Rn}. Wartość w rejestrze
\texttt{Rt} jest modyfikowana \emph{w miejscu}.
\begin{lstlisting}
bits %3 local 0x8e
integer %4 local 3
print %3 local
rol %3 local %4 local
print %3 local
\end{lstlisting}

Powyższy przykład spowoduje wydrukowanie na ekranie następujących wartości:
\begin{lstlisting}
10001110
01110100
\end{lstlisting}

\subsubsection{\texttt{ror}}

\begin{lstlisting}
ror Rt Rn
\end{lstlisting}

\paragraph*{Opis} Rotacja bitowa w prawo.

,,Obraca'' ciąg bitów w rejestrze \texttt{Rt} o ilość bitów określoną przez
liczbę całkowitą znajdującą się w rejestrze \texttt{Rn}. Wartość w rejestrze
\texttt{Rt} jest modyfikowana \emph{w miejscu}.
\begin{lstlisting}
bits %3 local 0x8e
integer %4 local 3
print %3 local
ror %3 local %4 local
print %3 local
\end{lstlisting}

Powyższy przykład spowoduje wydrukowanie na ekranie następujących wartości:
\begin{lstlisting}
10001110
11010001
\end{lstlisting}
