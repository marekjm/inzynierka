\subsection{Kontrola przepływu}
\label{viua_vm_ops_control_flow}

\paragraph*{Specyfikacje celów skoków}

\begin{lstlisting}
+%*\emph{<offset>}*)
-%*\emph{<offset>}*)
%*\emph{<marker>}*)
\end{lstlisting}

Pierwszy i drugi wariant przenoszą kontrolę o \emph{offset} instrukcji w przód
lub w tył. Trzeci wariant przenosi kontrolę do pierwszej instrukcji za markerem
\texttt{\emph{marker}}.

\subsubsection{\texttt{nop}}

\begin{lstlisting}
nop
\end{lstlisting}

\paragraph*{Opis} Instrukcja ,,pusta''.

\subsubsection{\texttt{jump}}

\begin{lstlisting}
jump %*\emph{<jump-spec>}*)
\end{lstlisting}

\paragraph*{Opis} Instrukcja skoku bezwarunkowego. Przenosi kontrolę do
instrukcji określonej przez \texttt{\emph{jump-spec}}.

\subsubsection{\texttt{if}}

\begin{lstlisting}
if Rcond %*\emph{<true-jump-spec>}*) %*\emph{<false-jump-spec>}*)
\end{lstlisting}

\paragraph*{Opis} Instrukcja skoku warunkowego odczytuje wartość z rejestru
\texttt{Rcond}, interpretuje ją jako wartość boolowską i podejmuje decyzję,
który skok wykonać. Jeśli warunek jest prawdziwy wykonywany jest skok
\texttt{\emph{true-jump-spec}}, jeśli warunek jest prawdziwy wykonywany jest
skok \texttt{\emph{false-jump-spec}}.

\subsubsection{\texttt{frame}}

\begin{lstlisting}
frame %*\emph{n}*)
\end{lstlisting}

\paragraph*{Opis} Przygotowuje do ramkę wywołania zawierającą \emph{n} rejestrów
do przekazania parametrów, np. \texttt{frame \%2} utworzy ramkę z dwoma
rejestrami.

Przygotowana ramka jest potem ,,konsumowana'' przez instrukcję \texttt{call},
\texttt{tailcall}, \texttt{process}, \texttt{defer} lub \texttt{watchdog}.

\subsubsection{\texttt{param}}

Instrukcja prywatna.
Przekazanie parametrów przez kopię odbywa się za pomocą instrukcji
\texttt{copy}.

\paragraph*{Uwagi}

Instrukcje prywatne nie są dostępne dla kodu użytkownika, a programy zawierające
je są odrzucane przez assembler jako nieprawidłowe.

\subsubsection{\texttt{pamv}}

Instrukcja prywatna.
Przekazanie parametrów przez przeniesienie odbywa się za pomocą instrukcji
\texttt{move}.

\paragraph*{Uwagi}

Instrukcje prywatne nie są dostępne dla kodu użytkownika, a programy zawierające
je są odrzucane przez assembler jako nieprawidłowe.

\subsubsection{\texttt{call}}

\begin{lstlisting}
call Rr %*\emph{function-name}*)
call Rr Rf
\end{lstlisting}

\paragraph*{Opis} Instrukcja wywołania funkcji lub instancji domknięcia.
Konsumuje utworzoną wcześniej ramkę wywołania i dodaje ją na szczyt stosu
wywołań.

Wariant 1 wywołuje funkcję określoną przez \texttt{\emph{function-name}}.
Wariant 2 wywołuje funkcję lub instancję domknięcia wskazywaną przez wartość w
rejestrze \texttt{Rf}. Tylko wariant drugi może być użyty do wywołania instancji
domknięcia.

W każdym wariancie wynik pracy funkcji jest umieszczany w rejestrze \texttt{Rr}.
Jeśli rejestr \texttt{Rr} jest podany jako \texttt{void} to wartość zwrotna
wywołanej funkcji jest natychmiast niszczona.

\subsubsection{\texttt{tailcall}}

\begin{lstlisting}
tailcall %*\emph{function-name}*)
tailcall Rf
\end{lstlisting}

\paragraph*{Opis} Instrukcja wywołania ogonowego funkcji lub instancji
domknięcia. Konsumuje utworzoną wcześniej ramkę wywołania i zamienia ją z ramką
znajdującą się na szczycie stosu wywołań -- nie zwiększając rozmiaru stosu
wywołań.

Wariant 1 wywołuje funkcję określoną przez \texttt{\emph{function-name}}.
Wariant 2 wywołuje funkcję lub instancję domknięcia wskazywaną przez wartość w
rejestrze \texttt{Rf}. Tylko wariant drugi może być użyty do wywołania instancji
domknięcia.

\subsubsection{\texttt{defer}}

\begin{lstlisting}
defer %*\emph{function-name}*)
defer Rf
\end{lstlisting}

\paragraph*{Opis} Instrukcja wywołania odroczonego funkcji. Konsumuje utworzoną
wcześniej ramkę wywołania.

Wariant 1 wywołuje funkcję określoną przez \texttt{\emph{function-name}}.
Wariant 2 wywołuje funkcję wskazywaną przez wartość w rejestrze \texttt{Rf}.

Wywołania odroczone będą wywołane w odwrotnej kolejności zgłaszania ich w
momencie, w którym ramka wywołania, w której zostały zarejestrowane jest
zdejmowana ze stosu wywołań (np. na skutek wykonania instrukcji \texttt{return},
podmiany ramki przez instrukcję \texttt{tailcall} lub odwinięcia stosu na skutek
nieobsłużonego wyjątku).

\subsubsection{\texttt{arg}}

Instrukcja prywatna.
Odczyt parametrów przez przeniesienie odbywa się za pomocą instrukcji
\texttt{move}.

\paragraph*{Uwagi}

Instrukcje prywatne nie są dostępne dla kodu użytkownika, a programy zawierające
je są odrzucane przez assembler jako nieprawidłowe.

\subsubsection{\texttt{allocate\_registers}}

\begin{lstlisting}
allocate_registers Rn
\end{lstlisting}

\paragraph*{Opis} Alokuje lokalny zestaw rejestrów w rozmiarze określonym przez
operand \texttt{Rn}. Jest to pierwsza instrukcja, która musi być wykonana przez
funkcję.

\paragraph*{Uwagi} Domknięcia \emph{nie mogą} używać tej instrukcji. Ich zestaw
rejestrów lokalnych jest alokowany przez instrukcję \texttt{closure}.

\subsubsection{\texttt{return}}

\begin{lstlisting}
return
\end{lstlisting}

\paragraph*{Opis} Kończy wykonywanie funkcji i zdejmuje jej ramkę ze stosu
wywołań.

\subsubsection{\texttt{halt}}

\begin{lstlisting}
halt
\end{lstlisting}

\paragraph*{Opis} Natychmiast przerywa wykonywanie programu i wyłącza maszynę
wirtualną.
