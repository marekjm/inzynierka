\subsection{Arytmetyka (Viua VM)}
\label{viua_vm_ops_arithmetic_vm}

Arytmetyka implementowana przez Viua VM, niezależna od fizycznej platformy, na
której maszyna wirtualna jest uruchomiona. Arytmetyka o zachowaniu zależnym od
fizycznej platformy jest opisana w rozdziale \ref{viua_vm_ops_arithmetic_cpu} na
stronie \pageref{viua_vm_ops_arithmetic_cpu}. Arytmetyka o zachowaniu
niezależnym od fizycznej platformy jest mniej wydajna (,,wolniesza''), ale
zapewnia przewidywalność, stabliność, i weryfikację wyników operacji (wykrywanie
błędów).

Operandami w instrukcjach arytmetycznych implementowanych przez VM są ciągi
bitów, których konstruktory są opisane w rozdziale \ref{viua_vm_ops_bit_ops} na
stronie \pageref{viua_vm_ops_bit_ops}.

\subsubsection{\texttt{wrapincrement}}

\begin{lstlisting}
wrapincrement Rt
\end{lstlisting}

\paragraph*{Opis} Inkrementacja modulo (ze znakiem).

Instrukcja interpretuje ciąg bitów jako liczbę całkowitą w kodowaniu z
dopełnieniem do dwóch\footnote{Kodowanie z dopełnieniem do dwóch
(ang. \emph{two's complement})} o szerokości równej długości ciągu bitów, na
którym instrukcja operuje.

\paragraph*{Uwagi}

W przypadku wystąpienia przepełnienia największa wartość dodatnia staje się
największą (tj. najbardziej oddaloną od zera) wartością ujemną.
\begin{lstlisting}
bits %1 local 0b01111111
print %1 local
wrapincrement %1 local
print %1 local
\end{lstlisting}
Ma to zastosowanie jedynie jeśli ciągi bitów są interpretowane przy konwersji
jako liczby ze znakiem.

\subsubsection{\texttt{wrapdecrement}}

\begin{lstlisting}
wrapdecrement Rt
\end{lstlisting}

\paragraph*{Opis} Dekrementacja modulo (ze znakiem).

Instrukcja interpretuje ciąg bitów jako liczbę całkowitą w kodowaniu z
dopełnieniem do dwóch o szerokości równej długości ciągu bitów, na którym
instrukcja operuje.

\paragraph*{Uwagi}

W przypadku wystąpienia przepełnienia największa (tj. najbardziej oddalona od
zera) wartość ujemna staje się największą wartością dodatnią.
\begin{lstlisting}
bits %1 local 0x80
print %1 local
wrapdecrement %1 local
print %1 local
\end{lstlisting}
Ma to zastosowanie jedynie jeśli ciągi bitów są interpretowane przy konwersji
jako liczby ze znakiem.

\subsubsection{\texttt{wrapadd}}

\begin{lstlisting}
wrapadd Rd Ra Rb
\end{lstlisting}

\paragraph*{Opis} Dodawanie modulo.

Instrukcja interpretuje liczby w rejestrach \texttt{Ra} i \texttt{Rb} jaki
liczby całkowite ze znakiem w kodowaniu z dopełnieniem do dwóch. Wynikiem
instrukcji jest ciąg bitów reprezentujący liczbe całkowitą ze znakiem w
kodowaniu z dopełnieniem do dwóch, o długości odpowiadającej długości ciągu
bitów znajdującemu się w rejestrze \texttt{Ra}.

Operacja jest wykonywana według wzoru $a + b$.

\paragraph*{Uwagi}

W przypadku przepełnienia wartość jest ,,zawijana'' i wynik z dodatniego może
stać się ujemy i \emph{vice versa}.

\subsubsection{\texttt{wrapsub}}

\begin{lstlisting}
wrapsub Rd Ra Rb
\end{lstlisting}

\paragraph*{Opis} Odejmowanie modulo.

Instrukcja interpretuje liczby w rejestrach \texttt{Ra} i \texttt{Rb} jaki
liczby całkowite ze znakiem w kodowaniu z dopełnieniem do dwóch. Wynikiem
instrukcji jest ciąg bitów reprezentujący liczbe całkowitą ze znakiem w
kodowaniu z dopełnieniem do dwóch, o długości odpowiadającej długości ciągu
bitów znajdującemu się w rejestrze \texttt{Ra}.

Operacja jest wykonywana według wzoru $a - b$.

\paragraph*{Uwagi}

W przypadku przepełnienia wartość jest ,,zawijana'' i wynik z dodatniego może
stać się ujemy i \emph{vice versa}.

\subsubsection{\texttt{wrapmul}}

\begin{lstlisting}
wrapmul Rd Ra Rb
\end{lstlisting}

\paragraph*{Opis} Mnożenie modulo.

Instrukcja interpretuje liczby w rejestrach \texttt{Ra} i \texttt{Rb} jaki
liczby całkowite ze znakiem w kodowaniu z dopełnieniem do dwóch. Wynikiem
instrukcji jest ciąg bitów reprezentujący liczbe całkowitą ze znakiem w
kodowaniu z dopełnieniem do dwóch, o długości odpowiadającej długości ciągu
bitów znajdującemu się w rejestrze \texttt{Ra}.

Operacja jest wykonywana według wzoru $a * b$.

\paragraph*{Uwagi}

W przypadku przepełnienia wartość jest ,,zawijana'' i wynik z dodatniego może
stać się ujemy i \emph{vice versa}.

\subsubsection{\texttt{wrapdiv}}

\begin{lstlisting}
wrapdiv Rd Ra Rb
\end{lstlisting}

\paragraph*{Opis} Dzielenie modulo.

Instrukcja interpretuje liczby w rejestrach \texttt{Ra} i \texttt{Rb} jaki
liczby całkowite ze znakiem w kodowaniu z dopełnieniem do dwóch. Wynikiem
instrukcji jest ciąg bitów reprezentujący liczbe całkowitą ze znakiem w
kodowaniu z dopełnieniem do dwóch, o długości odpowiadającej długości ciągu
bitów znajdującemu się w rejestrze \texttt{Ra}.

Operacja jest wykonywana według wzoru $a / b$.

\paragraph*{Uwagi}

W przypadku przepełnienia wartość jest ,,zawijana'' i wynik z dodatniego może
stać się ujemy i \emph{vice versa}.

\subsubsection{\texttt{checkedsincrement}}

\begin{lstlisting}
checkedsincrement Rt
\end{lstlisting}

\paragraph*{Opis} Inkrementacja weryfikowana ze znakiem.

Instrukcja interpretuje ciąg bitów jako liczbę całkowitą w kodowaniu z
dopełnieniem do dwóch o szerokości równej długości ciągu bitów, na którym
instrukcja operuje.

\paragraph*{Uwagi}

W przypadku przepełnienia zgłaszany jest wyjątek.

\subsubsection{\texttt{checkedsdecrement}}

\begin{lstlisting}
checkedsdecrement Rt
\end{lstlisting}

\paragraph*{Opis} Inkrementacja weryfikowana ze znakiem.

Instrukcja interpretuje ciąg bitów jako liczbę całkowitą w kodowaniu z
dopełnieniem do dwóch o szerokości równej długości ciągu bitów, na którym
instrukcja operuje.

\paragraph*{Uwagi}

W przypadku przepełnienia zgłaszany jest wyjątek.

\subsubsection{\texttt{checkedsadd}}

\begin{lstlisting}
checkedsadd Rd Ra Rb
\end{lstlisting}

\paragraph*{Opis} Dodawanie weryfikowane ze znakiem.

Instrukcja interpretuje liczby w rejestrach \texttt{Ra} i \texttt{Rb} jaki
liczby całkowite ze znakiem w kodowaniu z dopełnieniem do dwóch. Wynikiem
instrukcji jest ciąg bitów reprezentujący liczbe całkowitą ze znakiem w
kodowaniu z dopełnieniem do dwóch, o długości odpowiadającej długości ciągu
bitów znajdującemu się w rejestrze \texttt{Ra}.

Operacja jest wykonywana według wzoru $a + b$.

\paragraph*{Uwagi}

W przypadku przepełnienia zgłaszany jest wyjątek.

\subsubsection{\texttt{checkedssub}}

\begin{lstlisting}
checkedssub Rd Ra Rb
\end{lstlisting}

\paragraph*{Opis} Odejmowanie weryfikowane ze znakiem.

Instrukcja interpretuje liczby w rejestrach \texttt{Ra} i \texttt{Rb} jaki
liczby całkowite ze znakiem w kodowaniu z dopełnieniem do dwóch. Wynikiem
instrukcji jest ciąg bitów reprezentujący liczbe całkowitą ze znakiem w
kodowaniu z dopełnieniem do dwóch, o długości odpowiadającej długości ciągu
bitów znajdującemu się w rejestrze \texttt{Ra}.

Operacja jest wykonywana według wzoru $a - b$.

\paragraph*{Uwagi}

W przypadku przepełnienia zgłaszany jest wyjątek.

\subsubsection{\texttt{checkedsmul}}

\begin{lstlisting}
checkedsmul Rd Ra Rb
\end{lstlisting}

\paragraph*{Opis} Dzielenie weryfikowane ze znakiem.

Instrukcja interpretuje liczby w rejestrach \texttt{Ra} i \texttt{Rb} jaki
liczby całkowite ze znakiem w kodowaniu z dopełnieniem do dwóch. Wynikiem
instrukcji jest ciąg bitów reprezentujący liczbe całkowitą ze znakiem w
kodowaniu z dopełnieniem do dwóch, o długości odpowiadającej długości ciągu
bitów znajdującemu się w rejestrze \texttt{Ra}.

Operacja jest wykonywana według wzoru $a * b$.

\paragraph*{Uwagi}

W przypadku przepełnienia zgłaszany jest wyjątek.

\subsubsection{\texttt{checkedsdiv}}

\begin{lstlisting}
checkedsdiv Rd Ra Rb
\end{lstlisting}

\paragraph*{Opis} Dzielenie weryfikowane ze znakiem.

Instrukcja interpretuje liczby w rejestrach \texttt{Ra} i \texttt{Rb} jaki
liczby całkowite ze znakiem w kodowaniu z dopełnieniem do dwóch. Wynikiem
instrukcji jest ciąg bitów reprezentujący liczbe całkowitą ze znakiem w
kodowaniu z dopełnieniem do dwóch, o długości odpowiadającej długości ciągu
bitów znajdującemu się w rejestrze \texttt{Ra}.

Operacja jest wykonywana według wzoru $a / b$.

\paragraph*{Uwagi}

W przypadku przepełnienia zgłaszany jest wyjątek.

\subsubsection{\texttt{saturatingsincrement}}

Nasycająca inkrementacja ze znakiem.

\subsubsection{\texttt{saturatingsdecrement}}
\subsubsection{\texttt{saturatingsadd}}
\subsubsection{\texttt{saturatingssub}}
\subsubsection{\texttt{saturatingsmul}}
\subsubsection{\texttt{saturatingsdiv}}
