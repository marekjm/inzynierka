\subsection{Programowanie funkcyjne}
\label{viua_vm_ops_functional}

\subsubsection{\texttt{function}}

\begin{lstlisting}
function Rd %*\emph{function-name}*)
\end{lstlisting}

\paragraph*{Opis} Konstruuje w rejestrze \texttt{Rd} wartość reprezentującą
funkcję określoną przez \texttt{\emph{function-name}}.

Wartość ta może być wykorzystana w instrukcjach \texttt{call}, \texttt{tailcall}
i \texttt{process} aby wywołać funkcję niebezpośrednio. Przykład:
\begin{lstlisting}
function %1 local a_function/0
frame %0
call void %1 local
\end{lstlisting}

\subsubsection{\texttt{closure}}

\begin{lstlisting}
closure Rd %*\emph{closure-name}*)
\end{lstlisting}

\paragraph*{Opis} Konstruuje w rejestrze \texttt{Rd} wartość będącą instancją
domknięcia (ang. \emph{closure}) określonego przez \texttt{\emph{closure-name}}.

Zanim tak utworzona wartość będzie mogła być wywołana wszystkie wartości, do
których domknięcie \emph{closure-name} się odwołuje muszą zostać ,,schwytane''
wewnątrz utworzonego domknięcia za pomocą instrukcji \texttt{capture},
\texttt{capturecopy} lub \texttt{capturemove}.

\subsubsection{\texttt{capture}}

\begin{lstlisting}
capture Rc Ri Rv
\end{lstlisting}

\paragraph*{Opis} W rejestrze lokalnym o indeksie \texttt{Ri} wewnętrznym do
domknięcia znajdującego się w rejestrze \texttt{Rc} tworzy referencję do
wartości znajdującej się w rejestrze \texttt{Rv}.
\begin{lstlisting}
text %3 local "Hello World!"
capture %1 local %1 %3 local
\end{lstlisting}

Przykład powyżej ,,chwyta'' wartość tekstową znajdującą się w rejestrze
\texttt{\%3~local} i referencję do niej umieszcza w lokalnym rejestrze o
indeksie 1 wewnętrznym do domknięcia znajdującego się w rejestrze
\texttt{\%1~local}.

\paragraph*{Uwagi -- czas życia} Schwytanie wartości przez referencję powoduje,
że przestaje ona być śledzona przez standardowy deterministyczny mechanizm
zarządzania wartościami i jej czas życia może być trudny do określenia jeśli
domknięcie jest zwracane z funkcji lub przekazywane dalej jak parametr.

\paragraph*{Uwagi -- przekazywanie wiadomości} Wartości schwytane przez
referencję nie mogą być przekazywane jako wiadomości ponieważ naruszyło by to
ścisłą izolację procesów.

\subsubsection{\texttt{capturecopy}}

\begin{lstlisting}
capturecopy Rc Ri Rv
\end{lstlisting}

\paragraph*{Opis} W rejestrze lokalnym o indeksie \texttt{Ri} wewnętrznym do
domknięcia znajdującego się w rejestrze \texttt{Rc} tworzy kopię wartości
znajdującej się w rejestrze \texttt{Rv}.
\begin{lstlisting}
text %3 local "Hello World!"
capturecopy %1 local %1 %3 local
\end{lstlisting}

Przykład powyżej ,,chwyta'' wartość tekstową znajdującą się w rejestrze
\texttt{\%3~local} i jej kopię umieszcza w lokalnym rejestrze o indeksie 1
wewnętrznym do domknięcia znajdującego się w rejestrze \texttt{\%1~local}.

Odpowiada to instrukcji \texttt{copy}, dla której wynikowym zestawem rejestrów
jest lokalny zestaw rejestrów wewnętrzny dla domknięcia.

\subsubsection{\texttt{capturemove}}

\begin{lstlisting}
capturemove Rc Ri Rv
\end{lstlisting}

\paragraph*{Opis} Przenosi wartość z rejestru \texttt{Rv} do rejestru lokalnego
o indeksie \texttt{Ri} wewnętrznego do domknięcia znajdującego się w rejestrze
\texttt{Rc}.
\begin{lstlisting}
text %3 local "Hello World!"
capturemove %1 local %1 %3 local
\end{lstlisting}

Przykład powyżej ,,chwyta'' wartość tekstową znajdującą się w rejestrze
\texttt{\%3~local} i przenosi ją do lokalnego rejestru o indeksie 1
wewnętrznego do domknięcia znajdującego się w rejestrze \texttt{\%1~local}.

Odpowiada to instrukcji \texttt{move}, dla której wynikowym zestawem rejestrów
jest lokalny zestaw rejestrów wewnętrzny dla domknięcia.
