\subsection{Arytmetyka (CPU)}
\label{viua_vm_ops_arithmetic_cpu}

Arytmetyka implementowana przez fizyczne CPU, na którym działa Viua VM.
Specyfika tego fizycznego CPU wpływa na wyniki działania instrukcji z tej grupy.
Arytmetyka o zdefiniowanym zachowaniu niezależnym od fizycznej platformy jest
opisana w rozdziale \ref{viua_vm_ops_arithmetic_vm} na stronie
\pageref{viua_vm_ops_arithmetic_vm}. Arytmetyka oparta o CPU jest
bardziej wydajna (,,szybsza''), ale nie zawsze zapewnia przewidywalność,
stabliność, i weryfikację wyników operacji (wykrywanie błędów).

\subsubsection{\texttt{izero}}

\begin{lstlisting}
izero Rd
\end{lstlisting}

\paragraph*{Opis} Konstruuje w rejestrze \texttt{Rd} liczbę całkowitą o wartości 0.
Zwyczajowo wykorzystywana na końcu funkcji \texttt{main} do utworzenia domyślnej wartości zwracanej:

\begin{lstlisting}
	izero %0 local
	return
.end
\end{lstlisting}

\subsubsection{\texttt{integer}}

\begin{lstlisting}
integer Rd %*\emph{<integer>}*)
\end{lstlisting}

\paragraph*{Opis} Konstruuje w rejestrze \texttt{Rd} liczbę całkowitą o wartości \emph{\texttt{integer}}.

\subsubsection{\texttt{iinc}}

\begin{lstlisting}
iinc Rt
\end{lstlisting}

\paragraph*{Opis} Zwiększa wartość liczby całkowitej w rejestrze \texttt{Rt} o 1.
\paragraph*{Uwagi} Zachowanie w przypadku przepełnienia jest zdefiniowane przez
zachowanie platformy sprzętowej.

\subsubsection{\texttt{idec}}

\begin{lstlisting}
idec Rt
\end{lstlisting}

\paragraph*{Opis} Zmniejsza wartość liczby całkowitej w rejestrze \texttt{Rt} o 1.
\paragraph*{Uwagi} Zachowanie w przypadku przepełnienia jest zdefiniowane przez
zachowanie platformy sprzętowej.

\subsubsection{\texttt{float}}

\begin{lstlisting}
float Rd %*\emph{<float>}*)
\end{lstlisting}

\paragraph*{Opis} Konstruuje w rejestrze \texttt{Rd} liczbę zmiennoprzecinkową o wartości
\emph{\texttt{float}}.

\subsubsection{\texttt{itof}}

\begin{lstlisting}
itof Rd Rs
\end{lstlisting}

\paragraph*{Opis} Konwertuje wartość całkowitoliczbową z rejestru \texttt{Rs} na
wartość zmiennoprzecinkową i umieszcza wynik w rejestrze \texttt{Rd}.

\subsubsection{\texttt{ftoi}}

\begin{lstlisting}
ftoi Rd Rs
\end{lstlisting}

\paragraph*{Opis} Konwertuje wartość zmiennoprzecinkową z rejestru \texttt{Rs}
na wartość całkowitoliczbową i umieszcza wynik w rejestrze \texttt{Rd}.

\subsubsection{\texttt{stoi}}

\begin{lstlisting}
stoi Rd Rs
\end{lstlisting}

\paragraph*{Opis} Konwertuje string z rejestru \texttt{Rs} na wartość
całkowitoliczbową i umieszcza wynik w rejestrze \texttt{Rd}.

\subsubsection{\texttt{stof}}

\begin{lstlisting}
stof Rd Rs
\end{lstlisting}

\paragraph*{Opis} Konwertuje string z rejestru \texttt{Rs} na wartość
zmiennoprzecinkową i umieszcza wynik w rejestrze \texttt{Rd}.

\subsubsection{\texttt{add}}

\begin{lstlisting}
add Rd Ra Rb
\end{lstlisting}

\paragraph*{Opis} Dodaje dwie wartości liczbowe (całkowitoliczbową lub
zmiennoprzecinkową) znajdujące się w rejestrach \texttt{Ra} i \texttt{Rb} do
siebie, a wynik umieszcza w rejestrze \texttt{Rd}.

Wartość umieszczona w rejestrze \texttt{Rd} ma taki sam typ jak wartość w
rejestrze \texttt{Ra}. Wartości w rejestrach \texttt{Ra} i \texttt{Rb} pozostają
nienaruszone.

\subsubsection{\texttt{sub}}

\begin{lstlisting}
sub Rd Ra Rb
\end{lstlisting}

\paragraph*{Opis} Odejmuje dwie wartości liczbowe (całkowitoliczbową lub
zmiennoprzecinkową) znajdujące się w rejestrach \texttt{Ra} i \texttt{Rb} do
siebie, a wynik umieszcza w rejestrze \texttt{Rd}.

Wartość umieszczona w rejestrze \texttt{Rd} ma taki sam typ jak wartość w
rejestrze \texttt{Ra}. Wartości w rejestrach \texttt{Ra} i \texttt{Rb} pozostają
nienaruszone.

\subsubsection{\texttt{mul}}

\begin{lstlisting}
mul Rd Ra Rb
\end{lstlisting}

\paragraph*{Opis} Mnoży dwie wartości liczbowe (całkowitoliczbową lub
zmiennoprzecinkową) znajdujące się w rejestrach \texttt{Ra} i \texttt{Rb}, a
wynik umieszcza w rejestrze \texttt{Rd}. Działanie jest przeprowadzane zgodnie
ze wzorem $a * b$.

Wartość umieszczona w rejestrze \texttt{Rd} ma taki sam typ jak wartość w
rejestrze \texttt{Ra}. Wartości w rejestrach \texttt{Ra} i \texttt{Rb} pozostają
nienaruszone.

\subsubsection{\texttt{div}}

\begin{lstlisting}
add Rd Ra Rb
\end{lstlisting}

\paragraph*{Opis} Dzieli dwie wartości liczbowe (całkowitoliczbową lub
zmiennoprzecinkową) znajdujące się w rejestrach \texttt{Ra} i \texttt{Rb}, a
wynik umieszcza w rejestrze \texttt{Rd}. Działanie jest przeprowadzane zgodnie
ze wzorem $a / b$.

Wartość umieszczona w rejestrze \texttt{Rd} ma taki sam typ jak wartość w
rejestrze \texttt{Ra}. Wartości w rejestrach \texttt{Ra} i \texttt{Rb} pozostają
nienaruszone.

\subsubsection{\texttt{lt}}

\begin{lstlisting}
lt Rd Ra Rb
\end{lstlisting}

\paragraph*{Opis} Porównuje dwie wartości liczbowe (całkowitoliczbową lub
zmiennoprzecinkową) znajdujące się w rejestrach \texttt{Ra} i \texttt{Rb}, a
wynik umieszcza w rejestrze \texttt{Rd}. Działanie jest przeprowadzane zgodnie
ze wzorem $a < b$.

Wartość wynika jest typu boolowskiego.

\subsubsection{\texttt{lte}}

\begin{lstlisting}
lte Rd Ra Rb
\end{lstlisting}

\paragraph*{Opis} Porównuje dwie wartości liczbowe (całkowitoliczbową lub
zmiennoprzecinkową) znajdujące się w rejestrach \texttt{Ra} i \texttt{Rb}, a
wynik umieszcza w rejestrze \texttt{Rd}. Działanie jest przeprowadzane zgodnie
ze wzorem $a <= b$.

Wartość wynika jest typu boolowskiego.

\subsubsection{\texttt{gt}}

\begin{lstlisting}
gt Rd Ra Rb
\end{lstlisting}

\paragraph*{Opis} Porównuje dwie wartości liczbowe (całkowitoliczbową lub
zmiennoprzecinkową) znajdujące się w rejestrach \texttt{Ra} i \texttt{Rb}, a
wynik umieszcza w rejestrze \texttt{Rd}. Działanie jest przeprowadzane zgodnie
ze wzorem $a > b$.

Wartość wynika jest typu boolowskiego.

\subsubsection{\texttt{gte}}

\begin{lstlisting}
gte Rd Ra Rb
\end{lstlisting}

\paragraph*{Opis} Porównuje dwie wartości liczbowe (całkowitoliczbową lub
zmiennoprzecinkową) znajdujące się w rejestrach \texttt{Ra} i \texttt{Rb}, a
wynik umieszcza w rejestrze \texttt{Rd}. Działanie jest przeprowadzane zgodnie
ze wzorem $a >= b$.

Wartość wynika jest typu boolowskiego.

\subsubsection{\texttt{eq}}

\begin{lstlisting}
eq Rd Ra Rb
\end{lstlisting}

\paragraph*{Opis} Porównuje dwie wartości liczbowe (całkowitoliczbową lub
zmiennoprzecinkową) znajdujące się w rejestrach \texttt{Ra} i \texttt{Rb}, a
wynik umieszcza w rejestrze \texttt{Rd}. Działanie jest przeprowadzane zgodnie
ze wzorem $a = b$.

Wartość wynika jest typu boolowskiego.
