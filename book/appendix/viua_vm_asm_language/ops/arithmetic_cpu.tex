\subsection{Arytmetyka (CPU)}
\label{viua_vm_ops_arithmetic_cpu}

Arytmetyka implementowana przez fizyczne CPU, na którym działa Viua VM.
Specyfika tego fizycznego CPU wpływa na wyniki działania instrukcji z tej grupy.
Arytmetyka o zdefiniowanym zachowaniu niezależnym od fizycznej platformy jest
opisana w rozdziale \ref{viua_vm_ops_arithmetic_vm} na stronie
\pageref{viua_vm_ops_arithmetic_vm}. Arytmetyka oparta o CPU jest
bardziej wydajna (,,szybsza''), ale nie zawsze zapewnia przewidywalność,
stabliność, i weryfikację wyników operacji (wykrywanie błędów).

\subsubsection{\texttt{izero}}

\begin{lstlisting}
izero Rd
\end{lstlisting}

\paragraph*{Opis} Konstruuje w rejestrze \texttt{Rd} liczbę całkowitą o wartości 0.
Zwyczajowo wykorzystywana na końcu funkcji \texttt{main} do utworzenia domyślnej wartości zwracanej:

\begin{lstlisting}
	izero %0 local
	return
.end
\end{lstlisting}

\subsubsection{\texttt{integer}}

\begin{lstlisting}
integer Rd %*\emph{<integer>}*)
\end{lstlisting}

\paragraph*{Opis} Konstruuje w rejestrze \texttt{Rd} liczbę całkowitą o wartości \emph{\texttt{integer}}.

\subsubsection{\texttt{iinc}}
\subsubsection{\texttt{idec}}

\subsubsection{\texttt{float}}

\begin{lstlisting}
float Rd %*\emph{<float>}*)
\end{lstlisting}

\paragraph*{Opis} Konstruuje w rejestrze \texttt{Rd} liczbę zmiennoprzecinkową o wartości
\emph{\texttt{float}}.

\subsubsection{\texttt{itof}}
\subsubsection{\texttt{ftoi}}
\subsubsection{\texttt{stoi}}
\subsubsection{\texttt{stof}}

\subsubsection{\texttt{add}}
\subsubsection{\texttt{sub}}
\subsubsection{\texttt{mul}}
\subsubsection{\texttt{div}}
\subsubsection{\texttt{lt}}
\subsubsection{\texttt{lte}}
\subsubsection{\texttt{gt}}
\subsubsection{\texttt{gte}}
\subsubsection{\texttt{eq}}
