\subsection{Aktory}
\label{viua_vm_ops_actor}

\subsubsection{\texttt{process}}

\begin{lstlisting}
process Rd fn/0
process void fn/0
\end{lstlisting}

\paragraph*{Opis}

Wywołuje funkcję tworząc nowy proces.
Wariant 1 zapisuje PID nowego procesu w rejestrze \texttt{Rd}.
Wariant drugi nie zapisuje PID nowego procesu.
\begin{lstlisting}
process %4 local do_some_processing/1
process void a_free_actor/4
\end{lstlisting}

\subsubsection{\texttt{self}}

\begin{lstlisting}
self Rd
\end{lstlisting}

\paragraph*{Opis} Konstruuje PID procesu wewnątrz którego instrukcja jest
wykonywana.

\subsubsection{\texttt{pideq}}

\begin{lstlisting}
pideq Rd Ra Rb
\end{lstlisting}

\paragraph*{Opis} Porównuje wartości PID znajdujące się w rejestrach \texttt{Ra}
i \texttt{Rb}. Wynik porównania zapisuje jako wartość boolowską w rejestrze
\texttt{Rd}.

\subsubsection{\texttt{join}}

\begin{lstlisting}
join Rd Rp
join Rd Rp infinity
join Rd Rp %*\emph{<timeout>}*)
\end{lstlisting}

\paragraph*{Opis} Zatrzymuje wykonanie procesu wewnątrz którego wykonywana jest
instrukcja i oczekuje na zakończenie procesu identyfikowanego przez PID
znajdujący się w rejestrze \texttt{Rp}.

Wariant pierwszy jest identyczny w zachowaniu jak wariant drugi i powoduje, że
proces wykonujący instrukcję oczekuje w nieskończoność.

Wariant trzeci umożliwia podanie wartości \emph{timeout}, która określa
maksymalny czas oczekiwania na zakończenie pracy przez zadany proces. Czas może
być podany w sekundach (\texttt{\emph{N}s}) lub
milisekundach~(\texttt{\emph{N}ms}), a wartość zero jest poprawna. Przykład:
\begin{lstlisting}
join %1 local %2 local 1s
join %1 local %2 local 0ms
\end{lstlisting}

Jeśli rejestr \texttt{Rd} jest podany jako \texttt{void} wartość zwracana przez
zadany proces jest natychmiast niszczona.

\paragraph*{Uwagi}

Jeśli do momentu upłynięcia czasu oczekiwania zadany proces nie zakończył pracy
zgłaszany jest wyjątek.

Jeśli zadany proces zakończył pracę z powodu awarii to nieobsłużony wyjątek,
który tę awarię wywołał jest ponownie zgłaszany wewnątrz procesu, który wykonał
instrukcję \texttt{join}.

\subsubsection{\texttt{send}}

\begin{lstlisting}
send Rp Rs
\end{lstlisting}

\paragraph*{Opis} Wysyła wartość znajdującą się w rejestrze \texttt{Rs} jako
wiadomość do procesu identyfikowanego przez PID znajdujący się w rejestrze
\texttt{Rp}.

\subsubsection{\texttt{receive}}

\begin{lstlisting}
receive Rd
receive Rd infinity
receive Rd %*\emph{<timeout>}*)
\end{lstlisting}

\paragraph*{Opis} Blokuje proces wewnątrz którego wykonywana jest instrukcja i
oczekuje na nadejście wiadomości. Jeśli jakaś wiadomość jest już dostępna w
kolejce wiadomości procesu jest ona zwracana od razu, bez oczekiwania. Odebrana
wiadomość jest umieszczana w rejestrze \texttt{Rd}.

Pierwszy i drugi wariant są identyczne w zachowaniu i blokują proces oczekując
w nieskończoność na nadejście wiadomości.

Wariant trzeci umożliwia podanie wartości \emph{timeout} określającej maksymalny
czas oczekiwania na nadejście wiadomości w sekundach lub milisekundach. Czas
oczekiwania może wynosić zero sekund lub milisekund.

\paragraph*{Uwagi}

Jeśli po upłynięciu czasu oczekiwania żadna wiadomość nie będzie dostępna
zgłaszany jest wyjątek.
