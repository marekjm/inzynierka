\subsection{Obsługa tekstu}
\label{viua_vm_ops_text}

\subsubsection{\texttt{string}}

\begin{lstlisting}
string Rd "%*\emph{<str>}*)"
\end{lstlisting}

\paragraph*{Opis} Konstruuje w rejestrze \texttt{Rd} ciąg bajtów o wartości \emph{\texttt{str}}.

\subsubsection{\texttt{streq}}

\begin{lstlisting}
streq Rd Ra Rb
\end{lstlisting}

\paragraph*{Opis} Weryfikuje czy ciągi bajtów w rejestarch \texttt{Ra} i
\texttt{Rb} są sobie równe, a wynik sprawdzenia przechowuje w rejestrze
\texttt{Rd} jako wartość typu boolowskiego
\begin{enumerate*}[label=(\arabic*)]
    \item \texttt{true} jeśli ciągi są sobie równe
    \item \texttt{false} jeśli ciągi nie są sobie równe
\end{enumerate*}.

\subsubsection{\texttt{text}}

\begin{lstlisting}
text Rd "%*\emph{<txt>}*)"
text Rd Rs
\end{lstlisting}

\paragraph*{Opis} Wariant pierwszy konstruuje w rejestrze \texttt{Rd} tekst o wartości \emph{\texttt{txt}}.
Wariant drugi konwertuje wartość z rejestru \texttt{Rs} na tekst i umieszcza go w rejestrze \texttt{Rd}.

Utworzona wartość tekstowa jest ciągiem \emph{codepoint}-ów Unicode w kodowaniu UTF-8.

\subsubsection{\texttt{texteq}}

\begin{lstlisting}
texteq Rd Ra Rb
\end{lstlisting}

\paragraph*{Opis} Weryfikuje czy ciągi bajtów w rejestarch \texttt{Ra} i
\texttt{Rb} są sobie równe, a wynik sprawdzenia przechowuje w rejestrze
\texttt{Rd} jako wartość typu boolowskiego
\begin{enumerate*}[label=(\arabic*)]
    \item \texttt{true} jeśli ciągi są sobie równe
    \item \texttt{false} jeśli ciągi nie są sobie równe
\end{enumerate*}.

\subsubsection{\texttt{textat}}

\begin{lstlisting}
textat Rd Rs Rn
\end{lstlisting}

\paragraph*{Opis} Z tekstu w rejestrze \texttt{Rs} pobiera \emph{codepoint}
Unicode z indeksu określonego przez liczbę całkowitą znajdującą się w rejestrze
\texttt{Rn}. Wartość w rejestrze \texttt{Rs} pozostaje nienaruszona.

\subsubsection{\texttt{textsub}}

\begin{lstlisting}
textsub Rd Rs Rbegin Rend
\end{lstlisting}

\paragraph*{Opis} Z tekstu w rejestrze \texttt{Rs} kopiuje tekst pomiędzy
indeksem określonym przez liczbę całkowitą znajdującą się w rejestrze
\texttt{Rbegin}, a indeksem określonym przez liczbę całkowitą znajdującą się w
rejestrze \texttt{Rend} bez uwzględnienia znaku pod indeksem określonym przez
\texttt{Rend} -- czyli zakres wycinka tekstu to $[begin, end)$.

\subsubsection{\texttt{textlength}}

\begin{lstlisting}
textlength Rd Rs
\end{lstlisting}

\paragraph*{Opis} Odczytuje długość tekstu znajdującego się w rejestrze
\texttt{Rs} jako liczbę całkowitą, a wynik umieszcza w rejestrze \texttt{Rd}.
Długość tekstu jest określona przez liczbę \emph{codepoint}-ów Unicode, nie
liczbę bajtów.

\subsubsection{\texttt{textcommonprefix}}

\begin{lstlisting}
textcommonprefix Rd Ra Rb
\end{lstlisting}

\paragraph*{Opis} Określna wspólny prefiks tekstów znajdujących się w rejestrach
\texttt{Ra} i \texttt{Rb}. Wynik umieszcza w rejestrze \texttt{Rd}. Wynik jest
podany jako liczba całkowita określająca długość wspólnego prefiksu w
\emph{codepoint}-ach Unicode.

\subsubsection{\texttt{textcommonsuffix}}

\begin{lstlisting}
textcommonsuffix Rd Ra Rb
\end{lstlisting}

\paragraph*{Opis} Określna wspólny sufiks tekstów znajdujących się w rejestrach
\texttt{Ra} i \texttt{Rb}. Wynik umieszcza w rejestrze \texttt{Rd}. Wynik jest
podany jako liczba całkowita określająca długość wspólnego sufiksu w
\emph{codepoint}-ach Unicode.

\subsubsection{\texttt{textconcat}}

\begin{lstlisting}
textconcat Rd Ra Rb
\end{lstlisting}

\paragraph*{Opis} Łączy teksty znajdujące się w rejestrach \texttt{Ra} i
\texttt{Rb}, a wynik zapisuje w rejestrze \texttt{Rd}.
