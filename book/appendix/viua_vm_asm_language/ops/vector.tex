\subsection{Wektory}
\label{viua_vm_ops_vector}

\subsubsection{\texttt{vector}}

\begin{lstlisting}
vector Rd
vector Rd Rp %*\emph{n}*)
\end{lstlisting}

\paragraph*{Opis} Konstruuje wektor w rejestrze \texttt{Rd}.

Pierwszy wariant konstruuje pusty wektor.
Drugi wariant konstruuje wektor ,,pakując'' \emph{n} wartości rozpoczynając od
rejestru określonego przez \texttt{Rp} (wszystkie ,,spakowane'' wartości
pochodzą z tego samego zestawu rejestrów, określonego przez zestaw rejestrów
użyty w adresie rejestru \texttt{Rp}).

Przykład pakowania:
\begin{lstlisting}
integer %2 local 1
integer %3 local 2
integer %4 local 3
integer %5 local 4
vector %1 local %2 local %4
\end{lstlisting}

Wektor znajdujący się w rejestrze \texttt{\%1 local} spakuje wartości z
rejestrów lokalnych 2, 3, 4 i 5 (czyli będzie miał długość 4).

\paragraph*{Uwagi} Wektor nie może spakować:
\begin{enumerate}
    \item rejestru, w którym będzie skonstruowany
    \item rejestru, który nie zawiera żadnej wartości
    \item rejestru, którego indeks jest poza zakresem dla zestawu rejestrów, do
        którego odnosi się adres rejestru \texttt{Rp}
\end{enumerate}

\subsubsection{\texttt{vinsert}}

\begin{lstlisting}
vinsert Rd Rs void
vinsert Rd Rs Ri
\end{lstlisting}

\paragraph*{Opis} Przenosi wartość z rejestru \texttt{Rs} do wektora
znajdującego się w rejestrze \texttt{Rd}.

Pierwszy wariant przenosi wartość na początek wektora (na pozycję 0).
Drugi wariant przenosi wartość na pozycję określoną przez liczbę całkowitą
znajdującą się w rejestrze \texttt{Ri}.

\paragraph*{Wyjątki} Jeśli żądany indeks jest poza zakresem długości wektora
wygenerowany zostanie wyjątek.

\subsubsection{\texttt{vpush}}

\begin{lstlisting}
vpush Rd Rs
\end{lstlisting}

\paragraph*{Opis} Przenosi wartość z rejestru \texttt{Rs} na koniec wektora
znajdującego się w rejestrze \texttt{Rd}.

\subsubsection{\texttt{vpop}}

\begin{lstlisting}
vpop Rd Rs Ri
\end{lstlisting}

\paragraph*{Opis} Usuwa wartość znajdującą się pod indeksem określonym przez
liczbę całkowitą znajdującą się w rejestrze \texttt{Ri} w wektorze znajdującym
się w rejestrze \texttt{Rs} i zapisuje ją w rejestrze \texttt{Rd}.

Jeśli rejestr \texttt{Rd} jest podany jako \texttt{void} to wartość usunięta z
wektora nie jest zapisywana, a od razu niszczona.

Jeśli rejestr \texttt{Ri} jest podany jako \texttt{void} to wartość jest usuwana
z końca wektora.

Rejestr \texttt{Rd} i \texttt{Ri} mogą być podane jako \texttt{void}
jednocześnie (co spowoduje usunięcie ostatniej wartości w wektorze i jej
nastychmiastowe zniszczenie).

\paragraph*{Wyjątki} Jeśli żądany indeks jest poza zakresem długości wektora
wygenerowany zostanie wyjątek.

\subsubsection{\texttt{vat}}

\begin{lstlisting}
vat Rd Rs Ri
\end{lstlisting}

\paragraph*{Opis} Tworzy wskaźnik do wartości znajdującej się pod indeksem
określonym przez liczbę całkowitą znajdującą się w rejestrze \texttt{Ri} w
wektorze znajdującym się w rejestrze \texttt{Rs} i zapisuje go w rejestrze
\texttt{Rd}.

\paragraph*{Wyjątki} Jeśli żądany indeks jest poza zakresem długości wektora
wygenerowany zostanie wyjątek.

\subsubsection{\texttt{vlen}}

\begin{lstlisting}
vlen Rd Rs
\end{lstlisting}

\paragraph*{Opis} Odczytuje długość wektora znajdującego się w rejestrze
\texttt{Rd} i zapisuje ją jako liczbę całkowitą w rejestrze \texttt{Rd}.
