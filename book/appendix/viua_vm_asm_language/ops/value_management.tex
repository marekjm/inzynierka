\subsection{Zarządzanie wartościami}
\label{viua_vm_ops_value_management}

\subsubsection{\texttt{move}}

\begin{lstlisting}
move Rd Rs
\end{lstlisting}

\paragraph*{Opis} Przesuwa wartości między rejestrami. Jeśli w rejestrze
\texttt{\emph{Rd}} przed wykonaniem instrukcji znajduje się jakaś wartość to
będzie ona zniszczona. Po wykonaniu instrukcji wartość, która znajdowała się w
rejestrze \texttt{\emph{Rs}} znajduje się w rejestrze \texttt{\emph{Rd}}, a
rejestr \texttt{\emph{Rs}} jest pusty.

\begin{lstlisting}
move %0 local %4 local
move %0 parameters %1 local
move %1 local %1 static
\end{lstlisting}

Instrukcja z wariantu \emph{1} przesuwa wartość z lokalnego rejestru 4 do
lokalnego rejstru 0.
Instrukcja z wariantu \emph{2} przesuwa wartość z lokalnego rejestru 1 do
rejestru parametrów 0 (znajdującego się w ramce wywołania).
Instrukcja z wariantu \emph{3} przesuwa wartość ze statycznego rejestru 1 do
lokalnego rejestru 1.

\subsubsection{\texttt{copy}}

\begin{lstlisting}
copy Rd Rs
\end{lstlisting}

\paragraph*{Opis} Kopiuje wartość z rejestru \texttt{\emph{Rs}} do rejestru
\texttt{\emph{Rd}}. Jeśli w rejestrze \texttt{\emph{Rd}} przed wykonaniem
instrukcji znajduje się jakaś wartość to będzie ona zniszczona. Po wykonaniu
instrukcji wartość w rejestrze \texttt{\emph{Rs}} pozostaje nienaruszona.

\begin{lstlisting}
copy %0 local %4 local
copy %0 parameters %1 local
copy %1 local %1 static
\end{lstlisting}

Instrukcja z wariantu \emph{1} kopiuje wartość z lokalnego rejestru 4 do
lokalnego rejstru 0.
Instrukcja z wariantu \emph{2} kopiuje wartość z lokalnego rejestru 1 do
rejestru parametrów 0 (znajdującego się w ramce wywołania).
Instrukcja z wariantu \emph{3} kopiuje wartość ze statycznego rejestru 1 do
lokalnego rejestru 1.

\subsubsection{\texttt{ptr}}

\begin{lstlisting}
ptr Rd Rs
\end{lstlisting}

\paragraph*{Opis} Konstruuje w rejestrze \texttt{\emph{Rd}} wskaźnik do wartości
znajdującej się w rejestrze \texttt{\emph{Rs}}.

\begin{lstlisting}
ptr %1 local %2 local
\end{lstlisting}

\subsubsection{\texttt{ptrlive}}

\begin{lstlisting}
ptrlive Rd Rs
\end{lstlisting}

\paragraph*{Opis} Sprawdza poprawność (ważność) wskaźnika w rejestrze
\texttt{\emph{Rs}}. W rejestrze \texttt{\emph{Rd}} umieszcza wartość logiczną
\texttt{true} jeśli wskaźnik wskazuje na ,,żywą'' (niezniszczoną) wartość, lub
\texttt{false} jeśli wskaźnik wskazuje na ,,martwą'' (zniszczoną) wartość.

\begin{lstlisting}
ptrlive %1 local %2 local
\end{lstlisting}

\subsubsection{\texttt{swap}}

\begin{lstlisting}
swap Ra Rb
\end{lstlisting}

\paragraph*{Opis} Zamienia miejscami wartości w rejestrach \texttt{\emph{Ra}} i
\texttt{\emph{Rb}}.

\begin{lstlisting}
swap %1 local %2 local
\end{lstlisting}

\subsubsection{\texttt{delete}}

\begin{lstlisting}
delete Rt
\end{lstlisting}

\paragraph*{Opis} Niszczy wartość w rejestrze \texttt{\emph{Rt}}.

\begin{lstlisting}
	izero %0 local
	return
.end
\end{lstlisting}

Wywołanie destruktora wartości.

\subsubsection{\texttt{isnull}}

\begin{lstlisting}
izero Rd
\end{lstlisting}

\paragraph*{Opis} Konstruuje w rejestrze \texttt{Rd} liczbę całkowitą o wartości 0.
Zwyczajowo wykorzystywana na końcu funkcji \texttt{main} do utworzenia domyślnej wartości zwracanej:

\begin{lstlisting}
	izero %0 local
	return
.end
\end{lstlisting}
