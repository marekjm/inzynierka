\chapter{Język assemblera Viua VM}
\label{appendix_viua_vm_assembly_language}

W tym rozdziale zostanie po krótce omówiony język assemblera Viua VM.

\section{Składnia języka assemblera}

Składnia języka assemblera Viua VM jest prosta. Wyglądem przypomina składnię innych języków tego rodzaju
(np. języki assemblera x86 czy ARM).

\subsection{Ogólna składnia instrukcji}

Instrukcje składają się z mnemoniki, zera lub więcej adresów rejestrów, i co najwyżej jednego literału. Ogólną składnię można zapisać więc tak:

\begin{lstlisting}
mnemonic [<register>...] [<literal>]
\end{lstlisting}

Zanim pokazanych zostanie kilka przykładów ilustrujących różne warianty składni na konkretnych instrukcjach
należy zdefiniować czym jest ,,adres rejestru'' (lub, skrótowo, ,,rejestr''), a czym jest ,,literał''.

\begin{labeling}{\textbf{adres rejestru}}
	\item[\textbf{adres rejestru}] identyfikator informujący kernel VM skąd ma pobrać wartości do
		przetworzenia i gdzie zapisać wyniki działania instrukcji, w specjalnym przypadku adresem rejestru
		jest tzw. ,,\emph{rejestr pusty}'' -- \texttt{void}
	\item[\textbf{literał}] reprezentacja wartości wpisana dosłownie w kod źródłowy lub binarny,
		np. \texttt{0xdeadbeef}, \texttt{"Hello World!"}, \texttt{42}; do literałów zaliczane są też nazwy
		funkcji, bloków i modułów
\end{labeling}

\subsubsection{Adresy rejestrów}

\begin{lstlisting}
<access-operator> <index> <register-set>
void
\end{lstlisting}

\begin{labeling}{\texttt{access-operator}}
\item[\texttt{access-operator}] \texttt{\%} -- ,,dostęp bezpośredni'', lub
	\texttt{*} -- ,,dereferencja wskaźnika''
\item[\texttt{index}] indeks rejestru wewnątrz zestawu rejestrów
\item[\texttt{register-set}] nazwa zestawu rejestrów
\end{labeling}

Dostępne zestawy rejestrów to:

\begin{labeling}{\texttt{parameters}}
\item[\texttt{local}] zawiera wartości lokalne (,,zmienne lokalne'')
\item[\texttt{static}] zawiera wartości statyczne (,,zmienne statyczne'')
\item[\texttt{arguments}] zawiera wartości przekazane do aktywnej funkcji jako argumenty
\item[\texttt{parameters}] zawiera wartości przekazane do przygotowywanej ramki wywołania jako parametry
\end{labeling}

Podczas odczytu wartości adres \texttt{\%3 static} powoduje ,,dostęp do wartości w 3. statycznym rejestrze'';
\texttt{*4 local} powoduje ,,dostęp do wartości, na którą wskazuje wskaźnik w 4. lokalnym rejestrze''.

Wartości w rejestrach z zestawów \texttt{parameters} i \texttt{arguments} nie mogą być modyfikowane
bezpośrednio. Muszą być najpierw przeniesione lub skopiowane do rejestrów lokalnych lub statycznych.

\subsection{Definicje funkcji i bloków}

\begin{lstlisting}
.function: <name>/<arity>
	<instruction>
	[<instruction>...]
.end
\end{lstlisting}

\begin{lstlisting}
.block: <name>
	<instruction>
	[<instruction>...]
.end
\end{lstlisting}

\subsection{Deklaracje funkcji}

\begin{lstlisting}
.signature: <name>/<arity>
\end{lstlisting}

\subsection{Import modułów}

\begin{lstlisting}
.import: <module>[::<module>...]
\end{lstlisting}

\subsection{Markery}

\begin{lstlisting}
.mark: <marker-name>
\end{lstlisting}

\subsection{Nazywanie rejestrów}

\begin{lstlisting}
.name: <index> <name>
\end{lstlisting}

\section{Instrukcje Viua VM}

\subsection{Zarządzanie wartościami}
\label{viua_vm_ops_value_management}

\subsubsection{\texttt{move}}

Przesunięcie wartości między rejestrami.

\subsubsection{\texttt{copy}}

Skopiowanie wartości między rejestrami.

\subsubsection{\texttt{ptr}}

Konstruktor wskaźnika do wartości.

\subsubsection{\texttt{ptrlive}}

Sprawdzenie poprawności wskaźnika.

\subsubsection{\texttt{swap}}

Zamiana wartości w rejestrach.

\subsubsection{\texttt{delete}}

Wywołanie destruktora wartości.

\subsubsection{\texttt{isnull}}

\subsection{Operacje logiczne}
\label{viua_vm_ops_boolean_logic}

\subsubsection{\texttt{not}}

Negacja boolowska.

\subsubsection{\texttt{and}}

Iloczyn boolowski.

\subsubsection{\texttt{or}}

Suma boolowska.

\subsection{Operacje bitowe}
\label{viua_vm_ops_bit_ops}

\subsection{\texttt{bits\_of\_integer}}

Konstruktor bitów z liczby całkowitej.

\subsection{\texttt{integer\_of\_bits}}

Konstruktor liczby całkowitej z bitów.

\subsection{\texttt{bits}}

Konstruktor bitów.

\subsection{\texttt{bitand}}

Bitowa operacja \emph{\texttt{and}}.

\subsection{\texttt{bitor}}

Bitowa operacja \emph{\texttt{or}}.

\subsection{\texttt{bitnot}}

Bitowa operacja \emph{\texttt{not}}.

\subsection{\texttt{bitxor}}

Bitowa operacja \emph{\texttt{xor}}.

\subsection{\texttt{bitat}}

Sprawdzenie wartości pojedynczego bitu.

\subsection{\texttt{bitset}}

Ustawienie wartości pojedynczego bitu.

\subsection{\texttt{shl}}

Przesunięcie bitowe w lewo.

\subsection{\texttt{shr}}

Przesunięcie bitowe w prawo.

\subsection{\texttt{ashl}}

Arytmetyczne (z zachowaniem znaku) przesunięcie bitowe w lewo.

\subsection{\texttt{ashr}}

Arytmetyczne przesunięcie bitowe w prawo.

\subsection{\texttt{rol}}

Rotacja bitowa w lewo.

\subsection{\texttt{ror}}

Rotacja bitowa w prawo.

\subsection{Arytmetyka (CPU)}
\label{viua_vm_ops_arithmetic_cpu}

Arytmetyka implementowana przez fizyczne CPU, na którym działa Viua VM.
Specyfika tego fizycznego CPU wpływa na wyniki działania instrukcji z tej grupy.
Arytmetyka o zdefiniowanym zachowaniu niezależnym od fizycznej platformy jest
opisana w rozdziale \ref{viua_vm_ops_arithmetic_vm} na stronie
\pageref{viua_vm_ops_arithmetic_vm}. Arytmetyka oparta o CPU jest
bardziej wydajna (,,szybsza''), ale nie zawsze zapewnia przewidywalność,
stabliność, i weryfikację wyników operacji (wykrywanie błędów).

\subsubsection{\texttt{izero}}

\begin{lstlisting}
izero Rd
\end{lstlisting}

\paragraph*{Opis} Konstruuje w rejestrze \texttt{Rd} liczbę całkowitą o wartości 0.
Zwyczajowo wykorzystywana na końcu funkcji \texttt{main} do utworzenia domyślnej wartości zwracanej:

\begin{lstlisting}
	izero %0 local
	return
.end
\end{lstlisting}

\subsubsection{\texttt{integer}}

\begin{lstlisting}
integer Rd %*\emph{<integer>}*)
\end{lstlisting}

\paragraph*{Opis} Konstruuje w rejestrze \texttt{Rd} liczbę całkowitą o wartości \emph{\texttt{integer}}.

\subsubsection{\texttt{iinc}}

\begin{lstlisting}
iinc Rt
\end{lstlisting}

\paragraph*{Opis} Zwiększa wartość liczby całkowitej w rejestrze \texttt{Rt} o 1.
\paragraph*{Uwagi} Zachowanie w przypadku przepełnienia jest zdefiniowane przez
zachowanie platformy sprzętowej.

\subsubsection{\texttt{idec}}

\begin{lstlisting}
idec Rt
\end{lstlisting}

\paragraph*{Opis} Zmniejsza wartość liczby całkowitej w rejestrze \texttt{Rt} o 1.
\paragraph*{Uwagi} Zachowanie w przypadku przepełnienia jest zdefiniowane przez
zachowanie platformy sprzętowej.

\subsubsection{\texttt{float}}

\begin{lstlisting}
float Rd %*\emph{<float>}*)
\end{lstlisting}

\paragraph*{Opis} Konstruuje w rejestrze \texttt{Rd} liczbę zmiennoprzecinkową o wartości
\emph{\texttt{float}}.

\subsubsection{\texttt{itof}}

\begin{lstlisting}
itof Rd Rs
\end{lstlisting}

\paragraph*{Opis} Konwertuje wartość całkowitoliczbową z rejestru \texttt{Rs} na
wartość zmiennoprzecinkową i umieszcza wynik w rejestrze \texttt{Rd}.

\subsubsection{\texttt{ftoi}}

\begin{lstlisting}
ftoi Rd Rs
\end{lstlisting}

\paragraph*{Opis} Konwertuje wartość zmiennoprzecinkową z rejestru \texttt{Rs}
na wartość całkowitoliczbową i umieszcza wynik w rejestrze \texttt{Rd}.

\subsubsection{\texttt{stoi}}

\begin{lstlisting}
stoi Rd Rs
\end{lstlisting}

\paragraph*{Opis} Konwertuje string z rejestru \texttt{Rs} na wartość
całkowitoliczbową i umieszcza wynik w rejestrze \texttt{Rd}.

\subsubsection{\texttt{stof}}

\begin{lstlisting}
stof Rd Rs
\end{lstlisting}

\paragraph*{Opis} Konwertuje string z rejestru \texttt{Rs} na wartość
zmiennoprzecinkową i umieszcza wynik w rejestrze \texttt{Rd}.

\subsubsection{\texttt{add}}

\begin{lstlisting}
add Rd Ra Rb
\end{lstlisting}

\paragraph*{Opis} Dodaje dwie wartości liczbowe (całkowitoliczbową lub
zmiennoprzecinkową) znajdujące się w rejestrach \texttt{Ra} i \texttt{Rb} do
siebie, a wynik umieszcza w rejestrze \texttt{Rd}.

Wartość umieszczona w rejestrze \texttt{Rd} ma taki sam typ jak wartość w
rejestrze \texttt{Ra}. Wartości w rejestrach \texttt{Ra} i \texttt{Rb} pozostają
nienaruszone.

\subsubsection{\texttt{sub}}

\begin{lstlisting}
sub Rd Ra Rb
\end{lstlisting}

\paragraph*{Opis} Odejmuje dwie wartości liczbowe (całkowitoliczbową lub
zmiennoprzecinkową) znajdujące się w rejestrach \texttt{Ra} i \texttt{Rb} do
siebie, a wynik umieszcza w rejestrze \texttt{Rd}.

Wartość umieszczona w rejestrze \texttt{Rd} ma taki sam typ jak wartość w
rejestrze \texttt{Ra}. Wartości w rejestrach \texttt{Ra} i \texttt{Rb} pozostają
nienaruszone.

\subsubsection{\texttt{mul}}

\begin{lstlisting}
mul Rd Ra Rb
\end{lstlisting}

\paragraph*{Opis} Mnoży dwie wartości liczbowe (całkowitoliczbową lub
zmiennoprzecinkową) znajdujące się w rejestrach \texttt{Ra} i \texttt{Rb}, a
wynik umieszcza w rejestrze \texttt{Rd}. Działanie jest przeprowadzane zgodnie
ze wzorem $a * b$.

Wartość umieszczona w rejestrze \texttt{Rd} ma taki sam typ jak wartość w
rejestrze \texttt{Ra}. Wartości w rejestrach \texttt{Ra} i \texttt{Rb} pozostają
nienaruszone.

\subsubsection{\texttt{div}}

\begin{lstlisting}
add Rd Ra Rb
\end{lstlisting}

\paragraph*{Opis} Dzieli dwie wartości liczbowe (całkowitoliczbową lub
zmiennoprzecinkową) znajdujące się w rejestrach \texttt{Ra} i \texttt{Rb}, a
wynik umieszcza w rejestrze \texttt{Rd}. Działanie jest przeprowadzane zgodnie
ze wzorem $a / b$.

Wartość umieszczona w rejestrze \texttt{Rd} ma taki sam typ jak wartość w
rejestrze \texttt{Ra}. Wartości w rejestrach \texttt{Ra} i \texttt{Rb} pozostają
nienaruszone.

\subsubsection{\texttt{lt}}

\begin{lstlisting}
lt Rd Ra Rb
\end{lstlisting}

\paragraph*{Opis} Porównuje dwie wartości liczbowe (całkowitoliczbową lub
zmiennoprzecinkową) znajdujące się w rejestrach \texttt{Ra} i \texttt{Rb}, a
wynik umieszcza w rejestrze \texttt{Rd}. Działanie jest przeprowadzane zgodnie
ze wzorem $a < b$.

Wartość wynika jest typu boolowskiego.

\subsubsection{\texttt{lte}}

\begin{lstlisting}
lte Rd Ra Rb
\end{lstlisting}

\paragraph*{Opis} Porównuje dwie wartości liczbowe (całkowitoliczbową lub
zmiennoprzecinkową) znajdujące się w rejestrach \texttt{Ra} i \texttt{Rb}, a
wynik umieszcza w rejestrze \texttt{Rd}. Działanie jest przeprowadzane zgodnie
ze wzorem $a <= b$.

Wartość wynika jest typu boolowskiego.

\subsubsection{\texttt{gt}}

\begin{lstlisting}
gt Rd Ra Rb
\end{lstlisting}

\paragraph*{Opis} Porównuje dwie wartości liczbowe (całkowitoliczbową lub
zmiennoprzecinkową) znajdujące się w rejestrach \texttt{Ra} i \texttt{Rb}, a
wynik umieszcza w rejestrze \texttt{Rd}. Działanie jest przeprowadzane zgodnie
ze wzorem $a > b$.

Wartość wynika jest typu boolowskiego.

\subsubsection{\texttt{gte}}

\begin{lstlisting}
gte Rd Ra Rb
\end{lstlisting}

\paragraph*{Opis} Porównuje dwie wartości liczbowe (całkowitoliczbową lub
zmiennoprzecinkową) znajdujące się w rejestrach \texttt{Ra} i \texttt{Rb}, a
wynik umieszcza w rejestrze \texttt{Rd}. Działanie jest przeprowadzane zgodnie
ze wzorem $a >= b$.

Wartość wynika jest typu boolowskiego.

\subsubsection{\texttt{eq}}

\begin{lstlisting}
eq Rd Ra Rb
\end{lstlisting}

\paragraph*{Opis} Porównuje dwie wartości liczbowe (całkowitoliczbową lub
zmiennoprzecinkową) znajdujące się w rejestrach \texttt{Ra} i \texttt{Rb}, a
wynik umieszcza w rejestrze \texttt{Rd}. Działanie jest przeprowadzane zgodnie
ze wzorem $a = b$.

Wartość wynika jest typu boolowskiego.

\subsection{Arytmetyka (Viua VM)}
\label{viua_vm_ops_arithmetic_vm}

Arytmetyka implementowana przez Viua VM, niezależna od fizycznej platformy, na
której maszyna wirtualna jest uruchomiona. Arytmetyka o zachowaniu zależnym od
fizycznej platformy jest opisana w rozdziale \ref{viua_vm_ops_arithmetic_cpu} na
stronie \pageref{viua_vm_ops_arithmetic_cpu}. Arytmetyka o zachowaniu
niezależnym od fizycznej platformy jest mniej wydajna (,,wolniesza''), ale
zapewnia przewidywalność, stabliność, i weryfikację wyników operacji (wykrywanie
błędów).

Operandami w instrukcjach arytmetycznych implementowanych przez VM są ciągi
bitów, których konstruktory są opisane w rozdziale \ref{viua_vm_ops_bit_ops} na
stronie \pageref{viua_vm_ops_bit_ops}.

\subsubsection{\texttt{wrapincrement}}

\begin{lstlisting}
wrapincrement Rt
\end{lstlisting}

\paragraph*{Opis} Inkrementacja modulo (ze znakiem).

Instrukcja interpretuje ciąg bitów jako liczbę całkowitą w kodowaniu z
dopełnieniem do dwóch\footnote{Kodowanie z dopełnieniem do dwóch
(ang. \emph{two's complement})} o szerokości równej długości ciągu bitów, na
którym instrukcja operuje.

\paragraph*{Uwagi}

W przypadku wystąpienia przepełnienia największa wartość dodatnia staje się
największą (tj. najbardziej oddaloną od zera) wartością ujemną.
\begin{lstlisting}
bits %1 local 0b01111111
print %1 local
wrapincrement %1 local
print %1 local
\end{lstlisting}
Ma to zastosowanie jedynie jeśli ciągi bitów są interpretowane przy konwersji
jako liczby ze znakiem.

\subsubsection{\texttt{wrapdecrement}}

\begin{lstlisting}
wrapdecrement Rt
\end{lstlisting}

\paragraph*{Opis} Dekrementacja modulo (ze znakiem).

Instrukcja interpretuje ciąg bitów jako liczbę całkowitą w kodowaniu z
dopełnieniem do dwóch o szerokości równej długości ciągu bitów, na którym
instrukcja operuje.

\paragraph*{Uwagi}

W przypadku wystąpienia przepełnienia największa (tj. najbardziej oddalona od
zera) wartość ujemna staje się największą wartością dodatnią.
\begin{lstlisting}
bits %1 local 0x80
print %1 local
wrapdecrement %1 local
print %1 local
\end{lstlisting}
Ma to zastosowanie jedynie jeśli ciągi bitów są interpretowane przy konwersji
jako liczby ze znakiem.

\subsubsection{\texttt{wrapadd}}

\begin{lstlisting}
wrapadd Rd Ra Rb
\end{lstlisting}

\paragraph*{Opis} Dodawanie modulo.

Instrukcja interpretuje liczby w rejestrach \texttt{Ra} i \texttt{Rb} jaki
liczby całkowite ze znakiem w kodowaniu z dopełnieniem do dwóch. Wynikiem
instrukcji jest ciąg bitów reprezentujący liczbe całkowitą ze znakiem w
kodowaniu z dopełnieniem do dwóch, o długości odpowiadającej długości ciągu
bitów znajdującemu się w rejestrze \texttt{Ra}.

Operacja jest wykonywana według wzoru $a + b$.

\paragraph*{Uwagi}

W przypadku przepełnienia wartość jest ,,zawijana'' i wynik z dodatniego może
stać się ujemy i \emph{vice versa}.

\subsubsection{\texttt{wrapsub}}

\begin{lstlisting}
wrapsub Rd Ra Rb
\end{lstlisting}

\paragraph*{Opis} Odejmowanie modulo.

Instrukcja interpretuje liczby w rejestrach \texttt{Ra} i \texttt{Rb} jaki
liczby całkowite ze znakiem w kodowaniu z dopełnieniem do dwóch. Wynikiem
instrukcji jest ciąg bitów reprezentujący liczbe całkowitą ze znakiem w
kodowaniu z dopełnieniem do dwóch, o długości odpowiadającej długości ciągu
bitów znajdującemu się w rejestrze \texttt{Ra}.

Operacja jest wykonywana według wzoru $a - b$.

\paragraph*{Uwagi}

W przypadku przepełnienia wartość jest ,,zawijana'' i wynik z dodatniego może
stać się ujemy i \emph{vice versa}.

\subsubsection{\texttt{wrapmul}}

\begin{lstlisting}
wrapmul Rd Ra Rb
\end{lstlisting}

\paragraph*{Opis} Mnożenie modulo.

Instrukcja interpretuje liczby w rejestrach \texttt{Ra} i \texttt{Rb} jaki
liczby całkowite ze znakiem w kodowaniu z dopełnieniem do dwóch. Wynikiem
instrukcji jest ciąg bitów reprezentujący liczbe całkowitą ze znakiem w
kodowaniu z dopełnieniem do dwóch, o długości odpowiadającej długości ciągu
bitów znajdującemu się w rejestrze \texttt{Ra}.

Operacja jest wykonywana według wzoru $a * b$.

\paragraph*{Uwagi}

W przypadku przepełnienia wartość jest ,,zawijana'' i wynik z dodatniego może
stać się ujemy i \emph{vice versa}.

\subsubsection{\texttt{wrapdiv}}

\begin{lstlisting}
wrapdiv Rd Ra Rb
\end{lstlisting}

\paragraph*{Opis} Dzielenie modulo.

Instrukcja interpretuje liczby w rejestrach \texttt{Ra} i \texttt{Rb} jaki
liczby całkowite ze znakiem w kodowaniu z dopełnieniem do dwóch. Wynikiem
instrukcji jest ciąg bitów reprezentujący liczbe całkowitą ze znakiem w
kodowaniu z dopełnieniem do dwóch, o długości odpowiadającej długości ciągu
bitów znajdującemu się w rejestrze \texttt{Ra}.

Operacja jest wykonywana według wzoru $a / b$.

\paragraph*{Uwagi}

W przypadku przepełnienia wartość jest ,,zawijana'' i wynik z dodatniego może
stać się ujemy i \emph{vice versa}.

\subsubsection{\texttt{checkedsincrement}}

\begin{lstlisting}
checkedsincrement Rt
\end{lstlisting}

\paragraph*{Opis} Inkrementacja weryfikowana ze znakiem.

Instrukcja interpretuje ciąg bitów jako liczbę całkowitą w kodowaniu z
dopełnieniem do dwóch o szerokości równej długości ciągu bitów, na którym
instrukcja operuje.

\paragraph*{Uwagi}

W przypadku przepełnienia zgłaszany jest wyjątek.

\subsubsection{\texttt{checkedsdecrement}}

\begin{lstlisting}
checkedsdecrement Rt
\end{lstlisting}

\paragraph*{Opis} Inkrementacja weryfikowana ze znakiem.

Instrukcja interpretuje ciąg bitów jako liczbę całkowitą w kodowaniu z
dopełnieniem do dwóch o szerokości równej długości ciągu bitów, na którym
instrukcja operuje.

\paragraph*{Uwagi}

W przypadku przepełnienia zgłaszany jest wyjątek.

\subsubsection{\texttt{checkedsadd}}

\begin{lstlisting}
checkedsadd Rd Ra Rb
\end{lstlisting}

\paragraph*{Opis} Dodawanie weryfikowane ze znakiem.

Instrukcja interpretuje liczby w rejestrach \texttt{Ra} i \texttt{Rb} jaki
liczby całkowite ze znakiem w kodowaniu z dopełnieniem do dwóch. Wynikiem
instrukcji jest ciąg bitów reprezentujący liczbe całkowitą ze znakiem w
kodowaniu z dopełnieniem do dwóch, o długości odpowiadającej długości ciągu
bitów znajdującemu się w rejestrze \texttt{Ra}.

Operacja jest wykonywana według wzoru $a + b$.

\paragraph*{Uwagi}

W przypadku przepełnienia zgłaszany jest wyjątek.

\subsubsection{\texttt{checkedssub}}

\begin{lstlisting}
checkedssub Rd Ra Rb
\end{lstlisting}

\paragraph*{Opis} Odejmowanie weryfikowane ze znakiem.

Instrukcja interpretuje liczby w rejestrach \texttt{Ra} i \texttt{Rb} jaki
liczby całkowite ze znakiem w kodowaniu z dopełnieniem do dwóch. Wynikiem
instrukcji jest ciąg bitów reprezentujący liczbe całkowitą ze znakiem w
kodowaniu z dopełnieniem do dwóch, o długości odpowiadającej długości ciągu
bitów znajdującemu się w rejestrze \texttt{Ra}.

Operacja jest wykonywana według wzoru $a - b$.

\paragraph*{Uwagi}

W przypadku przepełnienia zgłaszany jest wyjątek.

\subsubsection{\texttt{checkedsmul}}

\begin{lstlisting}
checkedsmul Rd Ra Rb
\end{lstlisting}

\paragraph*{Opis} Dzielenie weryfikowane ze znakiem.

Instrukcja interpretuje liczby w rejestrach \texttt{Ra} i \texttt{Rb} jaki
liczby całkowite ze znakiem w kodowaniu z dopełnieniem do dwóch. Wynikiem
instrukcji jest ciąg bitów reprezentujący liczbe całkowitą ze znakiem w
kodowaniu z dopełnieniem do dwóch, o długości odpowiadającej długości ciągu
bitów znajdującemu się w rejestrze \texttt{Ra}.

Operacja jest wykonywana według wzoru $a * b$.

\paragraph*{Uwagi}

W przypadku przepełnienia zgłaszany jest wyjątek.

\subsubsection{\texttt{checkedsdiv}}

\begin{lstlisting}
checkedsdiv Rd Ra Rb
\end{lstlisting}

\paragraph*{Opis} Dzielenie weryfikowane ze znakiem.

Instrukcja interpretuje liczby w rejestrach \texttt{Ra} i \texttt{Rb} jaki
liczby całkowite ze znakiem w kodowaniu z dopełnieniem do dwóch. Wynikiem
instrukcji jest ciąg bitów reprezentujący liczbe całkowitą ze znakiem w
kodowaniu z dopełnieniem do dwóch, o długości odpowiadającej długości ciągu
bitów znajdującemu się w rejestrze \texttt{Ra}.

Operacja jest wykonywana według wzoru $a / b$.

\paragraph*{Uwagi}

W przypadku przepełnienia zgłaszany jest wyjątek.

\subsubsection{\texttt{saturatingsincrement}}

\begin{lstlisting}
saturatingsincrement Rt
\end{lstlisting}

\paragraph*{Opis} Inkrementacja nasyceniowa ze znakiem.

Instrukcja interpretuje ciąg bitów jako liczbę całkowitą w kodowaniu z
dopełnieniem do dwóch o szerokości równej długości ciągu bitów, na którym
instrukcja operuje.

\paragraph*{Uwagi}

W przypadku wykrycia przepełnienia następuje nasycenie i wynik jest wartością
maksymalną dla liczby całkowitej ze znakiem w kodowaniu do dwóch o szerokości
równej długości ciągu bitów użytego jako operand instrukcji.

\subsubsection{\texttt{saturatingsdecrement}}

\begin{lstlisting}
saturatingsdecrement Rt
\end{lstlisting}

\paragraph*{Opis} Dekrementacja nasyceniowa ze znakiem.

Instrukcja interpretuje ciąg bitów jako liczbę całkowitą w kodowaniu z
dopełnieniem do dwóch o szerokości równej długości ciągu bitów, na którym
instrukcja operuje.

\paragraph*{Uwagi}

W przypadku wykrycia przepełnienia następuje nasycenie i wynik jest wartością
minimalną dla liczby całkowitej ze znakiem w kodowaniu do dwóch o szerokości
równej długości ciągu bitów użytego jako operand instrukcji.

\subsubsection{\texttt{saturatingsadd}}

\begin{lstlisting}
saturatingsadd Rd Ra Rb
\end{lstlisting}

\paragraph*{Opis} Dodawanie nasyceniowe ze znakiem.

Instrukcja interpretuje liczby w rejestrach \texttt{Ra} i \texttt{Rb} jaki
liczby całkowite ze znakiem w kodowaniu z dopełnieniem do dwóch. Wynikiem
instrukcji jest ciąg bitów reprezentujący liczbe całkowitą ze znakiem w
kodowaniu z dopełnieniem do dwóch, o długości odpowiadającej długości ciągu
bitów znajdującemu się w rejestrze \texttt{Ra}.

Operacja jest wykonywana według wzoru $a + b$.

\paragraph*{Uwagi}

W przypadku wykrycia przepełnienia następuje nasycenie i
\begin{enumerate}
\item jeśli wynik miał być dodatni to jest wartością maksymalną dla liczby
    całkowitej ze znakiem w kodowaniu do dwóch o szerokości równej długości
    ciągu bitów użytego jako operand instrukcji
\item jeśli wynik miał być ujemny to jest wartością minimalną dla liczby
    całkowitej ze znakiem w kodowaniu do dwóch o szerokości równej długości
    ciągu bitów użytego jako operand instrukcji
\end{enumerate}

\subsubsection{\texttt{saturatingssub}}

\begin{lstlisting}
saturatingssub Rd Ra Rb
\end{lstlisting}

\paragraph*{Opis} Odejmowanie nasyceniowe ze znakiem.

Instrukcja interpretuje liczby w rejestrach \texttt{Ra} i \texttt{Rb} jaki
liczby całkowite ze znakiem w kodowaniu z dopełnieniem do dwóch. Wynikiem
instrukcji jest ciąg bitów reprezentujący liczbe całkowitą ze znakiem w
kodowaniu z dopełnieniem do dwóch, o długości odpowiadającej długości ciągu
bitów znajdującemu się w rejestrze \texttt{Ra}.

Operacja jest wykonywana według wzoru $a - b$.

\paragraph*{Uwagi}

W przypadku wykrycia przepełnienia następuje nasycenie i
\begin{enumerate}
\item jeśli wynik miał być dodatni to jest wartością maksymalną dla liczby
    całkowitej ze znakiem w kodowaniu do dwóch o szerokości równej długości
    ciągu bitów użytego jako operand instrukcji
\item jeśli wynik miał być ujemny to jest wartością minimalną dla liczby
    całkowitej ze znakiem w kodowaniu do dwóch o szerokości równej długości
    ciągu bitów użytego jako operand instrukcji
\end{enumerate}

\subsubsection{\texttt{saturatingsmul}}

\begin{lstlisting}
saturatingsmul Rd Ra Rb
\end{lstlisting}

\paragraph*{Opis} Mnożenie nasyceniowe ze znakiem.

Instrukcja interpretuje liczby w rejestrach \texttt{Ra} i \texttt{Rb} jaki
liczby całkowite ze znakiem w kodowaniu z dopełnieniem do dwóch. Wynikiem
instrukcji jest ciąg bitów reprezentujący liczbe całkowitą ze znakiem w
kodowaniu z dopełnieniem do dwóch, o długości odpowiadającej długości ciągu
bitów znajdującemu się w rejestrze \texttt{Ra}.

Operacja jest wykonywana według wzoru $a * b$.

\paragraph*{Uwagi}

W przypadku wykrycia przepełnienia następuje nasycenie i
\begin{enumerate}
\item jeśli wynik miał być dodatni to jest wartością maksymalną dla liczby
    całkowitej ze znakiem w kodowaniu do dwóch o szerokości równej długości
    ciągu bitów użytego jako operand instrukcji
\item jeśli wynik miał być ujemny to jest wartością minimalną dla liczby
    całkowitej ze znakiem w kodowaniu do dwóch o szerokości równej długości
    ciągu bitów użytego jako operand instrukcji
\end{enumerate}

\subsubsection{\texttt{saturatingsdiv}}

\begin{lstlisting}
checkedsdiv Rd Ra Rb
\end{lstlisting}

\paragraph*{Opis} Dzielenie nasyceniowe ze znakiem.

Instrukcja interpretuje liczby w rejestrach \texttt{Ra} i \texttt{Rb} jaki
liczby całkowite ze znakiem w kodowaniu z dopełnieniem do dwóch. Wynikiem
instrukcji jest ciąg bitów reprezentujący liczbe całkowitą ze znakiem w
kodowaniu z dopełnieniem do dwóch, o długości odpowiadającej długości ciągu
bitów znajdującemu się w rejestrze \texttt{Ra}.

Operacja jest wykonywana według wzoru $a / b$.

\paragraph*{Uwagi}

W przypadku wykrycia przepełnienia następuje nasycenie i
\begin{enumerate}
\item jeśli wynik miał być dodatni to jest wartością maksymalną dla liczby
    całkowitej ze znakiem w kodowaniu do dwóch o szerokości równej długości
    ciągu bitów użytego jako operand instrukcji
\item jeśli wynik miał być ujemny to jest wartością minimalną dla liczby
    całkowitej ze znakiem w kodowaniu do dwóch o szerokości równej długości
    ciągu bitów użytego jako operand instrukcji
\end{enumerate}

\subsection{Obsługa tekstu}
\label{viua_vm_ops_text}

\subsection{\texttt{string}}

\begin{lstlisting}
string Rd %*\emph{<str>}*)
\end{lstlisting}

\paragraph*{Opis} Konstruuje w rejestrze \texttt{Rd} ciąg bajtów o wartości \emph{\texttt{str}}.

\subsection{\texttt{streq}}

\subsection{\texttt{text}}

\begin{lstlisting}
text Rd %*\emph{<txt>}*)
text Rd Rs
\end{lstlisting}

\paragraph*{Opis} Wariant pierwszy konstruuje w rejestrze \texttt{Rd} tekst o wartości \emph{\texttt{txt}}.
Wariant drugi konwertuje wartość z rejestru \texttt{Rs} na tekst i umieszcza go w rejestrze \texttt{Rd}.

\subsection{\texttt{texteq}}
\subsection{\texttt{textat}}
\subsection{\texttt{textsub}}
\subsection{\texttt{textlength}}
\subsection{\texttt{textcommonprefix}}
\subsection{\texttt{textcommonsuffix}}
\subsection{\texttt{textconcat}}

\subsection{Wektory}
\label{viua_vm_ops_vector}

\subsubsection{\texttt{vector}}
\subsubsection{\texttt{vinsert}}
\subsubsection{\texttt{vpush}}
\subsubsection{\texttt{vpop}}
\subsubsection{\texttt{vat}}
\subsubsection{\texttt{vlen}}

\subsection{Atomy}
\label{viua_vm_ops_atom}

\subsection{\texttt{atom}}

Konstruktor atomu.

\subsection{\texttt{atomeq}}

\subsection{Struktury (rekordy)}
\label{viua_vm_ops_struct}

\subsection{\texttt{struct}}

Konstruktor struktury.

\subsection{\texttt{structinsert}}
\subsection{\texttt{structremove}}
\subsection{\texttt{structat}}
\subsection{\texttt{structkeys}}

\subsection{Aktory}
\label{viua_vm_ops_actor}

\subsubsection{\texttt{process}}

\begin{lstlisting}
process Rd fn/0
process void fn/0
\end{lstlisting}

\paragraph*{Opis}

Wywołuje funkcję tworząc nowy proces.
Wariant 1 zapisuje PID nowego procesu w rejestrze \texttt{Rd}.
Wariant drugi nie zapisuje PID nowego procesu.
\begin{lstlisting}
process %4 local do_some_processing/1
process void a_free_actor/4
\end{lstlisting}

\subsubsection{\texttt{self}}

\begin{lstlisting}
self Rd
\end{lstlisting}

\paragraph*{Opis} Konstruuje PID procesu wewnątrz którego instrukcja jest
wykonywana.

\subsubsection{\texttt{pideq}}

\begin{lstlisting}
pideq Rd Ra Rb
\end{lstlisting}

\paragraph*{Opis} Porównuje wartości PID znajdujące się w rejestrach \texttt{Ra}
i \texttt{Rb}. Wynik porównania zapisuje jako wartość boolowską w rejestrze
\texttt{Rd}.

\subsubsection{\texttt{join}}

\begin{lstlisting}
join Rd Rp
join Rd Rp infinity
join Rd Rp %*\emph{<timeout>}*)
\end{lstlisting}

\paragraph*{Opis} Zatrzymuje wykonanie procesu wewnątrz którego wykonywana jest
instrukcja i oczekuje na zakończenie procesu identyfikowanego przez PID
znajdujący się w rejestrze \texttt{Rp}.

Wariant pierwszy jest identyczny w zachowaniu jak wariant drugi i powoduje, że
proces wykonujący instrukcję oczekuje w nieskończoność.

Wariant trzeci umożliwia podanie wartości \emph{timeout}, która określa
maksymalny czas oczekiwania na zakończenie pracy przez zadany proces. Czas może
być podany w sekundach (\texttt{\emph{N}s}) lub
milisekundach~(\texttt{\emph{N}ms}), a wartość zero jest poprawna. Przykład:
\begin{lstlisting}
join %1 local %2 local 1s
join %1 local %2 local 0ms
\end{lstlisting}

Jeśli rejestr \texttt{Rd} jest podany jako \texttt{void} wartość zwracana przez
zadany proces jest natychmiast niszczona.

\paragraph*{Uwagi}

Jeśli do momentu upłynięcia czasu oczekiwania zadany proces nie zakończył pracy
zgłaszany jest wyjątek.

Jeśli zadany proces zakończył pracę z powodu awarii to nieobsłużony wyjątek,
który tę awarię wywołał jest ponownie zgłaszany wewnątrz procesu, który wykonał
instrukcję \texttt{join}.

\subsubsection{\texttt{send}}

\begin{lstlisting}
send Rp Rs
\end{lstlisting}

\paragraph*{Opis} Wysyła wartość znajdującą się w rejestrze \texttt{Rs} jako
wiadomość do procesu identyfikowanego przez PID znajdujący się w rejestrze
\texttt{Rp}.

\subsubsection{\texttt{receive}}

\begin{lstlisting}
receive Rd
receive Rd infinity
receive Rd %*\emph{<timeout>}*)
\end{lstlisting}

\paragraph*{Opis} Blokuje proces wewnątrz którego wykonywana jest instrukcja i
oczekuje na nadejście wiadomości. Jeśli jakaś wiadomość jest już dostępna w
kolejce wiadomości procesu jest ona zwracana od razu, bez oczekiwania. Odebrana
wiadomość jest umieszczana w rejestrze \texttt{Rd}.

Pierwszy i drugi wariant są identyczne w zachowaniu i blokują proces oczekując
w nieskończoność na nadejście wiadomości.

Wariant trzeci umożliwia podanie wartości \emph{timeout} określającej maksymalny
czas oczekiwania na nadejście wiadomości w sekundach lub milisekundach. Czas
oczekiwania może wynosić zero sekund lub milisekund.

\paragraph*{Uwagi}

Jeśli po upłynięciu czasu oczekiwania żadna wiadomość nie będzie dostępna
zgłaszany jest wyjątek.

\subsection{Programowanie funkcyjne}
\label{viua_vm_ops_functional}

\subsubsection{\texttt{function}}

\begin{lstlisting}
function Rd %*\emph{function-name}*)
\end{lstlisting}

\paragraph*{Opis} Konstruuje w rejestrze \texttt{Rd} wartość reprezentującą
funkcję określoną przez \texttt{\emph{function-name}}.

Wartość ta może być wykorzystana w instrukcjach \texttt{call}, \texttt{tailcall}
i \texttt{process} aby wywołać funkcję niebezpośrednio. Przykład:
\begin{lstlisting}
function %1 local a_function/0
frame %0
call void %1 local
\end{lstlisting}

\subsubsection{\texttt{closure}}

\begin{lstlisting}
closure Rd %*\emph{closure-name}*)
\end{lstlisting}

\paragraph*{Opis} Konstruuje w rejestrze \texttt{Rd} wartość będącą instancją
domknięcia (ang. \emph{closure}) określonego przez \texttt{\emph{closure-name}}.

Zanim tak utworzona wartość będzie mogła być wywołana wszystkie wartości, do
których domknięcie \emph{closure-name} się odwołuje muszą zostać ,,schwytane''
wewnątrz utworzonego domknięcia za pomocą instrukcji \texttt{capture},
\texttt{capturecopy} lub \texttt{capturemove}.

\subsubsection{\texttt{capture}}

\begin{lstlisting}
capture Rc Ri Rv
\end{lstlisting}

\paragraph*{Opis} W rejestrze lokalnym o indeksie \texttt{Ri} wewnętrznym do
domknięcia znajdującego się w rejestrze \texttt{Rc} tworzy referencję do
wartości znajdującej się w rejestrze \texttt{Rv}.
\begin{lstlisting}
text %3 local "Hello World!"
capture %1 local %1 %3 local
\end{lstlisting}

Przykład powyżej ,,chwyta'' wartość tekstową znajdującą się w rejestrze
\texttt{\%3~local} i referencję do niej umieszcza w lokalnym rejestrze o
indeksie 1 wewnętrznym do domknięcia znajdującego się w rejestrze
\texttt{\%1~local}.

\paragraph*{Uwagi -- czas życia} Schwytanie wartości przez referencję powoduje,
że przestaje ona być śledzona przez standardowy deterministyczny mechanizm
zarządzania wartościami i jej czas życia może być trudny do określenia jeśli
domknięcie jest zwracane z funkcji lub przekazywane dalej jak parametr.

\paragraph*{Uwagi -- przekazywanie wiadomości} Wartości schwytane przez
referencję nie mogą być przekazywane jako wiadomości ponieważ naruszyło by to
ścisłą izolację procesów.

\subsubsection{\texttt{capturecopy}}

\begin{lstlisting}
capturecopy Rc Ri Rv
\end{lstlisting}

\paragraph*{Opis} W rejestrze lokalnym o indeksie \texttt{Ri} wewnętrznym do
domknięcia znajdującego się w rejestrze \texttt{Rc} tworzy kopię wartości
znajdującej się w rejestrze \texttt{Rv}.
\begin{lstlisting}
text %3 local "Hello World!"
capturecopy %1 local %1 %3 local
\end{lstlisting}

Przykład powyżej ,,chwyta'' wartość tekstową znajdującą się w rejestrze
\texttt{\%3~local} i jej kopię umieszcza w lokalnym rejestrze o indeksie 1
wewnętrznym do domknięcia znajdującego się w rejestrze \texttt{\%1~local}.

Odpowiada to instrukcji \texttt{copy}, dla której wynikowym zestawem rejestrów
jest lokalny zestaw rejestrów wewnętrzny dla domknięcia.

\subsubsection{\texttt{capturemove}}

\begin{lstlisting}
capturemove Rc Ri Rv
\end{lstlisting}

\paragraph*{Opis} Przenosi wartość z rejestru \texttt{Rv} do rejestru lokalnego
o indeksie \texttt{Ri} wewnętrznego do domknięcia znajdującego się w rejestrze
\texttt{Rc}.
\begin{lstlisting}
text %3 local "Hello World!"
capturemove %1 local %1 %3 local
\end{lstlisting}

Przykład powyżej ,,chwyta'' wartość tekstową znajdującą się w rejestrze
\texttt{\%3~local} i przenosi ją do lokalnego rejestru o indeksie 1
wewnętrznego do domknięcia znajdującego się w rejestrze \texttt{\%1~local}.

Odpowiada to instrukcji \texttt{move}, dla której wynikowym zestawem rejestrów
jest lokalny zestaw rejestrów wewnętrzny dla domknięcia.

\subsection{Kontrola przepływu}
\label{viua_vm_ops_control_flow}

\paragraph*{Specyfikacje celów skoków}

\begin{lstlisting}
+%*\emph{<offset>}*)
-%*\emph{<offset>}*)
%*\emph{<marker>}*)
\end{lstlisting}

Pierwszy i drugi wariant przenoszą kontrolę o \emph{offset} instrukcji w przód
lub w tył. Trzeci wariant przenosi kontrolę do pierwszej instrukcji za markerem
\texttt{\emph{marker}}.

\subsubsection{\texttt{nop}}

\begin{lstlisting}
nop
\end{lstlisting}

\paragraph*{Opis} Instrukcja ,,pusta''.

\subsubsection{\texttt{jump}}

\begin{lstlisting}
jump %*\emph{<jump-spec>}*)
\end{lstlisting}

\paragraph*{Opis} Instrukcja skoku bezwarunkowego. Przenosi kontrolę do
instrukcji określonej przez \texttt{\emph{jump-spec}}.

\subsubsection{\texttt{if}}

\begin{lstlisting}
if Rcond %*\emph{<true-jump-spec>}*) %*\emph{<false-jump-spec>}*)
\end{lstlisting}

\paragraph*{Opis} Instrukcja skoku warunkowego odczytuje wartość z rejestru
\texttt{Rcond}, interpretuje ją jako wartość boolowską i podejmuje decyzję,
który skok wykonać. Jeśli warunek jest prawdziwy wykonywany jest skok
\texttt{\emph{true-jump-spec}}, jeśli warunek jest prawdziwy wykonywany jest
skok \texttt{\emph{false-jump-spec}}.

\subsubsection{\texttt{frame}}

\begin{lstlisting}
frame %*\emph{n}*)
\end{lstlisting}

\paragraph*{Opis} Przygotowuje do ramkę wywołania zawierającą \emph{n} rejestrów
do przekazania parametrów, np. \texttt{frame \%2} utworzy ramkę z dwoma
rejestrami.

Przygotowana ramka jest potem ,,konsumowana'' przez instrukcję \texttt{call},
\texttt{tailcall}, \texttt{process}, \texttt{defer} lub \texttt{watchdog}.

\subsubsection{\texttt{param}}

Instrukcja prywatna.
Przekazanie parametrów przez kopię odbywa się za pomocą instrukcji
\texttt{copy}.

\paragraph*{Uwagi}

Instrukcje prywatne nie są dostępne dla kodu użytkownika, a programy zawierające
je są odrzucane przez assembler jako nieprawidłowe.

\subsubsection{\texttt{pamv}}

Instrukcja prywatna.
Przekazanie parametrów przez przeniesienie odbywa się za pomocą instrukcji
\texttt{move}.

\paragraph*{Uwagi}

Instrukcje prywatne nie są dostępne dla kodu użytkownika, a programy zawierające
je są odrzucane przez assembler jako nieprawidłowe.

\subsubsection{\texttt{call}}

\begin{lstlisting}
call Rr %*\emph{function-name}*)
call Rr Rf
\end{lstlisting}

\paragraph*{Opis} Instrukcja wywołania funkcji lub instancji domknięcia.
Konsumuje utworzoną wcześniej ramkę wywołania i dodaje ją na szczyt stosu
wywołań.

Wariant 1 wywołuje funkcję określoną przez \texttt{\emph{function-name}}.
Wariant 2 wywołuje funkcję lub instancję domknięcia wskazywaną przez wartość w
rejestrze \texttt{Rf}. Tylko wariant drugi może być użyty do wywołania instancji
domknięcia.

W każdym wariancie wynik pracy funkcji jest umieszczany w rejestrze \texttt{Rr}.
Jeśli rejestr \texttt{Rr} jest podany jako \texttt{void} to wartość zwrotna
wywołanej funkcji jest natychmiast niszczona.

\subsubsection{\texttt{tailcall}}

\begin{lstlisting}
tailcall %*\emph{function-name}*)
tailcall Rf
\end{lstlisting}

\paragraph*{Opis} Instrukcja wywołania ogonowego funkcji lub instancji
domknięcia. Konsumuje utworzoną wcześniej ramkę wywołania i zamienia ją z ramką
znajdującą się na szczycie stosu wywołań -- nie zwiększając rozmiaru stosu
wywołań.

Wariant 1 wywołuje funkcję określoną przez \texttt{\emph{function-name}}.
Wariant 2 wywołuje funkcję lub instancję domknięcia wskazywaną przez wartość w
rejestrze \texttt{Rf}. Tylko wariant drugi może być użyty do wywołania instancji
domknięcia.

\subsubsection{\texttt{defer}}

\begin{lstlisting}
defer %*\emph{function-name}*)
defer Rf
\end{lstlisting}

\paragraph*{Opis} Instrukcja wywołania odroczonego funkcji. Konsumuje utworzoną
wcześniej ramkę wywołania.

Wariant 1 wywołuje funkcję określoną przez \texttt{\emph{function-name}}.
Wariant 2 wywołuje funkcję wskazywaną przez wartość w rejestrze \texttt{Rf}.

Wywołania odroczone będą wywołane w odwrotnej kolejności zgłaszania ich w
momencie, w którym ramka wywołania, w której zostały zarejestrowane jest
zdejmowana ze stosu wywołań (np. na skutek wykonania instrukcji \texttt{return},
podmiany ramki przez instrukcję \texttt{tailcall} lub odwinięcia stosu na skutek
nieobsłużonego wyjątku).

\subsubsection{\texttt{arg}}

Instrukcja prywatna.
Odczyt parametrów przez przeniesienie odbywa się za pomocą instrukcji
\texttt{move}.

\paragraph*{Uwagi}

Instrukcje prywatne nie są dostępne dla kodu użytkownika, a programy zawierające
je są odrzucane przez assembler jako nieprawidłowe.

\subsubsection{\texttt{allocate\_registers}}

\begin{lstlisting}
allocate_registers Rn
\end{lstlisting}

\paragraph*{Opis} Alokuje lokalny zestaw rejestrów w rozmiarze określonym przez
operand \texttt{Rn}. Jest to pierwsza instrukcja, która musi być wykonana przez
funkcję.

\paragraph*{Uwagi} Domknięcia \emph{nie mogą} używać tej instrukcji. Ich zestaw
rejestrów lokalnych jest alokowany przez instrukcję \texttt{closure}.

\subsubsection{\texttt{return}}

\begin{lstlisting}
return
\end{lstlisting}

\paragraph*{Opis} Kończy wykonywanie funkcji i zdejmuje jej ramkę ze stosu
wywołań.

\subsubsection{\texttt{halt}}

\begin{lstlisting}
halt
\end{lstlisting}

\paragraph*{Opis} Natychmiast przerywa wykonywanie programu i wyłącza maszynę
wirtualną.

\subsection{Obsługa błędów}
\label{viua_vm_ops_error_handling}

\subsection{\texttt{watchdog}}

\subsection{\texttt{throw}}

Rzucenie wartości jako wyjątku.

\subsection{\texttt{catch}}
\subsection{\texttt{draw}}
\subsection{\texttt{try}}
\subsection{\texttt{enter}}
\subsection{\texttt{leave}}

\subsection{Systemu modułów}
\label{viua_vm_ops_module_system}

\subsubsection{\texttt{import}}

\begin{lstlisting}
import %*\emph{module-name}*)
\end{lstlisting}

\paragraph*{Opis} Instrukcja dynamicznie dołącza moduł
\texttt{\emph{module-name}} do uruchomionego programu.

\paragraph*{Uwagi}

Użycie instrukcji \texttt{import} jest niezalecane.
Zamiast niej należy używać dyrektywy import opisanej w rozdziale
\ref{appendix_viua_vm_assembly_language_dir_import} na stronie
\pageref{appendix_viua_vm_assembly_language_dir_import}.

