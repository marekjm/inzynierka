\chapter{Język assemblera Viua VM}
\label{appendix_viua_vm_assembly_language}

W tym rozdziale zostanie po krótce omówiony język assemblera Viua VM.

\section{Składnia języka assemblera}

Składnia języka assemblera Viua VM jest prosta. Wyglądem przypomina składnię innych języków tego rodzaju
(np. języki assemblera x86 czy ARM).

\subsection{Ogólna składnia instrukcji}

Instrukcje składają się z mnemoniki, zera lub więcej adresów rejestrów, i co najwyżej jednego literału. Ogólną składnię można zapisać więc tak:

\begin{lstlisting}
mnemonic [<register>...] [<literal>]
\end{lstlisting}

Zanim pokazanych zostanie kilka przykładów ilustrujących różne warianty składni na konkretnych instrukcjach
należy zdefiniować czym jest ,,adres rejestru'' (lub, skrótowo, ,,rejestr''), a czym jest ,,literał''.

\begin{labeling}{\textbf{adres rejestru}}
	\item[\textbf{adres rejestru}] identyfikator informujący kernel VM skąd ma pobrać wartości do
		przetworzenia i gdzie zapisać wyniki działania instrukcji, w specjalnym przypadku adresem rejestru
		jest tzw. ,,\emph{rejestr pusty}'' -- \texttt{void}
	\item[\textbf{literał}] reprezentacja wartości wpisana dosłownie w kod źródłowy lub binarny,
		np. \texttt{0xdeadbeef}, \texttt{"Hello World!"}, \texttt{42}; do literałów zaliczane są też nazwy
		funkcji, bloków i modułów
\end{labeling}

\subsubsection{Adresy rejestrów}

\begin{lstlisting}
<access-operator> <index> <register-set>
void
\end{lstlisting}

\begin{labeling}{\texttt{access-operator}}
\item[\texttt{access-operator}] \texttt{\%} -- ,,dostęp bezpośredni'', lub
	\texttt{*} -- ,,dereferencja wskaźnika''
\item[\texttt{index}] indeks rejestru wewnątrz zestawu rejestrów
\item[\texttt{register-set}] nazwa zestawu rejestrów
\end{labeling}

Dostępne zestawy rejestrów to:

\begin{labeling}{\texttt{parameters}}
\item[\texttt{local}] zawiera wartości lokalne (,,zmienne lokalne'')
\item[\texttt{static}] zawiera wartości statyczne (,,zmienne statyczne'')
\item[\texttt{arguments}] zawiera wartości przekazane do aktywnej funkcji jako argumenty
\item[\texttt{parameters}] zawiera wartości przekazane do przygotowywanej ramki wywołania jako parametry
\end{labeling}

Podczas odczytu wartości adres \texttt{\%3 static} powoduje ,,dostęp do wartości w 3. statycznym rejestrze'';
\texttt{*4 local} powoduje ,,dostęp do wartości, na którą wskazuje wskaźnik w 4. lokalnym rejestrze''.

Wartości w rejestrach z zestawów \texttt{parameters} i \texttt{arguments} nie mogą być modyfikowane
bezpośrednio. Muszą być najpierw przeniesione lub skopiowane do rejestrów lokalnych lub statycznych.

\subsection{Definicje funkcji i bloków}

\begin{lstlisting}
.function: <name>/<arity>
	<instruction>
	[<instruction>...]
.end
\end{lstlisting}

\begin{lstlisting}
.block: <name>
	<instruction>
	[<instruction>...]
.end
\end{lstlisting}

\subsection{Deklaracje funkcji}

\begin{lstlisting}
.signature: <name>/<arity>
\end{lstlisting}

\subsection{Import modułów}

\begin{lstlisting}
.import: <module>[::<module>...]
\end{lstlisting}

\subsection{Markery}

\begin{lstlisting}
.mark: <marker-name>
\end{lstlisting}

\subsection{Nazywanie rejestrów}

\begin{lstlisting}
.name: <index> <name>
\end{lstlisting}

\section{Instrukcje Viua VM}

\subsection{\texttt{nop}}

\begin{lstlisting}
nop
\end{lstlisting}

\paragraph*{Opis} Instrukcja ,,pusta''.

\paragraph*{Wyjątki} Brak.

\paragraph*{Uwagi} Brak.

\subsection{\texttt{izero}}

\begin{lstlisting}
izero Rd
\end{lstlisting}

\paragraph*{Opis} Konstruuje w rejestrze \texttt{Rd} liczbę całkowitą o wartości 0.
Zwyczajowo wykorzystywana na końcu funkcji \texttt{main} do utworzenia domyślnej wartości zwracanej:

\begin{lstlisting}
	izero %0 local
	return
.end
\end{lstlisting}

\subsection{\texttt{integer}}

\begin{lstlisting}
integer Rd %*\emph{<integer>}*)
\end{lstlisting}

\paragraph*{Opis} Konstruuje w rejestrze \texttt{Rd} liczbę całkowitą o wartości \emph{\texttt{integer}}.

\subsection{\texttt{iinc}}
\subsection{\texttt{idec}}

\subsection{\texttt{float}}

\begin{lstlisting}
float Rd %*\emph{<float>}*)
\end{lstlisting}

\paragraph*{Opis} Konstruuje w rejestrze \texttt{Rd} liczbę zmiennoprzecinkową o wartości
\emph{\texttt{float}}.

\subsection{\texttt{itof}}
\subsection{\texttt{ftoi}}
\subsection{\texttt{stoi}}
\subsection{\texttt{stof}}

\subsection{\texttt{add}}
\subsection{\texttt{sub}}
\subsection{\texttt{mul}}
\subsection{\texttt{div}}
\subsection{\texttt{lt}}
\subsection{\texttt{lte}}
\subsection{\texttt{gt}}
\subsection{\texttt{gte}}
\subsection{\texttt{eq}}

\subsection{\texttt{string}}

\begin{lstlisting}
string Rd %*\emph{<str>}*)
\end{lstlisting}

\paragraph*{Opis} Konstruuje w rejestrze \texttt{Rd} ciąg bajtów o wartości \emph{\texttt{str}}.

\subsection{\texttt{streq}}

\subsection{\texttt{text}}

\begin{lstlisting}
text Rd %*\emph{<txt>}*)
text Rd Rs
\end{lstlisting}

\paragraph*{Opis} Wariant pierwszy konstruuje w rejestrze \texttt{Rd} tekst o wartości \emph{\texttt{txt}}.
Wariant drugi konwertuje wartość z rejestru \texttt{Rs} na tekst i umieszcza go w rejestrze \texttt{Rd}.

\subsection{\texttt{texteq}}
\subsection{\texttt{textat}}
\subsection{\texttt{textsub}}
\subsection{\texttt{textlength}}
\subsection{\texttt{textcommonprefix}}
\subsection{\texttt{textcommonsuffix}}
\subsection{\texttt{textconcat}}

\subsection{\texttt{vector}}
\subsection{\texttt{vinsert}}
\subsection{\texttt{vpush}}
\subsection{\texttt{vpop}}
\subsection{\texttt{vat}}
\subsection{\texttt{vlen}}

\subsection{\texttt{not}}

Negacja boolowska.

\subsection{\texttt{and}}

Iloczyn boolowski.

\subsection{\texttt{or}}

Suma boolowska.

\subsection{\texttt{bits\_of\_integer}}

Konstruktor bitów z liczby całkowitej.

\subsection{\texttt{integer\_of\_bits}}

Konstruktor liczby całkowitej z bitów.

\subsection{\texttt{bits}}

Konstruktor bitów.

\subsection{\texttt{bitand}}

Bitowa operacja \emph{\texttt{and}}.

\subsection{\texttt{bitor}}

Bitowa operacja \emph{\texttt{or}}.

\subsection{\texttt{bitnot}}

Bitowa operacja \emph{\texttt{not}}.

\subsection{\texttt{bitxor}}

Bitowa operacja \emph{\texttt{xor}}.

\subsection{\texttt{bitat}}

Sprawdzenie wartości pojedynczego bitu.

\subsection{\texttt{bitset}}

Ustawienie wartości pojedynczego bitu.

\subsection{\texttt{shl}}

Przesunięcie bitowe w lewo.

\subsection{\texttt{shr}}

Przesunięcie bitowe w prawo.

\subsection{\texttt{ashl}}

Arytmetyczne (z zachowaniem znaku) przesunięcie bitowe w lewo.

\subsection{\texttt{ashr}}

Arytmetyczne przesunięcie bitowe w prawo.

\subsection{\texttt{rol}}

Rotacja bitowa w lewo.

\subsection{\texttt{ror}}

Rotacja bitowa w prawo.

\subsection{\texttt{wrapincrement}}

Modulo inkrementacja ze znakiem.

\subsection{\texttt{wrapdecrement}}

Modulo dekrementacja ze znakiem.

\subsection{\texttt{wrapadd}}

Modulo dodawanie ze znakiem.

\subsection{\texttt{wrapsub}}

Modulo odejmowanie ze znakiem.

\subsection{\texttt{wrapmul}}

Modulo mnożenie ze znakiem.

\subsection{\texttt{wrapdiv}}

Modulo dzielenie ze znakiem.

\subsection{\texttt{checkedsincrement}}

Sprawdzana inkrementacja ze znakiem.

\subsection{\texttt{checkedsdecrement}}
\subsection{\texttt{checkedsadd}}
\subsection{\texttt{checkedssub}}
\subsection{\texttt{checkedsmul}}
\subsection{\texttt{checkedsdiv}}

\subsection{\texttt{saturatingsincrement}}

Nasycająca inkrementacja ze znakiem.

\subsection{\texttt{saturatingsdecrement}}
\subsection{\texttt{saturatingsadd}}
\subsection{\texttt{saturatingssub}}
\subsection{\texttt{saturatingsmul}}
\subsection{\texttt{saturatingsdiv}}

\subsection{\texttt{move}}

Przesunięcie wartości między rejestrami.

\subsection{\texttt{copy}}

Skopiowanie wartości między rejestrami.

\subsection{\texttt{ptr}}

Konstruktor wskaźnika do wartości.

\subsection{\texttt{ptrlive}}

Sprawdzenie poprawności wskaźnika.

\subsection{\texttt{swap}}

Zamiana wartości w rejestrach.

\subsection{\texttt{delete}}

Wywołanie destruktora wartości.

\subsection{\texttt{isnull}}

\subsection{\texttt{print}}
\subsection{\texttt{echo}}

\subsection{\texttt{capture}}
\subsection{\texttt{capturecopy}}
\subsection{\texttt{capturemove}}

\subsection{\texttt{closure}}
\subsection{\texttt{function}}

\subsection{\texttt{frame}}
\subsection{\texttt{param}}

Instrukcja prywatna.

\subsection{\texttt{pamv}}

Instrukcja prywatna.

\subsection{\texttt{call}}
\subsection{\texttt{tailcall}}
\subsection{\texttt{defer}}
\subsection{\texttt{arg}}

Instrukcja prywatna.

\subsection{\texttt{allocate\_registers}}

\subsection{\texttt{process}}
\subsection{\texttt{self}}
\subsection{\texttt{pideq}}
\subsection{\texttt{join}}
\subsection{\texttt{send}}
\subsection{\texttt{receive}}

\subsection{\texttt{watchdog}}

\subsection{\texttt{jump}}
\subsection{\texttt{if}}

\subsection{\texttt{throw}}

Rzucenie wartości jako wyjątku.

\subsection{\texttt{catch}}
\subsection{\texttt{draw}}
\subsection{\texttt{try}}
\subsection{\texttt{enter}}
\subsection{\texttt{leave}}

\subsection{\texttt{import}}

Import modułu.

\subsection{\texttt{atom}}

Konstruktor atomu.

\subsection{\texttt{atomeq}}

\subsection{\texttt{struct}}

Konstruktor struktury.

\subsection{\texttt{structinsert}}
\subsection{\texttt{structremove}}
\subsection{\texttt{structat}}
\subsection{\texttt{structkeys}}

\subsection{\texttt{return}}

Wyjście z funkcji.

\subsection{\texttt{halt}}

Zakończenie działania i wyłączenie VM.
