\chapter{Język assemblera Viua VM}
\label{appendix_viua_vm_assembly_language}

W tym rozdziale zostanie omówiony język assemblera Viua VM -- jego składnia,
dostępne dyrektywy, oraz instrukcje. Wyjaśnione zostaną pojęcia takie jak
,,adres~rejestru'' i ,,tryby dostępu''. Zostaną przedstawione istniejące i
możliwe do wykorzystania zestawy rejestrów.

W dalszej części rozdziału (na stronie \pageref{appendix_viua_vm_assembly_language_ops})
przedstawiona zostanie dokumentacja zestawu instrukcji jakie oferuje
programistom Viua~VM.

\section{Składnia języka assemblera}
\label{appendix_viua_vm_assembly_language_syntax}

Składnia języka assemblera Viua VM jest prosta. Wyglądem przypomina składnię innych języków tego rodzaju
(np. języki assemblera x86 czy ARM).

\subsection{Ogólna składnia instrukcji}
\label{appendix_viua_vm_assembly_language_syntax_generic}

Instrukcje składają się z mnemoniki, zera lub więcej adresów rejestrów, i co najwyżej jednego literału. Ogólną składnię można zapisać więc tak:

\begin{lstlisting}
mnemonic [<register>...] [<literal>]
\end{lstlisting}

Zanim pokazanych zostanie kilka przykładów ilustrujących różne warianty składni na konkretnych instrukcjach
należy zdefiniować czym jest ,,adres rejestru'' (lub, skrótowo, ,,rejestr''), a czym jest ,,literał''.

\begin{labeling}{\textbf{adres rejestru}}
	\item[\textbf{adres rejestru}] identyfikator informujący kernel VM skąd ma pobrać wartości do
		przetworzenia i gdzie zapisać wyniki działania instrukcji, w specjalnym przypadku adresem rejestru
		jest tzw. ,,\emph{rejestr pusty}'' -- \texttt{void}
	\item[\textbf{literał}] reprezentacja wartości wpisana dosłownie w kod źródłowy lub binarny,
		np. \texttt{0xdeadbeef}, \texttt{"Hello World!"}, \texttt{42}; do literałów zaliczane są też nazwy
		funkcji, bloków i modułów
\end{labeling}

\subsubsection{Adresy rejestrów}
\label{appendix_viua_vm_assembly_language_register_addrs}

\begin{lstlisting}
<access-operator> <index> <register-set>
void
\end{lstlisting}

\begin{labeling}{\texttt{access-operator}}
\item[\texttt{access-operator}] \texttt{\%} -- ,,dostęp bezpośredni'', lub
	\texttt{*} -- ,,dereferencja wskaźnika''
\item[\texttt{index}] indeks rejestru wewnątrz zestawu rejestrów
\item[\texttt{register-set}] nazwa zestawu rejestrów
\end{labeling}

Dostępne zestawy rejestrów ogólnego przeznaczenia to:

\begin{labeling}{\texttt{parameters}}
\item[\texttt{local}] zawiera wartości lokalne (,,zmienne lokalne'')
\item[\texttt{static}] zawiera wartości statyczne (,,zmienne statyczne'')
\item[\texttt{arguments}] zawiera wartości przekazane do aktywnej funkcji jako argumenty
\item[\texttt{parameters}] zawiera wartości przekazane do przygotowywanej ramki wywołania jako parametry
\end{labeling}

Pseudo-zestawy rejestrów specjalnego przeznaczenia to:

\begin{labeling}{\emph{exception register}}
\item[\emph{exception register}] rejestr przechowujący aktywny wyjątek; zapis do
    tego rejestru odbywa się za pomocą instrukcji \texttt{throw}, a odczyt z
    niego za pomocą instrukcji \texttt{draw}
\item[\emph{message queue}] kolejka wiadomości procesu; zapis do niej odbywa się
    za pomocą instrukcji \texttt{send}, a odczyt z niej za pomocą instrukcji
    \texttt{receive} (jest to jedyny ,,zestaw rejestrów'', który może być
    modyfikowany przez procesy nie będące jego ,,właścicielami'')
\end{labeling}

Podczas odczytu wartości adres \texttt{\%3 static} powoduje ,,dostęp do wartości w 3. statycznym rejestrze'';
\texttt{*4 local} powoduje ,,dostęp do wartości, na którą wskazuje wskaźnik w 4. lokalnym rejestrze''.

Wartości w rejestrach z zestawów \texttt{parameters} i \texttt{arguments} nie mogą być modyfikowane
bezpośrednio. Muszą być najpierw przeniesione lub skopiowane do rejestrów lokalnych lub statycznych.

\subsection{Definicje funkcji, domknięć i bloków}

\subsubsection{Funkcje}

Pierwszą instrukcją w funkcji zawsze \emph{musi} być \texttt{allocate\_registers}.
\begin{lstlisting}
.function: <name>/<arity>
    allocate_registers
	<instruction>
	[<instruction>...]
.end
\end{lstlisting}

Przykład:
\begin{lstlisting}
.function: main/0
    allocate_registers %2 local

    text %1 local "Hello, World!"
    print %1 local

    izero %0 local
    return
.end
\end{lstlisting}

\subsubsection{Domknięcia}

W przeciwieństwie do funkcji, pierwszą instrukcją w definicji domknięcia
\emph{nie może} być instrukcja \texttt{allocate\_registers}. Zestaw lokalnych
rejestrów dla domknięcia jest tworzony przez instrukcję \texttt{closure} w
momencie konstrukcji instacji domknięcia.
\begin{lstlisting}
.closure: <name>/<arity>
	<instruction>
	[<instruction>...]
.end
\end{lstlisting}

Przykład:
\begin{lstlisting}
.closure: print_whatever_was_captured/0
    print %1 local
    return
.end
\end{lstlisting}

\subsubsection{Bloki}

\begin{lstlisting}
.block: <name>
	<instruction>
	[<instruction>...]
.end
\end{lstlisting}

Przykład:
\begin{lstlisting}
.block: handle_an_exception
    draw void
    text %1 local "An error occured."
    print %1 local
    leave
.end
\end{lstlisting}

\subsection{Deklaracje funkcji}

\begin{lstlisting}
.signature: <name>/<arity>
\end{lstlisting}

\subsection{Import modułów}
\label{appendix_viua_vm_assembly_language_dir_import}

\begin{lstlisting}
.import: <module>[::<module>...]
\end{lstlisting}

\subsection{Markery}

\begin{lstlisting}
.mark: <marker-name>
\end{lstlisting}

\subsection{Nazywanie rejestrów}

\begin{lstlisting}
.name: <index> <name>
\end{lstlisting}

\newpage
\section{Instrukcje Viua VM}
\label{appendix_viua_vm_assembly_language_ops}

\subsection{Zarządzanie wartościami}
\label{viua_vm_ops_value_management}

\subsection{\texttt{move}}

Przesunięcie wartości między rejestrami.

\subsection{\texttt{copy}}

Skopiowanie wartości między rejestrami.

\subsection{\texttt{ptr}}

Konstruktor wskaźnika do wartości.

\subsection{\texttt{ptrlive}}

Sprawdzenie poprawności wskaźnika.

\subsection{\texttt{swap}}

Zamiana wartości w rejestrach.

\subsection{\texttt{delete}}

Wywołanie destruktora wartości.

\subsection{\texttt{isnull}}

\subsection{Operacje logiczne}
\label{viua_vm_ops_boolean_logic}

\subsection{\texttt{not}}

Negacja boolowska.

\subsection{\texttt{and}}

Iloczyn boolowski.

\subsection{\texttt{or}}

Suma boolowska.

\subsection{Operacje bitowe}
\label{viua_vm_ops_bit_ops}

\subsubsection{\texttt{bits\_of\_integer}}

\begin{lstlisting}
bits_of_integer Rd Rs
\end{lstlisting}

\paragraph*{Opis} Konstruktor bitów z liczby całkowitej.

Dokonuje konwersji liczby całkowitej znajdującej się w rejestrze \texttt{Rs} na
ciąg bitów o długości 64. Wynik konwersji jest zapisywany w rejestrze
\texttt{Rt}.

Wartość w rejestrze \texttt{Rs} pozostaje niezmieniona.

\begin{lstlisting}
integer %1 local 42
bits_of_integer %1 local %1 local
print %1 local
\end{lstlisting}

Powyższy przykład spowoduje wydrukowanie na ekran następującego ciągu bitów:
\newline
\texttt{0000000000000000000000000000000000000000000000000000000000101010}.

\subsubsection{\texttt{integer\_of\_bits}}

\begin{lstlisting}
integer_of_bits Rd Rs
\end{lstlisting}

\paragraph*{Opis} Konstruktor liczby całkowitej z bitów.

Dokonuje konwersji ciągu bitów znajdującej się w rejestrze \texttt{Rs} na liczbę
całkowitą. Wynik konwersji jest zapisywany w rejestrze \texttt{Rt}.

Wartość w rejestrze \texttt{Rs} pozostaje niezmieniona.

\begin{lstlisting}
bits %1 local 0b0101010
integer_of_bits %1 local %1 local
print %1 local
\end{lstlisting}

Powyższy przykład spowoduje wydrukowanie na ekran liczby 42.

\paragraph*{Uwagi}

Konwersja ciągów dłuższych niż 64 bity powinna spowodować błąd. Jednak nie jest
to weryfikowane i obecna wersja maszyny ,,po cichu'' powoduje zawinięcie liczby.

\subsubsection{\texttt{bits}}

\begin{lstlisting}
bits Rd 0x%*\emph{<hex-digits>}*)
bits Rd 0o%*\emph{<octal-digits>}*)
bits Rd 0b%*\emph{<binary-digits>}*)
bits Rd Rs
\end{lstlisting}

\paragraph*{Opis} Konstruktor bitów.

W wariantach 1-3 konstruuje ciąg bitów z literału podanego jako drugi operand.
W wariancie 4 konstruuje ciąg bitów o długości podanej w rejestrze \texttt{Rs}.
We wszystkich wariantach wynikowy ciąg bitów zapisywany jest w rejestrze
\texttt{Rd}.

\begin{lstlisting}
bits %1 local 0xdeadbeef
bits %2 local 0o1337
bits %3 local 0b01

integer %4 local 8
bits %4 local %4 local
\end{lstlisting}

\subsubsection{\texttt{bitand}}

\begin{lstlisting}
bitand Rd Ra Rb
\end{lstlisting}

\paragraph*{Opis} Bitowa operacja \emph{\texttt{and}}.
Długość wynikowego ciągu bitów jest równa długości ciągu, który w znadował się w
rejestrze \texttt{Ra}.

\subsubsection{\texttt{bitor}}

\begin{lstlisting}
bitor Rd Ra Rb
\end{lstlisting}

\paragraph*{Opis} Bitowa operacja \emph{\texttt{or}}.
Długość wynikowego ciągu bitów jest równa długości ciągu, który w znadował się w
rejestrze \texttt{Ra}.

\subsubsection{\texttt{bitnot}}

\begin{lstlisting}
bitxor Rd Rs
\end{lstlisting}

\paragraph*{Opis} Bitowa operacja \emph{\texttt{not}}.
Konstruuje nowy ciąg bitów będący negacją ciągu bitów znajdującego się w
rejestrze \texttt{Rs}.

\subsubsection{\texttt{bitxor}}

\begin{lstlisting}
bitxor Rd Ra Rb
\end{lstlisting}

\paragraph*{Opis} Bitowa operacja \emph{\texttt{xor}}.
Długość wynikowego ciągu bitów jest równa długości ciągu, który w znadował się w
rejestrze \texttt{Ra}.

\subsubsection{\texttt{bitat}}

\begin{lstlisting}
bitat Rd Rs Ri
\end{lstlisting}

\paragraph*{Opis} Sprawdzenie wartości pojedynczego bitu.
Po wykonaniu instrukcji w rejestrze \texttt{Rd} znajduje się wartość boolowska
określająca czy w ciągu bitów znajdującym się w rejestrze \texttt{Rs} bit pod
indeksem określonym przez liczbę całkowitą znajdującą się w rejestrze
\texttt{Ri} jest włączony czy wyłączony.

\begin{lstlisting}
integer %1 local 0
bitat %3 local %2 local %1 local
print %3 local

integer %1 local 1
bitat %3 local %2 local %1 local
print %3 local
\end{lstlisting}
Powyższy przykład wydrukuje na ekran najpierw \texttt{true}, a potem
\texttt{false}.

\subsubsection{\texttt{bitset}}

\begin{lstlisting}
bitset Rt Ri %*\emph{<boolean-literal>}*)
\end{lstlisting}

\paragraph*{Opis} Ustawienie wartości pojedynczego bitu.
Po wykonaniu instrukcji w ciągu bitów znajdującym się w rejestrze \texttt{Rt}
bit znajdujący się pod indeksem określonym przez liczbę całkowitą znajdującą
się w rejestrze \texttt{Ri} jest
\begin{enumerate*}[label=(\arabic*)]
\item włączony jeśli \texttt{\emph{<boolean-literal>}} był wartością
    \texttt{true}
\item wyłączony jeśli \texttt{\emph{<boolean-literal>}} był wartością
    \texttt{fale}
\end{enumerate*}

\subsubsection{\texttt{shl}}

\begin{lstlisting}
shl Ro Rt Rn
\end{lstlisting}

\paragraph*{Opis} Przesunięcie bitowe w lewo.

Przesuwa ciąg bitów w rejestrze \texttt{Rt} o ilość bitów określoną przez liczbę
całkowitą znajdującą się w rejestrze \texttt{Rn}. Wartość w rejestrze
\texttt{Rt} jest modyfikowana \emph{w miejscu}, a w rejestrze \texttt{Ro}
umieszczany jest ,,wysunięty'' ciąg bitów.
\begin{lstlisting}
bits %3 local 0x4a
print %3 local
integer %4 local 4
shl %4 local %3 local %4 local
print %3 local
print %4 local
\end{lstlisting}

Powyższy przykład spowoduje wydrukowanie na ekranie następujących wartości:
\begin{lstlisting}
01001010
10100000
0100
\end{lstlisting}

\subsubsection{\texttt{shr}}

\begin{lstlisting}
shr Rd Rs Rn
\end{lstlisting}

\paragraph*{Opis} Przesunięcie bitowe w prawo.

Przesuwa ciąg bitów w rejestrze \texttt{Rt} o ilość bitów określoną przez liczbę
całkowitą znajdującą się w rejestrze \texttt{Rn}. Wartość w rejestrze
\texttt{Rt} jest modyfikowana \emph{w miejscu}, a w rejestrze \texttt{Ro}
umieszczany jest ,,wysunięty'' ciąg bitów.
\begin{lstlisting}
bits %3 local 0x4a
print %3 local
integer %4 local 4
shr %4 local %3 local %4 local
print %3 local
print %4 local
\end{lstlisting}

Powyższy przykład spowoduje wydrukowanie na ekranie następujących wartości:
\begin{lstlisting}
01001010
00000100
1010
\end{lstlisting}

\subsubsection{\texttt{ashl}}

\begin{lstlisting}
ashl Rd Rs Rn
\end{lstlisting}

\paragraph*{Opis} Arytmetyczne (z zachowaniem znaku) przesunięcie bitowe w lewo.

Przesuwa ciąg bitów w rejestrze \texttt{Rt} o ilość bitów określoną przez liczbę
całkowitą znajdującą się w rejestrze \texttt{Rn}. Wartość w rejestrze
\texttt{Rt} jest modyfikowana \emph{w miejscu}, a w rejestrze \texttt{Ro}
umieszczany jest ,,wysunięty'' ciąg bitów.
\begin{lstlisting}
bits %3 local 0xa4
print %3 local
integer %4 local 4
ashl %4 local %3 local %4 local
print %3 local
print %4 local
\end{lstlisting}

Powyższy przykład spowoduje wydrukowanie na ekranie następujących wartości:
\begin{lstlisting}
10100100
11000000
1010
\end{lstlisting}

\subsubsection{\texttt{ashr}}

\begin{lstlisting}
ashr Rd Rs Rn
\end{lstlisting}

\paragraph*{Opis} Arytmetyczne przesunięcie bitowe w prawo.

Przesuwa ciąg bitów w rejestrze \texttt{Rt} o ilość bitów określoną przez liczbę
całkowitą znajdującą się w rejestrze \texttt{Rn}. Wartość w rejestrze
\texttt{Rt} jest modyfikowana \emph{w miejscu}, a w rejestrze \texttt{Ro}
umieszczany jest ,,wysunięty'' ciąg bitów.
\begin{lstlisting}
bits %3 local 0xa4
print %3 local
integer %4 local 4
ashr %4 local %3 local %4 local
print %3 local
print %4 local
\end{lstlisting}

Powyższy przykład spowoduje wydrukowanie na ekranie następujących wartości:
\begin{lstlisting}
10100100
11111010
0100
\end{lstlisting}

\subsubsection{\texttt{rol}}

\begin{lstlisting}
rol Rt Rn
\end{lstlisting}

\paragraph*{Opis} Rotacja bitowa w lewo.

,,Obraca'' ciąg bitów w rejestrze \texttt{Rt} o ilość bitów określoną przez
liczbę całkowitą znajdującą się w rejestrze \texttt{Rn}. Wartość w rejestrze
\texttt{Rt} jest modyfikowana \emph{w miejscu}.
\begin{lstlisting}
bits %3 local 0x8e
integer %4 local 3
print %3 local
rol %3 local %4 local
print %3 local
\end{lstlisting}

Powyższy przykład spowoduje wydrukowanie na ekranie następujących wartości:
\begin{lstlisting}
10001110
01110100
\end{lstlisting}

\subsubsection{\texttt{ror}}

\begin{lstlisting}
ror Rt Rn
\end{lstlisting}

\paragraph*{Opis} Rotacja bitowa w prawo.

,,Obraca'' ciąg bitów w rejestrze \texttt{Rt} o ilość bitów określoną przez
liczbę całkowitą znajdującą się w rejestrze \texttt{Rn}. Wartość w rejestrze
\texttt{Rt} jest modyfikowana \emph{w miejscu}.
\begin{lstlisting}
bits %3 local 0x8e
integer %4 local 3
print %3 local
ror %3 local %4 local
print %3 local
\end{lstlisting}

Powyższy przykład spowoduje wydrukowanie na ekranie następujących wartości:
\begin{lstlisting}
10001110
11010001
\end{lstlisting}

\subsection{Arytmetyka (CPU)}
\label{viua_vm_ops_arithmetic_cpu}

Arytmetyka implementowana przez fizyczne CPU, na którym działa Viua VM.
Specyfika tego fizycznego CPU wpływa na wyniki działania instrukcji z tej grupy.
Arytmetyka o zdefiniowanym zachowaniu niezależnym od fizycznej platformy jest
opisana w rozdziale \ref{viua_vm_ops_arithmetic_vm} na stronie
\pageref{viua_vm_ops_arithmetic_vm}. Arytmetyka oparta o CPU jest
bardziej wydajna (,,szybsza''), ale nie zawsze zapewnia przewidywalność,
stabliność, i weryfikację wyników operacji (wykrywanie błędów).

\subsubsection{\texttt{izero}}

\begin{lstlisting}
izero Rd
\end{lstlisting}

\paragraph*{Opis} Konstruuje w rejestrze \texttt{Rd} liczbę całkowitą o wartości 0.
Zwyczajowo wykorzystywana na końcu funkcji \texttt{main} do utworzenia domyślnej wartości zwracanej:

\begin{lstlisting}
	izero %0 local
	return
.end
\end{lstlisting}

\subsubsection{\texttt{integer}}

\begin{lstlisting}
integer Rd %*\emph{<integer>}*)
\end{lstlisting}

\paragraph*{Opis} Konstruuje w rejestrze \texttt{Rd} liczbę całkowitą o wartości \emph{\texttt{integer}}.

\subsubsection{\texttt{iinc}}
\subsubsection{\texttt{idec}}

\subsubsection{\texttt{float}}

\begin{lstlisting}
float Rd %*\emph{<float>}*)
\end{lstlisting}

\paragraph*{Opis} Konstruuje w rejestrze \texttt{Rd} liczbę zmiennoprzecinkową o wartości
\emph{\texttt{float}}.

\subsubsection{\texttt{itof}}
\subsubsection{\texttt{ftoi}}
\subsubsection{\texttt{stoi}}
\subsubsection{\texttt{stof}}

\subsubsection{\texttt{add}}
\subsubsection{\texttt{sub}}
\subsubsection{\texttt{mul}}
\subsubsection{\texttt{div}}
\subsubsection{\texttt{lt}}
\subsubsection{\texttt{lte}}
\subsubsection{\texttt{gt}}
\subsubsection{\texttt{gte}}
\subsubsection{\texttt{eq}}

\subsection{Arytmetyka (Viua VM)}
\label{viua_vm_ops_arithmetic_vm}

Arytmetyka implementowana przez Viua VM, niezależna od fizycznej platformy, na
której maszyna wirtualna jest uruchomiona. Arytmetyka o zachowaniu zależnym od
fizycznej platformy jest opisana w rozdziale \ref{viua_vm_ops_arithmetic_cpu} na
stronie \pageref{viua_vm_ops_arithmetic_cpu}. Arytmetyka o zachowaniu
niezależnym od fizycznej platformy jest mniej wydajna (,,wolniesza''), ale
zapewnia przewidywalność, stabliność, i weryfikację wyników operacji (wykrywanie
błędów).

Operandami w instrukcjach arytmetycznych implementowanych przez VM są ciągi
bitów, których konstruktory są opisane w rozdziale \ref{viua_vm_ops_bit_ops} na
stronie \pageref{viua_vm_ops_bit_ops}.

\subsubsection{\texttt{wrapincrement}}

\begin{lstlisting}
wrapincrement Rt
\end{lstlisting}

\paragraph*{Opis} Inkrementacja modulo (ze znakiem).

Instrukcja interpretuje ciąg bitów jako liczbę całkowitą w kodowaniu z
dopełnieniem do dwóch\footnote{Kodowanie z dopełnieniem do dwóch
(ang. \emph{two's complement})} o szerokości równej długości ciągu bitów, na
którym instrukcja operuje.

\paragraph*{Uwagi}

W przypadku wystąpienia przepełnienia największa wartość dodatnia staje się
największą (tj. najbardziej oddaloną od zera) wartością ujemną.
\begin{lstlisting}
bits %1 local 0b01111111
print %1 local
wrapincrement %1 local
print %1 local
\end{lstlisting}
Ma to zastosowanie jedynie jeśli ciągi bitów są interpretowane przy konwersji
jako liczby ze znakiem.

\subsubsection{\texttt{wrapdecrement}}

\begin{lstlisting}
wrapdecrement Rt
\end{lstlisting}

\paragraph*{Opis} Dekrementacja modulo (ze znakiem).

Instrukcja interpretuje ciąg bitów jako liczbę całkowitą w kodowaniu z
dopełnieniem do dwóch o szerokości równej długości ciągu bitów, na którym
instrukcja operuje.

\paragraph*{Uwagi}

W przypadku wystąpienia przepełnienia największa (tj. najbardziej oddalona od
zera) wartość ujemna staje się największą wartością dodatnią.
\begin{lstlisting}
bits %1 local 0x80
print %1 local
wrapdecrement %1 local
print %1 local
\end{lstlisting}
Ma to zastosowanie jedynie jeśli ciągi bitów są interpretowane przy konwersji
jako liczby ze znakiem.

\subsubsection{\texttt{wrapadd}}

\begin{lstlisting}
wrapadd Rd Ra Rb
\end{lstlisting}

\paragraph*{Opis} Dodawanie modulo.

Instrukcja interpretuje liczby w rejestrach \texttt{Ra} i \texttt{Rb} jaki
liczby całkowite ze znakiem w kodowaniu z dopełnieniem do dwóch. Wynikiem
instrukcji jest ciąg bitów reprezentujący liczbe całkowitą ze znakiem w
kodowaniu z dopełnieniem do dwóch, o długości odpowiadającej długości ciągu
bitów znajdującemu się w rejestrze \texttt{Ra}.

Operacja jest wykonywana według wzoru $a + b$.

\paragraph*{Uwagi}

W przypadku przepełnienia wartość jest ,,zawijana'' i wynik z dodatniego może
stać się ujemy i \emph{vice versa}.

\subsubsection{\texttt{wrapsub}}

\begin{lstlisting}
wrapsub Rd Ra Rb
\end{lstlisting}

\paragraph*{Opis} Odejmowanie modulo.

Instrukcja interpretuje liczby w rejestrach \texttt{Ra} i \texttt{Rb} jaki
liczby całkowite ze znakiem w kodowaniu z dopełnieniem do dwóch. Wynikiem
instrukcji jest ciąg bitów reprezentujący liczbe całkowitą ze znakiem w
kodowaniu z dopełnieniem do dwóch, o długości odpowiadającej długości ciągu
bitów znajdującemu się w rejestrze \texttt{Ra}.

Operacja jest wykonywana według wzoru $a - b$.

\paragraph*{Uwagi}

W przypadku przepełnienia wartość jest ,,zawijana'' i wynik z dodatniego może
stać się ujemy i \emph{vice versa}.

\subsubsection{\texttt{wrapmul}}

\begin{lstlisting}
wrapmul Rd Ra Rb
\end{lstlisting}

\paragraph*{Opis} Mnożenie modulo.

Instrukcja interpretuje liczby w rejestrach \texttt{Ra} i \texttt{Rb} jaki
liczby całkowite ze znakiem w kodowaniu z dopełnieniem do dwóch. Wynikiem
instrukcji jest ciąg bitów reprezentujący liczbe całkowitą ze znakiem w
kodowaniu z dopełnieniem do dwóch, o długości odpowiadającej długości ciągu
bitów znajdującemu się w rejestrze \texttt{Ra}.

Operacja jest wykonywana według wzoru $a * b$.

\paragraph*{Uwagi}

W przypadku przepełnienia wartość jest ,,zawijana'' i wynik z dodatniego może
stać się ujemy i \emph{vice versa}.

\subsubsection{\texttt{wrapdiv}}

\begin{lstlisting}
wrapdiv Rd Ra Rb
\end{lstlisting}

\paragraph*{Opis} Dzielenie modulo.

Instrukcja interpretuje liczby w rejestrach \texttt{Ra} i \texttt{Rb} jaki
liczby całkowite ze znakiem w kodowaniu z dopełnieniem do dwóch. Wynikiem
instrukcji jest ciąg bitów reprezentujący liczbe całkowitą ze znakiem w
kodowaniu z dopełnieniem do dwóch, o długości odpowiadającej długości ciągu
bitów znajdującemu się w rejestrze \texttt{Ra}.

Operacja jest wykonywana według wzoru $a / b$.

\paragraph*{Uwagi}

W przypadku przepełnienia wartość jest ,,zawijana'' i wynik z dodatniego może
stać się ujemy i \emph{vice versa}.

\subsubsection{\texttt{checkedsincrement}}

\begin{lstlisting}
checkedsincrement Rt
\end{lstlisting}

\paragraph*{Opis} Inkrementacja weryfikowana ze znakiem.

Instrukcja interpretuje ciąg bitów jako liczbę całkowitą w kodowaniu z
dopełnieniem do dwóch o szerokości równej długości ciągu bitów, na którym
instrukcja operuje.

\paragraph*{Uwagi}

W przypadku przepełnienia zgłaszany jest wyjątek.

\subsubsection{\texttt{checkedsdecrement}}

\begin{lstlisting}
checkedsdecrement Rt
\end{lstlisting}

\paragraph*{Opis} Inkrementacja weryfikowana ze znakiem.

Instrukcja interpretuje ciąg bitów jako liczbę całkowitą w kodowaniu z
dopełnieniem do dwóch o szerokości równej długości ciągu bitów, na którym
instrukcja operuje.

\paragraph*{Uwagi}

W przypadku przepełnienia zgłaszany jest wyjątek.

\subsubsection{\texttt{checkedsadd}}

\begin{lstlisting}
checkedsadd Rd Ra Rb
\end{lstlisting}

\paragraph*{Opis} Dodawanie weryfikowane ze znakiem.

Instrukcja interpretuje liczby w rejestrach \texttt{Ra} i \texttt{Rb} jaki
liczby całkowite ze znakiem w kodowaniu z dopełnieniem do dwóch. Wynikiem
instrukcji jest ciąg bitów reprezentujący liczbe całkowitą ze znakiem w
kodowaniu z dopełnieniem do dwóch, o długości odpowiadającej długości ciągu
bitów znajdującemu się w rejestrze \texttt{Ra}.

Operacja jest wykonywana według wzoru $a + b$.

\paragraph*{Uwagi}

W przypadku przepełnienia zgłaszany jest wyjątek.

\subsubsection{\texttt{checkedssub}}

\begin{lstlisting}
checkedssub Rd Ra Rb
\end{lstlisting}

\paragraph*{Opis} Odejmowanie weryfikowane ze znakiem.

Instrukcja interpretuje liczby w rejestrach \texttt{Ra} i \texttt{Rb} jaki
liczby całkowite ze znakiem w kodowaniu z dopełnieniem do dwóch. Wynikiem
instrukcji jest ciąg bitów reprezentujący liczbe całkowitą ze znakiem w
kodowaniu z dopełnieniem do dwóch, o długości odpowiadającej długości ciągu
bitów znajdującemu się w rejestrze \texttt{Ra}.

Operacja jest wykonywana według wzoru $a - b$.

\paragraph*{Uwagi}

W przypadku przepełnienia zgłaszany jest wyjątek.

\subsubsection{\texttt{checkedsmul}}

\begin{lstlisting}
checkedsmul Rd Ra Rb
\end{lstlisting}

\paragraph*{Opis} Dzielenie weryfikowane ze znakiem.

Instrukcja interpretuje liczby w rejestrach \texttt{Ra} i \texttt{Rb} jaki
liczby całkowite ze znakiem w kodowaniu z dopełnieniem do dwóch. Wynikiem
instrukcji jest ciąg bitów reprezentujący liczbe całkowitą ze znakiem w
kodowaniu z dopełnieniem do dwóch, o długości odpowiadającej długości ciągu
bitów znajdującemu się w rejestrze \texttt{Ra}.

Operacja jest wykonywana według wzoru $a * b$.

\paragraph*{Uwagi}

W przypadku przepełnienia zgłaszany jest wyjątek.

\subsubsection{\texttt{checkedsdiv}}

\begin{lstlisting}
checkedsdiv Rd Ra Rb
\end{lstlisting}

\paragraph*{Opis} Dzielenie weryfikowane ze znakiem.

Instrukcja interpretuje liczby w rejestrach \texttt{Ra} i \texttt{Rb} jaki
liczby całkowite ze znakiem w kodowaniu z dopełnieniem do dwóch. Wynikiem
instrukcji jest ciąg bitów reprezentujący liczbe całkowitą ze znakiem w
kodowaniu z dopełnieniem do dwóch, o długości odpowiadającej długości ciągu
bitów znajdującemu się w rejestrze \texttt{Ra}.

Operacja jest wykonywana według wzoru $a / b$.

\paragraph*{Uwagi}

W przypadku przepełnienia zgłaszany jest wyjątek.

\subsubsection{\texttt{saturatingsincrement}}

\begin{lstlisting}
saturatingsincrement Rt
\end{lstlisting}

\paragraph*{Opis} Inkrementacja nasyceniowa ze znakiem.

Instrukcja interpretuje ciąg bitów jako liczbę całkowitą w kodowaniu z
dopełnieniem do dwóch o szerokości równej długości ciągu bitów, na którym
instrukcja operuje.

\paragraph*{Uwagi}

W przypadku wykrycia przepełnienia następuje nasycenie i wynik jest wartością
maksymalną dla liczby całkowitej ze znakiem w kodowaniu do dwóch o szerokości
równej długości ciągu bitów użytego jako operand instrukcji.

\subsubsection{\texttt{saturatingsdecrement}}

\begin{lstlisting}
saturatingsdecrement Rt
\end{lstlisting}

\paragraph*{Opis} Dekrementacja nasyceniowa ze znakiem.

Instrukcja interpretuje ciąg bitów jako liczbę całkowitą w kodowaniu z
dopełnieniem do dwóch o szerokości równej długości ciągu bitów, na którym
instrukcja operuje.

\paragraph*{Uwagi}

W przypadku wykrycia przepełnienia następuje nasycenie i wynik jest wartością
minimalną dla liczby całkowitej ze znakiem w kodowaniu do dwóch o szerokości
równej długości ciągu bitów użytego jako operand instrukcji.

\subsubsection{\texttt{saturatingsadd}}

\begin{lstlisting}
saturatingsadd Rd Ra Rb
\end{lstlisting}

\paragraph*{Opis} Dodawanie nasyceniowe ze znakiem.

Instrukcja interpretuje liczby w rejestrach \texttt{Ra} i \texttt{Rb} jaki
liczby całkowite ze znakiem w kodowaniu z dopełnieniem do dwóch. Wynikiem
instrukcji jest ciąg bitów reprezentujący liczbe całkowitą ze znakiem w
kodowaniu z dopełnieniem do dwóch, o długości odpowiadającej długości ciągu
bitów znajdującemu się w rejestrze \texttt{Ra}.

Operacja jest wykonywana według wzoru $a + b$.

\paragraph*{Uwagi}

W przypadku wykrycia przepełnienia następuje nasycenie i
\begin{enumerate}
\item jeśli wynik miał być dodatni to jest wartością maksymalną dla liczby
    całkowitej ze znakiem w kodowaniu do dwóch o szerokości równej długości
    ciągu bitów użytego jako operand instrukcji
\item jeśli wynik miał być ujemny to jest wartością minimalną dla liczby
    całkowitej ze znakiem w kodowaniu do dwóch o szerokości równej długości
    ciągu bitów użytego jako operand instrukcji
\end{enumerate}

\subsubsection{\texttt{saturatingssub}}

\begin{lstlisting}
saturatingssub Rd Ra Rb
\end{lstlisting}

\paragraph*{Opis} Odejmowanie nasyceniowe ze znakiem.

Instrukcja interpretuje liczby w rejestrach \texttt{Ra} i \texttt{Rb} jaki
liczby całkowite ze znakiem w kodowaniu z dopełnieniem do dwóch. Wynikiem
instrukcji jest ciąg bitów reprezentujący liczbe całkowitą ze znakiem w
kodowaniu z dopełnieniem do dwóch, o długości odpowiadającej długości ciągu
bitów znajdującemu się w rejestrze \texttt{Ra}.

Operacja jest wykonywana według wzoru $a - b$.

\paragraph*{Uwagi}

W przypadku wykrycia przepełnienia następuje nasycenie i
\begin{enumerate}
\item jeśli wynik miał być dodatni to jest wartością maksymalną dla liczby
    całkowitej ze znakiem w kodowaniu do dwóch o szerokości równej długości
    ciągu bitów użytego jako operand instrukcji
\item jeśli wynik miał być ujemny to jest wartością minimalną dla liczby
    całkowitej ze znakiem w kodowaniu do dwóch o szerokości równej długości
    ciągu bitów użytego jako operand instrukcji
\end{enumerate}

\subsubsection{\texttt{saturatingsmul}}

\begin{lstlisting}
saturatingsmul Rd Ra Rb
\end{lstlisting}

\paragraph*{Opis} Mnożenie nasyceniowe ze znakiem.

Instrukcja interpretuje liczby w rejestrach \texttt{Ra} i \texttt{Rb} jaki
liczby całkowite ze znakiem w kodowaniu z dopełnieniem do dwóch. Wynikiem
instrukcji jest ciąg bitów reprezentujący liczbe całkowitą ze znakiem w
kodowaniu z dopełnieniem do dwóch, o długości odpowiadającej długości ciągu
bitów znajdującemu się w rejestrze \texttt{Ra}.

Operacja jest wykonywana według wzoru $a * b$.

\paragraph*{Uwagi}

W przypadku wykrycia przepełnienia następuje nasycenie i
\begin{enumerate}
\item jeśli wynik miał być dodatni to jest wartością maksymalną dla liczby
    całkowitej ze znakiem w kodowaniu do dwóch o szerokości równej długości
    ciągu bitów użytego jako operand instrukcji
\item jeśli wynik miał być ujemny to jest wartością minimalną dla liczby
    całkowitej ze znakiem w kodowaniu do dwóch o szerokości równej długości
    ciągu bitów użytego jako operand instrukcji
\end{enumerate}

\subsubsection{\texttt{saturatingsdiv}}

\begin{lstlisting}
checkedsdiv Rd Ra Rb
\end{lstlisting}

\paragraph*{Opis} Dzielenie nasyceniowe ze znakiem.

Instrukcja interpretuje liczby w rejestrach \texttt{Ra} i \texttt{Rb} jaki
liczby całkowite ze znakiem w kodowaniu z dopełnieniem do dwóch. Wynikiem
instrukcji jest ciąg bitów reprezentujący liczbe całkowitą ze znakiem w
kodowaniu z dopełnieniem do dwóch, o długości odpowiadającej długości ciągu
bitów znajdującemu się w rejestrze \texttt{Ra}.

Operacja jest wykonywana według wzoru $a / b$.

\paragraph*{Uwagi}

W przypadku wykrycia przepełnienia następuje nasycenie i
\begin{enumerate}
\item jeśli wynik miał być dodatni to jest wartością maksymalną dla liczby
    całkowitej ze znakiem w kodowaniu do dwóch o szerokości równej długości
    ciągu bitów użytego jako operand instrukcji
\item jeśli wynik miał być ujemny to jest wartością minimalną dla liczby
    całkowitej ze znakiem w kodowaniu do dwóch o szerokości równej długości
    ciągu bitów użytego jako operand instrukcji
\end{enumerate}

\subsection{Obsługa tekstu}
\label{viua_vm_ops_text}

\subsubsection{\texttt{string}}

\begin{lstlisting}
string Rd %*\emph{<str>}*)
\end{lstlisting}

\paragraph*{Opis} Konstruuje w rejestrze \texttt{Rd} ciąg bajtów o wartości \emph{\texttt{str}}.

\subsubsection{\texttt{streq}}

\subsubsection{\texttt{text}}

\begin{lstlisting}
text Rd %*\emph{<txt>}*)
text Rd Rs
\end{lstlisting}

\paragraph*{Opis} Wariant pierwszy konstruuje w rejestrze \texttt{Rd} tekst o wartości \emph{\texttt{txt}}.
Wariant drugi konwertuje wartość z rejestru \texttt{Rs} na tekst i umieszcza go w rejestrze \texttt{Rd}.

\subsubsection{\texttt{texteq}}
\subsubsection{\texttt{textat}}
\subsubsection{\texttt{textsub}}
\subsubsection{\texttt{textlength}}
\subsubsection{\texttt{textcommonprefix}}
\subsubsection{\texttt{textcommonsuffix}}
\subsubsection{\texttt{textconcat}}

\subsection{Wektory}
\label{viua_vm_ops_vector}

\subsubsection{\texttt{vector}}

\begin{lstlisting}
vector Rd
vector Rd Rp %*\emph{n}*)
\end{lstlisting}

\paragraph*{Opis} Konstruuje wektor w rejestrze \texttt{Rd}.

Pierwszy wariant konstruuje pusty wektor.
Drugi wariant konstruuje wektor ,,pakując'' \emph{n} wartości rozpoczynając od
rejestru określonego przez \texttt{Rp} (wszystkie ,,spakowane'' wartości
pochodzą z tego samego zestawu rejestrów, określonego przez zestaw rejestrów
użyty w adresie rejestru \texttt{Rp}).

Przykład pakowania:
\begin{lstlisting}
integer %2 local 1
integer %3 local 2
integer %4 local 3
integer %5 local 4
vector %1 local %2 local %4
\end{lstlisting}

Wektor znajdujący się w rejestrze \texttt{\%1 local} spakuje wartości z
rejestrów lokalnych 2, 3, 4 i 5 (czyli będzie miał długość 4).

\paragraph*{Uwagi} Wektor nie może spakować:
\begin{enumerate}
    \item rejestru, w którym będzie skonstruowany
    \item rejestru, który nie zawiera żadnej wartości
    \item rejestru, którego indeks jest poza zakresem dla zestawu rejestrów, do
        którego odnosi się adres rejestru \texttt{Rp}
\end{enumerate}

\subsubsection{\texttt{vinsert}}

\begin{lstlisting}
vinsert Rd Rs void
vinsert Rd Rs Ri
\end{lstlisting}

\paragraph*{Opis} Przenosi wartość z rejestru \texttt{Rs} do wektora
znajdującego się w rejestrze \texttt{Rd}.

Pierwszy wariant przenosi wartość na początek wektora (na pozycję 0).
Drugi wariant przenosi wartość na pozycję określoną przez liczbę całkowitą
znajdującą się w rejestrze \texttt{Ri}.

\paragraph*{Wyjątki} Jeśli żądany indeks jest poza zakresem długości wektora
wygenerowany zostanie wyjątek.

\subsubsection{\texttt{vpush}}

\begin{lstlisting}
vpush Rd Rs
\end{lstlisting}

\paragraph*{Opis} Przenosi wartość z rejestru \texttt{Rs} na koniec wektora
znajdującego się w rejestrze \texttt{Rd}.

\subsubsection{\texttt{vpop}}

\begin{lstlisting}
vpop Rd Rs Ri
\end{lstlisting}

\paragraph*{Opis} Usuwa wartość znajdującą się pod indeksem określonym przez
liczbę całkowitą znajdującą się w rejestrze \texttt{Ri} w wektorze znajdującym
się w rejestrze \texttt{Rs} i zapisuje ją w rejestrze \texttt{Rd}.

Jeśli rejestr \texttt{Rd} jest podany jako \texttt{void} to wartość usunięta z
wektora nie jest zapisywana, a od razu niszczona.

Jeśli rejestr \texttt{Ri} jest podany jako \texttt{void} to wartość jest usuwana
z końca wektora.

Rejestr \texttt{Rd} i \texttt{Ri} mogą być podane jako \texttt{void}
jednocześnie (co spowoduje usunięcie ostatniej wartości w wektorze i jej
nastychmiastowe zniszczenie).

\paragraph*{Wyjątki} Jeśli żądany indeks jest poza zakresem długości wektora
wygenerowany zostanie wyjątek.

\subsubsection{\texttt{vat}}

\begin{lstlisting}
vat Rd Rs Ri
\end{lstlisting}

\paragraph*{Opis} Tworzy wskaźnik do wartości znajdującej się pod indeksem
określonym przez liczbę całkowitą znajdującą się w rejestrze \texttt{Ri} w
wektorze znajdującym się w rejestrze \texttt{Rs} i zapisuje go w rejestrze
\texttt{Rd}.

\paragraph*{Wyjątki} Jeśli żądany indeks jest poza zakresem długości wektora
wygenerowany zostanie wyjątek.

\subsubsection{\texttt{vlen}}

\begin{lstlisting}
vlen Rd Rs
\end{lstlisting}

\paragraph*{Opis} Odczytuje długość wektora znajdującego się w rejestrze
\texttt{Rd} i zapisuje ją jako liczbę całkowitą w rejestrze \texttt{Rd}.

\subsection{Atomy}
\label{viua_vm_ops_atom}

\subsection{\texttt{atom}}

Konstruktor atomu.

\subsection{\texttt{atomeq}}

\subsection{Struktury (rekordy)}
\label{viua_vm_ops_struct}

\subsection{\texttt{struct}}

Konstruktor struktury.

\subsection{\texttt{structinsert}}
\subsection{\texttt{structremove}}
\subsection{\texttt{structat}}
\subsection{\texttt{structkeys}}

\subsection{Aktory}
\label{viua_vm_ops_actor}

\subsubsection{\texttt{process}}
\subsubsection{\texttt{self}}
\subsubsection{\texttt{pideq}}
\subsubsection{\texttt{join}}
\subsubsection{\texttt{send}}
\subsubsection{\texttt{receive}}

\subsection{Programowanie funkcyjne}
\label{viua_vm_ops_functional}

\subsubsection{\texttt{function}}

\begin{lstlisting}
function Rd %*\emph{function-name}*)
\end{lstlisting}

\paragraph*{Opis}

\subsubsection{\texttt{closure}}

\begin{lstlisting}
closure Rd %*\emph{function-name}*)
\end{lstlisting}

\paragraph*{Opis}

\subsubsection{\texttt{capture}}

\begin{lstlisting}
capture Rc Ri Rv
\end{lstlisting}

\paragraph*{Opis}

\subsubsection{\texttt{capturecopy}}

\begin{lstlisting}
capturecopy Rc Ri Rv
\end{lstlisting}

\paragraph*{Opis}

\subsubsection{\texttt{capturemove}}

\begin{lstlisting}
capturemove Rc Ri Rv
\end{lstlisting}

\paragraph*{Opis}

\subsection{Kontrola przepływu}
\label{viua_vm_ops_control_flow}

\subsubsection{\texttt{nop}}

\begin{lstlisting}
nop
\end{lstlisting}

\paragraph*{Opis} Instrukcja ,,pusta''.

\paragraph*{Wyjątki} Brak.

\paragraph*{Uwagi} Brak.

\subsection{\texttt{jump}}
\subsection{\texttt{if}}

\subsection{\texttt{frame}}
\subsection{\texttt{param}}

Instrukcja prywatna.

\subsection{\texttt{pamv}}

Instrukcja prywatna.

\subsection{\texttt{call}}
\subsection{\texttt{tailcall}}
\subsection{\texttt{defer}}
\subsection{\texttt{arg}}

Instrukcja prywatna.

\subsection{\texttt{allocate\_registers}}

\subsection{\texttt{return}}

Wyjście z funkcji.

\subsection{\texttt{halt}}

Zakończenie działania i wyłączenie VM.

\subsection{Obsługa błędów}
\label{viua_vm_ops_error_handling}

\subsubsection{\texttt{watchdog}}

\subsubsection{\texttt{throw}}

Rzucenie wartości jako wyjątku.

\subsubsection{\texttt{catch}}
\subsubsection{\texttt{draw}}
\subsubsection{\texttt{try}}
\subsubsection{\texttt{enter}}
\subsubsection{\texttt{leave}}

\subsection{Systemu modułów}
\label{viua_vm_ops_module_system}

\subsubsection{\texttt{import}}

\begin{lstlisting}
import %*\emph{module-name}*)
\end{lstlisting}

\paragraph*{Opis} Instrukcja dynamicznie dołącza moduł
\texttt{\emph{module-name}} do uruchomionego programu.

\paragraph*{Uwagi}

Użycie instrukcji \texttt{import} jest niezalecane.
Zamiast niej należy używać dyrektywy import opisanej w rozdziale
\ref{appendix_viua_vm_assembly_language_dir_import} na stronie
\pageref{appendix_viua_vm_assembly_language_dir_import}.

