\section{Instrukcja użytkownika kompilatora języka Viuact}
\label{viuact_manual}

Tradycja nakazuje, aby pierwszym programem jaki pisze się w nowym języku był program, który wypisze na ekran
napis ,,\emph{Hello World!}''. W ViuAct ten program wygląda następująco:

\begin{lstlisting}
(let main () {
    (print "Hello World!")
    0
})
\end{lstlisting}

Aby skopilować ten program, należy wykonać w konsoli następujące polecenia:

\begin{lstlisting}
$ viuact-cc --mode exec ./hello_world.lisp
$ viuact-opt ./build/_default/hello_world.asm
\end{lstlisting}

Kod wykonywalny (\emph{bytecode} wykonywalny przez Viua VM) będzie umieszczony w pliku
\texttt{hello\_world.bc} w katalogu \texttt{./build/\_default}.
Aby go uruchomić należy użyć jądra Viua VM:

\begin{lstlisting}
$ viua-vm ./build/_default/hello_world.bc
%*\emph{Hello World!}*)
$
\end{lstlisting}

Nazwy plików pośrednich są wywodzone z nazwy pliku źródłowego:

\begin{description}
    \item[\texttt{\emph{example}.lisp}] plik z kodem źródłowym w języku ViuAct
    \item[\texttt{\emph{example}.asm}] plik wynikowy kompilatora, z kodem źródłowym w języku assemblera Viua
        VM
    \item[\texttt{\emph{example}.bc}] plik wynikowy assemblera Viua VM, zawierający wykonywalny bytecode
\end{description}

Pliki \texttt{.asm} i \texttt{.bc} są umieszczane w katalogu \texttt{./build/\_default}.

\subsection{Opcje kompilatora}

Jedyną opcją kompilatora jest \texttt{-\phantom{}-mode}, która przyjmuje dwie możliwe wartości:

\begin{description}
    \item[\texttt{exec}] jeśli plik źródłowy definiuje moduł wykonywalny
    \item[\texttt{module}] jeśli plik źródłowy definiuje moduł biblioteczny
\end{description}

\subsection{Zmienne środowiskowe}

Zachowanie kompilatora można częściowo zmodyfikować ustawiając zmienne środowiskowe.

\subsubsection{\texttt{DEFAULT\_OUTPUT\_DIRECTORY}}

Kontroluje katalog, w którym kompilator składuje pliki wynikowe. Domyślnie pliki wynikowe są składowane w
katalogu \texttt{./build/\_default}.

\subsubsection{\texttt{VIUAC\_LIBRARY\_PATH}}

Jak \texttt{LD\_LIBRARY\_PATH}.

\subsubsection{\texttt{VIUA\_ASM\_PATH}}

Kontroluje ścieżkę do assemblera Viua VM.

\subsubsection{\texttt{VIUAC\_VERBOSE}}

Wartość \texttt{true} powoduje wyświetlenie komunikatów podczas kompilacji.

\subsubsection{\texttt{VIUAC\_DEBUGGING}}

Wartość \texttt{true} włącza komunikaty debugowania.

\subsubsection{\texttt{VIUAC\_INFO}}

Wartość \texttt{true} włącza dodatkowe komunikaty informacyjne.

\subsubsection{\texttt{VIUAC\_DUMP\_INTERMEDIATE}}

Wartość \texttt{tokens} powoduje zrzut strumienia tokenów do pliku \texttt{\emph{example}.tokens}.
Wartość \texttt{exprs} powoduje zrzut drzewa składni do pliku \texttt{\emph{example}.expressions}.
Można podać obie wartości, oddzielone przecinkiem.
