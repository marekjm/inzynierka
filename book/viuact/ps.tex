\section{Projekt struktury}

\subsection{Wykorzystywane struktury danych}

Brak jest rozbudowanego diagramu klas, ponieważ program nie jest pisany w stylu obiektowym.
W programie istnieją dwie główne grupy struktur opisujące elementy języka -- typy tokenów i typy grup
(reprezentujących konstrukcje językowe), struktura reprezentująca adres rejestru (abstrakcyjny ,,slot'' na
wartości), struktura reprezentująca stan kompilowanego programu, oraz cztery struktury reprezentujące
niskopoziomowe abstrakcje linii programu w języku assemblera -- konstruktor, przeniesienie, wywołanie, i
,,\emph{verbatim}''.

\begin{quote}
    Listingi przedstawiające wykorzystywane struktury danych jest podany w języku OCaml.
    Kompilator jest napisany w języku Python, który nie pozwala na tak łatwe i czytelne definiowanie
    nowych struktur danych -- stąd decyzja o użyciu innego języka w pracy.

    Zachowane zostały typy danych, nazwy pól i całych struktur. OCaml pozwala na duże bardziej
    przejrzyste i czytelne opisanie typów danych poszczególnych pól niż Python, co ma dużą
    wartość dokumentacyjną.
\end{quote}

\subsubsection{Tokeny}
\label{diagram_klas_tokeny}

Każdy typ tokenu jest reprezentowany przez osobną strukturę. Tokeny, oprócz swojego typu mają atrybuty
określające ich lokalizację w pliku (wiersz i kolumna), oraz pole z leksemem. Typy tokenów wymagane do
reprezentacji języka \ViuAct są opisane w specyfikacji języka.

Definicje struktur reprezentujących tokeny są zawarte w pliku \texttt{viuact/token\_types.py}.

\subsubsection{Grupy}
\label{diagram_klas_grupy}

Każda konstrukcja językowa jest reprezentowana przez osobną strukturę. Konstrukcje wymagane do reprezentacji
języka \ViuAct są opisane w specyfikacji języka.

Definicje struktur reprezentujących grupy są zawarte w pliku \texttt{viuact/group\_types.py}.

\subsubsection{Slot}
\label{diagram_klas_slot}

\begin{small}
\begin{lstlisting}
type register_set =
    | Local
    | Parameters
    | Arguments
    | Closure_local

type slot = {
    name         : string ;
    index        : int ;
    register_set : register_set ;
}
\end{lstlisting}
\end{small}

Struktura \texttt{slot} reprezentuje adres rejestru. Z punktu widzenia języka \ViuAct istotne jest pole
\texttt{name} (określające nazwę slotu jaką posługuje się programista); z punktu widzenia emitera kodu istotne
są pola \texttt{index} i \texttt{register\_set} określające adres rejestru jakim posługuje się Viua VM.

\subsubsection{Stan programu}
\label{diagram_klas_stan_programu}

Stan programu (struktura \texttt{State}) zawiera pola śledzące ilość zaalokowanych rejestrów, widoczne
funkcje, a w przypadkach śledzenia stanu funkcji zagnieżdżonej -- także wykorzystywane sloty z otaczającego
zakresu leksykalnego.

\subsubsection{Konstruktor}
\label{diagram_klas_konstruktor}

\begin{small}
\begin{lstlisting}
type ctor = {
    of_type : string ;
    slot    : slot ;
    value   : string ;
}
\end{lstlisting}
\end{small}

Konstruktor reprezentuje instrukcję bezpośrednio tworzącą wartość w rejestrze. Przykładowo, aby wyemitować
instrukcję \texttt{integer \%1 local 42} utworzony zostanie:

\begin{small}
\begin{lstlisting}
{
    of_type = "integer" ;
    slot = { name = "x"; index = 1 ; register_set = Local } ;
    value = "42"
}
\end{lstlisting}
\end{small}

\subsubsection{Przeniesienie}
\label{diagram_klas_przeniesienie}

\begin{small}
\begin{lstlisting}
type move_kind =
    | Move
    | Copy

type move = {
    kind   : move_kind ;
    source : slot ;
    dest   : slot option ;
}
\end{lstlisting}
\end{small}

Przeniesienie opisuje przesunięcie (instrukcja \emph{\texttt{move}}) lub kopię wartości (instrukcja
\emph{\texttt{copy}}). Slot docelowy nie musi być obecny - np. wtedy wartość jest przenoszona do slotu
\texttt{void}.

\subsubsection{Wywołanie}
\label{diagram_klas_wywolanie}

\begin{small}
\begin{lstlisting}
type call_kind =
    | Synchronous
    | Actor
    | Tail
    | Deferred

type call = {
    kind : call_kind ;
    slot : slot option ;
    to   : string ;
}
\end{lstlisting}
\end{small}

Wywołanie opisuje wywołanie funkcji w każdy sposób dostępny w języku \ViuAct: zwykłe wywołanie funkcji,
wywołanie tworzące aktora, wywołąnie \emph{tail call}, i wywołanie ,,odroczone''.

Typy wywołań opisane są w specyfikacji języka;
\texttt{Synchronous} w \ref{viuact_spec_fn_call} (strona \pageref{viuact_spec_fn_call}),
\texttt{Actor} w \ref{viuact_spec_actor_call} (strona \pageref{viuact_spec_actor_call}),
\texttt{Tail} w \ref{viuact_spec_tail_call} (strona \pageref{viuact_spec_tail_call}),
\texttt{Deferred} w \ref{viuact_spec_deferred_call} (strona \pageref{viuact_spec_deferred_call}).

\subsubsection{Linia ,,\emph{verbatim}''}
\label{diagram_klas_linia_verbatim}

\begin{small}
\begin{lstlisting}
type verbatim = {
    text : string ;
}
\end{lstlisting}
\end{small}

Linia \emph{verbatim} opisuje dowolną linię języka assemblera (m.in. dyrektywy \texttt{.import:} czy
\texttt{.function:}).

\emph{Emitter} (rozdział \ref{opis_etapow_kompilacji_emisja_kodu_wynikowego} na stronie
\pageref{opis_etapow_kompilacji_emisja_kodu_wynikowego}) większość instrukcji tworzy za pomocą linii
\emph{verbatim}. Jest to zabieg o tyle ,,brzydki'' co efektywny; na etapie prototypowania bardzo szybko
można w ten sposób wyemitować spory zakres instrukcji bez potrzeby projektowania struktury dla każdego typu
instrukcji.

\section{Decyzje projektowe}

\subsection{Środowisko docelowe}

Środowiskiem docelowym, na którym będą uruchamiane programy napisane w języku
\ViuAct\phantom{} jest maszyna wirtualna Viua VM. Środowisko docelowe musi
spełniać wymagania jakie ma Viua VM (m.in. musi to być system zgodny ze
standardem POSIX).

\subsection{Środowisko implementacji}

Środowiskiem implmentacji jest Linux z dostępnymi standardowymi narzędziami GNU,
interpreterem języka Python 3, i umożliwiający uruchomienie assemblera
dostarczanego przez Viua VM.

\subsection{Priorytety implementacyjne}

Maksymalizacja prostoty budowy kompilatora i języka.
Marginalizacja obsługi błędów w kompilatorze z uwagi na brak czasu.
Marginalizacja optymalizacji z uwagi na brak czasu.

\section{Projekt algorytmów i przyjętych protokołów}

Dyskusja na temat algorytmów i sposobu implementacji jest częściowo przeprowadzona w rozdziale
\ref{opis_etapow_kompilacji} na stronie \pageref{opis_etapow_kompilacji}, szczególnie w rozdziale
\ref{opis_etapow_kompilacji_analiza_skladniowa}.

\section{Projekt interfejsu}

\subsection{Interfejs kompilatora}

Kompilator składa się z dwóch programów: \texttt{viuact-cc} (kompilatora właściwego) i \texttt{viuact-opt}
(programu łączącego). Programy te są konfigurowane za pomocą zmiennych środowiskowych, które kontrolują poziom
''głośności'' logów, położenie assemblera Viua VM, a także włączają bądź wyłączają serializację formy
pośredniej.

Interfejs użytkownika opisany jest w rozdziale \ref{viuact_manual}~,,\nameref{viuact_manual}''
na stronie \pageref{viuact_manual}.

\subsection{Interfejs języka}

Interfejsem języka jest jego składnia.
Jest ona opisana w rozdziale \ref{viuact_spec} na stronie \pageref{viuact_spec}.

\subsection{Inne interfejsy}

\subsubsection{Pliki interfejsów modułów (\texttt{.i})}
\label{pliki_interfejsow_modulow}

Pliki interfejsów modułów wyliczają funkcje eksportowane przez dany moduł, oraz prezentują metadane wymagane
do połączenia plików w sposób, który będzie mógł działać na Viua VM. Pliki interfejsu dla modułów ''własnych''
nie różnią się zasadniczą strukturą od plików interfejsu dla modułów ''obcych'', ale pliki interfejsu dla
modułów ''obcych'' muszą być uzupełnione o kilka dodatkowych pól. Jest to dokładniej opisane w rozdziale
\ref{pliki_interfejsow_modulow_obcych} na stronie \pageref{pliki_interfejsow_modulow_obcych}.

Pliki interfejsów są zapisywane w formacie JSON.

\paragraph{Plik interfejsu dla modułów ''własnych''}

Moduły ''własne'' języka \ViuAct to moduły napisane w języku \ViuAct.

\begin{small}
\begin{lstlisting}
{
    "foreign": false,
    "real_name": "A_module",
    "fns": [
        {
            "arity": 1,
            "name": "f",
            "real_name": "A_module::f",
            "from_module": "A_module"
        }
    ]
}
\end{lstlisting}
\end{small}

Atrybut \texttt{foreign} określa czy moduł jest ''obcy'' (\texttt{true}) czy ''własny'' (\texttt{false}).
Atrybut \texttt{real\_name} określa nazwę modułu tak jak będzie prezentowana na poziomie bytecode'u.
Atrybut \texttt{fns} jest listą funkcji, które są przez dany moduł eksportowane.

W elementach listy \texttt{fns} atrybuty mają następujące znaczenie:

\begin{enumerate}
    \item \texttt{arity} określa ''moc'' funkcji
    \item \texttt{name} określa nazwę funkcji widoczną z poziomu języka \ViuAct
    \item \texttt{real\_name} określa pełną nazwę funkcji widoczną z poziomu bytecode'u
    \item \texttt{from\_module} określa pełną nazwę modułu, z którego pochodzi dana funkcja
\end{enumerate}

\paragraph{Plik interfejsu dla modułów ''obcych''}
\label{pliki_interfejsow_modulow_obcych}

Moduły ''obce'' języka \ViuAct to moduły napisane w języku assemblera Viua VM lub w C++.

\begin{small}
\begin{lstlisting}
{
    "foreign": true,
    "real_name": "std::posix::network",
    "fns": [
        {
            "arity": 0,
            "name": "socket",
            "real_name": "Std::Posix::Network::socket",
            "bytecode_name": "std::posix::network::socket",
            "from_module": "Std::Posix::Network"
        }
    ]
}
\end{lstlisting}
\end{small}

Różnicą w stosunku do plików interfejsu dla modułow ''własnych'' jest dodatkowy atrybut w deklaracji
eksportowanej funkcji -- \texttt{bytecode\_name} określający nazwę funkcji na poziomie bytecode'u. Nazwa ta
nie musi pokrywać się z nazwą w atrybucie \texttt{real\_name}, który określa pełną nazwę funkcji widoczną na
poziomie języka \ViuAct.

Dwa atrybuty są potrzebne ponieważ na tym poziomie następuje łączenie dwóch ''światów''; języka \ViuAct, który
narzuca reguły tego jak muszą wyglądać nazwy modułów i funkcji, oraz języka assemblera Viua VM, który takich
reguł nie narzuca. Dla modułów ''własnych'' atrybuty \texttt{bytecode\_name} jest zbędny ponieważ dla nich
nazwa widoczna na poziomie \ViuAct i języka assemblera Viua VM jest taka sama.

\subsubsection{Pliki zależności modułów (\texttt{.d})}
\label{pliki_zaleznosci_modulow}

Pliki zależności są kodowane w formacie JSON.

\begin{small}
\begin{lstlisting}
{
    "imports": [
        {
            "module_name": "Std::Random",
            "real_name": "std::random",
            "foreign": true
        }
    ]
}
\end{lstlisting}
\end{small}

Atrybut \texttt{imports} przechowuje listę modułów importowanych przez moduł, którego zależności definiuje
dany plik (co jest określane na podstawie nazwy pliku).

W elementach listy \texttt{imports} atrybuty mają następujące znaczenie:

\begin{enumerate}
    \item \texttt{module\_name} określa pełną nazwę modułu z punktu widzenia języka \ViuAct
    \item \texttt{real\_name} określa pełną nazwę modułu widoczną z poziomu bytecode'u
    \item \texttt{foreign} określa czy moduł jest ''własny'' (\texttt{false}) czy ''obcy'' (\texttt{true})
\end{enumerate}

\section{Opis implementacji}
\label{viuact_cc_implementation_desc}

Implementacja kompilatora jest bardzo prosta, jak na standardy tego typu projektów.
Rysunek \ref{basic_compiler_flow} na stronie \pageref{basic_compiler_flow} bardzo dobrze oddaje strukturę
kompilatora dostarczonego jako wynik tej pracy.

Kompilacja rozpoczyna się od funkcji \texttt{main()} w pliku \texttt{cc.py},
która wywołuje funkcję\newline\texttt{viuact.driver.compile\_file()} w sposób
określony przez argumenty podane w wierszu poleceń.
\texttt{compile\_file()} przygotowuje katalog kompilacji (domyślnie jest to
\texttt{./build/\_default}), wczytuje plik z kodem źródłowym przekazany do
kompilacji i wywołuje funkcję \texttt{compile\_text()} z tego samego modułu.

\subsection{Analiza leksykalna}
\label{viuact_cc_impl_desc_lex}

Funkcja \texttt{viuact.driver.compile\_text()} na początek dokonuje analizy
leksykalnej kodu źródłowego. Ta pierwsza faza przygotowania tekstu programu ,,do
obróbki'' jest wykonywana przez funkcję \texttt{lex()} z modułu
\texttt{viuact.lexer}.

\subsection{Analiza składniowa}
\label{viuact_cc_impl_desc_parse}

\subsubsection{Grupowanie}

Po przeprowadzeniu analizy leksykalnej, funkcja \texttt{compile\_text()} oddaje
jej wynik (czyli listę tokenów) do funkcji \texttt{viuact.parser.group()}, która
przeprowadza grupowanie tokenów odpowiadających różnym konstrukcjom językowym.
Algorytm sterujący tym procesem jest trywialny:

\begin{enumerate}
    \item utwórz pustą listę
    \item rozważ pierwszy nieprzenalizowany token
    \item jeśli ten token to \texttt{(} lub \texttt{\{}, rozpocznij
        rekurencyjnie przetwarzanie strumienia tokenów od następnego tokenu, a
        wynik analizy dodaj jako pojedynczy element do listy
    \item w innym przypadku dodaj token do listy
    \item jeśli strumień tokenów jest pusty, zakończ algorytm
    \item w innym przypadku kontynuuj od punktu 2.
\end{enumerate}

W ten sposób tworzona jest lista list i pojedynczych tokenów reprezentujących
konstrukcje językowe, z których składa się analizowany kod źródłowy. Na tym
etapie łączone są również rozbudowane identyfikatory (np. \texttt{foo.bar.baz})
oraz wstawiane markery wyrażeń złożonych.

To, że wstępna faza analizy składniowej sprowadza się w dużej mierze do
rekurencyjnego ,,odbijania się'' od nawiasów, bez konieczności sprawdzania co
jest pomiędzy nimi, i okazjonalnej konkatenacji jeśli parser napotka znak
\emph{.} jest zasługą wykorzystania składni wzorowanej na języku Lisp. Jest ona
bardzo regularna co pozwala przeprowadzić proste parsowanie małym nakładem pracy
i sprawdzając niewielką ilość informacji.

% Żeby ten ostatni akapit był nieco oddzielony od głównego tekstu. On ma tylko
% powiedzieć co jest wynikiem tej fazy.
\vspace{1em}

W wyniku grupowania otrzymujemy prymitywne drzewo składni abstrakcyjnej
(\emph{abstract syntax tree}, \emph{AST}), które następnie jest ,,etykietowane''
podczas następnej fazy: klasyfikacji grup.

\subsubsection{Klasyfikacja grup}

Klasyfikacją grup otrzymanych podczas fazy grupowania jako wynik funkcji
\texttt{group()} zajmuje się funkcja \texttt{parse()} z modułu \texttt{viuact.parser}.

Również w tym przypadku algorytm jest bardzo prosty. Funkcja \texttt{parse()}
sprawdza każdy element podstawowego AST i konwertuje go na pewien typ z modułu
\texttt{viuact.group\_types}. Jeśli jest to pojedynczy token konwersja następuje
bezpośrednio; jeśli sprawdzanym elementem jest to lista algorytm determinuje co
ta lista reprezentuje (na podstawie pierwszego elementu tejże, którym zawsze
jest pojedynczy token) i postępuje rekurencyjnie etykietując każdy jej element,
pozostawiając konwersję podstawowego elementu na koniec -- po przeanalizowaniu
elementów składowych.

Wykorzystanie algorytmu o tak niskim poziomie skomplikowania jest możliwe dzięki
uważnemu zaprojektowaniu składni języka. Każda konstrukcja językowa jest
ograniczona nawiasami -- klamrowymi bądź okrągłymi -- lub ma stałe miejsce w
grupie. Poniżej, dla przykładu, zaprezentowane są wzorce odpowiadające wybranym
konstrukcjom językowym:

\begin{lstlisting}
(let   %*\emph{name}*) %*\emph{expression}*))
(if    %*\emph{condition-expr}*) %*\emph{true-expr}*) %*\emph{false-expr}*))
(try   %*\emph{guarded-expr}*) ...)
(actor %*\emph{id}*) %*\emph{expr}*)*)
(+     %*\emph{lhs-expr}*) %*\emph{rhs-expr}*))
\end{lstlisting}

Jeśli całość oprócz pierwszego tokenu zostanie ,,ukryta'' to dalej oczywistym
jest z jaką konstrukcją mamy do czynienia:

\begin{lstlisting}
(let   ...)
(if    ...)
(try   ...)
(actor ...)
(+     ...)
\end{lstlisting}

Wykorzystanie notacji polskiej (notacji prefiksowej) i takie zaprojektowanie
składni, że każda konstrukcja ma stałą ,,szerokość'' pozwala na zastosowanie
chwytu ,,\emph{pierwszy token decyduje}''. Sprawia to też, że język ma
rozpoznawalne tempo i styl, a konsekwencja układu nadaje mu elegancji.

% Żeby ten ostatni akapit był nieco oddzielony od głównego tekstu. On ma tylko
% powiedzieć co jest wynikiem tej fazy.
\vspace{1em}

W wyniku procesu ,,etykietowania'' otrzymujemy ,,wzbogacone'' AST składające się
ze sklasyfikowanych konstrukcji językowych reprezentowanych przez wartości
konkretnego typu zamiast ubogich w informację (bez dodatkowej weryfikacji) list
i ,,samotnych'' tokenów. Zamiast ,,listy list'' otrzymujemy listę modułów,
funkcji, wyliczeń itd. Takie AST znacząco upraszcza implementację kolejnej fazy
kompilacji: generowania kodu.

\subsection{Generowanie kodu wynikowego}
\label{viuact_cc_impl_desc_codegen}

Po przeprowadzeniu analizy składniowej i wygenerowaniu AST kompilator od razu
przystępuje do generowania kodu. Jest to uwarunkowane pomienięciem w pracy
etapów analizy semantycznej i optymalizacji. Te fazy są nieodzownymi elementami
w kompilatorach ,,produkcyjnych'' jednak zostały pominięte z uwagi na
ograniczoną ilość czasu na wykonanie projektu gdyż zarówno optymalizacje jak i
analiza semantyczna są elementami kompilatora, które mogą być rozwijane ,,w
nieskończoność'' -- implementując kolejne algorytmy i metody optimalizacji, i
tworząc coraz bardziej ,,inteligentne'' systemy analizy statycznej (kończąc na
systemach będących w stanie udowodnić poprawność bądź niepoprawność programu).

Analiza semantyczna (m.in. weryfikacja typów i analiza poprawności kodu) jest
częściowo oddelegowana do assemblera Viua VM, który potrafi ją, w ograniczonym
zakresie, przeprowadzić.

W kompilatorze języka \ViuAct\phantom{} będącym wynikiem tej pracy generowanie
kodu wynikowego jest implementowane przez dwa procesy działające naprzemiennie:
\emph{obniżanie poziomu} i \emph{emisję instrukcji}.

Modułem implementującym fazę obniżenia poziomu jest \texttt{viuact.lowerer}, a
fazę emisji instrukcji \texttt{viuact.emitter}. Funkcja \texttt{compile\_text()}
przekazuje AST otrzymane po etykietowaniu do funkcji \texttt{lower\_file()} z
modułu \texttt{viuact.lowerer} czym rozpoczyna proces generowania kodu
wynikowego.

\subsection{Obniżenie poziomu}
\label{viuact_cc_impl_desc_lower}

,,Obniżenie poziomu'' polega na ,,rozbiciu'' elementów AST na takie fragmenty,
które mają swój bezpośredni (lub bliski bezpośredniemu) odpowiednik w języku
assemblera Viua VM. Wygenerowanie kodu wynikowego dla takich fragmentów jest
względnie proste.

Na tym etapie ,,znikają'' konstrukcje istniejące w języku \ViuAct, ale nie w
języku assemblera Viua VM; są to na przykład wyliczenia i dowiązania \emph{let}.
Wyrażenia, które mają swoje odpowiedniki, ale są zbyt skomplikowane do prostego
przetworzenia są rozbijane na prostsze formy (np. wyrażenia złożone, wywołania
funkcji, instrukcje warunkowe).

To ,,rozbijanie'' i ,,znikanie'' konstrukcji języka daje tej fazie jej nazwę:
konstrukcje języka wysokiego poziomu są tłumaczone na konstrukcje języka
niskiego poziomu.

\subsubsection{Obniżenie modułów}

W pierwszej kolejności obniżane są moduły. Jest to wykonywane rekurencyjnie,
a zagnieżdżone moduły są obniżane przed modułem, który je zawiera. Ten proces
jest implementowany przez funkcję \texttt{viuact.lowerer.lower\_module()}.

\paragraph*{Importowanie} Po obniżeniu modułów przetwarzane są importy
modułów. Obniżenie musi być wykonane najpierw żeby umożliwić importowanie
zagnieżdżonych modułów. Przetworzeniem importów zajmuje się funkcja
\texttt{viuact.lowerer.perform\_imports()}.

\subsubsection{,,Obniżenie'' wyliczeń}

Następnie ,,obniżane'' są wyliczenia. Jest to jeden z najprostszych elementów
kompilatora. Obniżenie wyliczeń sprowadza się do przypisania każdej wartości
składowej wyliczenia pewnej wartości całkowitoliczbowej, a następnie zapisania
wyniku tego przypisania i udostępnienia go do podglądu podczas dalszej
kompilacji.

Przypisywanie wartości całkowitoliczbowych rozpoczyna się od 0, a każdej
kolejnej wartości składowej pojedynczego wyliczenia jest przypisywana liczba
całkowita o 1 większa od poprzedniej.
Wyliczenie w języku \ViuAct\phantom{} może mieć jedynie $2^{63}$ elementów z
uwagi na użycie typu liczby całkowitej ze znakiem (rozdział \ref{viuact_spec_datatypes_simple_integer}
na stronie \pageref{viuact_spec_datatypes_simple_integer}) jako typu bazowego
dla wyliczeń.

\subsubsection{Obniżenie funkcji} Obniżanie funkcji jest w założeniu prostym
procesem podzielonym na kilka faz:

\begin{enumerate}
    \item przesunięcia argumentów do rejestrów lokalnych
    \item emisja wyrażenia będącego ciałem funkcji
    \item przesunięcie wyniku do rejestru 0 (rejestru wartości zwracanych)
\end{enumerate}

Skomplikowanie tego procesu jest ,,ukryte'' przez fakt, że wyrażenie będące
ciałem funkcji może być dowolnie złożone, ale zawsze jest emitowane przy użyciu
tylko jednego (z punktu widzenia modułu obniżającego) wywołania funkcji
\texttt{viuact.emitter.emit\_expr()}. To, że wewnątrze \texttt{emit\_expr()}
może nastąpić ,,eksplozja'' wywołań i działania nie jest widoczne z tego
poziomu.

\paragraph{,,Sklejanie'' funkcji}

Funkcje są jedynymi elementami języka \ViuAct\phantom{}, które po obniżeniu są
wyraźnie widoczne w postaci linii kodu. Generowanie tekstu dla funkcji następuje
w kilku etapach, których kolejność nie pokrywa się z tym co widać w pliku
wynikowym.

Całość procesu spina stan (reprezentowany przez typ \texttt{State}) nazdorujący
ilość alokowanych rejestrów, funkcje zagnieżdżone, oraz -- w przypadku domknięć
-- informację o domykanym zakresie leksykalnym, do którego funkcja może się
odwoływać.

Po utworzeniu stanu funkcji wykonywane są etapy opisane poniżej.

\subparagraph{Przesunięcia argumentów}

Dla każdego argumentu emitowana jest instrukcja \texttt{move} (opisana dokładnie
w \ref{viua_vm_ops_value_management_move} na stronie \pageref{viua_vm_ops_value_management_move})
przesuwająca wartość przekazaną jako argument z rejestru argumentów do
rejestru lokalnego.

Wiersze dopisywane są do listy instrukcji składajacych się na funkcję.

\subparagraph{Emisja ciała funkcji}

Po przesunięciu argumentów emitowane jest wyrażenie będące ciałem funkcji.
Wiersze dopisywane są na końcu listy instrukcji składajacych się na funkcję.

\subparagraph{Utworzenie wiersza sygnatury funkcji i całościowej listy wierszy}

Po wyemitowaniu wyrażenia będącego ciałem funkcji tworzona jest całościowa lista
wierszy funkcji. Jako jej pierwszy element wstawiana jest linia będąca syganturą
funkcji zgodną z formatem oczekiwanym przez assembler Viua VM:
,,\texttt{.function: \emph{nazwa\_funkcji}/\emph{ilość\_argumentów}}''.
Na przykład ,,\texttt{.function:~say\_hello\_to/1}''.

\subparagraph{Zapis alokacji rejestrów}

Po sygnaturze, jak pierwszy wiersz ciała właściwego (instrukcji składających się
na funkcję) dopisywana jest instrukcja \texttt{allocate\_registers} alokująca
taką ilość rejestrów lokalnych jaka jest potrzebna emitowanej funkcji do
działania.

Ilość potrzebnych funkcji rejestrów lokalnych jest śledzona przez obiekt
reprezentujący stan funkcji. Każdy nowy potrzebny rejestr jest ,,oddawany''
przez funkcję \texttt{State.get\_slot()} co powoduje, że stan funkcji zna
dokładną ilość rejestrów ,,zaalokowanych'' podczas emisji kodu funkcji.

\subparagraph{Zapis ciała wewnętrznego}

Po zapisie zaalokowanych rejestrów do całościowej listy wierszy dopisywane są
wiersze wygenerowane przez wyrażenie będące ciałem funkcji.

\subparagraph{Przesunięcie wartości zwracanej i epilog funkcji}

Po zapisie ciała wewnętrznego do całościowej listy wierszy dopisywana jest
instrukcja \texttt{move} przesuwająca wynik wyrażenia będącego ciałem funkcji do
rejestru 0 (rejestru wartości zwracanej).

Następnie do całościowej listy wierszy dopisywana jest instrukcja \texttt{return}
oraz dyrektywa \texttt{.end}, a proces generowania funkcji jest zakończony.

\paragraph{Funkcje zagnieżdżone}

Funkcje w języku \ViuAct\phantom{} mogą być zagnieżdżane. Z tego powodu funkcja
\texttt{lower\_function()} może zwrócić jeden lub więcej wyników. Aby ułatwić
obsługę takich sytuacji \texttt{lower\_function()} deleguje swoją pracę do
funkcji \texttt{output\_function\_body()}, która zwraca listę wszystkich funkcji
wyemitowanych jako wynik obniżenia funkcji wejściowych. Następnie każdy element
(tj. każda funkcja) tej listy jest przekazywany do funkcji \texttt{lower\_function\_body()},
która przetwarza ,,surowe'' wyniki na formę tekstową gotową do zapisu w pliku z
kodem wynikowym (tj. programem przetłumaczonym z języka \ViuAct\phantom{} na
jezyk assemblera Viua VM).

\vspace{1em}

Obniżenie funkcji jest ostatnim elementem fazy ,,obniżania poziomu''. Jest też
punktem styku modułu \texttt{viuact.lowerer} z modułem \texttt{viuact.emitter}
-- funkcja \texttt{output\_function\_body()} jest jedyną funkcją wywołującą
funkcje modułu \texttt{emitter} podczas tej fazy.

Emisja instrukcji jest szczegółowo opisana w rozdziale \ref{viuact_cc_impl_desc_emit}.

\subsection{Emisja instrukcji}
\label{viuact_cc_impl_desc_emit}

Emisja instrukcji tłumaczy przetworzone konstrukcje języka \ViuAct\phantom{} na
ich bezpośrednie odpowiedniki w języku assemblera Viua VM.

\subsubsection{Emisja wyrażeń prostych}
\paragraph{Emisja konstruktorów}
\subsubsection{Emisja wyrażeń złożonych}
\subsubsection{Emisja dowiązań \emph{let}}
\subsubsection{Emisja wywołań funkcji wbudowanych}
\subsubsection{Emisja wywołań funkcji}
\subsubsection{Emisja wywołań aktorów}
\subsubsection{Emisja wywołań odroczonych}
\subsubsection{Emisja wywołań watchdog}
\subsubsection{Emisja wywołań ogonowych}
\subsubsection{Emisja wyrażeń warunkowych}
\subsubsection{Emisja funkcji zagnieżdżonych}
\subsubsection{Emisja wyrażeń \emph{try}}
\subsubsection{Emisja przypisania do pola struktury}
\subsubsection{Emisja dostępu do pola struktury}
