\section{Projekt struktury}

\subsection{Wykorzystywane struktury danych}

Brak jest rozbudowanego diagramu klas, ponieważ program nie jest pisany w stylu obiektowym.
W programie istnieją dwie główne grupy struktur opisujące elementy języka -- typy tokenów i typy grup
(reprezentujących konstrukcje językowe), struktura reprezentująca adres rejestru (abstrakcyjny ,,slot'' na
wartości), struktura reprezentująca stan kompilowanego programu, oraz cztery struktury reprezentujące
niskopoziomowe abstrakcje linii programu w języku assemblera -- konstruktor, przeniesienie, wywołanie, i
,,\emph{verbatim}''.

\begin{quote}
    Listingi przedstawiające wykorzystywane struktury danych jest podany w języku OCaml.
    Kompilator jest napisany w języku Python, który nie pozwala na tak łatwe i czytelne definiowanie
    nowych struktur danych -- stąd decyzja o użyciu innego języka w pracy.

    Zachowane zostały typy danych, nazwy pól i całych struktur. OCaml pozwala na duże bardziej
    przejrzyste i czytelne opisanie typów danych poszczególnych pól niż Python, co ma dużą
    wartość dokumentacyjną.
\end{quote}

\subsubsection{Tokeny}
\label{diagram_klas_tokeny}

Każdy typ tokenu jest reprezentowany przez osobną strukturę. Tokeny, oprócz swojego typu mają atrybuty
określające ich lokalizację w pliku (wiersz i kolumna), oraz pole z leksemem. Typy tokenów wymagane do
reprezentacji języka ViuAct są opisane w specyfikacji języka.

Definicje struktur reprezentujących tokeny są zawarte w pliku \texttt{viuact/token\_types.py}.

\subsubsection{Grupy}
\label{diagram_klas_grupy}

Każda konstrukcja językowa jest reprezentowana przez osobną strukturę. Konstrukcje wymagane do reprezentacji
języka ViuAct są opisane w specyfikacji języka.

Definicje struktur reprezentujących grupy są zawarte w pliku \texttt{viuact/group\_types.py}.

\subsubsection{Slot}
\label{diagram_klas_slot}

\begin{small}
\begin{lstlisting}
type register_set =
    | Local
    | Parameters
    | Arguments
    | Closure_local

type slot = {
    name         : string ;
    index        : int ;
    register_set : register_set ;
}
\end{lstlisting}
\end{small}

Struktura \texttt{slot} reprezentuje adres rejestru. Z punktu widzenia języka ViuAct istotne jest pole
\texttt{name} (określające nazwę slotu jaką posługuje się programista); z punktu widzenia emitera kodu istotne
są pola \texttt{index} i \texttt{register\_set} określające adres rejestru jakim posługuje się Viua VM.

\subsubsection{Stan programu}
\label{diagram_klas_stan_programu}

Stan programu (struktura \texttt{State}) zawiera pola śledzące ilość zaalokowanych rejestrów, widoczne
funkcje, a w przypadkach śledzenia stanu funkcji zagnieżdżonej -- także wykorzystywane sloty z otaczającego
zakresu leksykalnego.

\subsubsection{Konstruktor}
\label{diagram_klas_konstruktor}

\begin{small}
\begin{lstlisting}
type ctor = {
    of_type : string ;
    slot    : slot ;
    value   : string ;
}
\end{lstlisting}
\end{small}

Konstruktor reprezentuje instrukcję bezpośrednio tworzącą wartość w rejestrze. Przykładowo, aby wyemitować
instrukcję \texttt{integer \%1 local 42} utworzony zostanie:

\begin{small}
\begin{lstlisting}
{
    of_type = "integer" ;
    slot = { name = "x"; index = 1 ; register_set = Local } ;
    value = "42"
}
\end{lstlisting}
\end{small}

\subsubsection{Przeniesienie}
\label{diagram_klas_przeniesienie}

\begin{small}
\begin{lstlisting}
type move_kind =
    | Move
    | Copy

type move = {
    kind   : move_kind ;
    source : slot ;
    dest   : slot option ;
}
\end{lstlisting}
\end{small}

Przeniesienie opisuje przesunięcie (instrukcja \emph{\texttt{move}}) lub kopię wartości (instrukcja
\emph{\texttt{copy}}). Slot docelowy nie musi być obecny - np. wtedy wartość jest przenoszona do slotu
\texttt{void}.

\subsubsection{Wywołanie}
\label{diagram_klas_wywolanie}

\begin{small}
\begin{lstlisting}
type call_kind =
    | Synchronous
    | Actor
    | Tail
    | Deferred

type call = {
    kind : call_kind ;
    slot : slot option ;
    to   : string ;
}
\end{lstlisting}
\end{small}

Wywołanie opisuje wywołanie funkcji w każdy sposób dostępny w języku ViuAct: zwykłe wywołanie funkcji,
wywołanie tworzące aktora, wywołąnie \emph{tail call}, i wywołanie ''odroczone''.

Typy wywołań opisane są w specyfikacji języka.

\subsubsection{Linia ''\emph{verbatim}''}
\label{diagram_klas_linia_verbatim}

\begin{small}
\begin{lstlisting}
type verbatim = {
    text : string ;
}
\end{lstlisting}
\end{small}

Linia \emph{verbatim} opisuje dowolną linię języka assemblera (m.in. dyrektywy \texttt{.import:} czy
\texttt{.function:}).

\emph{Emitter} (rozdział \ref{opis_etapow_kompilacji_emisja_kodu_wynikowego} na stronie
\pageref{opis_etapow_kompilacji_emisja_kodu_wynikowego}) większość instrukcji tworzy za pomocą linii
\emph{verbatim}. Jest to zabieg o tyle ''brzydki'' co efektywny; na etapie prototypowania bardzo szybko
można w ten sposób wyemitować spory zakres instrukcji bez potrzeby projektowania struktury dla każdego typu
instrukcji.

\section{Decyzje projektowe}

\subsection{Środowisko docelowe}

Środowiskiem docelowym, na którym będą uruchamiane programy napisane w języku \ViuAct jest maszyna wirtualna
Viua VM. Środowisko docelowa musi spełniać wymagania jakie ma Viua VM (m.in. musi to być system zgodny ze
standardem POSIX).

\subsection{Środowisko implementacji}

Środowiskiem implmentacji jest Linux z dostępnymi standardowymi narzędziami GNU, językiem Python 3, i
umożliwiającym uruchomienie assemblera dostarczanego przez Viua VM.

\subsection{Priorytety implementacyjne}

Maksymalizacja prostoty budowy kompilatora i języka.
Marginalizacja obsługi błędów w kompilatorze z uwagi na brak czasu.
Marginalizacja optymalizacji z uwagi na brak czasu.

\section{Projekt algorytmów i przyjętych protokołów}

Dyskusja na temat algorytmów i sposobu implementacji jest częściowo przeprowadzona w rozdziale
\ref{opis_etapow_kompilacji} na stronie \pageref{opis_etapow_kompilacji}, szczególnie w rozdziale
\ref{opis_etapow_kompilacji_analiza_skladniowa}.

\section{Projekt rozwiązań sprzętowych}

Brak w tym projekcie. Jest on wyłącznie softwareowy.

\section{Projekt interfejsu}

\subsection{Interfejs użytkownika}

Interfejs użytkownika opisany jest w osobnym dokumencie -- podręczniku użytkownika kompilatora ViuAct (plik
''viuact-manual.pdf'').

\subsubsection{Założenia konstrukcji interfejsu}

\subsection{Interfejs kompilatora}

Kompilator składa się z dwóch programów: \texttt{viuact-cc} (kompilatora właściwego) i \texttt{viuact-opt}
(programu łączącego). Programy te są konfigurowane za pomocą zmiennych środowiskowych, które kontrolują poziom
''głośności'' logów, położenie assemblera Viua VM, a także włączają bądź wyłączają serializację formy
pośredniej.

Interfejs kompilatora opisany jest w osobnym dokumencie.

\subsection{Interfejs języka}

Interfejsem języka jest jego składnia.
Jest ona opisana w specyfikacji języka.

\subsection{Inne interfejsy}

\subsubsection{Pliki interfejsów modułów (\texttt{.i})}
\label{pliki_interfejsow_modulow}

Pliki interfejsów modułów wyliczają funkcje eksportowane przez dany moduł, oraz prezentują metadane wymagane
do połączenia plików w sposób, który będzie mógł działać na Viua VM. Pliki interfejsu dla modułów ''własnych''
nie różnią się zasadniczą strukturą od plików interfejsu dla modułów ''obcych'', ale pliki interfejsu dla
modułów ''obcych'' muszą być uzupełnione o kilka dodatkowych pól. Jest to dokładniej opisane w rozdziale
\ref{pliki_interfejsow_modulow_obcych} na stronie \pageref{pliki_interfejsow_modulow_obcych}.

Pliki interfejsów są zapisywane w formacie JSON.

\paragraph{Plik interfejsu dla modułów ''własnych''}

Moduły ''własne'' języka ViuAct to moduły napisane w języku ViuAct.

\begin{lstlisting}
{
    "foreign": false,
    "real_name": "A_module",
    "fns": [
        {
            "arity": 1,
            "name": "f",
            "real_name": "A_module::f",
            "from_module": "A_module"
        }
    ]
}
\end{lstlisting}

Atrybut \texttt{foreign} określa czy moduł jest ''obcy'' (\texttt{true}) czy ''własny'' (\texttt{false}).
Atrybut \texttt{real\_name} określa nazwę modułu tak jak będzie prezentowana na poziomie bytecode'u.
Atrybut \texttt{fns} jest listą funkcji, które są przez dany moduł eksportowane.

W elementach listy \texttt{fns} atrybuty mają następujące znaczenie:

\begin{enumerate}
    \item \texttt{arity} określa ''moc'' funkcji
    \item \texttt{name} określa nazwę funkcji widoczną z poziomu języka ViuAct
    \item \texttt{real\_name} określa pełną nazwę funkcji widoczną z poziomu bytecode'u
    \item \texttt{from\_module} określa pełną nazwę modułu, z którego pochodzi dana funkcja
\end{enumerate}

\paragraph{Plik interfejsu dla modułów ''obcych''}
\label{pliki_interfejsow_modulow_obcych}

Moduły ''obce'' języka ViuAct to moduły napisane w języku assemblera Viua VM lub w C++.

\begin{lstlisting}
{
    "foreign": true,
    "real_name": "std::posix::network",
    "fns": [
        {
            "arity": 0,
            "name": "socket",
            "real_name": "Std::Posix::Network::socket",
            "bytecode_name": "std::posix::network::socket",
            "from_module": "Std::Posix::Network"
        }
    ]
}
\end{lstlisting}

Różnicą w stosunku do plików interfejsu dla modułow ''własnych'' jest dodatkowy atrybut w deklaracji
eksportowanej funkcji -- \texttt{bytecode\_name} określający nazwę funkcji na poziomie bytecode'u. Nazwa ta
nie musi pokrywać się z nazwą w atrybucie \texttt{real\_name}, który określa pełną nazwę funkcji widoczną na
poziomie języka ViuAct.

Dwa atrybuty są potrzebne ponieważ na tym poziomie następuje łączenie dwóch ''światów''; języka ViuAct, który
narzuca reguły tego jak muszą wyglądać nazwy modułów i funkcji, oraz języka assemblera Viua VM, który takich
reguł nie narzuca. Dla modułów ''własnych'' atrybuty \texttt{bytecode\_name} jest zbędny ponieważ dla nich
nazwa widoczna na poziomie ViuAct i języka assemblera Viua VM jest taka sama.

\subsubsection{Pliki zależności modułów (\texttt{.d})}
\label{pliki_zaleznosci_modulow}

Pliki zależności są kodowane w formacie JSON.

\begin{lstlisting}
{
    "imports": [
        {
            "module_name": "Std::Random",
            "real_name": "std::random",
            "foreign": true
        }
    ]
}
\end{lstlisting}

Atrybut \texttt{imports} przechowuje listę modułów importowanych przez moduł, którego zależności definiuje
dany plik (co jest określane na podstawie nazwy pliku).

W elementach listy \texttt{imports} atrybuty mają następujące znaczenie:

\begin{enumerate}
    \item \texttt{module\_name} określa pełną nazwę modułu z punktu widzenia języka ViuAct
    \item \texttt{real\_name} określa pełną nazwę modułu widoczną z poziomu bytecode'u
    \item \texttt{foreign} określa czy moduł jest ''własny'' (\texttt{false}) czy ''obcy'' (\texttt{true})
\end{enumerate}

\section{Projekt bazy danych}

Brak bazy danych jako takiej w projekcie.
Dane przetwarzane przez kompilator są trzymane w plikach, których struktura opisana jest w rozdziałach
\ref{pliki_interfejsow_modulow} na stronie \pageref{pliki_interfejsow_modulow} i
\ref{pliki_zaleznosci_modulow} na stronie \pageref{pliki_zaleznosci_modulow}.

\section{Opis implementacji}

Implementacja kompilatora jest bardzo prosta, jak na standardy tego typu projektów.
Rysunek \ref{basic_compiler_flow} na stronie \pageref{basic_compiler_flow} bardzo dobrze oddaje strukturę
kompilatora dostarczonego jako wynik tej pracy.

\paragraph*{Analiza leksykalna i składniowa}
W fazie przygotowania tekstu programu ''do obróbki'' najpierw przeprowadzana jest analiza leksykalna (podział
tekstu na tokeny), później analiza składniowa -- która w dużej mierze polega na uporządkowaniu ''płaskiej''
listy tokenów w grupy ograniczone nawiasami. Algorytm dokonujący analizy składniowej jest trywialny:

\begin{enumerate}
    \item utwórz pustą listę przetworzoną
    \item rozważ token pierwszy nieprzenalizowany token
    \item jeśli ten token to \texttt{(} lub \texttt{\{}, rozpocznij rekurencyjnie przetwarzanie strumienia
        tokenów od następnego tokenu, a wynik analizy dodaj jako pojedynczy element do listy przetworzonej
    \item w innym przypadku dodaj token do listy przetworzonej
    \item jeśli na strumień tokenów jest pusty, zakończ algorytm
    \item w innym przypadku kontynuuj od punktu 2.
\end{enumerate}

Na pierwszy rzut oka widać, że ten algorytm nie analizuje wiele -- jego jedynym zadaniem jest pogrupowanie
płaskiego, jednowymiarowego strumienia tokenów w grupy reprezentujące konkretne konstrukcje językowe. Prostota
algorytmu sprawia, że jest on też bardzo szybki w działaniu.

Faktycznie, na tym etapie łączone są również rozbudowane identyfikatory (np. \texttt{foo.bar.baz}), ale nie
wpływa to znacząco na skomplikowanie kodu.

\paragraph*{Klasyfikacja grup}
Po tym etapie ''wstępnej'' analizy następuje klasyfikacja grup. W tym przypadku algorytm również jest bardzo
prosty. Klasyfikator rekurencyjnie sprawdza wszystkie listy i nadaje im typ oznaczający jaką konstrukcję
językową reprezentują. W tym celu musi jedynie sprawdzić typ pierwszego tokenu, i ewentualnie (w przypadku
dowiązań \emph{\texttt{let}} i definicji funkcji) długość listy.

Wykorzystanie algorytmu o tak niskim poziomie złożoności jest możliwe dzięki uważnemu zaprojektowaniu składni
języka. Każda konstrukcja językowa jest grupowana nawiasami -- klamrowymi bądź okrągłymi -- lub ma stałe
miejsce w grupie. Poniżej, dla przykładu, zaprezentowane są wzorce odpowiadające wybranym konstrukcjom
językowym:
\begin{lstlisting}
(let   %*\emph{name}*) %*\emph{expression}*))
(if    %*\emph{condition-expr}*) %*\emph{true-expr}*) %*\emph{false-expr}*))
(try   %*\emph{guarded-expr}*) ...)
(actor %*\emph{id}*) %*\emph{expr}*)*)
(+     %*\emph{lhs-expr}*) %*\emph{rhs-expr}*))
\end{lstlisting}

Jeśli całość oprócz pierwszego tokenu zostanie ''ukryta'' to dalej oczywistym jest z jaką konstrukcją mamy do
czynienia:
\begin{lstlisting}
(let   ...)
(if    ...)
(try   ...)
(actor ...)
(+     ...)
\end{lstlisting}

Wykorzystanie notacji polskiej (notacji prefiksowej) i takie zaprojektowanie składni, że każda konstrukcja ma
stałą ''szerokość'' pozwala na zastosowanie chwytu ''\emph{pierwszy token decyduje}''. Sprawia to też, że
język ma rozpoznawalne tempo i styl, a konsekwencja układu nadaje mu elegancji.
