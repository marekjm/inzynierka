\chapter{Specyfikacja języka \ViuAct}
\label{viuact_spec}
\label{specyfikacja_jezyka_viuact} % FIXME zbędny label

\ViuAct\ jest językiem kładącym silny nacisk na funkcjonalność związaną z współbieżnością opartą na modelu
aktorów. Większość konstrukcji językowych jest wyrażeniami (\emph{expressions}) zamiast ,,stwierdzeniami''
(\emph{statements}). Składnia języka przypomina składnię języka Lisp.

Najciekawsze funkcjonalności języka \ViuAct\ to:

\begin{enumerate}
    \item aktory\footnote{Dlaczego używamy słowa ,,aktory'' zamiast ,,aktorzy''
        wyjaśnione jest na stronie \pageref{glossary_actors}.}
    \item przekazywanie wiadomości
    \item system modułów
    \item funkcje zagnieżdżone i niemutowalne domknięcia
    \item dowiązania \emph{\texttt{let}}
\end{enumerate}

\paragraph*{Aktory}

Aktory są \emph{wewnętrznie sekwencyjnymi} (każdy aktor wykonuje sekwencyjny strumień instrukcji bez
jakiejkolwiek współbieżności), izolowanymi od siebie procesami, które komunikują się za pomocą
\emph{przekazywania wiadomości}.

\paragraph*{Przekazywanie wiadomości}

Przekazywanie wiadomości w \ViuAct\ jest \emph{bezpośrednie} (wiadomości są
przekazywane bezpośrednio między dwoma aktorami, bez użycia jakiegokolwiek
jawnego kanału pośredniczącego), \emph{asynchroniczne} (wiadomość jest wysyłana,
a kontrola natychmiast wraca do aktora nadawcy bez oczekiwania na odpowiedź od
odbiorcy). Przekazywanie wiadomości nie daje gwarancji dostarczenia ani
kolejności -- implementacja protokołu zapewniającego poprawność komunikacji leży
po stronie ,,przestrzeni użytkownika''.

\paragraph*{System modułów}

\ViuAct\ zapewnia programiście system modułów. Moduły są automatycznie kompilowane i linkowane, w większości z
poziomu języka, co redukuje poziom skomplikowania konfiguracji systemów budowania projektu.

\paragraph*{Funkcje zagnieżdżone i niemutowalne domknięcia}

Funkcje w \ViuAct\ mogą być zagnieżdżane, i tworzyć niemutowalne domknięcia (ang. \emph{closures}) -- funkcje,
które odwołują się do zmiennych zdefiniowanych w otaczającej je przestrzeni leksykalnej.

\paragraph*{Dowiązania \emph{\texttt{let}}}

Zmienne w \ViuAct\ są typowane dynamicznie (czyli zmienna może być dowiązana do wartości wielu typów wewnątrz
jednej funkcji), a wartości statycznie (każda wartość ma konkretny typ, który jest śledzony w czasie
kompilacji). Dowiązania \emph{\texttt{let}} do jednej nazwy mogą być tworzone wiele razy, na przykład:

\begin{lstlisting}
(let conf (Config.from_file "/etc/some.config"))
(let conf (Config.load_section conf "daemon"))
\end{lstlisting}

Po ponownym dowiązaniu nie jest możliwe dotarcie do poprzednio dowiązanej wartości, ale użycia zmiennej
,,przed'' ponownym dowiązaniem zachowują swoje znaczenie (\emph{prior art}: na takiej samej zasadzie działają
dowiązania \emph{\texttt{let}} w językach OCaml i Rust).

\section{Model programowania}
\label{specyfikacja_jezyka_viuact_model_programowania}

\subsection{Podział programu}

Najmniejszym elementem możliwym do wykonania jest wyrażenie.
Wyrażenia proste (\emph{simple expression}) są grupowane w wyrażenia złożone (\emph{compound expression}).
Funkcje parametryzują wyrażenia.
Funkcje są zebrane w modułach.

\subsection{Wykonywanie programu}

Program składa się z jednego lub większej liczby ,,aktorów'' (procesów).
Aktory są izolowane od siebie nawzajem i działają równolegle (wszystkie naraz).
Wewnętrznie, każdy z aktorów wykonywany jest sekwencyjnie, bez jakiejkolwiek
współbieżności.
Wykonanie programu rozpoczyna się od funkcji \texttt{main}.

\subsection{Komunikacja między aktorami}

Aktory komunikują się ze sobą za pomocą ,,wymiany wiadomości''. Wymiana
wiadomości ma następujące cechy:

\begin{enumerate}
    \item jest asynchroniczna (nadawca nie jest blokowany do momentu aż odbiorca otrzyma
        wiadomość)
    \item nie daje gwarancji dostarczenia (nadawca nie ma gwarancji, że jego wiadomość
        dotrze do odbiorcy)
    \item nie daje gwarancji kolejności dostarczenia (nadawca nie ma gwarancji w jakiej
        kolejności wysłane przez niego wiadomości dotrą do odbiorcy)
    \item wiadomością może być dowolna wartość, która może być reprezentowana w
        języku \ViuAct. Typy wartości opisane są w rozdziale
        \ref{viuact_spec_datatypes} na stronie \pageref{viuact_spec_datatypes}.
\end{enumerate}

Ten model wymiany wiadomości jest warunkowany faktem, że nie jest fizycznie
możliwe zagwarantowanie poprawności komunikacji.

\subsection{Obsługa błędów}

Obsługa błędów jest realizowana za pomocą wyjątków. \ViuAct\ oferuje typową
składnię \texttt{try-catch}.

\section{Typy danych}
\label{viuact_spec_datatypes}

Ten rozdział opisuje typy danych jakie mogą być używane w języku \ViuAct. Każda
wartość użyta w programie napisanym w języku \ViuAct\ jest wartością jednego z
typów wymienionych w tym rozdziale -- czasem będąc kombinacją niektórych z nich,
na przykład ,,wskaźnik do liczby całkowitej'' lub ,,wektor PID-ów''.

\subsection{Typy proste}

\subsubsection{Liczba całkowita}
\label{viuact_spec_datatypes_simple_integer}

Liczba całkowita ze znakiem, o szerokości 64 bitów, kodowana z dopełnienem do
dwóch. Liczba całkowita jest typem liczbowym.

Przy konwersji do typu boolowskiego przyjmuje fałsz dla wartości
całkowitoliczbowej \texttt{0}.

\subsubsection{Liczba zmiennoprzecinkowa}

Liczba zmiennoprzecinkowa podwójnej precyzji o szerokości 64 bitów kodowana w
standardzie IEEE~754-2008. Liczba całkowita jest typem liczbowym.

Przy konwersji do typu boolowskiego przyjmuje fałsz dla wartości
zmiennoprzecinkowej \texttt{0.0}.

\subsubsection{Wartość boolowska}

Wartość logiczna o szerokości 8 bitów reprezentująca prawdę lub fałsz.

\subsubsection{String}

Ciąg znaków Unicode, kodowany w formacie UTF-8. Maksymalna długość stringu to
$2^{64}$ znaków.

Przy konwersji do typu boolowskiego przyjmuje fałsz dla wartości
reprezentujących puste napisy (napisy o długości 0).

\subsection{Typy złożone}

\subsubsection{Wektor}

Sekwencja wartości dowolnych typów danych, indeksowana od 0. Maksymalna długość
wektora to $2^{64}$ elementów. Wektory zawierają elementy jednego typu.
Program, który wykorzystuje wektory przechowujące elementy różnych typów jest
niepoprawny.

Przy konwersji do typu boolowskiego przyjmuje fałsz dla wartości
reprezentujących puste wektory (wektory o długości 0).

\subsubsection{Struktura}

Struktura agregująca wartości dowolnych (różnych) typów danych. Struktury
ewoluują dynamicznie, a przypisanie wartości do pól struktury odbywa się w
czasie wykonywania.

Przy konwersji do typu boolowskiego przyjmuje fałsz dla wartości
reprezentujących puste struktury (struktury nie mające żadnego pola).

\subsection{Typy referencyjne}

\subsubsection{Wskaźnik}

Wskaźnik umożliwia niebezpośrednie przekazanie informacji o pewnej innej
wartości i może być użyty do odczytania bądź zmodyfikowania wartości, na którą
wskazuje.

W przeciwieństwie do wskaźników w językach takich jak C i C++ wskaźniki w języku
\ViuAct\ są bezpieczne -- nie wskazują na lokalizacje w pamięci lub rejestry, a
na \emph{wartości}.

Przy konwersji do typu boolowskiego przyjmuje fałsz dla wartości
reprezentujących ,,martwe'' wskaźniki (wskaźniki wskazujące na wartość, która
już nie istnieje).

\subsection{Typy platformy}

\subsubsection{PID}

Wartość identyfikująca aktora w programie.
Jest to typ dostarczany przez Viua VM. Dla programów pisanych w \ViuAct\ jest to
typ nieprzezroczysty\footnote{Typ nieprzezroczysty czyli tzw. \emph{opaque~type}
to typ, który nie może być analizowany i modyfikowany bezpośredni z poziomu
języka.}.

Przy konwersji do typu boolowskiego przyjmuje fałsz dla wszystkich wartości.
Używanie wartości typu PID jako wartości typu boolowskiego jest niezalecane
ponieważ taka konwersja nie ma sensu praktycznego, a udostępniana jest jedynie
dla zachowania spójności języka.

\section{Konstrukcje języka}

W tym rozdziale zaprezentowane są wszystkie konstrukcje językowe dostępne w języku \ViuAct.
Schemat opisu każdego z wyrażeń jest następujący:

\begin{enumerate}
    \item abstrakcyjna składnia w formacie EBNF
    \item opis wyrażenia
    \item przykład kodu wykorzystujący dane wyrażenie
\end{enumerate}

Przykład kodu to zazwyczaj kilka przykładów jak dana konstrukcja może wyglądać, nie kawałek działającego
programu.

\subsection{Wyrażenia}
\label{language_expressions}

\texttt{expression := \emph{literal} \\
\phantom{expression :}| \emph{name} \\
\phantom{expression :}| \emph{fn-call} \\
\phantom{expression :}| \emph{actor-call} \\
\phantom{expression :}| \emph{operator-call} \\
\phantom{expression :}| \emph{try-expr} \\
\phantom{expression :}| \emph{if-expr}}
\\
\texttt{expression := \{ \emph{expression}+ \}}
\vspace{1em}

Wyrażenia zwracają wartość, którą reprezentują. W przypadku literałów (linie od
1 do 4 na listingu \ref{lst:language_expressions_example}) jest to wartość,
którą bezpośrednio reprezentują; w przypadku innych wyrażeń (linie od 5 do 8 na
listingu \ref{lst:language_expressions_example}) wartość ta jest określana na
etapie wykonania zależnie od wcześniejszego stanu programu. Dla wyrażeń
złożonych (listing \ref{lst:language_expressions_compound_example})
obowiązują specjalne zasady opisane w rozdziale \ref{language_expressions_compound}
na stronie \pageref{language_expressions_compound}.

Większość konstrukcji jezykowych oferowanych przez \ViuAct\ może być użyta jako
wyrażenie. Wyjątki to definicje zmiennych, definicje funkcji, definicje modułów,
i przypisanie do pola struktury.

\begin{lstlisting}[caption={Przykłady wyrażeń},label={lst:language_expressions_example}]
42
true
3.14
"Hello World!"
x
(print x)
(actor server_loop sock)
(+ x 1)
\end{lstlisting}

\subsubsection{Wyrażenia złożone}
\label{language_expressions_compound}

Wyrażenia złożone są wyrażeniami składającymi się z kilku wyrażeń ograniczonych nawiasami klamrowymi.
Wyrażenia składowe wyrażenia złożonego są wykonywane po kolei, a końcowa wartość wyrażenia złożonego jest
wartością ostatniego wyrażenia składowego.

Wartością zwracaną wyrażenia złożonego przedstawionego poniżej będzie \texttt{42}:

\begin{lstlisting}[caption={Wyrażenie złożone},label={lst:language_expressions_compound_example}]
{
    (let x 42)
    (print x)
    x   ; %*wartość zwracana z tego wyrażenia jest wartością*)
        ; %*przechowywaną pod nazwą x*)
}
\end{lstlisting}

Wyrażenia złożone mogą być dowolnie zagnieżdżane. Mimo iż składają się z
potencjalnie wielu wyrażeń wyrażenia złożone są traktowane jako jedno wyrażenie
-- mogą być wykorzystywane wszędzie tam gdzie składnia języka wymaga użycia
pojedynczego wyrażenia (definicje funkcji, wyrażenia kontrolne w konstrukcji
warunkowej, wyrażenia obsługi w konstrukcji \texttt{try}, itp.). Na przykładzie
poniżej prezentowane jest dowiązanie \emph{\texttt{let}} (przedstawione
dokładniej w rozdziale \ref{viuact_spec_let_binding}) do wartości wyrażenia
złożonego:

\begin{lstlisting}
(let x {
    (let z (Std.Random.random))
    (print z)
    z
})
\end{lstlisting}

\subsection{Definicje zmiennych}
\label{viuact_spec_let_binding}

\texttt{let-binding := (let \emph{name} \emph{expression})}
\vspace{1em}

Definicje zmiennych przypisują wartości wyrażeń do nazw.

\begin{lstlisting}
(let i 42)
(let f -3.14)
(let t "Hello World!")
\end{lstlisting}

\subsection{Wywołanie operatora}

\texttt{(\emph{operator} \emph{lhs} \emph{rhs})}
\vspace{1em}

Operatory są zapisywane w konwencji prefiksowej (tj. najpierw operator, potem operandy).

\begin{lstlisting}
(+ x 1)
(* a b)
\end{lstlisting}

\subsubsection{Operatory binarne (dwuargumentowe) logiczne}

Binarne operatory logiczne pobierają dwie wartości, i produkują wartość logiczną.

\begin{labeling}{and}
    \item[\texttt{or}] pobiera dwie dowolne wartości; \texttt{(or a b)}
    \item[\texttt{and}] pobiera dwie dowolne wartości; \texttt{(and a b)}
    \item[\texttt{=}] pobiera dwie wartości liczbowe; \texttt{(= a b)}
    \item[\texttt{!=}] pobiera dwie wartości liczbowe; \texttt{(!= a b)}
    \item[\texttt{<}] pobiera dwie wartości liczbowe; \texttt{(< a b)}
    \item[\texttt{<=}] pobiera dwie wartości liczbowe; \texttt{(<= a b)}
    \item[\texttt{>}] pobiera dwie wartości liczbowe; \texttt{(> a b)}
    \item[\texttt{>=}] pobiera dwie wartości liczbowe; \texttt{(>= a b)}
\end{labeling}

\subsubsection{Operatory binarne (dwuargumentowe) arytmetyczne}

Operatory arytmetyczne pobierają dwie wartości liczbowe i produkują wartość
liczbową zgodną z typem operandu lewej strony (\emph{left-hand side}).

\begin{lstlisting}
let op : Ta -> Tb -> Ta
\end{lstlisting}

Przed wykonaniem operacji wartość operandu prawej strony (\emph{right-hand side})
jest konwertowana na wartość odpowiadającą typowi operandu lewej strony. Liczby
zmiennoprzecinkowe są zaokrąglane w dół (\emph{floor}) podczas konwersji na
liczby całkowite.

\begin{labeling}{+}
    \item[\texttt{+}] dodaje dwie wartości liczbowe
    \item[\texttt{-}] odejmuje dwie wartości liczbowe, \emph{\texttt{rhs}} jest odejmowany od
        \emph{\texttt{lhs}}
    \item[\texttt{*}] mnoży dwie wartości liczbowe
    \item[\texttt{/}] dzieli dwie wartości liczbowe, \emph{\texttt{lhs}} jest dzielony przez
        \emph{\texttt{rhs}}
\end{labeling}

\subsubsection{Operator dostępu}

\texttt{id := \emph{name}} \\
\texttt{id := \emph{id} . \emph{name}}

\begin{lstlisting}
(let x a_struct.some_field)
(let sock (Std.Posix.Network.socket))
\end{lstlisting}

\begin{description}
    \item[\texttt{.}] operator dostępu do składowych modułów i pól struktur, w zależności od kontekstu
\end{description}

\subsubsection{Operatory unarne (jednoargumentowe)}

\begin{description}
    \item[\texttt{not}] negacja logiczna, pobiera dowolną wartość i produkuje jej odwrotność logiczną
    \item[\texttt{\^}] dereferencja wskaźnika
\end{description}

\subsection{Przypisanie do pola struktury}

\texttt{(:= \emph{field} \emph{expression})}
\vspace{1em}

Operator przypisania do pola struktury powoduje utworzenie żądanego pola w strukturze (jeśli wcześniej nie
istniało) lub aktualizację wartości przechowywanej w tym polu (jeśli wcześniej istniało).

\begin{lstlisting}
(let x (struct))
(:= x.field 42)
\end{lstlisting}

\subsection{Definicje funkcji}

\texttt{function-definition := (let \emph{name} (\emph{formal-parameters}) \emph{expression})}
\vspace{1em}

O funkcjach w \ViuAct\ można myśleć jak o parametryzowanych wyrażeniach.

\begin{lstlisting}
(let f (x) (+ x 1))
(let f_with_print (x) {
    (print x)
    (+ x 1)
})
(let apply (f x) (f x))
\end{lstlisting}

\subsubsection{Parametry formalne}

\texttt{formal-parameters := \emph{name}*}
\vspace{1em}

Parametry formalne (\emph{formal parameters}) są nazwami zmiennych widocznymi
wewnątrz funkcji. Parametry faktyczne (\emph{actual parameters}) są wartościami
przypisywanymi parametrom formalnym na etapie wykonania przez wywołującego.
Parametry formalne przyjęło się nazywać \emph{parametrami}, a parametry
faktyczne \emph{argumentami}.

Powyższe definicje zostały zaczerpnięte z rozdziału 8.3 ,,Parameter~Passing'' w
\cite{ProgrammingLanguagePragmatics}.

\subsubsection{Przekazywanie parametrów}

Przekazywanie parametrów do funkcji odbywa się przez kopiowanie
(\emph{call-by-value} w \cite{ProgrammingLanguagePragmatics}). Kopia każdego
argumentu (parametru faktycznego) jest dowiązywana (jak w dowiązaniach
\emph{let}) do nazwy odpowiadającego mu parametru (parametru formalnego).

Możliwe jest przekazanie ,,przez referencję'' (\emph{call-by-reference} w
\cite{ProgrammingLanguagePragmatics}) za pomocą wskaźników -- w takim wypadku to
wskaźnik będzie skopiowany, nie wartość, na którą on wskazuje.

\subsubsection{Ciało funkcji}

Ciało funkcji jest reprezentowane jednym wyrażeniem (może to być wyrażenie
złożone). Wyrażenia są opisane w rozdziale \ref{language_expressions} na stronie
\pageref{language_expressions}.

\subsubsection{Wartość zwracana}

Wartość zwracana z funkcji jest wartością, do której ewaluuje wyrażenie będące
jej ciałem.

\subsubsection{Rekurencja}

W języku \ViuAct\ nie istnieją pętle. Sposobem na uzyskanie iteracji jest
rekurencja, ze szczególnym uwzględnieniem rekurencji ogonowej (opisanej w
rozdziale \ref{viuact_spec_tail_call} na stronie
\pageref{viuact_spec_tail_call}).

\subsection{Definicje modułów}
\label{viuact_spec_module_definition}

\texttt{inline-module-definition := (module \emph{module-name} (
\newline
\phantom{inline-module-definition := ~}\emph{module-definition}*
\newline
\phantom{inline-module-definition := ~}\emph{module-import}*
\newline
\phantom{inline-module-definition := ~}\emph{function-definition}+
))}
\newline
\texttt{module-definition := (module \emph{module-name}) \\
\phantom{module-definition :}| \emph{inline-module-definition}}
\newline
\texttt{module-import := (import \emph{module-name})}
\newline

Moduł jest zbiorem definicji modułów, deklaracji import, i definicji funkcji.
Taka kolejność w kodzie źródłowym nie jest wymagana, ale kompilator w takiej kolejności przetwarza konstrukcje
składające się na definicję modułu.

\subsection{Wywołanie funkcji}
\label{viuact_spec_fn_call}

\texttt{fn-call := (\emph{id} \emph{expression}*)}
\newline

Wywołanie funkcji jest reprezentowane przez podanie jej nazwy w nawiasach
okrągłych, po której podane jest zero lub więcej wyrażeń, których wartości
zostaną przypisane do odpowiednich parametrów aktualnych.

Funkcje, których definicje znajdują się w innych modułach niż aktualnie
przetwarzany muszą być poprzedzone pełną ścieżką do modułu, w którym są
zdefiniowane (np. \texttt{Std.Posix.Network.socket}). Jedynie funkcje znajdujące
się w tym samym module muszą być wywoływane bez pełnej ścieżki.

\subsection{Wywołanie ,,tailcall''}
\label{viuact_spec_tail_call}

\texttt{(tailcall \emph{id} \emph{expression}*)}
\newline

Wywołania \emph{tailcall} powodują wywołanie funkcji ,,wykorzystując'' ramkę
obecnej funkcji. W praktyce powodują natychmiastowe zdjęcie obecnej ramki ze
stosu (co powoduje destrukcję wszystkich wartości w rejestrach lokalnych obecnej
ramki, oraz uruchomienie wszelkich wywołań odroczonych) i wepchnięcie na szczyt
stosu ramki dla funkcji wywołanej \emph{tailcall}.

Wywołanie \emph{tailcall} nie ma bezpośredniej wartości zwracanej, ponieważ nie
da się jej zaobserwować. Wywołanie \emph{tailcall} ,,porywa'' (ang. ,,hijacks'')
punkt, do którego powinna zwrócić wartość funkcja, która wykonała takie
wywołanie -- i dopiero ten punkt zaobserwuje de facto wartość zwróconą przez
wywołanie \emph{tailcall}, ale z jego punktu widzenia wygląda to tak jakby tą
wartość zwróciła funkcja, którą on bezpośrednio wywołał.

\begin{lstlisting}
(let g (x) (+ x 1))

(let f (x) (tailcall g x))  ; ramka %*wywołania*) funkcji f()
                            ; %*zastępowana ramką*) g()
(let fa (x) (g x))          ; funkcja f() bez tail cal

(let main () {
    (let i (f 41))  ; w tym miejscu "i" jest przypisane do
                    ; wyniku g()
    0
})
\end{lstlisting}

Podmiana ramki jest niedostrzegalna dla wywołującego oryginalną funkcję.

\subsubsection{Zastosowania}

Wywołania \emph{tailcall} są wykorzystywane do implementacji iteracji (co jest
niezbędne przy przetwarzaniu wektorów), oraz przy obsłudze błędów za pomocą
funkcji \emph{watchdog} (opisanych w rozdziale \ref{viuact_spec_watchdog_call}
na stronie \pageref{viuact_spec_watchdog_call}) gdzie pozwala na podmianę
funkcji głównej bez zmiany PID aktora.

Wywołania \emph{tailcall} są również przydatne w typowym wzorcu wykorzystywanym
w języku \ViuAct\ czyli \emph{init-impl} -- ,,inicjalizacja i implementacja''.
Funkcja inicjalizująca tworzy i konfiguruje nowego aktora, a następnie wykonuje
wywołanie w pozycji ogonowej do funkcji implementacji. Po zakończeniu
inicjalizacji funkcja przeprowadzająca ją nie jest już potrzebna więc jej ramka
może zostać zastąpiona ramką innej funkcji bez żadnego problemu.

\subsection{Wywołanie odroczone}
\label{viuact_spec_deferred_call}

\texttt{(defer \emph{name} \emph{expression}*)}
\newline

Wywołania odroczone nie są wywoływane od razu (jak zwykłe wywołania funkcji, bądź wywołania \emph{tailcall}),
ale dopiero w momencie, w którym ramka, w której zostały zarejestrowane jest zdejmowana ze stosu, niezależnie
od powodu dla którego jest ona zdejmowana -- czy będzie to zwykły powrót, odwinięcie stosu wywołane przez
wyjątek, czy wywołanie \emph{tailcall}.

Wywołania odroczone są wykonywane w odwrotnej kolejności rejestracji. Jeśli odroczone wywołanie \texttt{f()}
zostanie zarejestrowane przed odroczonym wywołaniem \texttt{g()}, to zostanie wykonane po nim. Dla przykładu:
\begin{lstlisting}
(let f () (print "from f()"))
(let g () (print "from g()"))
(let h () {
    (defer f)
    (defer g)
    (print "from h()")
    0
})
\end{lstlisting}

Wywołując funkcję \texttt{h()} na standardowym wyjściu pojawi się następujący tekst:
\begin{lstlisting}
from h()
from g()
from f()
\end{lstlisting}

Funkcja uruchomiona w wywołaniu odroczonym może zarejestrować swoje wywołania
odroczone; takie wywołania mogą być zagnieżdżane bez ograniczeń.

Wywołania odroczone są uruchamiane po zakończeniu działania funkcji, w której
zostały zarejestrowane, ale przed zniszczeniem wartości, które były umieszczone
w jej zmiennych lokalnych. Ten fakt może zostać wykorzystany do implementacji
zarządzania zasobami -- jeśli wywołanie odroczone ma jakikolwiek uchwyt
(np.~wskaźnik) do zasobu to może wykonać dla niego jakąś logikę finalizacji.
W takim celu po utworzeniu zasobu najlepiej od razu zarejestrować wywołanie
odroczone, które dokona jego ,,finalizacji''.

\begin{lstlisting}
(let f () {
    ; ...

    (let x (Some_module.reserve_resource))
    (defer Some_module.free_resource (Std.pointer x))

    ; ...
})
\end{lstlisting}

\subsubsection{Wywołania odroczone, a \emph{tailcall}}

Wywołania odroczone są związane z ramką wywołania, w której zostały
zarejestrowane. Jeśli ta ramka jest zastępowana inną z powodu wywołania
\emph{tailcall} tu tuż \textbf{przed} zastąpieniem ramki uruchamiane są
wszystkie zarejestrowane wywołania odroczone.

\subsection{Wywołanie aktora}
\label{viuact_spec_actor_call}

\texttt{(actor \emph{name} \emph{expression}*)}

\subsubsection{Fork-join}

Metoda \emph{fork-join} może być zrealizowana w \ViuAct\ za pomocą aktorów.
Jeśli wynik wywołania aktora jest przypisywany do zmiennej, to aktor będzie
działał jako \emph{aktor potomny} aktora, który go wywołał. Taki aktor potomny
może być włączony (\emph{join}) za pomocą wywołania funkcji wbudowanej
\texttt{Std.Actor.join()} z biblioteki standardowej (opisanej na
stronie~\pageref{Std_Actor_join}).

\begin{lstlisting}
(let p (actor foo x))           ; %*aktor stworzony z funkcji foo*)
                                ; %*wykonuje się równolegle*)
(let result (Std.Actor.join p)) ; %*wynik pracy aktora będzie dostępny*)
                                ; %*w zmiennej result*)
\end{lstlisting}

Funkcja \texttt{Std.Actor.join()} zwraca wartość, którą zwróciła funkcja
uruchomiona jako aktor lub wyjątek, który tego aktora zabił. Aktor pochodny nie
utworzy zrzutu stosu -- w razie awarii jego wyjątek jest zachowywyany i czeka na
włączenie.

\subsubsection{Wolne aktory}

Wolne aktory to aktory, których adres nie był przypisany do zmiennej w momencie ich utworzenia. Są one
niezwiązane w żaden sposób z aktorem, który je utworzył. W przeciwieństwie do aktorów potomnych aktorów
wolnych nie można włączyć za pomocą funkcji \texttt{Std.Actor.join()}.

Aktory wolne w razie awarii tworzą zrzuty stosu.

Typowym sposobem na uzyskanie adresu aktora wolnego jest takie zaprojektowanie programu żeby aktor wolny jako
jeden z parametrów przyjmował PID, na który powinien wysłać \emph{swój} PID po wystartowaniu. Dla przykładu:

\begin{lstlisting}
(actor Some_module.free_actor (Std.Actor.self))
(let p (Std.Actor.receive))
\end{lstlisting}

Funkcja \texttt{Std.Actor.self()} pozwala aktoromi uzyskać swój własny PID.

\subsection{Konstrukcja warunkowa}

\texttt{(if \emph{condition-expression} \emph{true-arm} \emph{false-arm})}
\newline

Konstrukcja warunkowa składa się z trzech wyrażeń:

\begin{description}
    \item[\texttt{condition-expression}] wyrażenie, które po obliczeniu warunkuje, które wyrażenie będzie
        obliczane w następnym kroku
    \item[\texttt{true-arm}] wyrażenie będzie obliczone jeśli \texttt{condition-expression} ewaluuje do prawdy
    \item[\texttt{false-arm}] wyrażenie będzie obliczone jeśli \texttt{condition-expression} ewaluuje do
        fałszu
\end{description}

\begin{lstlisting}
(if (= x 42)
    (print "%*Odpowiedź na sens życia, Wszechświata, i w ogóle wszystkiego.*)")
    (print "Nie bardzo.")
)
\end{lstlisting}

\subsection{Wywołanie ,,watchdog''}
\label{viuact_spec_watchdog_call}

\texttt{(watchdog \emph{name} \emph{expression}*)}
\newline

Watchdog jest ostatnią linią obrony danego aktora przed niekontrolowanym zamknięciem w wyniku błędu programu.
Jeśli aktor nie obsłużył błędu (reprezentowanego przez wyjątek) i jego stos wywołań został całkowicie
odwinięty, ale aktor ma ustawiony watchdog to VM ,,w miejscu'' podmieni odwinięty stos na wywołanie funkcji
ustawionej jako watchdog.

Oznacza to, że aktor zamiast zostać zabity zacznie wykonywać funkcję ustawioną jako watchdog -- jednocześnie
zachowując PID, co może być bardzo przydatne.

Watchdog jest normalną funkcją i może korzystać ze wszystkich dozwolonych konstrukcji językowych.
Jedynym ograniczeniem jest to, że watchdog musi być napisany w języku \ViuAct\ lub w języku assemblera Viua VM.

\paragraph*{Argumenty funkcji watchdog}

\begin{lstlisting}
(let use_as_watchdog (error controller) {
    (let report (struct))
    (:= report.error error)
    (:= report.pid (Std.Actor.self))
    (Std.Actor.send controller error)
})

(let some_other_function (controller) {
    (watchdog use_as_watchdog controller)
})
\end{lstlisting}

Jako pierwszy argument watchdog zawsze otrzymuje \texttt{struct} zawierający informacje o błędzie, który
spowodował awarię aktora. Jako pozostałe argumenty otrzymuje to co zostało przekazane w wywołaniu.
Czyli ,,pierwszy'' argument użyty w wywołaniu będzie tak naprawdę drugi z punktu widzenia funkcji użytej jako
watchdog.

\paragraph*{Błędy wewnątrz watchdog}

Jeśli podczas wykonywania funkcji ustawionej jako watchdog wystąpi kolejny nieobsłużony błąd aktor zostanie
natychmiast zabity, a watchdog nie będzie uruchomiony przez VM ,,z powodu samego siebie''.

\subsection{Obsługa wyjątków}

\texttt{(try \emph{expression} ((catch \emph{exception-tag} \emph{name} \emph{expression})+))}
\vspace{1em}

Obsługa wyjątków w \ViuAct\ jest mechanizmem bardzo prostym. W momencie, w którym wyjątek jest zgłaszany
stos wywołań jest odwijany w poszukiwaniu pierwszego bloku \texttt{catch} z odpowiadającym mu
tagiem (\texttt{\emph{exception-tag}}).

Kiedy taki blok jest znaleziony, wyłapana wartość wyjątku jest dowiązywana do nazwy (\texttt{\emph{name}})
określonej w bloku. Wyjątek jest odrzucany (jeśli nie jest potrzebny) jeśli nazwa to
,,\texttt{\_}''\footnote{Podobne zachowanie, czyli odrzucenie wartości jeśli nazwa dowiązania to
\texttt{\_}, można odnaleźć w języku Erlang.}

\begin{lstlisting}
(let x (try (Std.Actor.receive 1s)) (
    (catch Bad_error _ {
        (print "Something bad happened.")
        0   ; return a default value
    })
    (catch Worse_error _ {
        (print "Something worse happened!")
        0   ; return a default value
    })
    (catch The_worst_error _ {
        (print "Apocalypse.")
        0   ; return a default value
    })
))
\end{lstlisting}

\subsection{Wyliczenia}

\texttt{(enum \emph{enum-name} (\emph{enum-member}+))}
\newline

Wyliczenia przydatne są do reprezentowania pewnego skończonego zbioru możliwych wartości. Ułatwiają one
zachowanie porządku w programie dzięki umożliwieniu wyliczenia możliwych stanów pewnych zmiennych, oraz
redukują możliwość popełnienia błędu dzięki utrzymaniu jednej definicji tych stanów.

\begin{lstlisting}
(enum Good_languages (
    Cpp
    OCaml
    ViuAct
))

(enum Bad_languages (
    Perl
    Bash
    JavaScript
))
\end{lstlisting}

Wyliczenia mogą pojawić się jedynie na poziomie modułu, w tym na poziomie modułu anonimowego niejawnie
dodawanego do plików z kodem źródłowym kompilowanym do plików wykonywalnych. Wyliczeń nie można tworzyć
wewnątrz funkcji.

Wyliczenia są symbolicznymi wartościami liczbowymi, dlatego można porównywać je
za pomocą normalnego operatora porównania:

\begin{lstlisting}
(let is_viuact (x) (= x Good_languages.ViuAct))
\end{lstlisting}

\subsection{Komentarze}

\texttt{; \emph{treść komentarza}}
\newline

Komentarz rozpoczyna się od znaku \texttt{;} i kończy wraz z końcem linii.
Nie istnieją komentarze ,,blokowe'' takie jak \texttt{/* komentarz */} z C++ czy
\texttt{(* komentarz *)} z OCaml.

\section{Biblioteka standardowa}

Biblioteka standardowa języka \ViuAct\ składa się z modułów implementujących
wybrane algorytmy (np. sortowanie) i funkcjonalności (np. konwersje między
typami danych), oraz umożliwia interakcję ze środowiskiem zewnętrznym gdyż
\ViuAct\ nie posiada wbudowanych w język mechanizmów takiej interakcji.

Część funkcji udostępnianych przez bibliotekę standardową jest niemożliwa do zaimplementowania w języku
\ViuAct. Takie funkcje są napisane albo w języku assemblera Viua VM, albo w języku C++.

Nie wszystkie funkcje biblioteki standardowej są ,,normalnymi'' funkcjami, możliwymi do uruchomienia jako
aktor ponieważ niektóre z nich kompilują się do pojedynczych instrukcji maszyny (funkcje ,,wbudowane'') albo
są napisane w języku C++, który jest niemożliwy do wywłaszczenia przez Viua VM (funkcje ,,obce'').

Dla takich funkcji programista musi napisać \emph{wrappery} jeśli chce użyć ich jak w niektórych kontekstach.
Jednak w typowym programie nie będzie to potrzebne, gdyż funkcje biblioteki standardowej zazwyczaj
implementują atomowe zadania, których sens uruchamiania w osobnych aktorach jest ograniczony.

\begin{center}
\emph{Należy pamiętać, że opis biblioteki standardowej \textbf{nie} jest ostateczny i może ulec zmianie w
trakcie trwania projektu. Z tego powodu funkcje są opisane skrótowo, a dokumentacja nie jest wyczerpująca.}
\end{center}

Moduł \texttt{Std.Actor} jest opisany na stronie \pageref{stdlib_Std_Actor}.
Moduł \texttt{Std.Io} jest opisany na stronie \pageref{stdlib_Std_Io}.
Moduł \texttt{Std.Posix.Network} jest opisany na stronie \pageref{stdlib_Std_Posix_Network}.
Moduł \texttt{Std.Random} jest opisany na stronie \pageref{stdlib_Std_Random}.

\subsection{\texttt{Std.Actor}}
\label{stdlib_Std_Actor}

Moduł udostępniający funkcjonalność związaną z wymianą wiadomości i modelem aktorów.

\subsubsection{\texttt{Std.Actor.join()}}
\label{Std_Actor_join}

\begin{small}
\begin{lstlisting}
Std.Actor.join(Pid) -> Value
\end{lstlisting}
\end{small}

Oczekuje na zakończenie wykonywania przez aktora o podanym PID, po czym zwraca wynik jego pracy.
Może powodować zgłoszenie wyjątku jeśli aktor o podanym PID zakończył wykonywanie z powodu błędu, lub jeśli aktor
o podanym PID nie istnieje.

\subsubsection{\texttt{Std.Actor.self()}}

\begin{small}
\begin{lstlisting}
Std.Actor.self() -> Pid
\end{lstlisting}
\end{small}

Zwraca PID obecnego aktora.

\subsubsection{\texttt{Std.Actor.receive()}}

\begin{small}
\begin{lstlisting}
Std.Actor.receive() -> Value
Std.Actor.receive(Timeout) -> Value
\end{lstlisting}
\end{small}

Oczekuje na nadejście wiadomości; wariant 1. oczekuje w nieskończoność,
wariant 2. oczekuje do momentu upłynięcia czasu podanego w parametrze.

\begin{small}
\begin{lstlisting}
(let wait_1_second        (Std.Actor.receive 1s))
(let wait_16_milliseconds (Std.Actor.receive 16ms))
(let wait_forever         (Std.Actor.receive))
\end{lstlisting}
\end{small}

Parametr \texttt{Timeout} jest liczbą całkowitą podającą ile sekund
(suffix~\texttt{s}) lub milisekund (suffix~\texttt{ms}) aktor powinien
maksymalnie oczekiwać na wiadomość. Z uwagi ma nieprzewidywalność zarządzania
czasem procesora realny czas oczekiwania może się wydłużyć, jednak nigdy nie
będzie krótszy niż podany w parametrze.

\subsection{\texttt{Std.Io}}
\label{stdlib_Std_Io}

Moduł udostępniający mechanizmy I/O (\emph{wejścia-wyjścia}).
Umożliwia on wykonywanie operacji I/O na plikach, oraz prostą interakcję z
konsolą użytkownika (\texttt{tty}). I/O jest buforowane. Ten moduł nie jest
bezpośrednim opakowaniem dla wywołań systemowych I/O definiowanych przez
standard POSIX (np. \texttt{read(3)} lub \texttt{write(3)}).

\subsubsection{\texttt{Std.Io.stdin\_getline()}}

\begin{small}
\begin{lstlisting}
Std.Io.stdin_getline() -> String
\end{lstlisting}
\end{small}

Umożliwia odczytanie pojedynczej linii ze strumienia standardowego wejścia (\emph{\texttt{stdin}}).

\subsubsection{\texttt{Std.Io.open()}}

\begin{small}
\begin{lstlisting}
Std.Io.open(String) -> Fstream
\end{lstlisting}
\end{small}

Otwiera plik zdefiniowany ścieżką podaną w parametrze.

\subsubsection{\texttt{Std.Io.peek()}}

\begin{small}
\begin{lstlisting}
Std.Io.peek(Fstream) -> String
\end{lstlisting}
\end{small}

Zwraca pierwszy znak w strumieniu.

\subsubsection{\texttt{Std.Io.getline()}}

\begin{small}
\begin{lstlisting}
Std.Io.getline(Fstream) -> String
Std.Io.getline(Fstream, String) -> String
\end{lstlisting}
\end{small}

Wariant 1. odczytuje znaki ze strumienia do napotkania bajtu białej linii (\texttt{\\n}).
Wariant 2. odczytuje znaki ze strumienia do napotkania bajtu podanego w parametrze.

\subsubsection{\texttt{Std.Io.read()}}

\begin{small}
\begin{lstlisting}
Std.Io.read(Fstream) -> String
Std.Io.read(Fstream, Integer) -> String
\end{lstlisting}
\end{small}

Wariant 1. odczytuje naraz całą zawartość strumienia (cały plik).
Wariant 2. odczytuje maksymalnie \emph{n} bajtów.

\subsubsection{\texttt{Std.Io.write()}}

\begin{small}
\begin{lstlisting}
Std.Io.write(Fstream, String) -> void
\end{lstlisting}
\end{small}

Zapisuje string do strumienia.

\subsection{\texttt{Std.Posix.Network}}
\label{stdlib_Std_Posix_Network}

Moduł udostępniający implementację ,,POSIX sockets''. Jest to cienka abstrakcja nad API dostarczanym przez
system operacyjny; w przypadku braków w tym dokumencie ich dokumentację można wydedukować ze stron manuala
sekcji 3 dostarczanych przez program \texttt{man(1)} (np. dokumentację dla funkcji
\texttt{Std.Posix.Network.socket} można uzyskać wykonując polecenie \texttt{man 3 socket}).

\subsubsection{\texttt{Std.Posix.Network.socket()}}

\begin{small}
\begin{lstlisting}
Std.Posix.Network.socket() -> Integer
\end{lstlisting}
\end{small}

Tworzy socket za pomocą wywołania funkcji \texttt{socket(3)}.
Socket jest w rodzinie \texttt{AF\_INET}, rodzaju \texttt{SOCK\_STREAM}, i jest tworzony bez flag.

Socket zwracany przez tą funkcję jest blokujący.

\subsubsection{\texttt{Std.Posix.Network.connect()}}

\begin{small}
\begin{lstlisting}
Std.Posix.Network.connect(
      socket : Integer
    , addr   : Text
    , port   : Integer
) -> Void
\end{lstlisting}
\end{small}

Wrapper na funkcję \texttt{connect(3)}.

\subsubsection{\texttt{Std.Posix.Network.bind()}}

\begin{small}
\begin{lstlisting}
Std.Posix.Network.bind(
      socket : Integer
    , addr   : Text
    , port   : Integer
) -> Void
\end{lstlisting}
\end{small}

Wrapper na funkcję \texttt{bind(3)}.

\subsubsection{\texttt{Std.Posix.Network.listen()}}

\begin{small}
\begin{lstlisting}
Std.Posix.Network.listen(
      socket  : Integer
    , backlog : Integer
) -> Void
\end{lstlisting}
\end{small}

Wrapper na funkcję \texttt{listen(3)}.

\subsubsection{\texttt{Std.Posix.Network.accept()}}

\begin{small}
\begin{lstlisting}
Std.Posix.Network.accept(socket : Integer) -> Void
\end{lstlisting}
\end{small}

Wrapper na funkcję \texttt{accept(3)}.

Sockety zwracane przez tą funkcję są nieblokujące. Ich timeout jest ustawiony na 500ms.

\subsubsection{\texttt{Std.Posix.Network.write()}}

\begin{small}
\begin{lstlisting}
Std.Posix.Network.write(
      socket : Integer
    , value  : Value
) -> Int
\end{lstlisting}
\end{small}

Niezależnie od tego jaka wartość jest przekazana do funkcji, będzie najpierw przekonwertowana na
\texttt{String}, a potem wpisana do socketu.

\subsubsection{\texttt{Std.Posix.Network.read()}}

\begin{small}
\begin{lstlisting}
Std.Posix.Network.read(
    socket  : Integer
) -> String
\end{lstlisting}
\end{small}

Wrapper na funkcję \texttt{read(3)}. Odczytuje z socketu maksymalnie 1024 bajty.

\subsubsection{\texttt{Std.Posix.Network.recv()}}

\begin{small}
\begin{lstlisting}
Std.Posix.Network.recv(
      socket        : Integer
    , buffer_length : Integer
) -> String
\end{lstlisting}
\end{small}

Wrapper na funkcję \texttt{recv(3)}. Odczytuje z socketu maksymalnie \texttt{buffer\_length} bajtów.

\subsubsection{\texttt{Std.Posix.Network.shutdown()}}

\begin{small}
\begin{lstlisting}
Std.Posix.Network.listen(
    socket  : Integer
) -> Void
\end{lstlisting}
\end{small}

Wrapper na funkcję \texttt{shutdown(3)}.
Wyłącza socket z flagą \texttt{SHUT\_RDWR}.

\subsubsection{\texttt{Std.Posix.Network.close()}}

\begin{small}
\begin{lstlisting}
Std.Posix.Network.listen(
      socket  : Integer
) -> Void
\end{lstlisting}
\end{small}

Wrapper na funkcję \texttt{close(3)}.


\subsection{\texttt{Std.Random}}
\label{stdlib_Std_Random}

Moduł udostępniający dostęp do liczb losowych ogólnego przeznaczenia (\texttt{/dev/urandom}) oraz zdatnych do
zastosowań kryptograficznych (\texttt{/dev/random}).

\subsubsection{\texttt{Std.Random.random()}}

\begin{small}
\begin{lstlisting}
Std.Random.random() -> Float
\end{lstlisting}
\end{small}

Zwraca losową liczbę zmiennoprzecinkową z przedziału $[0.0, 1.0)$.

\subsubsection{\texttt{Std.Random.randint()}}

\begin{small}
\begin{lstlisting}
Std.Random.randint(lower : Integer, upper : Integer) -> Integer
\end{lstlisting}
\end{small}

Zwraca losową liczbę całkowitą z przedziału [\texttt{\emph{lower}}, \texttt{\emph{upper}}).

\section{Interakcja z platformą}

,,Platformą'' w przypadku \ViuAct\ jest Viua VM. Narzuca to pewne ograniczenia związane ze środowiskiem, w
którym programy napisane w języku \ViuAct\ można skompilować i uruchomić. Wynika to z faktu, że Viua VM sama w
sobie ma pewną platformę i wymagania.

\subsection{Platforma kompilacji}

Do kompilacji programów napisanych w \ViuAct\ potrzebny jest interpreter języka Python 3 w wersji co najmniej
3.6. Jest to jedyne ograniczenie, ponieważ na etapie kompilacji \ViuAct\ nie odwołuje się do żadnego elementu
Viua VM.

\subsection{Platforma asemblacji i linkowania}

Asemblacja i linkowanie odbywa się z użyciem narzędzi dostarczanych przez Viua VM.
Oprócz interpretera języka Python 3 potrzebny jest assembler dostarczany przez Viua VM. Oznacza to, że etap
asemblacji i linkowania może odbywać się jedynie na platformie zgodnej z POSIX-2008, wyposażonej w kompilator
obsługujący standard C++17 (są to wymagania Viua VM).

\subsection{Platforma uruchomieniowa}

Uruchomienie odbywa się w całości przy użyciu narzędzi dostarczanych przez Viua VM.
Ograniczenia i zależności są na tym etapie całkowicie zależne od wymagań narzucanych przez Viua VM.
