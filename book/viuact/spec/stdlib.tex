\section{Biblioteka standardowa}

Biblioteka standardowa języka \ViuAct\ składa się z modułów implementujących
wybrane algorytmy (np. sortowanie) i funkcjonalności (np. konwersje między
typami danych), oraz umożliwia interakcję ze środowiskiem zewnętrznym gdyż
\ViuAct\ nie posiada wbudowanych w język mechanizmów takiej interakcji.

Część funkcji udostępnianych przez bibliotekę standardową jest niemożliwa do
zaimplementowania w języku \ViuAct. Takie funkcje są napisane albo w języku
assemblera Viua VM, albo w języku C++.

Nie wszystkie funkcje biblioteki standardowej są ,,normalnymi'' funkcjami,
możliwymi do uruchomienia jako aktor lub przekazania jako argument. Niektóre z
nich kompilują się do pojedynczych instrukcji maszyny (są to tzw. funkcje
,,wbudowane'' opisane w rozdziale \ref{stdlib_builtin_fns} na
stronie~\pageref{stdlib_builtin_fns}).

Dla takich funkcji programista musi napisać \emph{wrappery} jeśli chce użyć ich jak w niektórych kontekstach.
Jednak w typowym programie nie będzie to potrzebne, gdyż funkcje biblioteki standardowej zazwyczaj
implementują atomowe zadania, których sens uruchamiania w osobnych aktorach jest ograniczony.

Moduł \texttt{Std.Io} jest opisany na stronie \pageref{stdlib_Std_Io}.
Moduł \texttt{Std.Posix.Network} jest opisany na stronie \pageref{stdlib_Std_Posix_Network}.
Moduł \texttt{Std.Random} jest opisany na stronie \pageref{stdlib_Std_Random}.

\subsection*{Konwencje}

Konwencją wykorzystywaną w specyfikacji funkcji biblioteki standardowej jest
następujący sposób zapisu:
\texttt{\emph{nazwa-funkcji}(\emph{typ-parametru}*) -> \emph{typ-zwracany}}
Jeśli jest ich więcej niż jeden, typy parametrów są oddzielane od siebie
przecinkiem, np.
\texttt{funkcja(Ta, Tb) -> Tc}.

Jeśli funkcja przyjmuje dowolny typ oznaczany jest on jako \texttt{T}. Jeśli
takich dowolnych typów jest więcej oznaczane są jako \texttt{Ta}, \texttt{Tb}
itd.

Jeśli ma znaczenie nazwa parametru formalnego jego nazwa jest podawana przed
jego typem i oddzielona od niego znakiem dwukropka:
\texttt{\emph{nazwa}~:~T}.

\subsection{Funkcje wbudowane}
\label{stdlib_builtin_fns}

Funkcje wbudowane są funkcjami, które kompilują się do pojedynczej instrukcji
Viua~VM. Są one dostępne zawsze, bez potrzeby importowania modułów, w których
się znajdują.

\subsubsection{\texttt{print()}}
\label{stdlib_builtin_fns_print}

\begin{small}
\begin{lstlisting}
print(T) -> void
\end{lstlisting}
\end{small}

Drukuje pojedynczą wartość na ekran. Wydruk kończy znakiem nowej linii
(\texttt{\textbackslash{}n}).

\subsubsection{\texttt{Std.Actor.join()}}
\label{Std_Actor_join}

\begin{small}
\begin{lstlisting}
Std.Actor.join(Pid) -> T
\end{lstlisting}
\end{small}

Oczekuje na zakończenie wykonywania przez aktora o podanym PID, po czym zwraca
wynik jego pracy. Może powodować zgłoszenie wyjątku jeśli aktor o podanym PID
zakończył wykonywanie z powodu błędu, lub jeśli aktor o podanym PID nie
istnieje.

\subsubsection{\texttt{Std.Actor.receive()}}

\begin{small}
\begin{lstlisting}
Std.Actor.receive() -> T
Std.Actor.receive(%*\emph{timeout-literal}*)) -> T
\end{lstlisting}
\end{small}

Zwraca pierwszą dostępną wartość w kolejce wiadomości procesu wewnątrz którego
ta funkcja została wywołana. Jeśli żadna wiadomość nie jest dostępna funkcja
oczekuje w nieskończoność na nadejście wiadomości.

W wariancie drugim, oczekuje na nadejście wiadomości tylko przez okres czasu
określony przez \emph{timeout-literal}. Taki literał jest w postaci:
\texttt{\emph{n}s} (oczekuje \emph{n} sekund), lub \texttt{\emph{n}ms} (oczekuje
\emph{n} milisekund).

\begin{small}
\begin{lstlisting}
(let wait_1_second        (Std.Actor.receive 1s))
(let wait_16_milliseconds (Std.Actor.receive 16ms))
(let wait_forever         (Std.Actor.receive))
\end{lstlisting}
\end{small}

Parametr \texttt{\emph{timeout-literal}} jest liczbą całkowitą podającą ile
sekund (suffix~\texttt{s}) lub milisekund (suffix~\texttt{ms}) aktor powinien
maksymalnie oczekiwać na wiadomość. Z uwagi ma nieprzewidywalność zarządzania
czasem procesora realny czas oczekiwania może się wydłużyć, jednak nigdy nie
będzie krótszy niż podany w parametrze.

\subsubsection{\texttt{Std.Actor.self()}}

\begin{small}
\begin{lstlisting}
Std.Actor.self() -> Pid
\end{lstlisting}
\end{small}

Zwraca PID procesu, w którym ta funkcja została wywołana.

\subsubsection{\texttt{Std.Actor.send()}}

\begin{small}
\begin{lstlisting}
Std.Actor.send(Pid, T) -> void
\end{lstlisting}
\end{small}

Wysyła do procesu określonego podanego przez PID kopię wartości przekazanej w
drugim parametrze.

\subsubsection{\texttt{Std.Pid.eq()}}

\begin{small}
\begin{lstlisting}
Std.Pid.eq(Pid, Pid) -> Bool
\end{lstlisting}
\end{small}

Sprawdza czy dwie wartości PID są sobie równe.

\subsubsection{\texttt{Std.copy()}}

\begin{small}
\begin{lstlisting}
Std.copy(Ta) -> Ta
\end{lstlisting}
\end{small}

Kopiuje wartość podaną jako parametr funkcji.

\subsubsection{\texttt{Std.String.at()}}

\begin{small}
\begin{lstlisting}
Std.String.at(String, Integer) -> String
\end{lstlisting}
\end{small}

Pobiera znak spod indeksu w stringu.

\subsubsection{\texttt{Std.String.concat()}}

\begin{small}
\begin{lstlisting}
Std.String.concat(String, String) -> String
\end{lstlisting}
\end{small}

Pobiera dwa stringi i zwraca nowy string będący ich połączeniem.

\subsubsection{\texttt{Std.String.eq()}}

\begin{small}
\begin{lstlisting}
Std.String.eq(String, String) -> Bool
\end{lstlisting}
\end{small}

Sprawdza czy dwie wartości typu string są sobie równe.

\subsubsection{\texttt{Std.String.size()}}

\begin{small}
\begin{lstlisting}
Std.String.size(String) -> Integer
\end{lstlisting}
\end{small}

Zwraca długość stringu przekazanego jako parametr.

\subsubsection{\texttt{Std.String.substr()}}

\begin{small}
\begin{lstlisting}
Std.String.substr(String, a : Integer, b : Integer) -> String
\end{lstlisting}
\end{small}

Zwraca wycinek stringu pomiędzy \texttt{a} i \texttt{b}.

\subsubsection{\texttt{Std.String.to\_string()}}

\begin{small}
\begin{lstlisting}
Std.String.to_string(T) -> String
\end{lstlisting}
\end{small}

Konwertuje dowolną wartość na string.

\subsubsection{\texttt{Std.Integer.of\_bytes()}}

\begin{small}
\begin{lstlisting}
Std.Integer.of_bytes(Byte_string) -> Integer
\end{lstlisting}
\end{small}

Konwertuje string bajtów na liczbę całkowitą. Stringi bajtów są zwracane m.in. z
funkcji \texttt{Std.Io}.

\subsubsection{\texttt{Std.Vector.at()}}

\begin{small}
\begin{lstlisting}
Std.Vector.at(v : Vector, i : Integer) -> T
\end{lstlisting}
\end{small}

Zwraca wskaźnik do wartości w wektorze \texttt{v} pod indeksem \texttt{i}. Jeśli
wektor jest krótszy niż \texttt{i} to zgłaszany jest wyjątek.

\subsubsection{\texttt{Std.Vector.push()}}

\begin{small}
\begin{lstlisting}
Std.Vector.push(Vector, T) -> void
\end{lstlisting}
\end{small}

Dopisuje wartość do wektora.

\subsubsection{\texttt{Std.Vector.size()}}

\begin{small}
\begin{lstlisting}
Std.Vector.size(Vector) -> Integer
\end{lstlisting}
\end{small}

Zwraca długość wektora.

\subsubsection{\texttt{Std.Pointer.take()}}

\begin{small}
\begin{lstlisting}
Std.Pointer.take(T) -> Tp
\end{lstlisting}
\end{small}

Zwraca wskaźnik do wartości podanej w parametrze.

\subsection{\texttt{Std.Io}}
\label{stdlib_Std_Io}

Moduł udostępniający mechanizmy I/O (\emph{wejścia-wyjścia}). Umożliwia on
wykonywanie operacji I/O na plikach, oraz prostą interakcję z konsolą
użytkownika (\texttt{tty}). I/O jest buforowane. Ten moduł nie jest bezpośrednim
opakowaniem dla wywołań systemowych I/O definiowanych przez standard POSIX
(np.~\texttt{read(3)} lub \texttt{write(3)}).

Funkcje z tego modułu wykorzystują typ \texttt{Fstream} (\emph{file stream})
umożliwiający dostęp do pliku.

\subsubsection{\texttt{Std.Io.stdin\_getline()}}

\begin{small}
\begin{lstlisting}
Std.Io.stdin_getline() -> String
\end{lstlisting}
\end{small}

Umożliwia odczytanie pojedynczej linii ze strumienia standardowego wejścia (\emph{\texttt{stdin}}).

\subsubsection{\texttt{Std.Io.open()}}

\begin{small}
\begin{lstlisting}
Std.Io.open(String) -> Fstream
\end{lstlisting}
\end{small}

Otwiera plik zdefiniowany ścieżką podaną w parametrze.

\subsubsection{\texttt{Std.Io.peek()}}

\begin{small}
\begin{lstlisting}
Std.Io.peek(Fstream) -> String
\end{lstlisting}
\end{small}

Zwraca pierwszy znak w strumieniu.

\subsubsection{\texttt{Std.Io.getline()}}

\begin{small}
\begin{lstlisting}
Std.Io.getline(Fstream) -> String
Std.Io.getline(Fstream, String) -> String
\end{lstlisting}
\end{small}

Wariant 1. odczytuje znaki ze strumienia do napotkania bajtu białej linii (\texttt{\\n}).
Wariant 2. odczytuje znaki ze strumienia do napotkania bajtu podanego w parametrze.

\subsubsection{\texttt{Std.Io.read()}}

\begin{small}
\begin{lstlisting}
Std.Io.read(Fstream) -> String
Std.Io.read(Fstream, Integer) -> String
\end{lstlisting}
\end{small}

Wariant 1. odczytuje naraz całą zawartość strumienia (cały plik).
Wariant 2. odczytuje maksymalnie \emph{n} bajtów.

\subsubsection{\texttt{Std.Io.write()}}

\begin{small}
\begin{lstlisting}
Std.Io.write(Fstream, String) -> void
\end{lstlisting}
\end{small}

Zapisuje string do strumienia.

\subsection{\texttt{Std.Posix.Network}}
\label{stdlib_Std_Posix_Network}

Moduł udostępniający implementację ,,POSIX sockets''. Jest to cienka abstrakcja nad API dostarczanym przez
system operacyjny; w przypadku braków w tym dokumencie ich dokumentację można wydedukować ze stron manuala
sekcji 3 dostarczanych przez program \texttt{man(1)} (np. dokumentację dla funkcji
\texttt{Std.Posix.Network.socket} można uzyskać wykonując polecenie \texttt{man 3 socket}).

\subsubsection{\texttt{Std.Posix.Network.socket()}}

\begin{small}
\begin{lstlisting}
Std.Posix.Network.socket() -> Integer
\end{lstlisting}
\end{small}

Tworzy socket za pomocą wywołania funkcji \texttt{socket(3)}.
Socket jest w rodzinie \texttt{AF\_INET}, rodzaju \texttt{SOCK\_STREAM}, i jest tworzony bez flag.

Socket zwracany przez tą funkcję jest blokujący.

\subsubsection{\texttt{Std.Posix.Network.connect()}}

\begin{small}
\begin{lstlisting}
Std.Posix.Network.connect(
      socket : Integer
    , addr   : String
    , port   : Integer
) -> void
\end{lstlisting}
\end{small}

Wrapper na funkcję \texttt{connect(3)}.

\subsubsection{\texttt{Std.Posix.Network.bind()}}

\begin{small}
\begin{lstlisting}
Std.Posix.Network.bind(
      socket : Integer
    , addr   : String
    , port   : Integer
) -> void
\end{lstlisting}
\end{small}

Wrapper na funkcję \texttt{bind(3)}.

\subsubsection{\texttt{Std.Posix.Network.listen()}}

\begin{small}
\begin{lstlisting}
Std.Posix.Network.listen(
      socket  : Integer
    , backlog : Integer
) -> void
\end{lstlisting}
\end{small}

Wrapper na funkcję \texttt{listen(3)}.

\subsubsection{\texttt{Std.Posix.Network.accept()}}

\begin{small}
\begin{lstlisting}
Std.Posix.Network.accept(socket : Integer) -> void
\end{lstlisting}
\end{small}

Wrapper na funkcję \texttt{accept(3)}.

Sockety zwracane przez tą funkcję są nieblokujące. Ich timeout jest ustawiony na 500ms.

\subsubsection{\texttt{Std.Posix.Network.write()}}

\begin{small}
\begin{lstlisting}
Std.Posix.Network.write(
      socket : Integer
    , value  : T
) -> Integer
\end{lstlisting}
\end{small}

Niezależnie od tego jaka wartość jest przekazana do funkcji, będzie najpierw przekonwertowana na
\texttt{String}, a potem wpisana do socketu.

\subsubsection{\texttt{Std.Posix.Network.read()}}

\begin{small}
\begin{lstlisting}
Std.Posix.Network.read(
    socket  : Integer
) -> String
\end{lstlisting}
\end{small}

Wrapper na funkcję \texttt{read(3)}. Odczytuje z socketu maksymalnie 1024 bajty.

\subsubsection{\texttt{Std.Posix.Network.recv()}}

\begin{small}
\begin{lstlisting}
Std.Posix.Network.recv(
      socket        : Integer
    , buffer_length : Integer
) -> String
\end{lstlisting}
\end{small}

Wrapper na funkcję \texttt{recv(3)}. Odczytuje z socketu maksymalnie \texttt{buffer\_length} bajtów.

\subsubsection{\texttt{Std.Posix.Network.shutdown()}}

\begin{small}
\begin{lstlisting}
Std.Posix.Network.listen(
    socket  : Integer
) -> void
\end{lstlisting}
\end{small}

Wrapper na funkcję \texttt{shutdown(3)}.
Wyłącza socket z flagą \texttt{SHUT\_RDWR}.

\subsubsection{\texttt{Std.Posix.Network.close()}}

\begin{small}
\begin{lstlisting}
Std.Posix.Network.listen(
      socket  : Integer
) -> void
\end{lstlisting}
\end{small}

Wrapper na funkcję \texttt{close(3)}.


\subsection{\texttt{Std.Random}}
\label{stdlib_Std_Random}

Moduł udostępniający dostęp do liczb losowych ogólnego przeznaczenia (\texttt{/dev/urandom}) oraz zdatnych do
zastosowań kryptograficznych (\texttt{/dev/random}).

\subsubsection{\texttt{Std.Random.random()}}

\begin{small}
\begin{lstlisting}
Std.Random.random() -> Float
\end{lstlisting}
\end{small}

Zwraca losową liczbę zmiennoprzecinkową z przedziału $[0.0, 1.0)$.

\subsubsection{\texttt{Std.Random.randint()}}

\begin{small}
\begin{lstlisting}
Std.Random.randint(lower : Integer, upper : Integer) -> Integer
\end{lstlisting}
\end{small}

Zwraca losową liczbę całkowitą z przedziału [\texttt{\emph{lower}}, \texttt{\emph{upper}}).
