\section{Biblioteka standardowa}

Biblioteka standardowa języka \ViuAct\ składa się z modułów implementujących
wybrane algorytmy (np. sortowanie) i funkcjonalności (np. konwersje między
typami danych), oraz umożliwia interakcję ze środowiskiem zewnętrznym gdyż
\ViuAct\ nie posiada wbudowanych w język mechanizmów takiej interakcji.

Część funkcji udostępnianych przez bibliotekę standardową jest niemożliwa do zaimplementowania w języku
\ViuAct. Takie funkcje są napisane albo w języku assemblera Viua VM, albo w języku C++.

Nie wszystkie funkcje biblioteki standardowej są ,,normalnymi'' funkcjami, możliwymi do uruchomienia jako
aktor ponieważ niektóre z nich kompilują się do pojedynczych instrukcji maszyny (funkcje ,,wbudowane'') albo
są napisane w języku C++, który jest niemożliwy do wywłaszczenia przez Viua VM (funkcje ,,obce'').

Dla takich funkcji programista musi napisać \emph{wrappery} jeśli chce użyć ich jak w niektórych kontekstach.
Jednak w typowym programie nie będzie to potrzebne, gdyż funkcje biblioteki standardowej zazwyczaj
implementują atomowe zadania, których sens uruchamiania w osobnych aktorach jest ograniczony.

\begin{center}
\emph{Należy pamiętać, że opis biblioteki standardowej \textbf{nie} jest ostateczny i może ulec zmianie w
trakcie trwania projektu. Z tego powodu funkcje są opisane skrótowo, a dokumentacja nie jest wyczerpująca.}
\end{center}

Moduł \texttt{Std.Actor} jest opisany na stronie \pageref{stdlib_Std_Actor}.
Moduł \texttt{Std.Io} jest opisany na stronie \pageref{stdlib_Std_Io}.
Moduł \texttt{Std.Posix.Network} jest opisany na stronie \pageref{stdlib_Std_Posix_Network}.
Moduł \texttt{Std.Random} jest opisany na stronie \pageref{stdlib_Std_Random}.

\subsection{\texttt{Std.Actor}}
\label{stdlib_Std_Actor}

Moduł udostępniający funkcjonalność związaną z wymianą wiadomości i modelem aktorów.

\subsubsection{\texttt{Std.Actor.join()}}
\label{Std_Actor_join}

\begin{small}
\begin{lstlisting}
Std.Actor.join(Pid) -> Value
\end{lstlisting}
\end{small}

Oczekuje na zakończenie wykonywania przez aktora o podanym PID, po czym zwraca wynik jego pracy.
Może powodować zgłoszenie wyjątku jeśli aktor o podanym PID zakończył wykonywanie z powodu błędu, lub jeśli aktor
o podanym PID nie istnieje.

\subsubsection{\texttt{Std.Actor.self()}}

\begin{small}
\begin{lstlisting}
Std.Actor.self() -> Pid
\end{lstlisting}
\end{small}

Zwraca PID obecnego aktora.

\subsubsection{\texttt{Std.Actor.receive()}}

\begin{small}
\begin{lstlisting}
Std.Actor.receive() -> Value
Std.Actor.receive(Timeout) -> Value
\end{lstlisting}
\end{small}

Oczekuje na nadejście wiadomości; wariant 1. oczekuje w nieskończoność,
wariant 2. oczekuje do momentu upłynięcia czasu podanego w parametrze.

\begin{small}
\begin{lstlisting}
(let wait_1_second        (Std.Actor.receive 1s))
(let wait_16_milliseconds (Std.Actor.receive 16ms))
(let wait_forever         (Std.Actor.receive))
\end{lstlisting}
\end{small}

Parametr \texttt{Timeout} jest liczbą całkowitą podającą ile sekund
(suffix~\texttt{s}) lub milisekund (suffix~\texttt{ms}) aktor powinien
maksymalnie oczekiwać na wiadomość. Z uwagi ma nieprzewidywalność zarządzania
czasem procesora realny czas oczekiwania może się wydłużyć, jednak nigdy nie
będzie krótszy niż podany w parametrze.

\subsection{\texttt{Std.Io}}
\label{stdlib_Std_Io}

Moduł udostępniający mechanizmy I/O (\emph{wejścia-wyjścia}).
Umożliwia on wykonywanie operacji I/O na plikach, oraz prostą interakcję z
konsolą użytkownika (\texttt{tty}). I/O jest buforowane. Ten moduł nie jest
bezpośrednim opakowaniem dla wywołań systemowych I/O definiowanych przez
standard POSIX (np. \texttt{read(3)} lub \texttt{write(3)}).

\subsubsection{\texttt{Std.Io.stdin\_getline()}}

\begin{small}
\begin{lstlisting}
Std.Io.stdin_getline() -> String
\end{lstlisting}
\end{small}

Umożliwia odczytanie pojedynczej linii ze strumienia standardowego wejścia (\emph{\texttt{stdin}}).

\subsubsection{\texttt{Std.Io.open()}}

\begin{small}
\begin{lstlisting}
Std.Io.open(String) -> Fstream
\end{lstlisting}
\end{small}

Otwiera plik zdefiniowany ścieżką podaną w parametrze.

\subsubsection{\texttt{Std.Io.peek()}}

\begin{small}
\begin{lstlisting}
Std.Io.peek(Fstream) -> String
\end{lstlisting}
\end{small}

Zwraca pierwszy znak w strumieniu.

\subsubsection{\texttt{Std.Io.getline()}}

\begin{small}
\begin{lstlisting}
Std.Io.getline(Fstream) -> String
Std.Io.getline(Fstream, String) -> String
\end{lstlisting}
\end{small}

Wariant 1. odczytuje znaki ze strumienia do napotkania bajtu białej linii (\texttt{\\n}).
Wariant 2. odczytuje znaki ze strumienia do napotkania bajtu podanego w parametrze.

\subsubsection{\texttt{Std.Io.read()}}

\begin{small}
\begin{lstlisting}
Std.Io.read(Fstream) -> String
Std.Io.read(Fstream, Integer) -> String
\end{lstlisting}
\end{small}

Wariant 1. odczytuje naraz całą zawartość strumienia (cały plik).
Wariant 2. odczytuje maksymalnie \emph{n} bajtów.

\subsubsection{\texttt{Std.Io.write()}}

\begin{small}
\begin{lstlisting}
Std.Io.write(Fstream, String) -> void
\end{lstlisting}
\end{small}

Zapisuje string do strumienia.

\subsection{\texttt{Std.Posix.Network}}
\label{stdlib_Std_Posix_Network}

Moduł udostępniający implementację ,,POSIX sockets''. Jest to cienka abstrakcja nad API dostarczanym przez
system operacyjny; w przypadku braków w tym dokumencie ich dokumentację można wydedukować ze stron manuala
sekcji 3 dostarczanych przez program \texttt{man(1)} (np. dokumentację dla funkcji
\texttt{Std.Posix.Network.socket} można uzyskać wykonując polecenie \texttt{man 3 socket}).

\subsubsection{\texttt{Std.Posix.Network.socket()}}

\begin{small}
\begin{lstlisting}
Std.Posix.Network.socket() -> Integer
\end{lstlisting}
\end{small}

Tworzy socket za pomocą wywołania funkcji \texttt{socket(3)}.
Socket jest w rodzinie \texttt{AF\_INET}, rodzaju \texttt{SOCK\_STREAM}, i jest tworzony bez flag.

Socket zwracany przez tą funkcję jest blokujący.

\subsubsection{\texttt{Std.Posix.Network.connect()}}

\begin{small}
\begin{lstlisting}
Std.Posix.Network.connect(
      socket : Integer
    , addr   : Text
    , port   : Integer
) -> Void
\end{lstlisting}
\end{small}

Wrapper na funkcję \texttt{connect(3)}.

\subsubsection{\texttt{Std.Posix.Network.bind()}}

\begin{small}
\begin{lstlisting}
Std.Posix.Network.bind(
      socket : Integer
    , addr   : Text
    , port   : Integer
) -> Void
\end{lstlisting}
\end{small}

Wrapper na funkcję \texttt{bind(3)}.

\subsubsection{\texttt{Std.Posix.Network.listen()}}

\begin{small}
\begin{lstlisting}
Std.Posix.Network.listen(
      socket  : Integer
    , backlog : Integer
) -> Void
\end{lstlisting}
\end{small}

Wrapper na funkcję \texttt{listen(3)}.

\subsubsection{\texttt{Std.Posix.Network.accept()}}

\begin{small}
\begin{lstlisting}
Std.Posix.Network.accept(socket : Integer) -> Void
\end{lstlisting}
\end{small}

Wrapper na funkcję \texttt{accept(3)}.

Sockety zwracane przez tą funkcję są nieblokujące. Ich timeout jest ustawiony na 500ms.

\subsubsection{\texttt{Std.Posix.Network.write()}}

\begin{small}
\begin{lstlisting}
Std.Posix.Network.write(
      socket : Integer
    , value  : Value
) -> Int
\end{lstlisting}
\end{small}

Niezależnie od tego jaka wartość jest przekazana do funkcji, będzie najpierw przekonwertowana na
\texttt{String}, a potem wpisana do socketu.

\subsubsection{\texttt{Std.Posix.Network.read()}}

\begin{small}
\begin{lstlisting}
Std.Posix.Network.read(
    socket  : Integer
) -> String
\end{lstlisting}
\end{small}

Wrapper na funkcję \texttt{read(3)}. Odczytuje z socketu maksymalnie 1024 bajty.

\subsubsection{\texttt{Std.Posix.Network.recv()}}

\begin{small}
\begin{lstlisting}
Std.Posix.Network.recv(
      socket        : Integer
    , buffer_length : Integer
) -> String
\end{lstlisting}
\end{small}

Wrapper na funkcję \texttt{recv(3)}. Odczytuje z socketu maksymalnie \texttt{buffer\_length} bajtów.

\subsubsection{\texttt{Std.Posix.Network.shutdown()}}

\begin{small}
\begin{lstlisting}
Std.Posix.Network.listen(
    socket  : Integer
) -> Void
\end{lstlisting}
\end{small}

Wrapper na funkcję \texttt{shutdown(3)}.
Wyłącza socket z flagą \texttt{SHUT\_RDWR}.

\subsubsection{\texttt{Std.Posix.Network.close()}}

\begin{small}
\begin{lstlisting}
Std.Posix.Network.listen(
      socket  : Integer
) -> Void
\end{lstlisting}
\end{small}

Wrapper na funkcję \texttt{close(3)}.


\subsection{\texttt{Std.Random}}
\label{stdlib_Std_Random}

Moduł udostępniający dostęp do liczb losowych ogólnego przeznaczenia (\texttt{/dev/urandom}) oraz zdatnych do
zastosowań kryptograficznych (\texttt{/dev/random}).

\subsubsection{\texttt{Std.Random.random()}}

\begin{small}
\begin{lstlisting}
Std.Random.random() -> Float
\end{lstlisting}
\end{small}

Zwraca losową liczbę zmiennoprzecinkową z przedziału $[0.0, 1.0)$.

\subsubsection{\texttt{Std.Random.randint()}}

\begin{small}
\begin{lstlisting}
Std.Random.randint(lower : Integer, upper : Integer) -> Integer
\end{lstlisting}
\end{small}

Zwraca losową liczbę całkowitą z przedziału [\texttt{\emph{lower}}, \texttt{\emph{upper}}).
