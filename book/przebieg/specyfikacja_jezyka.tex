\section{Specyfikacja języka \ViuAct}

Tworzenie formalnej specyfikacji języka \ViuAct\ było, paradoksalnie, procesem
mało sformalizowanym. Polegało na określeniu głównych założeń języka (opisanych
w rozdziale
\ref{specyfikacja_jezyka_viuact_model_programowania}~\nameref{specyfikacja_jezyka_viuact_model_programowania}
na stronie \pageref{specyfikacja_jezyka_viuact_model_programowania}), co zostało
wykonane na jednym z pierwszych spotkań i nadało kierunek dalszemu rozwojowi
języka, a następnie na zaprojektowaniu poszczególnych konstrukcji jezykowych w
sposób zgodny z założoną wizją.

Równolegle z tworzeniem specyfikacji poszczególnych konstrukcji w języku
powstawały ich prototypowe implementacje mające na celu szybką i jak
najwcześniejszą weryfikację tego czy dana konstrukcja jest możliwa do
zaimplementowania w założonym budżecie czasowym. Jeśli dana konstrukcja
okazywała się niemożliwa do zaimplementowania w satysfakcjonującym czasie, lub
możliwa do zaimplementowania jedynie w pewnym niewystarczającym fragmencie to
była usuwana ze specyfikacji.
