\chapter{Przebieg prac}
\label{przebieg_prac}

\section*{Środowisko pracy}

Aby ułatwić prace rozwojowe nad składnikami tej pracy inżynierskiej został
napisany skrypt przygotowujący środowisko wytwórcze. Zapewnia on spójną
strukturę katalogów pomiędzy maszynami członków zespołu co eliminuje
niepotrzebne różnice między środowiskami wytwórczymi. Skrypt pozwala na
przygotowanie środowiska za pomocą jednego polecenia wydanego w powłoce:
\begin{lstlisting}
wget -O https://viuavm.org/viuact-bootstrap.sh | bash -s
\end{lstlisting}

Skrypt ten utworzy w katalogu roboczym katalog \texttt{viuact-workspace},
pobierze najnowsze wersje każdego z komponentów projektu (klonując ich
repozytoria Git) i skompiluje je w odpowiedniej kolejności.

Aby aktywować środowisko (ustawić zmienne środowiskowe istotne dla kompilatora
języka \ViuAct\ oraz jądra i assemblera dostarczanych z Viua~VM) należy wykonać
następujące polecenie wewnątrz utworzonego przez skrypt katalogu
\texttt{viuact-workspace}:
\begin{lstlisting}
source ./activate.sh
\end{lstlisting}

Takie podejście (oskryptowanie i automatyzacja przygotowanie środowiska pracy)
znacząco zredukowało problemy wynikające z różnic, które członkowie zespołu
mogli -- nawet omyłkowo -- wprowadzić do swojego lokalnego środowiska.

\section{Specyfikacja języka ViuAct}

Tworzenie formalnej specyfikacji języka ViuAct było, paradoksalnie, procesem mało sformalizowanym. Polegało
na określeniu głównych założeń języka (opisanych w rozdziale
\ref{specyfikacja_jezyka_viuact_model_programowania}
\nameref{specyfikacja_jezyka_viuact_model_programowania}).


\section{Implementacja kompilatora}

Śledzenie prac nad rozwojem kompilatora było wspierane przez dwa narzędzia:
system kontroli wersji Git\footnote{\url{https://git-scm.com/}} i narzędzie do
śledzenia zadań Issue\footnote{\url{https://github.com/marekjm/issue}}.

\subsection{Zakończone zadania}

Poniżej przedstawiona jest lista zakończonych zadań zarejestrowanych do
wykonania podczas projektu.
Na dzień 2019-05-19 wykonane zostało $64.44\%$ zaplanowanych zadań. Tak niska
wartość wiąże się z tym, że część zadań zostało otwartych z założeniem, że mogą
nie zostać wykonane (na zasadzie ,,jeśli wystarczy czasu''). Mediana czasów
całkowitego czasu (od zgłoszenia do zamknięcia) każdego z ukończonych zadań to
nieco ponad sześć dni (6 dni i 11 godzin).

Repozytorium programu Issue jest dostępne wraz z repozytorium Git dołączonym
jako załącznik do projektu.

\subsubsection{Add an assembly driver tool for executables}

Identyfikator zadania: \texttt{5b8f77f4714be2c16d1f87a02b03692c96b983c1}
\newline
Rozpoczęte 2018-12-18 23:20:11, zakończone 2018-12-20 16:00:54.

\subsubsection{Basic CLI chat}

Identyfikator zadania: \texttt{b8cea1dc8ac19f63ad579ac28196c82f6a9eb7aa}
\newline
Rozpoczęte 2019-01-19 22:15:28, porzucone 2019-03-11 19:04:01.
\newline

Zadanie zostało porzucone ponieważ powstał czat oparty o protokół WebSocket.

\subsubsection{Create a module system}

Identyfikator zadania: \texttt{509ee8af1c2f503e402b34c18224a84caf4c9252}
\newline
Rozpoczęte 2018-11-22 19:26:21, zakończone 2018-11-29 06:54:16.
\newline

Język \ViuAct\phantom{} posiada działający system modułów opisany w rozdziale
\ref{viuact_spec_module_definition}~\nameref{viuact_spec_module_definition}.

\subsubsection{Create an EBNF notation to describe Viuact syntax}

Identyfikator zadania: \texttt{3295295f3d3b08cd0a3ae46c1adebe40bb3ebad4}
\newline
Rozpoczęte 2018-12-19 00:04:24, zakończone 2019-03-26 20:00:00.

Zadanie zostało zakończone po umieszczeniu w pracy specyfikacji języka \ViuAct,
która opisuje składnię w notacji EBNF.

\subsubsection{Dump intermediate representations}

Identyfikator zadania: \texttt{902f}

\subsubsection{Emit dependency files for modules and executables}

Identyfikator zadania: \texttt{f841}

\subsubsection{Emit interface files for modules}

Identyfikator zadania: \texttt{86fd}

\subsubsection{Expose pointers}

Identyfikator zadania: \texttt{3208}

\subsubsection{Expose the mechanism to set a process watchdog}

Identyfikator zadania: \texttt{cc03}

\subsubsection{Implement an exception catching mechanism}

Identyfikator zadania: \texttt{d4bd}

\subsubsection{Implement an impressive example program}

Identyfikator zadania: \texttt{b28a}

\subsubsection{Implement boolean values}

Identyfikator zadania: \texttt{0050}

\subsubsection{Implement compound expressions}

Identyfikator zadania: \texttt{d452}

\subsubsection{Implement proper imports inside Viua VM}

Identyfikator zadania: \texttt{6f4f}

\subsubsection{Implement structs}

Identyfikator zadania: \texttt{1f45}

\subsubsection{Implement tail calls}

Identyfikator zadania: \texttt{b680}

\subsubsection{Implement vectors}

Identyfikator zadania: \texttt{c499}

\subsubsection{Integrate FFI imports}

Identyfikator zadania: \texttt{713e}

\subsubsection{Make function bodies a single expression}

Identyfikator zadania: \texttt{0d91}

\subsubsection{Make functions first class values}

Identyfikator zadania: \texttt{04dd}

\subsubsection{Make the compiler installable}

Identyfikator zadania: \texttt{6b5d}

\subsubsection{Remove the parentheses around function args}

Identyfikator zadania: \texttt{bb64}

\subsubsection{Rework README for the Issue project}

Identyfikator zadania: \texttt{7348}

\subsubsection{Specify tail calls}

Identyfikator zadania: \texttt{9774}

\subsubsection{Specify the module system}

Identyfikator zadania: \texttt{242e}

\subsubsection{Suppress unused register errors thrown by Viua VM assembler}

Identyfikator zadania: \texttt{cd43}

\subsubsection{Test the compiler and assembly driver}

Identyfikator zadania: \texttt{561c}

\subsubsection{Write some tests}

Identyfikator zadania: \texttt{2925}

\subsubsection{Write websockets FFI module}

Identyfikator zadania: \texttt{631b}


\section{Implementacja ViuaChat}
 Poniżej przedstawiono zadania, jakie realizowano w ramach kolejnych
 sprintów. Część z nich odnosi się do pierwotnych wymagań (w tym
 \textit{user stories}, zaś niektóre zostały sformułowane w trakcie samej
 realizacji, pod wpływem pierwotnych potrzeb.

\subsection{Sprint 1}

\textbf{Termin realizacji:} 2-9 marca 2019 r.

\subsubsection{Cel sprintu}
Celem sprintu było utworzenie podstawowej funkcjonalności związanej z
komunikacją frontend-backend -- ustanawiania połączenia WebSocket,
dekodowania komunikatów bo obu stronach aplikacji, a także podstawowa
walidacja funkcjonalności.

\subsubsection{Zadania oparte o pierwotne wymagania}
\leavevmode\hbox{}

\begin{tabular}{ | l | l | }
	\hline
		\textbf{Identyfikator} &
	WS-01
		\\

	\hline
		\textbf{Treść} & \parbox[t]{11.5cm}{\strut
			W rozwiązaniu należy wykorzystać środowisko Viua VM i
			język \ViuAct
		\strut}\\

	\hline
  \parbox[t]{4cm}{\textbf{Nakład godzinowy (planowany / włożony)}} & \parbox[t]{11.5cm}{\strut
    Nakład godzinowy związany z zadaniem ujęto przy realizacji zadań ZZ-01 i
    ZZ-02.
  \strut}\\

\hline
  \parbox[t]{4cm}{\textbf{Ukończono?}} &
  \parbox[t]{11.5cm}{\strut
    Tak.
  \strut}\\

\hline
\end{tabular}

\vspace{1em}

\begin{tabular}{ | l | l | }
	\hline
		\textbf{Identyfikator} &
	WS-02
		\\

	\hline
		\textbf{Treść} & \parbox[t]{11.5cm}{\strut
			Czat będzie użytkowany jako aplikacja webowa typu
			\textit{single page application}
		\strut}\\

	\hline
  \parbox[t]{4cm}{\textbf{Nakład godzinowy (planowany / włożony)}} & \parbox[t]{11.5cm}{\strut
    Nakład godzinowy związany z zadaniem ujęto przy realizacji zadań ZZ-01 i
    ZZ-02.
  \strut}\\

\hline
  \parbox[t]{4cm}{\textbf{Ukończono?}} &
  \parbox[t]{11.5cm}{\strut
    Tak.
  \strut}\\

\hline
\end{tabular}

\subsubsection{Zadania wykraczające poza pierwotne wymagania}

\leavevmode\hbox{}

\begin{tabular}{ | l | l | }
 \hline
   \textbf{Identyfikator} &
   ZZ-01
   \\

 \hline
   \textbf{Treść} & \parbox[t]{11.5cm}{\strut
     Frontend aplikacji jest w stanie nawiązać połączenie WebSocket z backendem.
   \strut}\\

 \hline
   \parbox[t]{4cm}{\textbf{Kryteria akceptacji}} & \parbox[t]{11.5cm}{\strut
     \begin{enumreq}
       \item Frontend nawiązał połączenie z serwerem (tj. doszło do
       zmiany protokołu z HTTP na WS).
       \item Frontend wysłał wiadomość do backendu, a wiadomość została
       odebrana.
       \item Backend wysłał wiadomość do frontendu, a wiadomość została
       odebrana.
     \end{enumreq}
     \strut}
   \\

   \hline
     \parbox[t]{4cm}{\textbf{Nakład godzinowy (planowany / włożony)}} & \parbox[t]{11.5cm}{\strut
       5h / 5.5h
     \strut}\\

   \hline
     \parbox[t]{4cm}{\textbf{Ukończono?}} &
     \parbox[t]{11.5cm}{\strut
       Tak.
     \strut}\\

   \hline
 \end{tabular}

 \vspace{1em}

\begin{tabular}{ | l | l | }
 \hline
   \textbf{Identyfikator} &
   ZZ-02
   \\

 \hline
   \textbf{Treść} & \parbox[t]{11.5cm}{\strut
     Obie strony (tj. frontend i backend) potrafią zakodować wiadomości w
     ustalonym formacie JSON przed ich wysyłką i odebrać je po ich odebraniu.
   \strut}\\

 \hline
   \parbox[t]{4cm}{\textbf{Kryteria akceptacji}} & \parbox[t]{11.5cm}{\strut
     \begin{enumreq}
       \item Frontend potrafi zakodować wiadomość w formacie JSON -- wiadomość
       synchroniczną oraz wiadomość asynchroniczną.
       \item Frontend potrafi odkodować wiadomość w formacie JSON.
       \item Frontend potrafi ustalić, czy odebrana wiadomość jest synchroniczna,
       czy asynchroniczna, a w przypadku synchronicznej -- dopasować ją do
       wysłanej wcześniej wiadomości.
       \item Frontend potrafi właściwie obsługiwać wyjątki wynikające z błędów
       w składni JSON odbieranych wiadomości oraz z błędów w strukturze obiektów
       JSON: braku obowiązkowych własności oraz nieprawidłowych typów tych
       wiadomości. Błędy te nie powodują krytycznego przerwania pracy działania
       programu.
       \item Backend potrafi odkodować otrzymywane wiadomości w  formacie JSON.
       \item Backend potrafi ustalić, czy otrzymana wiadomość jest synchroniczna,
       czy asynchroniczna.
       \item Backend potrafi odesłać odpowiedź pasującą do otrzymanej wiadomości
       synchronicznej.
       \item Backend potrafi właściwie obsługiwać wyjątki z tytułu niewłaściwie
       skonstruowanych wiadomości. W razie otrzymania takiej wiadomości, nie
       dochodzi do przerwania pracy serwera (tj. aktora \texttt{WsConnector}).
     \end{enumreq}
     \strut}
   \\

   \hline
     \parbox[t]{4cm}{\textbf{Nakład godzinowy (planowany / włożony)}} & \parbox[t]{11.5cm}{\strut
       10h / 12h
     \strut}\\

       \hline
         \parbox[t]{4cm}{\textbf{Ukończono?}} &
         \parbox[t]{11.5cm}{\strut
           Tak.
         \strut}\\

         \hline
     \end{tabular}


\subsection{Sprint 2}

\textbf{Termin realizacji:} 10-23 marca 2019 r.

\subsubsection{Cel sprintu}
W ramach tego sprintu zaplanowano opracowanie mechanizm autoryzacji do czatu i
możliwość przejrzenia listy pokojów.

\subsubsection{Zadania oparte o pierwotne wymagania}

\leavevmode\hbox{}

\begin{tabular}{ | l | l | }
	\hline
		\textbf{Identyfikator} &
		WF-01
		\\

	\hline
		\textbf{Treść} & \parbox[t]{11.5cm}{\strut
			Jako użytkownik serwera czatu, chcę się do niego zalogować, aby zobaczyć listę pokojów dyskusyjnych.
		\strut}\\

	\hline
		\parbox[t]{4cm}{\textbf{Kryteria akceptacji}} & \parbox[t]{11.5cm}{\strut
			\begin{enumreq}
				\item Po wejściu na czat bez rozpoczętej sesji, pokazuje się monit o podanie nazwy użytkownika.
				\item Po wpisaniu nazwy użytkownika i zatwierdzeniu, użytkownik rozpocznie sesję na serwerze czatu.
				\item Tuż po rozpoczęciu sesji czatu, użytkownik zobaczy listę pokojów.
			\end{enumreq}
			\strut}
		\\

    \hline
      \parbox[t]{4cm}{\textbf{Nakład godzinowy (planowany / włożony)}} & \parbox[t]{11.5cm}{\strut
        15h / 21h
      \strut}\\
	\hline

    \hline
      \parbox[t]{4cm}{\textbf{Ukończono?}} &
      \parbox[t]{11.5cm}{\strut
        Tak.
      \strut}\\

      \hline
  \end{tabular}

\vspace{1em}

  \begin{tabular}{ | l | l | }
  	\hline
  		\textbf{Identyfikator} &
  		HN-01
  		\\

  	\hline
  		\textbf{Treść} & \parbox[t]{11.5cm}{\strut
  			Długość nazwy użytkownika jest ograniczona od 2 do 32 znaków alfanumerycznych, w celu uniknięcia problemów z identyfikacją użytkownika na serwerze.
  		\strut}\\

  	\hline
  		\parbox[t]{4cm}{\textbf{Powiązane zasady biznesowe}} & \parbox[t]{11.5cm}{\strut
  			ZU-03 Nazwa użytkownika to ciąg od 2 do 32 alfanumerycznych znaków.
  		\strut}\\

  	\hline
  		\parbox[t]{4cm}{\textbf{Kryteria akceptacji}} & \parbox[t]{11.5cm}{\strut
  			\begin{enumreq}
  				\item Po wpisaniu do pola użytkownika nazwy krótszej niż 2 znaki, dłużej niż 32 znaki lub zawierającej inne znaki niż alfanumeryczne, zwracany jest błąd.
  			\end{enumreq}
  			\strut}
  		\\

  	\hline
    \parbox[t]{4cm}{\textbf{Nakład godzinowy (planowany / włożony)}} & \parbox[t]{11.5cm}{\strut
      Czas wynikający z tego zadania ujęto przy realizacji zadania WF-01.
    \strut}\\
\hline

  \hline
    \parbox[t]{4cm}{\textbf{Ukończono?}} &
    \parbox[t]{11.5cm}{\strut
      Tak.
    \strut}\\

    \hline
  \end{tabular}

  \vspace{1em}

  \begin{tabular}{ | l | l | }
  	\hline
  		\textbf{Identyfikator} &
  		HN-02
  		\\

  	\hline
  		\textbf{Treść} & \parbox[t]{11.5cm}{\strut
  			Loginy i hasła administratorów są gromadzone w plikach
        konfiguracyjnych, a po uruchomieniu serwera -- w jego
        pamięci operacyjnej.
  		\strut}\\

  	\hline
  		\parbox[t]{4cm}{\textbf{Powiązane zasady biznesowe}} & \parbox[t]{11.5cm}{\strut
  			ZU-07 Konta administratorów są utrzymywane na serwerze w postaci par wartości: nazwa użytkownika i hasło.
  		\strut}\\

  	\hline
  		\parbox[t]{4cm}{\textbf{Kryteria akceptacji}} & \parbox[t]{11.5cm}{\strut
  			\begin{enumreq}
  				\item Serwer jest wyposażony w pliki konfiguracyjne
          \item Po załadowaniu serwera, z plików konfiguracyjnych
          są odczytywane dane kont administracyjnych
          \item Serwer po uruchomieniu jest wyposażony w konta o
          nazwach i hasłach zgodnych z wpisami w plikach konfiguracyjnych.
  			\end{enumreq}
  			\strut}
  		\\

  	\hline
    \parbox[t]{4cm}{\textbf{Nakład godzinowy (planowany / włożony)}} & \parbox[t]{11.5cm}{\strut
      Czas wynikający z tego zadania ujęto przy realizacji zadania WF-01.
    \strut}\\
\hline

  \hline
    \parbox[t]{4cm}{\textbf{Ukończono?}} &
    \parbox[t]{11.5cm}{\strut
      Tak.
    \strut}\\

    \hline
  \end{tabular}

\subsubsection{Zadania wykraczające poza pierwotne wymagania}

\leavevmode\hbox{}

\begin{tabular}{ | l | l | }
	\hline
		\textbf{Identyfikator} &
		ZZ-03
		\\

	\hline
		\textbf{Treść} & \parbox[t]{11.5cm}{\strut
			Połączenie po autoryzacji powinno zostać zabezpieczone losowym łańcuchem
      znaków -- kluczem sesji. Klucz sesji powinien być generowany
      automatycznie po stronie backendu i wysyłany frontendowi tuż po
      autoryzacji użytkownika. Od tej pory backend nie zaakceptuje żadnej
      wiadomości od frontendu, dopóki nie
      będzie zawierała dodatkowego parametru \texttt{session\_key} z treścią
      klucza sesji. Celem tego środka jest uniknięcie sytuacji zwielokrotnienia
      klientów na tym samym kanale w razie zresetowania lub wielokrotnego
      nawiązywania połączenia z tego samego urządzenia.
		\strut}\\

	\hline
		\parbox[t]{4cm}{\textbf{Kryteria akceptacji}} & \parbox[t]{11.5cm}{\strut
			\begin{enumreq}
				\item W momencie autoryzacji, backend wygeneruje 32-znakowy,
        alfanumeryczny, losowy klucz.
        \item Backend odeśle klucz z informacją o autoryzacji do frontendu.
        \item Frontent zapisze i utrzyma otrzymany klucz sesji
        \item Frontent będzie automatycznie dodawać parametr \texttt{session\_key},
        otrzymany podczas autoryzacji, do każdej kolejnej, dodawanej wiadomości.
			\end{enumreq}
			\strut}
		\\

    \hline
  		\parbox[t]{4cm}{\textbf{Nakład godzinowy (planowany / włożony)}} &
      \parbox[t]{11.5cm}{\strut
  			2h / 1h
  		\strut}\\

        \hline
          \parbox[t]{4cm}{\textbf{Ukończono?}} &
          \parbox[t]{11.5cm}{\strut
            Tak.
          \strut}\\

          \hline
      \end{tabular}

      \vspace{1em}


\subsection{Sprint 3}

\textbf{Termin realizacji:} 24 marca -- 6 kwietnia 2019 r.

\subsubsection{Cel sprintu}
Po zakończeniu sprintu, powinno być możliwe podpięcie się do pokoju i rozmowa z
innymi użytkownikami, którzy są podpięci do tego samego pokoju.

\subsubsection{Zadania oparte o pierwotne wymagania}

\leavevmode\hbox{}

\begin{tabular}{ | l | l | }
 \hline
   \textbf{Identyfikator} &
   WF-02
   \\

 \hline
   \textbf{Treść} & \parbox[t]{11.5cm}{\strut
     Jako użytkownik serwera czatu, chcę wpiąć się do pokoju,
     aby wziąć udział w dyskusji.
   \strut}\\

 \hline
   \parbox[t]{4cm}{\textbf{Kryteria akceptacji}} & \parbox[t]{11.5cm}{\strut
     \begin{enumreq}
       \item Użytkownik, który ma otwartą sesję
       połączenia z serwerem czatu i nie jest wpięty
       do żadnego pokoju, zobaczy listę pokojów.
       \item Użytkownik, po kilknięciu w liście pokojów
       na nazwę pokoju, zostanie do niego podpięty
       \item Użytkownik po wpięciu się do pokoju zobaczy
       okno pokoju
       \item Użytkownik, który ma otwartą sesję
       połączenia z serwerem i jest wpięty do pokoju,
       po odświeżeniu przeglądarki zobaczy okno pokoju,
       do którego jest wpięty
     \end{enumreq}
     \strut}
   \\

   \hline
     \parbox[t]{4cm}{\textbf{Nakład godzinowy (planowany / włożony)}} &
     \parbox[t]{11.5cm}{\strut
       7h / 5h
     \strut}\\

       \hline
         \parbox[t]{4cm}{\textbf{Ukończono?}} &
         \parbox[t]{11.5cm}{\strut
           Tak.
         \strut}\\

         \hline
     \end{tabular}

\vspace{1em}

\begin{tabular}{ | l | l | }
 \hline
   \textbf{Identyfikator} &
   WF-03
   \\

 \hline
   \textbf{Treść} & \parbox[t]{11.5cm}{\strut
     Jako użytkownik serwera czatu, chcę po wpięciu
     do pokoju zobaczyć ostatnie wiadomości wysłane
     przed moim dołączeniem, aby dowiedzieć się, co
     tam się obecnie dzieje.
   \strut}\\

 \hline
   \parbox[t]{4cm}{\textbf{Kryteria akceptacji}} & \parbox[t]{11.5cm}{\strut
     \begin{enumreq}
       \item Użytkownik po wpięciu się do pokoju zobaczy
       10 najnowszych wiadomości wysłanych do pokoju
       przed jego dołączeniem (lub mniej, jeżeli
       dotychczas nie wysłano do pokoju co najmniej
       10 wiadomości)
     \end{enumreq}
     \strut}
   \\

   \hline
     \parbox[t]{4cm}{\textbf{Nakład godzinowy (planowany / włożony)}} &
     \parbox[t]{11.5cm}{\strut
       6h / 5h
     \strut}\\

 \hline
   \parbox[t]{4cm}{\textbf{Ukończono?}} &
   \parbox[t]{11.5cm}{\strut
     Tak.
   \strut}\\

   \hline
\end{tabular}

\vspace{1em}

\begin{tabular}{ | l | l | }
 \hline
   \textbf{Identyfikator} &
   WF-04
   \\

 \hline
   \textbf{Treść} & \parbox[t]{11.5cm}{\strut
     Jako użytkownik serwera czatu, chcę wysłać
     wiadomość do pokoju w który jestem wpięty, aby
     zobaczyli ją inni uczestnicy dyskusji.
   \strut}\\

 \hline
   \parbox[t]{4cm}{\textbf{Kryteria akceptacji}} & \parbox[t]{11.5cm}{\strut
     \begin{enumreq}
       \item Użytkownik wpisze tekst wiadomości w polu
       tekstowym u dołu czatu
       \item Wiadomość wpisana w polu tekstowym zostanie
       wysłana po wciśnięciu klawisza ,,Enter'', gdy aktywne
       będzie pole tekstowe
       \item Wiadomość wpisana w polu tekstowym zostanie
       wysłana po naciśnięciu przycisku ,,Wyślij'',
       widocznego obok pola tekstowego
       \item Po wysłaniu wiadomości, pole tekstowe zostanie
       wyczyszczone (niezależnie od tego czy wiadomość
       zostanie doręczona)
       \item Wiadomość wysłana do pokoju jest pokazywana
       wszystkim użytkownikom podpiętym do czatu u dołu
       strony
       \item Nowa wiadomość jest pokazywana wraz z nazwą
       użytkownika wysyłającego u dołu konwersacji
     \end{enumreq}
     \strut}
   \\

   \hline
     \parbox[t]{4cm}{\textbf{Nakład godzinowy (planowany / włożony)}} &
     \parbox[t]{11.5cm}{\strut
       3.5h / 5h
     \strut}\\
 \hline

   \parbox[t]{4cm}{\textbf{Ukończono?}} &
   \parbox[t]{11.5cm}{\strut
     Tak.
   \strut}\\

   \hline
\end{tabular}

\vspace{1em}

\begin{tabular}{ | l | l | }
 \hline
   \textbf{Identyfikator} &
   WF-07
   \\

 \hline
   \textbf{Treść} & \parbox[t]{11.5cm}{\strut
     Jako użytkownik serwera czatu, chcę odpiąć się od pokoju,
     aby wpiąć się do innego pokoju.
   \strut}\\

 \hline
   \parbox[t]{4cm}{\textbf{Powiązane zasady biznesowe}} & \parbox[t]{11.5cm}{\strut
     ZP-06 Użytkownik może się samodzielnie wypiąć z pokoju,
     do którego jest wpięty

   \strut}\\

 \hline
   \parbox[t]{4cm}{\textbf{Kryteria akceptacji}} & \parbox[t]{11.5cm}{\strut
     \begin{enumreq}
       \item W oknie pokoju użytkownik zobaczy przycisk
       lub link ,,Opuść pokój''.
       \item Po kliknięciu w ,,Opuść pokój'', użytkownik
       zobaczy listę pokojów.
     \end{enumreq}
     \strut}
   \\

   \hline
     \parbox[t]{4cm}{\textbf{Nakład godzinowy (planowany / włożony)}} &
     \parbox[t]{11.5cm}{\strut
       2h / 1h
     \strut}\\

     \hline
       \parbox[t]{4cm}{\textbf{Ukończono?}} &
       \parbox[t]{11.5cm}{\strut
         Tak.
       \strut}\\
 \hline
\end{tabular}

\vspace{1em}

\begin{tabular}{ | l | l | }
	\hline
		\textbf{Identyfikator} &
		HN-07
		\\

	\hline
		\textbf{Treść} & \parbox[t]{11.5cm}{\strut
			Bufor pokoju niebędącego dedykowanym do wiadomości prywatnych zawiera do
      10 wiadomości.
		\strut}\\

	\hline
		\parbox[t]{4cm}{\textbf{Kryteria akceptacji}} & \parbox[t]{11.5cm}{\strut
			\begin{enumreq}
				\item Po przekroczeniu liczby 10 wiadomości w pokoju, bufor ulega
        ,,zawinięciu'', usuwając najstarsze wiadomości.
			\end{enumreq}
			\strut}
		\\
    \hline
      \parbox[t]{4cm}{\textbf{Nakład godzinowy (planowany / włożony)}} &
      \parbox[t]{11.5cm}{\strut
        Nakład czasowy ujęto podczas realizacji zadania WF-03.
      \strut}\\

      \hline
        \parbox[t]{4cm}{\textbf{Ukończono?}} &
        \parbox[t]{11.5cm}{\strut
          Tak.
        \strut}\\

	\hline
\end{tabular}

\subsubsection{Zadania wykraczające poza pierwotne wymagania}

Brak.


\subsection{Sprint 4}

\textbf{Termin realizacji:} 7-27 kwietnia 2019 r.

\subsubsection{Cel sprintu}
Sprint ma na celu wykonanie mechanizmu prywatnych wiadomości. Wlicza się do
niego lista użytkowników obecnych na czacie, z którymi ma być prowadzona
rozmowa, a także nawiązywanie rozmowy w pokoju PW oraz wysyłanie i odbiór
wiadomości w pokoju PW.

\subsubsection{Zadania oparte o pierwotne wymagania}

\leavevmode\hbox{}

\begin{tabular}{ | l | l | }
	\hline
		\textbf{Identyfikator} &
		WF-08
		\\

	\hline
		\textbf{Treść} & \parbox[t]{11.5cm}{\strut
			Jako użytkownik serwera czatu, chcę zobaczyć
			okno wiadomości prywatnych, aby odczytać wiadomości,
			które wysłano specjalnie do mnie.
		\strut}\\


	\hline
		\parbox[t]{4cm}{\textbf{Kryteria akceptacji}} & \parbox[t]{11.5cm}{\strut
			\begin{enumreq}
				\item Po kliknięciu w link ,,PW'', użytkownik
				zobaczy okno prywatnych wiadomości
				\item W oknie wiadomości prywatnych, użytkownik
				zobaczy listę użytkowników, od których otrzymał
				wiadomości prywatne.
				\item Po kliknięciu w link z nazwą użytkownika,
				użytkownik zobaczy prywatne wiadomości, których
				nadawcą i odbiorcą jest wskazana osoba.
			\end{enumreq}
			\strut}
		\\
    \hline
      \parbox[t]{4cm}{\textbf{Nakład godzinowy (planowany / włożony)}} &
      \parbox[t]{11.5cm}{\strut
        6h / 5h
      \strut}\\

      \hline
        \parbox[t]{4cm}{\textbf{Ukończono?}} &
        \parbox[t]{11.5cm}{\strut
          Tak.
        \strut}\\
	\hline
\end{tabular}

\vspace{1em}

\begin{tabular}{ | l | l | }
	\hline
		\textbf{Identyfikator} &
		WF-09
		\\

	\hline
		\textbf{Treść} & \parbox[t]{11.5cm}{\strut
			Jako użytkownik serwera czatu, chcę wysłać
			wiadomość prywatną do jednego użytkownika, aby
			prowadzić z nim ciągłą konwersację.
		\strut}\\

	\hline
		\parbox[t]{4cm}{\textbf{Kryteria akceptacji}} & \parbox[t]{11.5cm}{\strut
			\begin{enumreq}
				\item Użytkownik wpisze tekst wiadomości w polu
				tekstowym u dołu okna wiadomości prywatnych
				\item Wiadomość wpisana w polu tekstowym zostanie
				wysłana po wciśnięciu klawisza ,,Enter'', gdy
				aktywne
				będzie pole tekstowe
				\item Wiadomość wpisana w polu tekstowym zostanie
				wysłana po naciśnięciu przycisku ,,Wyślij'',
				widocznego obok pola tekstowego
				\item Po wysłaniu wiadomości, pole tekstowe zostanie
				wyczyszczone (niezależnie od tego czy wiadomość
				zostanie doręczona)
				\item Wiadomość wysłana w oknie zostanie pokazana
				tylko użytkownikowi, z którym trwa otwarta
				konwersacja
				\item Nowa wiadomość jest pokazywana wraz z nazwą
				użytkownika wysyłającego u dołu konwersacji
			\end{enumreq}
			\strut}
		\\

    \hline
      \parbox[t]{4cm}{\textbf{Nakład godzinowy (planowany / włożony)}} &
      \parbox[t]{11.5cm}{\strut
        2h / 1h
      \strut}\\

      \hline
        \parbox[t]{4cm}{\textbf{Ukończono?}} &
        \parbox[t]{11.5cm}{\strut
          Tak.
        \strut}\\
	\hline
\end{tabular}

\vspace{1em}

\begin{tabular}{ | l | l | }
	\hline
		\textbf{Identyfikator} &
		WF-11
		\\

	\hline
		\textbf{Treść} & \parbox[t]{11.5cm}{\strut
			Jako użytkownik serwera czatu, chcę wysłać wiadomość
			prywatną do innego użytkownika, z którym wcześniej nie
			wymieniałem takich wiadomości, aby rozpocząć z nim
			prywatną konwersację.
		\strut}\\

	\hline
		\parbox[t]{4cm}{\textbf{Kryteria akceptacji}} & \parbox[t]{11.5cm}{\strut
			\begin{enumreq}
				\item Użytkownik kliknie w oknie wiadomości
				prywatnych w przyciski ,,Nowy''.
				\item Użytkownik zobaczy monit o podanie nazwy
				użytkownika, z którym chce rozpocząć rozmowę
				\item Jeżeli użytkownik jest aktywny, wówczas
				\item Wiadomość wpisana w polu tekstowym zostanie
				wysłana po wciśnięciu klawisza ,,Enter'', gdy
				aktywne
				będzie pole tekstowe
				\item Wiadomość wpisana w polu tekstowym zostanie
				wysłana po naciśnięciu przycisku ,,Wyślij'',
				widocznego obok pola tekstowego
				\item Po wysłaniu wiadomości, pole tekstowe zostanie
				wyczyszczone (niezależnie od tego czy wiadomość
				zostanie doręczona)
				\item Wiadomość wysłana w oknie zostanie pokazana
				tylko użytkownikowi, z którym trwa otwarta
				konwersacja
				\item Nowa wiadomość jest pokazywana wraz z nazwą
				użytkownika wysyłającego u dołu konwersacji
			\end{enumreq}
			\strut}
		\\
    \hline
      \parbox[t]{4cm}{\textbf{Nakład godzinowy (planowany / włożony)}} &
      \parbox[t]{11.5cm}{\strut
        11h / 5h
      \strut}\\

      \hline
        \parbox[t]{4cm}{\textbf{Ukończono?}} &
        \parbox[t]{11.5cm}{\strut
          Nie -- przeniesiono do kolejnego sprintu.
        \strut}\\

	\hline
\end{tabular}

\vspace{1em}

\begin{tabular}{ | l | l | }
	\hline
		\textbf{Identyfikator} &
		HN-05
		\\

	\hline
		\textbf{Treść} & \parbox[t]{11.5cm}{\strut
			Wiadomości prywatne są czyszczone niezwłocznie po rozłączeniu się przez dowolnego z rozmówców.
		\strut}\\

	\hline
		\parbox[t]{4cm}{\textbf{Powiązane zasady biznesowe}} & \parbox[t]{11.5cm}{\strut
			ZW-10 Wiadomości prywatne są utrzymywane dopóki nadawca i odbiorca mają aktywną sesję na serwerze.
		\strut}\\

	\hline
		\parbox[t]{4cm}{\textbf{Kryteria akceptacji}} & \parbox[t]{11.5cm}{\strut
			\begin{enumreq}
				\item Po zamknięciu sesji użytkownika, wiadomości prywatne których był nadawcą lub odbiorcą ulegają
        usunięciu.
			\end{enumreq}
			\strut}
		\\

	\hline
  \parbox[t]{4cm}{\textbf{Nakład godzinowy (planowany / włożony)}} &
  \parbox[t]{11.5cm}{\strut
   --
  \strut}\\

  \hline
    \parbox[t]{4cm}{\textbf{Ukończono?}} &
    \parbox[t]{11.5cm}{\strut
      Nie -- zadanie przeniesiono do kolejnego sprintu.
    \strut}\\
\hline
\end{tabular}

\vspace{1em}

\begin{tabular}{ | l | l | }
	\hline
		\textbf{Identyfikator} &
		HN-06
		\\

	\hline
		\textbf{Treść} & \parbox[t]{11.5cm}{\strut
			Bufor pokoju wiadomości prywatnych zawiera do 100 wiadomości.
		\strut}\\

	\hline
		\parbox[t]{4cm}{\textbf{Powiązane zasady biznesowe}} & \parbox[t]{11.5cm}{\strut
			ZW-11 Dla każdej pary użytkowników, na serwerze jest gromadzone co najwyżej 100 wiadomości prywatnych.
		\strut}\\

	\hline
		\parbox[t]{4cm}{\textbf{Kryteria akceptacji}} & \parbox[t]{11.5cm}{\strut
			\begin{enumreq}
				\item Po przekroczeniu liczby 100 wiadomości prywatnych w pokoju, bufor
        ulega ,,zawinięciu'', usuwając najstarsze wiadomości.
			\end{enumreq}
			\strut}
		\\

	\hline
  \parbox[t]{4cm}{\textbf{Nakład godzinowy (planowany / włożony)}} &
  \parbox[t]{11.5cm}{\strut
    --
  \strut}\\

  \hline
    \parbox[t]{4cm}{\textbf{Ukończono?}} &
    \parbox[t]{11.5cm}{\strut
      Nie -- zadanie przeniesiono do kolejnego sprintu.
    \strut}\\
\hline
\end{tabular}

\subsubsection{Zadania wykraczające poza pierwotne wymagania}

Brak.


\subsection{Sprint 5}

\textbf{Termin realizacji:} 28 kwietnia -- 11 maja 2019 r.

\subsubsection{Cel sprintu}
Celem jest dokończenie mechanizmu prywatnych wiadomości, których nie udało się
dokończyć w poprzednim sprincie -- łączenia z wybranym użytkownikiem, z którym
wcześniej nie prowadzono korespondencji.

\subsubsection{Zadania oparte o pierwotne wymagania}

\leavevmode\hbox{}

\begin{tabular}{ | l | l | }
	\hline
		\textbf{Identyfikator} &
		WF-09
		\\

	\hline
		\textbf{Treść} & \parbox[t]{11.5cm}{\strut
			Jako użytkownik serwera czatu, chcę wysłać
			wiadomość prywatną do jednego użytkownika, aby
			prowadzić z nim ciągłą konwersację.
		\strut}\\

	\hline
		\parbox[t]{4cm}{\textbf{Kryteria akceptacji}} & \parbox[t]{11.5cm}{\strut
			\begin{enumreq}
				\item Użytkownik wpisze tekst wiadomości w polu
				tekstowym u dołu okna wiadomości prywatnych
				\item Wiadomość wpisana w polu tekstowym zostanie
				wysłana po wciśnięciu klawisza ,,Enter'', gdy
				aktywne
				będzie pole tekstowe
				\item Wiadomość wpisana w polu tekstowym zostanie
				wysłana po naciśnięciu przycisku ,,Wyślij'',
				widocznego obok pola tekstowego
				\item Po wysłaniu wiadomości, pole tekstowe zostanie
				wyczyszczone (niezależnie od tego czy wiadomość
				zostanie doręczona)
				\item Wiadomość wysłana w oknie zostanie pokazana
				tylko użytkownikowi, z którym trwa otwarta
				konwersacja
				\item Nowa wiadomość jest pokazywana wraz z nazwą
				użytkownika wysyłającego u dołu konwersacji
			\end{enumreq}
			\strut}
		\\

    \hline
      \parbox[t]{4cm}{\textbf{Nakład godzinowy (planowany / włożony)}} &
      \parbox[t]{11.5cm}{\strut
        4h / 5h
      \strut}\\

      \hline
        \parbox[t]{4cm}{\textbf{Ukończono?}} &
        \parbox[t]{11.5cm}{\strut
          Tak.
        \strut}\\

\hline
\end{tabular}

\vspace{1em}

\begin{tabular}{ | l | l | }
	\hline
		\textbf{Identyfikator} &
		WF-11
		\\

	\hline
		\textbf{Treść} & \parbox[t]{11.5cm}{\strut
			Jako użytkownik serwera czatu, chcę wysłać wiadomość
			prywatną do innego użytkownika, z którym wcześniej nie
			wymieniałem takich wiadomości, aby rozpocząć z nim
			prywatną konwersację.
		\strut}\\

	\hline
		\parbox[t]{4cm}{\textbf{Kryteria akceptacji}} & \parbox[t]{11.5cm}{\strut
			\begin{enumreq}
				\item Użytkownik kliknie w oknie wiadomości
				prywatnych w przyciski ,,Nowy''.
				\item Użytkownik zobaczy monit o podanie nazwy
				użytkownika, z którym chce rozpocząć rozmowę
				\item Jeżeli użytkownik jest aktywny, wówczas
				\item Wiadomość wpisana w polu tekstowym zostanie
				wysłana po wciśnięciu klawisza ,,Enter'', gdy
				aktywne
				będzie pole tekstowe
				\item Wiadomość wpisana w polu tekstowym zostanie
				wysłana po naciśnięciu przycisku ,,Wyślij'',
				widocznego obok pola tekstowego
				\item Po wysłaniu wiadomości, pole tekstowe zostanie
				wyczyszczone (niezależnie od tego czy wiadomość
				zostanie doręczona)
				\item Wiadomość wysłana w oknie zostanie pokazana
				tylko użytkownikowi, z którym trwa otwarta
				konwersacja
				\item Nowa wiadomość jest pokazywana wraz z nazwą
				użytkownika wysyłającego u dołu konwersacji
			\end{enumreq}
			\strut}
		\\


	\hline
  \parbox[t]{4cm}{\textbf{Nakład godzinowy (planowany / włożony)}} &
  \parbox[t]{11.5cm}{\strut
    3h / 5h
  \strut}\\

  \hline
    \parbox[t]{4cm}{\textbf{Ukończono?}} &
    \parbox[t]{11.5cm}{\strut
      Tak.
    \strut}\\
\hline
\end{tabular}

\vspace{1em}

\begin{tabular}{ | l | l | }
	\hline
		\textbf{Identyfikator} &
		HN-07
		\\

	\hline
		\textbf{Treść} & \parbox[t]{11.5cm}{\strut
			Bufor pokoju niebędącego dedykowanym do wiadomości prywatnych zawiera do
      10 wiadomości.
		\strut}\\

	\hline
		\parbox[t]{4cm}{\textbf{Kryteria akceptacji}} & \parbox[t]{11.5cm}{\strut
			\begin{enumreq}
				\item Po przekroczeniu liczby 10 wiadomości w pokoju, bufor ulega
        ,,zawinięciu'', usuwając najstarsze wiadomości.
			\end{enumreq}
			\strut}
		\\
    \hline
      \parbox[t]{4cm}{\textbf{Nakład godzinowy (planowany / włożony)}} &
      \parbox[t]{11.5cm}{\strut
        Nakład czasu ujęto w zadaniu WF-11.
      \strut}\\

      \hline
        \parbox[t]{4cm}{\textbf{Ukończono?}} &
        \parbox[t]{11.5cm}{\strut
          Tak.
        \strut}\\

	\hline
\end{tabular}

\subsubsection{Zadania wykraczające poza pierwotne wymagania}

Brak.


\subsection{Sprint 6}

\textbf{Termin realizacji:} 12-26 maja 2019 r.

\subsubsection{Cel sprintu}
W tym sprincie zaplanowano wykonanie narzędzi administracyjnych, służących
do zarządzania serwerem.

\subsubsection{Zadania oparte o pierwotne wymagania}

\leavevmode\hbox{}

\begin{tabular}{ | l | l | }
	\hline
		\textbf{Identyfikator} &
		WF-12
		\\

	\hline
		\textbf{Treść} & \parbox[t]{11.5cm}{\strut
			Jako administrator, chcę utworzyć nowy pokój, aby umożliwić użytkownikom konwersację w węższym gronie.
		\strut}\\

	\hline
		\parbox[t]{4cm}{\textbf{Kryteria akceptacji}} & \parbox[t]{11.5cm}{\strut
			\begin{enumreq}
				\item Administrator kliknie w oknie z listą pokojów w przycisk ,,Nowy''.
				\item Administrator zobaczy monit o podanie nazwy
				nowego pokoju.
				\item Administrator po podaniu nazwy i zaakceptowaniu,
        zostanie przeniesiony do listy pokojów, na której
        będzie widoczna nazwa dodanego pokoju.
				\item Administrator i inni użytkownicy mogą wpiąć się do nowoutworzonego pokoju.
			\end{enumreq}
			\strut}
		\\

	\hline
\end{tabular}

\vspace{1em}

\begin{tabular}{ | l | l | }
	\hline
		\textbf{Identyfikator} &
		WF-13
		\\

	\hline
		\textbf{Treść} & \parbox[t]{11.5cm}{\strut
			Jako administrator, chcę usunąć zbędny pokój, aby utrzymać porządek na swoim serwerze czatu.
		\strut}\\

	\hline
		\parbox[t]{4cm}{\textbf{Kryteria akceptacji}} & \parbox[t]{11.5cm}{\strut
			\begin{enumreq}
				\item Administrator wejdzie do pokoju, który chce
        usunąć.
        \item Administrator zobaczy obok tytułu z nazwą pokoju
        przycisk ,,Usuń''.
				\item Administrator po kliknięciu przycisku zobaczy
        monit z potwierdzeniem działania.
        \item Administrator potwierdzi decyzję w monicie.
        \item Po potwierdzeniu decyzji o usunięciu, administrator
        zostanie przeniesiony do listy pokojów, na której nie
         będzie już widniała nazwa usuniętego pokoju.
				\item Pozostali użytkownicy w usuniętym pokoju zostaną niezwłocznie od niego odpięci i zobaczą monit systemowy informujący o usunięciu pokoju.
			\end{enumreq}
			\strut}
		\\

	\hline
  \parbox[t]{4cm}{\textbf{Nakład godzinowy (planowany / włożony)}} &
  \parbox[t]{11.5cm}{\strut
    2h / 3h
  \strut}\\

  \hline
    \parbox[t]{4cm}{\textbf{Ukończono?}} &
    \parbox[t]{11.5cm}{\strut
      Tak.
    \strut}\\
\hline
\end{tabular}

\vspace{1em}

\begin{tabular}{ | l | l | }
	\hline
		\textbf{Identyfikator} &
		WF-14
		\\

	\hline
		\textbf{Treść} & \parbox[t]{11.5cm}{\strut
			Jako administrator, chcę usunąć użytkownika z pokoju,
      aby utrzymać należyty poziom konwersacji.
		\strut}\\

	\hline
		\parbox[t]{4cm}{\textbf{Kryteria akceptacji}} & \parbox[t]{11.5cm}{\strut
			\begin{enumreq}
				\item Administrator wejdzie do pokoju.
        \item Administrator najedzie na nazwę użytkownika którego chce usunąć z pokoju i kliknie na przycisk z
        nazwą ,,Usuń z pokoju''.
				\item Administrator po kliknięciu przycisku zobaczy
        monit z potwierdzeniem działania.
        \item Administrator potwierdzi decyzję w monicie.
        \item Po potwierdzeniu decyzji o usunięciu, administrator
        (tak samo jak każdy inny użytkownik podpięty do pokoju) zobaczy wiadomość systemową o usunięciu z konwersacji.
				\item Usunięty użytkownik zostanie niezwłocznie wypięty
        z pokoju, a także zobaczy monit o przyczynie wypięcia.
			\end{enumreq}
			\strut}
		\\

	\hline
  \parbox[t]{4cm}{\textbf{Nakład godzinowy (planowany / włożony)}} &
  \parbox[t]{11.5cm}{\strut
    6h / 5h
  \strut}\\

  \hline
    \parbox[t]{4cm}{\textbf{Ukończono?}} &
    \parbox[t]{11.5cm}{\strut
      Tak.
    \strut}\\
\hline
\end{tabular}

\vspace{1em}

\begin{tabular}{ | l | l | }
	\hline
		\textbf{Identyfikator} &
		WF-15
		\\

	\hline
		\textbf{Treść} & \parbox[t]{11.5cm}{\strut
			Jako administrator, chcę usunąć użytkownika z serwera,
      aby ukarać go za łamanie zasad netykiety.
		\strut}\\

	\hline
		\parbox[t]{4cm}{\textbf{Kryteria akceptacji}} & \parbox[t]{11.5cm}{\strut
			\begin{enumreq}
				\item Administrator wejdzie do pokoju.
        \item Administrator najedzie na nazwę użytkownika którego chce usunąć z pokoju i kliknie na przycisk z
        nazwą ,,Usuń z serwera''.
				\item Administrator po kliknięciu przycisku zobaczy
        monit z potwierdzeniem działania.
        \item Administrator potwierdzi decyzję w monicie.
        \item Po potwierdzeniu decyzji o usunięciu, administrator
        (tak samo jak każdy inny użytkownik podpięty do pokoju) zobaczy wiadomość systemową o usunięciu z serwera.
				\item Usunięty użytkownik zostanie niezwłocznie wypięty
        z pokoju i jego sesja zostanie zakończona, a także pokazany zostanie monit o przyczynie tych zdarzeń (usunięcie z serwera czatu).
			\end{enumreq}
			\strut}
		\\

	\hline
  \parbox[t]{4cm}{\textbf{Nakład godzinowy (planowany / włożony)}} &
  \parbox[t]{11.5cm}{\strut
    2h / 5h
  \strut}\\

  \hline
    \parbox[t]{4cm}{\textbf{Ukończono?}} &
    \parbox[t]{11.5cm}{\strut
      Tak.
    \strut}\\
\hline
\end{tabular}

\vspace{1em}

\begin{tabular}{ | l | l | }
	\hline
		\textbf{Identyfikator} &
		HN-03
		\\

	\hline
		\textbf{Treść} & \parbox[t]{11.5cm}{\strut
			Loginy administratorów w oknach czatu są pogrubione i pokolorowane na czerwono.
		\strut}\\

	\hline
		\parbox[t]{4cm}{\textbf{Kryteria akceptacji}} & \parbox[t]{11.5cm}{\strut
			\begin{enumreq}
				\item Nazwy administratorów w oknach czatu są pogrubione
        i pokolorowane na czerowono.
			\end{enumreq}
			\strut}
		\\

	\hline
  \parbox[t]{4cm}{\textbf{Nakład godzinowy (planowany / włożony)}} &
  \parbox[t]{11.5cm}{\strut
    1h / 1h
  \strut}\\

  \hline
    \parbox[t]{4cm}{\textbf{Ukończono?}} &
    \parbox[t]{11.5cm}{\strut
      Tak.
    \strut}\\
\hline
\end{tabular}


\vspace{1em}

\begin{tabular}{ | l | l | }
	\hline
		\textbf{Identyfikator} &
		HN-04
		\\

	\hline
		\textbf{Treść} & \parbox[t]{11.5cm}{\strut
			Nazwy pokojów mają od 3 do 32 znaków alfanumerycznych długości,
      przy czym nigdy dwa pokoje nie mają tej samej nazwy.
		\strut}\\

	\hline
		\parbox[t]{4cm}{\textbf{Powiązane zasady biznesowe}} & \parbox[t]{11.5cm}{\strut
			ZP-02 Każdy pokój ma unikalną nazwę będącą ciągiem
      alfanumerycznym od 3 do 32 znaków.
		\strut}\\

	\hline
		\parbox[t]{4cm}{\textbf{Kryteria akceptacji}} & \parbox[t]{11.5cm}{\strut
			\begin{enumreq}
				\item Nie jest możlwe utworzenie pokoju o nazwie, która
        już wcześniej się pojawiała
        \item Nie jest możliwe utworzenie pokoju o nazwie krótszej niż 3 znaki i dłuższej niż 32 znaki.
        \item Nie jest możliwe utworzenie pokoju o nazwie zawierającej znaki inne niż litery alfabetu łacińskiego, cyfry i znak podkreślenia.
			\end{enumreq}
			\strut}
		\\

	\hline
  \parbox[t]{4cm}{\textbf{Nakład godzinowy (planowany / włożony)}} &
  \parbox[t]{11.5cm}{\strut
    Nakład czasowy ujęto w ramach zadania WF-13.
  \strut}\\

  \hline
    \parbox[t]{4cm}{\textbf{Ukończono?}} &
    \parbox[t]{11.5cm}{\strut
      Tak.
    \strut}\\
\hline
\end{tabular}

\subsubsection{Zadania wykraczające poza pierwotne wymagania}

Brak.


\subsection{Sprint 7}

\textbf{Termin realizacji:} 27 maja -- 1 czerwca 2019 r.

\subsubsection{Cel sprintu}
Celem sprintu jest wykonanie testów całego systemu, a także wykrycie i
poprawienie niedoróbek.

\subsubsection{Zadania oparte o pierwotne wymagania}

\leavevmode\hbox{}

\begin{tabular}{ | l | l | }
	\hline
		\textbf{Identyfikator} &
		WF-05
		\\

	\hline
		\textbf{Treść} & \parbox[t]{11.5cm}{\strut
			Jako użytkownik serwera czatu, chcę chcę zobaczyć
			powiadomienie o wpięciu się nowego użytkownika do
			pokoju w którym sam jestem obecnie wpięty, aby powitać
			nowego dyskutanta
		\strut}\\

	\hline
		\parbox[t]{4cm}{\textbf{Kryteria akceptacji}} & \parbox[t]{11.5cm}{\strut
			\begin{enumreq}
				\item Niezwłocznie po wpięciu się użytkownika do
				pokoju, serwer wyśle wiadomość systemową o treści
				,,Użytkownik ... dołączył do pokoju'', widoczną
				dla wszystkich użytkowników wpiętych do tego pokoju
			\end{enumreq}
			\strut}
		\\

	\hline

  \parbox[t]{4cm}{\textbf{Nakład godzinowy (planowany / włożony)}} &
  \parbox[t]{11.5cm}{\strut
    3h / 3h
  \strut}\\

  \hline
    \parbox[t]{4cm}{\textbf{Ukończono?}} &
    \parbox[t]{11.5cm}{\strut
      Tak.
    \strut}\\
\hline
\end{tabular}

\vspace{1em}

\begin{tabular}{ | l | l | }
	\hline
		\textbf{Identyfikator} &
		WF-06
		\\

	\hline
		\textbf{Treść} & \parbox[t]{11.5cm}{\strut
			Jako użytkownik serwera czatu, chcę zobaczyć
			powiadomienie o opuszczeniu pokoju przez użytkownika,
			aby łatwo zorientować się, że nie bierze już udziału
			w dyskusji.
		\strut}\\

	\hline
		\parbox[t]{4cm}{\textbf{Kryteria akceptacji}} & \parbox[t]{11.5cm}{\strut
			\begin{enumreq}
				\item Niezwłocznie po wypięciu się użytkownika z
				pokoju, serwer wyśle wiadomość systemową, widoczną
				dla wszystkich użytkowników wpiętych do tego pokoju,
				o treści:
				\begin{enumerate}
					\item ,,Użytkownik ... opuścił pokój'', gdy
					użytkownik samodzielnie wypiął się z pokoju
					\item ,,Użytkownik ... stracił połączenie'',
					gdy użytkownik został wypięty z pokoju na skutek
					przerwania sesji z uwagi na zerwanie połączenia
					\item ,,Użytkownik ... został wyrzucony'', gdy
					użytkownik został wypięty wskutek interwencji
					administratora
				\end{enumerate}
			\end{enumreq}
			\strut}
		\\

	\hline

  \parbox[t]{4cm}{\textbf{Nakład godzinowy (planowany / włożony)}} &
  \parbox[t]{11.5cm}{\strut
    2h / 5h
  \strut}\\

  \hline
    \parbox[t]{4cm}{\textbf{Ukończono?}} &
    \parbox[t]{11.5cm}{\strut
      Tak.
    \strut}\\
\hline
\end{tabular}

\vspace{1em}

\begin{tabular}{ | l | l | }
	\hline
		\textbf{Identyfikator} &
		WF-16
		\\

	\hline
		\textbf{Treść} & \parbox[t]{11.5cm}{\strut
			Jako administrator, chcę zmienić swoje hasło, aby zabezpieczyć swoje hasło w razie ujawnienia go osobie niepowołanej, bez zmiany plików konfiguracyjnych i restartowania całego serwera.
		\strut}\\

	\hline
		\parbox[t]{4cm}{\textbf{Kryteria akceptacji}} & \parbox[t]{11.5cm}{\strut
			\begin{enumreq}
				\item Administrator wejdzie na kartę ,,Moje konto''.
        \item Administrator kliknie na przycisk ,,Zmień hasło'',
        widoczny pod nazwą użytkownika.
				\item Administrator zobaczy monit zmiany hasła,
        zawierający jedno pole tekstowe na stare hasło i dwa na
        nowe hasło (wszystkie trzy ukryte przed podglądaniem
        treści podczas ich wprowadzania).
        \item Administrator potwierdzi decyzję o zmianie hasła w monicie.
        \item Po potwierdzeniu decyzji, administrator zobaczy wiadomość systemową o zmianie hasła.
        \item Administrator rozłączy się z serwerem.
        \item Administrator spróbuje rozpocząć nową sesję z
        serwerem, autoryzując się nowym hasłem.
        \item Nowe hasło zostanie zaakceptowane przez serwer,
        sesja zostanie rozpoczęta prawidłowo.
			\end{enumreq}
			\strut}
		\\

	\hline
  \parbox[t]{4cm}{\textbf{Nakład godzinowy (planowany / włożony)}} &
  \parbox[t]{11.5cm}{\strut
    2h / 2h
  \strut}\\

  \hline
    \parbox[t]{4cm}{\textbf{Ukończono?}} &
    \parbox[t]{11.5cm}{\strut
      Tak.
    \strut}\\
\hline
\end{tabular}

\vspace{1em}

\begin{tabular}{ | l | l | }
	\hline
		\textbf{Identyfikator} &
	WS-03
		\\

	\hline
		\textbf{Treść} & \parbox[t]{11.5cm}{
			Aplikacja czatu będzie dostosowana przede wszystkim
			do obsługi z wykorzystaniem urządzeń mobilnych.
		}\\

	\hline      \parbox[t]{4cm}{\textbf{Nakład godzinowy (planowany / włożony)}} &
        \parbox[t]{11.5cm}{\strut
          2h / 0h
        \strut}\\

        \hline
          \parbox[t]{4cm}{\textbf{Ukończono?}} &
          \parbox[t]{11.5cm}{\strut
            Nie -- idea zarzucona, pozostano przy wersji dla ekranów
						jednego typu urządzenia - smartfonu. Pozostałe ekrany pokazują
						przeskalowany interfejs.
          \strut}\\
  	\hline
\end{tabular}

\subsubsection{Zadania wykraczające poza pierwotne wymagania}

Brak.


\section{Wyzwania i niepowodzenia}
W trakcie realizacji projektu nie wszystko przebiegło po myśli
zespołu. Przede wszystkim, wiele do życzenia pozostawiała prędkość kompilacji,
znacznie spadająca wraz z przyrastającym kodem źródłowym. Co ciekawe, w początkowej
fazie prac, za źródło problemów uważano implementację kompilatora ViuAct w języku Python.
Ustępuje on wydajnością programom opracowanym w językach takich jak C, czy też OCaml. Ku
zaskoczeniu studentów, wraz z kolejnymi linijkami kodu, widocznemu wydłużeniu
ulegała faza asemblacji, za którą odpowiadał asembler maszyny wirtualnej ViuaVM.
Wreszcie, w końcowych wersjach serwera ViuaChat, asemblacja była do 5 razy
bardziej czasochłonna od kompilacji. Niestety, poprawianie wydajności asemblera
pozostawało poza zakresem projektu, a wydłużony czas asemblacji musiał
zostać zaakceptowany.

Niee udało się również zrealizować kompilatora języka ViuAct w
docelowym języku programowania, jakim miał zostać OCaml. Przeciwnikiem okazał się tu
harmonogram, zdecydowanie zbyt napięty, by pozwolić na przepisanie prototypu.
Pierwsze tygodnie prac nad systemem ViuaChat dodatkowo ujawniły niezgodności 
pomiędzy zakładanymi rezultatami kompilacji a jej faktycznymi wynikami. Ich ujawnienie
nie byłoby możliwe, gdyby nie zastosowanie teoretycznych konstrukcji składniowych
do zaimplementowania realnych programów o wartości użytkowej. Wobec tego, nie chcąc
ryzykować dalszych problemów, pozostano przy dalszym rozwoju ,,pythonowego'' prototypu.

