\subsection{Sprint 3}

\textbf{Termin realizacji:} 24 marca -- 6 kwietnia 2019 r.

\subsubsection{Cel sprintu}
Po zakończeniu sprintu, powinno być możliwe podpięcie się do pokoju i rozmowa z
innymi użytkownikami, którzy są podpięci do tego samego pokoju.

\subsubsection{Zadania oparte o pierwotne wymagania}

\leavevmode\hbox{}

\begin{tabular}{ | l | l | }
 \hline
   \textbf{Identyfikator} &
   WF-02
   \\

 \hline
   \textbf{Treść} & \parbox[t]{11.5cm}{\strut
     Jako użytkownik serwera czatu, chcę wpiąć się do pokoju,
     aby wziąć udział w dyskusji.
   \strut}\\

 \hline
   \parbox[t]{4cm}{\textbf{Kryteria akceptacji}} & \parbox[t]{11.5cm}{\strut
     \begin{enumreq}
       \item Użytkownik, który ma otwartą sesję
       połączenia z serwerem czatu i nie jest wpięty
       do żadnego pokoju, zobaczy listę pokojów.
       \item Użytkownik, po kilknięciu w liście pokojów
       na nazwę pokoju, zostanie do niego podpięty
       \item Użytkownik po wpięciu się do pokoju zobaczy
       okno pokoju
       \item Użytkownik, który ma otwartą sesję
       połączenia z serwerem i jest wpięty do pokoju,
       po odświeżeniu przeglądarki zobaczy okno pokoju,
       do którego jest wpięty
     \end{enumreq}
     \strut}
   \\

   \hline
     \parbox[t]{4cm}{\textbf{Nakład godzinowy (planowany / włożony)}} &
     \parbox[t]{11.5cm}{\strut
       7h / 5h
     \strut}\\

       \hline
         \parbox[t]{4cm}{\textbf{Ukończono?}} &
         \parbox[t]{11.5cm}{\strut
           Tak.
         \strut}\\

         \hline
     \end{tabular}

\vspace{1em}

\begin{tabular}{ | l | l | }
 \hline
   \textbf{Identyfikator} &
   WF-03
   \\

 \hline
   \textbf{Treść} & \parbox[t]{11.5cm}{\strut
     Jako użytkownik serwera czatu, chcę po wpięciu
     do pokoju zobaczyć ostatnie wiadomości wysłane
     przed moim dołączeniem, aby dowiedzieć się, co
     tam się obecnie dzieje.
   \strut}\\

 \hline
   \parbox[t]{4cm}{\textbf{Kryteria akceptacji}} & \parbox[t]{11.5cm}{\strut
     \begin{enumreq}
       \item Użytkownik po wpięciu się do pokoju zobaczy
       10 najnowszych wiadomości wysłanych do pokoju
       przed jego dołączeniem (lub mniej, jeżeli
       dotychczas nie wysłano do pokoju co najmniej
       10 wiadomości)
     \end{enumreq}
     \strut}
   \\

   \hline
     \parbox[t]{4cm}{\textbf{Nakład godzinowy (planowany / włożony)}} &
     \parbox[t]{11.5cm}{\strut
       6h / 5h
     \strut}\\

 \hline
   \parbox[t]{4cm}{\textbf{Ukończono?}} &
   \parbox[t]{11.5cm}{\strut
     Tak.
   \strut}\\

   \hline
\end{tabular}

\vspace{1em}

\begin{tabular}{ | l | l | }
 \hline
   \textbf{Identyfikator} &
   WF-04
   \\

 \hline
   \textbf{Treść} & \parbox[t]{11.5cm}{\strut
     Jako użytkownik serwera czatu, chcę wysłać
     wiadomość do pokoju w który jestem wpięty, aby
     zobaczyli ją inni uczestnicy dyskusji.
   \strut}\\

 \hline
   \parbox[t]{4cm}{\textbf{Kryteria akceptacji}} & \parbox[t]{11.5cm}{\strut
     \begin{enumreq}
       \item Użytkownik wpisze tekst wiadomości w polu
       tekstowym u dołu czatu
       \item Wiadomość wpisana w polu tekstowym zostanie
       wysłana po wciśnięciu klawisza ,,Enter'', gdy aktywne
       będzie pole tekstowe
       \item Wiadomość wpisana w polu tekstowym zostanie
       wysłana po naciśnięciu przycisku ,,Wyślij'',
       widocznego obok pola tekstowego
       \item Po wysłaniu wiadomości, pole tekstowe zostanie
       wyczyszczone (niezależnie od tego czy wiadomość
       zostanie doręczona)
       \item Wiadomość wysłana do pokoju jest pokazywana
       wszystkim użytkownikom podpiętym do czatu u dołu
       strony
       \item Nowa wiadomość jest pokazywana wraz z nazwą
       użytkownika wysyłającego u dołu konwersacji
     \end{enumreq}
     \strut}
   \\

   \hline
     \parbox[t]{4cm}{\textbf{Nakład godzinowy (planowany / włożony)}} &
     \parbox[t]{11.5cm}{\strut
       3.5h / 5h
     \strut}\\
 \hline

   \parbox[t]{4cm}{\textbf{Ukończono?}} &
   \parbox[t]{11.5cm}{\strut
     Tak.
   \strut}\\

   \hline
\end{tabular}

\vspace{1em}

\begin{tabular}{ | l | l | }
 \hline
   \textbf{Identyfikator} &
   WF-07
   \\

 \hline
   \textbf{Treść} & \parbox[t]{11.5cm}{\strut
     Jako użytkownik serwera czatu, chcę odpiąć się od pokoju,
     aby wpiąć się do innego pokoju.
   \strut}\\

 \hline
   \parbox[t]{4cm}{\textbf{Powiązane zasady biznesowe}} & \parbox[t]{11.5cm}{\strut
     ZP-06 Użytkownik może się samodzielnie wypiąć z pokoju,
     do którego jest wpięty

   \strut}\\

 \hline
   \parbox[t]{4cm}{\textbf{Kryteria akceptacji}} & \parbox[t]{11.5cm}{\strut
     \begin{enumreq}
       \item W oknie pokoju użytkownik zobaczy przycisk
       lub link ,,Opuść pokój''.
       \item Po kliknięciu w ,,Opuść pokój'', użytkownik
       zobaczy listę pokojów.
     \end{enumreq}
     \strut}
   \\

   \hline
     \parbox[t]{4cm}{\textbf{Nakład godzinowy (planowany / włożony)}} &
     \parbox[t]{11.5cm}{\strut
       2h / 1h
     \strut}\\

     \hline
       \parbox[t]{4cm}{\textbf{Ukończono?}} &
       \parbox[t]{11.5cm}{\strut
         Tak.
       \strut}\\
 \hline
\end{tabular}

\vspace{1em}

\begin{tabular}{ | l | l | }
	\hline
		\textbf{Identyfikator} &
		HN-07
		\\

	\hline
		\textbf{Treść} & \parbox[t]{11.5cm}{\strut
			Bufor pokoju niebędącego dedykowanym do wiadomości prywatnych zawiera do
      10 wiadomości.
		\strut}\\

	\hline
		\parbox[t]{4cm}{\textbf{Kryteria akceptacji}} & \parbox[t]{11.5cm}{\strut
			\begin{enumreq}
				\item Po przekroczeniu liczby 10 wiadomości w pokoju, bufor ulega
        ,,zawinięciu'', usuwając najstarsze wiadomości.
			\end{enumreq}
			\strut}
		\\
    \hline
      \parbox[t]{4cm}{\textbf{Nakład godzinowy (planowany / włożony)}} &
      \parbox[t]{11.5cm}{\strut
        Nakład czasowy ujęto podczas realizacji zadania WF-03.
      \strut}\\

      \hline
        \parbox[t]{4cm}{\textbf{Ukończono?}} &
        \parbox[t]{11.5cm}{\strut
          Tak.
        \strut}\\

	\hline
\end{tabular}

\subsubsection{Zadania wykraczające poza pierwotne wymagania}

Brak.
