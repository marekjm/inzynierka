\subsection{Sprint 5}

\textbf{Termin realizacji:} 28 kwietnia -- 11 maja 2019 r.

\subsubsection{Cel sprintu}
Celem jest dokończenie mechanizmu prywatnych wiadomości, których nie udało się
dokończyć w poprzednim sprincie -- łączenia z wybranym użytkownikiem, z którym
wcześniej nie prowadzono korespondencji.

\subsubsection{Zadania oparte o pierwotne wymagania}

\leavevmode\hbox{}

\begin{tabular}{ | l | l | }
	\hline
		\textbf{Identyfikator} &
		WF-09
		\\

	\hline
		\textbf{Treść} & \parbox[t]{11.5cm}{\strut
			Jako użytkownik serwera czatu, chcę wysłać
			wiadomość prywatną do jednego użytkownika, aby
			prowadzić z nim ciągłą konwersację.
		\strut}\\

	\hline
		\parbox[t]{4cm}{\textbf{Kryteria akceptacji}} & \parbox[t]{11.5cm}{\strut
			\begin{enumreq}
				\item Użytkownik wpisze tekst wiadomości w polu
				tekstowym u dołu okna wiadomości prywatnych
				\item Wiadomość wpisana w polu tekstowym zostanie
				wysłana po wciśnięciu klawisza ,,Enter'', gdy
				aktywne
				będzie pole tekstowe
				\item Wiadomość wpisana w polu tekstowym zostanie
				wysłana po naciśnięciu przycisku ,,Wyślij'',
				widocznego obok pola tekstowego
				\item Po wysłaniu wiadomości, pole tekstowe zostanie
				wyczyszczone (niezależnie od tego czy wiadomość
				zostanie doręczona)
				\item Wiadomość wysłana w oknie zostanie pokazana
				tylko użytkownikowi, z którym trwa otwarta
				konwersacja
				\item Nowa wiadomość jest pokazywana wraz z nazwą
				użytkownika wysyłającego u dołu konwersacji
			\end{enumreq}
			\strut}
		\\

    \hline
      \parbox[t]{4cm}{\textbf{Nakład godzinowy (planowany / włożony)}} &
      \parbox[t]{11.5cm}{\strut
        4h / 5h
      \strut}\\

      \hline
        \parbox[t]{4cm}{\textbf{Ukończono?}} &
        \parbox[t]{11.5cm}{\strut
          Tak.
        \strut}\\

\hline
\end{tabular}

\vspace{1em}

\begin{tabular}{ | l | l | }
	\hline
		\textbf{Identyfikator} &
		WF-11
		\\

	\hline
		\textbf{Treść} & \parbox[t]{11.5cm}{\strut
			Jako użytkownik serwera czatu, chcę wysłać wiadomość
			prywatną do innego użytkownika, z którym wcześniej nie
			wymieniałem takich wiadomości, aby rozpocząć z nim
			prywatną konwersację.
		\strut}\\

	\hline
		\parbox[t]{4cm}{\textbf{Kryteria akceptacji}} & \parbox[t]{11.5cm}{\strut
			\begin{enumreq}
				\item Użytkownik kliknie w oknie wiadomości
				prywatnych w przyciski ,,Nowy''.
				\item Użytkownik zobaczy monit o podanie nazwy
				użytkownika, z którym chce rozpocząć rozmowę
				\item Jeżeli użytkownik jest aktywny, wówczas
				\item Wiadomość wpisana w polu tekstowym zostanie
				wysłana po wciśnięciu klawisza ,,Enter'', gdy
				aktywne
				będzie pole tekstowe
				\item Wiadomość wpisana w polu tekstowym zostanie
				wysłana po naciśnięciu przycisku ,,Wyślij'',
				widocznego obok pola tekstowego
				\item Po wysłaniu wiadomości, pole tekstowe zostanie
				wyczyszczone (niezależnie od tego czy wiadomość
				zostanie doręczona)
				\item Wiadomość wysłana w oknie zostanie pokazana
				tylko użytkownikowi, z którym trwa otwarta
				konwersacja
				\item Nowa wiadomość jest pokazywana wraz z nazwą
				użytkownika wysyłającego u dołu konwersacji
			\end{enumreq}
			\strut}
		\\


	\hline
  \parbox[t]{4cm}{\textbf{Nakład godzinowy (planowany / włożony)}} &
  \parbox[t]{11.5cm}{\strut
    3h / 5h
  \strut}\\

  \hline
    \parbox[t]{4cm}{\textbf{Ukończono?}} &
    \parbox[t]{11.5cm}{\strut
      Tak.
    \strut}\\
\hline
\end{tabular}

\vspace{1em}

\begin{tabular}{ | l | l | }
	\hline
		\textbf{Identyfikator} &
		HN-07
		\\

	\hline
		\textbf{Treść} & \parbox[t]{11.5cm}{\strut
			Bufor pokoju niebędącego dedykowanym do wiadomości prywatnych zawiera do
      10 wiadomości.
		\strut}\\

	\hline
		\parbox[t]{4cm}{\textbf{Kryteria akceptacji}} & \parbox[t]{11.5cm}{\strut
			\begin{enumreq}
				\item Po przekroczeniu liczby 10 wiadomości w pokoju, bufor ulega
        ,,zawinięciu'', usuwając najstarsze wiadomości.
			\end{enumreq}
			\strut}
		\\
    \hline
      \parbox[t]{4cm}{\textbf{Nakład godzinowy (planowany / włożony)}} &
      \parbox[t]{11.5cm}{\strut
        Nakład czasu ujęto w zadaniu WF-11.
      \strut}\\

      \hline
        \parbox[t]{4cm}{\textbf{Ukończono?}} &
        \parbox[t]{11.5cm}{\strut
          Tak.
        \strut}\\

	\hline
\end{tabular}

\subsubsection{Zadania wykraczające poza pierwotne wymagania}

Brak.
