\subsection{Sprint 6}

\textbf{Termin realizacji:} 12-26 maja 2019 r.

\subsubsection{Cel sprintu}
W tym sprincie zaplanowano wykonanie narzędzi administracyjnych, służących
do zarządzania serwerem.

\subsubsection{Zadania oparte o pierwotne wymagania}

\leavevmode\hbox{}

\begin{tabular}{ | l | l | }
	\hline
		\textbf{Identyfikator} &
		WF-12
		\\

	\hline
		\textbf{Treść} & \parbox[t]{11.5cm}{\strut
			Jako administrator, chcę utworzyć nowy pokój, aby umożliwić użytkownikom konwersację w węższym gronie.
		\strut}\\

	\hline
		\parbox[t]{4cm}{\textbf{Kryteria akceptacji}} & \parbox[t]{11.5cm}{\strut
			\begin{enumreq}
				\item Administrator kliknie w oknie z listą pokojów w przycisk ,,Nowy''.
				\item Administrator zobaczy monit o podanie nazwy
				nowego pokoju.
				\item Administrator po podaniu nazwy i zaakceptowaniu,
        zostanie przeniesiony do listy pokojów, na której
        będzie widoczna nazwa dodanego pokoju.
				\item Administrator i inni użytkownicy mogą wpiąć się do nowoutworzonego pokoju.
			\end{enumreq}
			\strut}
		\\

	\hline
\end{tabular}

\vspace{1em}

\begin{tabular}{ | l | l | }
	\hline
		\textbf{Identyfikator} &
		WF-13
		\\

	\hline
		\textbf{Treść} & \parbox[t]{11.5cm}{\strut
			Jako administrator, chcę usunąć zbędny pokój, aby utrzymać porządek na swoim serwerze czatu.
		\strut}\\

	\hline
		\parbox[t]{4cm}{\textbf{Kryteria akceptacji}} & \parbox[t]{11.5cm}{\strut
			\begin{enumreq}
				\item Administrator wejdzie do pokoju, który chce
        usunąć.
        \item Administrator zobaczy obok tytułu z nazwą pokoju
        przycisk ,,Usuń''.
				\item Administrator po kliknięciu przycisku zobaczy
        monit z potwierdzeniem działania.
        \item Administrator potwierdzi decyzję w monicie.
        \item Po potwierdzeniu decyzji o usunięciu, administrator
        zostanie przeniesiony do listy pokojów, na której nie
         będzie już widniała nazwa usuniętego pokoju.
				\item Pozostali użytkownicy w usuniętym pokoju zostaną niezwłocznie od niego odpięci i zobaczą monit systemowy informujący o usunięciu pokoju.
			\end{enumreq}
			\strut}
		\\

	\hline
  \parbox[t]{4cm}{\textbf{Nakład godzinowy (planowany / włożony)}} &
  \parbox[t]{11.5cm}{\strut
    2h / 3h
  \strut}\\

  \hline
    \parbox[t]{4cm}{\textbf{Ukończono?}} &
    \parbox[t]{11.5cm}{\strut
      Tak.
    \strut}\\
\hline
\end{tabular}

\vspace{1em}

\begin{tabular}{ | l | l | }
	\hline
		\textbf{Identyfikator} &
		WF-14
		\\

	\hline
		\textbf{Treść} & \parbox[t]{11.5cm}{\strut
			Jako administrator, chcę usunąć użytkownika z pokoju,
      aby utrzymać należyty poziom konwersacji.
		\strut}\\

	\hline
		\parbox[t]{4cm}{\textbf{Kryteria akceptacji}} & \parbox[t]{11.5cm}{\strut
			\begin{enumreq}
				\item Administrator wejdzie do pokoju.
        \item Administrator najedzie na nazwę użytkownika którego chce usunąć z pokoju i kliknie na przycisk z
        nazwą ,,Usuń z pokoju''.
				\item Administrator po kliknięciu przycisku zobaczy
        monit z potwierdzeniem działania.
        \item Administrator potwierdzi decyzję w monicie.
        \item Po potwierdzeniu decyzji o usunięciu, administrator
        (tak samo jak każdy inny użytkownik podpięty do pokoju) zobaczy wiadomość systemową o usunięciu z konwersacji.
				\item Usunięty użytkownik zostanie niezwłocznie wypięty
        z pokoju, a także zobaczy monit o przyczynie wypięcia.
			\end{enumreq}
			\strut}
		\\

	\hline
  \parbox[t]{4cm}{\textbf{Nakład godzinowy (planowany / włożony)}} &
  \parbox[t]{11.5cm}{\strut
    6h / 5h
  \strut}\\

  \hline
    \parbox[t]{4cm}{\textbf{Ukończono?}} &
    \parbox[t]{11.5cm}{\strut
      Tak.
    \strut}\\
\hline
\end{tabular}

\vspace{1em}

\begin{tabular}{ | l | l | }
	\hline
		\textbf{Identyfikator} &
		WF-15
		\\

	\hline
		\textbf{Treść} & \parbox[t]{11.5cm}{\strut
			Jako administrator, chcę usunąć użytkownika z serwera,
      aby ukarać go za łamanie zasad netykiety.
		\strut}\\

	\hline
		\parbox[t]{4cm}{\textbf{Kryteria akceptacji}} & \parbox[t]{11.5cm}{\strut
			\begin{enumreq}
				\item Administrator wejdzie do pokoju.
        \item Administrator najedzie na nazwę użytkownika którego chce usunąć z pokoju i kliknie na przycisk z
        nazwą ,,Usuń z serwera''.
				\item Administrator po kliknięciu przycisku zobaczy
        monit z potwierdzeniem działania.
        \item Administrator potwierdzi decyzję w monicie.
        \item Po potwierdzeniu decyzji o usunięciu, administrator
        (tak samo jak każdy inny użytkownik podpięty do pokoju) zobaczy wiadomość systemową o usunięciu z serwera.
				\item Usunięty użytkownik zostanie niezwłocznie wypięty
        z pokoju i jego sesja zostanie zakończona, a także pokazany zostanie monit o przyczynie tych zdarzeń (usunięcie z serwera czatu).
			\end{enumreq}
			\strut}
		\\

	\hline
  \parbox[t]{4cm}{\textbf{Nakład godzinowy (planowany / włożony)}} &
  \parbox[t]{11.5cm}{\strut
    2h / 5h
  \strut}\\

  \hline
    \parbox[t]{4cm}{\textbf{Ukończono?}} &
    \parbox[t]{11.5cm}{\strut
      Tak.
    \strut}\\
\hline
\end{tabular}

\vspace{1em}

\begin{tabular}{ | l | l | }
	\hline
		\textbf{Identyfikator} &
		HN-03
		\\

	\hline
		\textbf{Treść} & \parbox[t]{11.5cm}{\strut
			Loginy administratorów w oknach czatu są pogrubione i pokolorowane na czerwono.
		\strut}\\

	\hline
		\parbox[t]{4cm}{\textbf{Kryteria akceptacji}} & \parbox[t]{11.5cm}{\strut
			\begin{enumreq}
				\item Nazwy administratorów w oknach czatu są pogrubione
        i pokolorowane na czerowono.
			\end{enumreq}
			\strut}
		\\

	\hline
  \parbox[t]{4cm}{\textbf{Nakład godzinowy (planowany / włożony)}} &
  \parbox[t]{11.5cm}{\strut
    1h / 1h
  \strut}\\

  \hline
    \parbox[t]{4cm}{\textbf{Ukończono?}} &
    \parbox[t]{11.5cm}{\strut
      Tak.
    \strut}\\
\hline
\end{tabular}


\vspace{1em}

\begin{tabular}{ | l | l | }
	\hline
		\textbf{Identyfikator} &
		HN-04
		\\

	\hline
		\textbf{Treść} & \parbox[t]{11.5cm}{\strut
			Nazwy pokojów mają od 3 do 32 znaków alfanumerycznych długości,
      przy czym nigdy dwa pokoje nie mają tej samej nazwy.
		\strut}\\

	\hline
		\parbox[t]{4cm}{\textbf{Powiązane zasady biznesowe}} & \parbox[t]{11.5cm}{\strut
			ZP-02 Każdy pokój ma unikalną nazwę będącą ciągiem
      alfanumerycznym od 3 do 32 znaków.
		\strut}\\

	\hline
		\parbox[t]{4cm}{\textbf{Kryteria akceptacji}} & \parbox[t]{11.5cm}{\strut
			\begin{enumreq}
				\item Nie jest możlwe utworzenie pokoju o nazwie, która
        już wcześniej się pojawiała
        \item Nie jest możliwe utworzenie pokoju o nazwie krótszej niż 3 znaki i dłuższej niż 32 znaki.
        \item Nie jest możliwe utworzenie pokoju o nazwie zawierającej znaki inne niż litery alfabetu łacińskiego, cyfry i znak podkreślenia.
			\end{enumreq}
			\strut}
		\\

	\hline
  \parbox[t]{4cm}{\textbf{Nakład godzinowy (planowany / włożony)}} &
  \parbox[t]{11.5cm}{\strut
    Nakład czasowy ujęto w ramach zadania WF-13.
  \strut}\\

  \hline
    \parbox[t]{4cm}{\textbf{Ukończono?}} &
    \parbox[t]{11.5cm}{\strut
      Tak.
    \strut}\\
\hline
\end{tabular}

\subsubsection{Zadania wykraczające poza pierwotne wymagania}

Brak.
