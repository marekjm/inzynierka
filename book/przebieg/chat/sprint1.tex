\subsection{Sprint 1}

\textbf{Termin realizacji:} 2-9 marca 2019 r.

\subsubsection{Cel sprintu}
Celem sprintu było utworzenie podstawowej funkcjonalności związanej z
komunikacją frontend-backend -- ustanawiania połączenia WebSocket,
dekodowania komunikatów bo obu stronach aplikacji, a także podstawowa
walidacja funkcjonalności.

\subsubsection{Zadania oparte o pierwotne wymagania}
\leavevmode\hbox{}

\begin{tabular}{ | l | l | }
	\hline
		\textbf{Identyfikator} &
	WS-01
		\\

	\hline
		\textbf{Treść} & \parbox[t]{11.5cm}{\strut
			W rozwiązaniu należy wykorzystać środowisko Viua VM i
			język \ViuAct
		\strut}\\

	\hline
  \parbox[t]{4cm}{\textbf{Nakład godzinowy (planowany / włożony)}} & \parbox[t]{11.5cm}{\strut
    Nakład godzinowy związany z zadaniem ujęto przy realizacji zadań ZZ-01 i
    ZZ-02.
  \strut}\\

\hline
  \parbox[t]{4cm}{\textbf{Ukończono?}} &
  \parbox[t]{11.5cm}{\strut
    Tak.
  \strut}\\

\hline
\end{tabular}

\vspace{1em}

\begin{tabular}{ | l | l | }
	\hline
		\textbf{Identyfikator} &
	WS-02
		\\

	\hline
		\textbf{Treść} & \parbox[t]{11.5cm}{\strut
			Czat będzie użytkowany jako aplikacja webowa typu
			\textit{single page application}
		\strut}\\

	\hline
  \parbox[t]{4cm}{\textbf{Nakład godzinowy (planowany / włożony)}} & \parbox[t]{11.5cm}{\strut
    Nakład godzinowy związany z zadaniem ujęto przy realizacji zadań ZZ-01 i
    ZZ-02.
  \strut}\\

\hline
  \parbox[t]{4cm}{\textbf{Ukończono?}} &
  \parbox[t]{11.5cm}{\strut
    Tak.
  \strut}\\

\hline
\end{tabular}

\subsubsection{Zadania wykraczające poza pierwotne wymagania}

\leavevmode\hbox{}

\begin{tabular}{ | l | l | }
 \hline
   \textbf{Identyfikator} &
   ZZ-01
   \\

 \hline
   \textbf{Treść} & \parbox[t]{11.5cm}{\strut
     Frontend aplikacji jest w stanie nawiązać połączenie WebSocket z backendem.
   \strut}\\

 \hline
   \parbox[t]{4cm}{\textbf{Kryteria akceptacji}} & \parbox[t]{11.5cm}{\strut
     \begin{enumreq}
       \item Frontend nawiązał połączenie z serwerem (tj. doszło do
       zmiany protokołu z HTTP na WS).
       \item Frontend wysłał wiadomość do backendu, a wiadomość została
       odebrana.
       \item Backend wysłał wiadomość do frontendu, a wiadomość została
       odebrana.
     \end{enumreq}
     \strut}
   \\

   \hline
     \parbox[t]{4cm}{\textbf{Nakład godzinowy (planowany / włożony)}} & \parbox[t]{11.5cm}{\strut
       5h / 5.5h
     \strut}\\

   \hline
     \parbox[t]{4cm}{\textbf{Ukończono?}} &
     \parbox[t]{11.5cm}{\strut
       Tak.
     \strut}\\

   \hline
 \end{tabular}

 \vspace{1em}

\begin{tabular}{ | l | l | }
 \hline
   \textbf{Identyfikator} &
   ZZ-02
   \\

 \hline
   \textbf{Treść} & \parbox[t]{11.5cm}{\strut
     Obie strony (tj. frontend i backend) potrafią zakodować wiadomości w
     ustalonym formacie JSON przed ich wysyłką i odebrać je po ich odebraniu.
   \strut}\\

 \hline
   \parbox[t]{4cm}{\textbf{Kryteria akceptacji}} & \parbox[t]{11.5cm}{\strut
     \begin{enumreq}
       \item Frontend potrafi zakodować wiadomość w formacie JSON -- wiadomość
       synchroniczną oraz wiadomość asynchroniczną.
       \item Frontend potrafi odkodować wiadomość w formacie JSON.
       \item Frontend potrafi ustalić, czy odebrana wiadomość jest synchroniczna,
       czy asynchroniczna, a w przypadku synchronicznej -- dopasować ją do
       wysłanej wcześniej wiadomości.
       \item Frontend potrafi właściwie obsługiwać wyjątki wynikające z błędów
       w składni JSON odbieranych wiadomości oraz z błędów w strukturze obiektów
       JSON: braku obowiązkowych własności oraz nieprawidłowych typów tych
       wiadomości. Błędy te nie powodują krytycznego przerwania pracy działania
       programu.
       \item Backend potrafi odkodować otrzymywane wiadomości w  formacie JSON.
       \item Backend potrafi ustalić, czy otrzymana wiadomość jest synchroniczna,
       czy asynchroniczna.
       \item Backend potrafi odesłać odpowiedź pasującą do otrzymanej wiadomości
       synchronicznej.
       \item Backend potrafi właściwie obsługiwać wyjątki z tytułu niewłaściwie
       skonstruowanych wiadomości. W razie otrzymania takiej wiadomości, nie
       dochodzi do przerwania pracy serwera (tj. aktora \texttt{WsConnector}).
     \end{enumreq}
     \strut}
   \\

   \hline
     \parbox[t]{4cm}{\textbf{Nakład godzinowy (planowany / włożony)}} & \parbox[t]{11.5cm}{\strut
       10h / 12h
     \strut}\\

       \hline
         \parbox[t]{4cm}{\textbf{Ukończono?}} &
         \parbox[t]{11.5cm}{\strut
           Tak.
         \strut}\\

         \hline
     \end{tabular}
