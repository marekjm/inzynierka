\subsection{Sprint 7}

\textbf{Termin realizacji:} 27 maja -- 1 czerwca 2019 r.

\subsubsection{Cel sprintu}
Celem sprintu jest wykonanie testów całego systemu, a także wykrycie i
poprawienie niedoróbek.

\subsubsection{Zadania oparte o pierwotne wymagania}

\leavevmode\hbox{}

\begin{tabular}{ | l | l | }
	\hline
		\textbf{Identyfikator} &
		WF-05
		\\

	\hline
		\textbf{Treść} & \parbox[t]{11.5cm}{\strut
			Jako użytkownik serwera czatu, chcę chcę zobaczyć
			powiadomienie o wpięciu się nowego użytkownika do
			pokoju w którym sam jestem obecnie wpięty, aby powitać
			nowego dyskutanta
		\strut}\\

	\hline
		\parbox[t]{4cm}{\textbf{Kryteria akceptacji}} & \parbox[t]{11.5cm}{\strut
			\begin{enumreq}
				\item Niezwłocznie po wpięciu się użytkownika do
				pokoju, serwer wyśle wiadomość systemową o treści
				,,Użytkownik ... dołączył do pokoju'', widoczną
				dla wszystkich użytkowników wpiętych do tego pokoju
			\end{enumreq}
			\strut}
		\\

	\hline

  \parbox[t]{4cm}{\textbf{Nakład godzinowy (planowany / włożony)}} &
  \parbox[t]{11.5cm}{\strut
    3h / 3h
  \strut}\\

  \hline
    \parbox[t]{4cm}{\textbf{Ukończono?}} &
    \parbox[t]{11.5cm}{\strut
      Tak.
    \strut}\\
\hline
\end{tabular}

\vspace{1em}

\begin{tabular}{ | l | l | }
	\hline
		\textbf{Identyfikator} &
		WF-06
		\\

	\hline
		\textbf{Treść} & \parbox[t]{11.5cm}{\strut
			Jako użytkownik serwera czatu, chcę zobaczyć
			powiadomienie o opuszczeniu pokoju przez użytkownika,
			aby łatwo zorientować się, że nie bierze już udziału
			w dyskusji.
		\strut}\\

	\hline
		\parbox[t]{4cm}{\textbf{Kryteria akceptacji}} & \parbox[t]{11.5cm}{\strut
			\begin{enumreq}
				\item Niezwłocznie po wypięciu się użytkownika z
				pokoju, serwer wyśle wiadomość systemową, widoczną
				dla wszystkich użytkowników wpiętych do tego pokoju,
				o treści:
				\begin{enumerate}
					\item ,,Użytkownik ... opuścił pokój'', gdy
					użytkownik samodzielnie wypiął się z pokoju
					\item ,,Użytkownik ... stracił połączenie'',
					gdy użytkownik został wypięty z pokoju na skutek
					przerwania sesji z uwagi na zerwanie połączenia
					\item ,,Użytkownik ... został wyrzucony'', gdy
					użytkownik został wypięty wskutek interwencji
					administratora
				\end{enumerate}
			\end{enumreq}
			\strut}
		\\

	\hline

  \parbox[t]{4cm}{\textbf{Nakład godzinowy (planowany / włożony)}} &
  \parbox[t]{11.5cm}{\strut
    2h / 5h
  \strut}\\

  \hline
    \parbox[t]{4cm}{\textbf{Ukończono?}} &
    \parbox[t]{11.5cm}{\strut
      Tak.
    \strut}\\
\hline
\end{tabular}

\vspace{1em}

\begin{tabular}{ | l | l | }
	\hline
		\textbf{Identyfikator} &
		WF-16
		\\

	\hline
		\textbf{Treść} & \parbox[t]{11.5cm}{\strut
			Jako administrator, chcę zmienić swoje hasło, aby zabezpieczyć swoje hasło w razie ujawnienia go osobie niepowołanej, bez zmiany plików konfiguracyjnych i restartowania całego serwera.
		\strut}\\

	\hline
		\parbox[t]{4cm}{\textbf{Kryteria akceptacji}} & \parbox[t]{11.5cm}{\strut
			\begin{enumreq}
				\item Administrator wejdzie na kartę ,,Moje konto''.
        \item Administrator kliknie na przycisk ,,Zmień hasło'',
        widoczny pod nazwą użytkownika.
				\item Administrator zobaczy monit zmiany hasła,
        zawierający jedno pole tekstowe na stare hasło i dwa na
        nowe hasło (wszystkie trzy ukryte przed podglądaniem
        treści podczas ich wprowadzania).
        \item Administrator potwierdzi decyzję o zmianie hasła w monicie.
        \item Po potwierdzeniu decyzji, administrator zobaczy wiadomość systemową o zmianie hasła.
        \item Administrator rozłączy się z serwerem.
        \item Administrator spróbuje rozpocząć nową sesję z
        serwerem, autoryzując się nowym hasłem.
        \item Nowe hasło zostanie zaakceptowane przez serwer,
        sesja zostanie rozpoczęta prawidłowo.
			\end{enumreq}
			\strut}
		\\

	\hline
  \parbox[t]{4cm}{\textbf{Nakład godzinowy (planowany / włożony)}} &
  \parbox[t]{11.5cm}{\strut
    2h / 2h
  \strut}\\

  \hline
    \parbox[t]{4cm}{\textbf{Ukończono?}} &
    \parbox[t]{11.5cm}{\strut
      Tak.
    \strut}\\
\hline
\end{tabular}

\vspace{1em}

\begin{tabular}{ | l | l | }
	\hline
		\textbf{Identyfikator} &
	WS-03
		\\

	\hline
		\textbf{Treść} & \parbox[t]{11.5cm}{
			Aplikacja czatu będzie dostosowana przede wszystkim
			do obsługi z wykorzystaniem urządzeń mobilnych.
		}\\

	\hline      \parbox[t]{4cm}{\textbf{Nakład godzinowy (planowany / włożony)}} &
        \parbox[t]{11.5cm}{\strut
          2h / 0h
        \strut}\\

        \hline
          \parbox[t]{4cm}{\textbf{Ukończono?}} &
          \parbox[t]{11.5cm}{\strut
            Nie -- idea zarzucona, pozostano przy wersji dla ekranów
						jednego typu urządzenia - smartfonu. Pozostałe ekrany pokazują
						przeskalowany interfejs.
          \strut}\\
  	\hline
\end{tabular}

\subsubsection{Zadania wykraczające poza pierwotne wymagania}

Brak.
