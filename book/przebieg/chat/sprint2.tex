\subsection{Sprint 2}

\textbf{Termin realizacji:} 10-23 marca 2019 r.

\subsubsection{Cel sprintu}
W ramach tego sprintu zaplanowano opracowanie mechanizm autoryzacji do czatu i
możliwość przejrzenia listy pokojów.

\subsubsection{Zadania oparte o pierwotne wymagania}

\leavevmode\hbox{}

\begin{tabular}{ | l | l | }
	\hline
		\textbf{Identyfikator} &
		WF-01
		\\

	\hline
		\textbf{Treść} & \parbox[t]{11.5cm}{\strut
			Jako użytkownik serwera czatu, chcę się do niego zalogować, aby zobaczyć listę pokojów dyskusyjnych.
		\strut}\\

	\hline
		\parbox[t]{4cm}{\textbf{Kryteria akceptacji}} & \parbox[t]{11.5cm}{\strut
			\begin{enumreq}
				\item Po wejściu na czat bez rozpoczętej sesji, pokazuje się monit o podanie nazwy użytkownika.
				\item Po wpisaniu nazwy użytkownika i zatwierdzeniu, użytkownik rozpocznie sesję na serwerze czatu.
				\item Tuż po rozpoczęciu sesji czatu, użytkownik zobaczy listę pokojów.
			\end{enumreq}
			\strut}
		\\

    \hline
      \parbox[t]{4cm}{\textbf{Nakład godzinowy (planowany / włożony)}} & \parbox[t]{11.5cm}{\strut
        15h / 21h
      \strut}\\
	\hline

    \hline
      \parbox[t]{4cm}{\textbf{Ukończono?}} &
      \parbox[t]{11.5cm}{\strut
        Tak.
      \strut}\\

      \hline
  \end{tabular}

\vspace{1em}

  \begin{tabular}{ | l | l | }
  	\hline
  		\textbf{Identyfikator} &
  		HN-01
  		\\

  	\hline
  		\textbf{Treść} & \parbox[t]{11.5cm}{\strut
  			Długość nazwy użytkownika jest ograniczona od 2 do 32 znaków alfanumerycznych, w celu uniknięcia problemów z identyfikacją użytkownika na serwerze.
  		\strut}\\

  	\hline
  		\parbox[t]{4cm}{\textbf{Powiązane zasady biznesowe}} & \parbox[t]{11.5cm}{\strut
  			ZU-03 Nazwa użytkownika to ciąg od 2 do 32 alfanumerycznych znaków.
  		\strut}\\

  	\hline
  		\parbox[t]{4cm}{\textbf{Kryteria akceptacji}} & \parbox[t]{11.5cm}{\strut
  			\begin{enumreq}
  				\item Po wpisaniu do pola użytkownika nazwy krótszej niż 2 znaki, dłużej niż 32 znaki lub zawierającej inne znaki niż alfanumeryczne, zwracany jest błąd.
  			\end{enumreq}
  			\strut}
  		\\

  	\hline
    \parbox[t]{4cm}{\textbf{Nakład godzinowy (planowany / włożony)}} & \parbox[t]{11.5cm}{\strut
      Czas wynikający z tego zadania ujęto przy realizacji zadania WF-01.
    \strut}\\
\hline

  \hline
    \parbox[t]{4cm}{\textbf{Ukończono?}} &
    \parbox[t]{11.5cm}{\strut
      Tak.
    \strut}\\

    \hline
  \end{tabular}

  \vspace{1em}

  \begin{tabular}{ | l | l | }
  	\hline
  		\textbf{Identyfikator} &
  		HN-02
  		\\

  	\hline
  		\textbf{Treść} & \parbox[t]{11.5cm}{\strut
  			Loginy i hasła administratorów są gromadzone w plikach
        konfiguracyjnych, a po uruchomieniu serwera -- w jego
        pamięci operacyjnej.
  		\strut}\\

  	\hline
  		\parbox[t]{4cm}{\textbf{Powiązane zasady biznesowe}} & \parbox[t]{11.5cm}{\strut
  			ZU-07 Konta administratorów są utrzymywane na serwerze w postaci par wartości: nazwa użytkownika i hasło.
  		\strut}\\

  	\hline
  		\parbox[t]{4cm}{\textbf{Kryteria akceptacji}} & \parbox[t]{11.5cm}{\strut
  			\begin{enumreq}
  				\item Serwer jest wyposażony w pliki konfiguracyjne
          \item Po załadowaniu serwera, z plików konfiguracyjnych
          są odczytywane dane kont administracyjnych
          \item Serwer po uruchomieniu jest wyposażony w konta o
          nazwach i hasłach zgodnych z wpisami w plikach konfiguracyjnych.
  			\end{enumreq}
  			\strut}
  		\\

  	\hline
    \parbox[t]{4cm}{\textbf{Nakład godzinowy (planowany / włożony)}} & \parbox[t]{11.5cm}{\strut
      Czas wynikający z tego zadania ujęto przy realizacji zadania WF-01.
    \strut}\\
\hline

  \hline
    \parbox[t]{4cm}{\textbf{Ukończono?}} &
    \parbox[t]{11.5cm}{\strut
      Tak.
    \strut}\\

    \hline
  \end{tabular}

\subsubsection{Zadania wykraczające poza pierwotne wymagania}

\leavevmode\hbox{}

\begin{tabular}{ | l | l | }
	\hline
		\textbf{Identyfikator} &
		ZZ-03
		\\

	\hline
		\textbf{Treść} & \parbox[t]{11.5cm}{\strut
			Połączenie po autoryzacji powinno zostać zabezpieczone losowym łańcuchem
      znaków -- kluczem sesji. Klucz sesji powinien być generowany
      automatycznie po stronie backendu i wysyłany frontendowi tuż po
      autoryzacji użytkownika. Od tej pory backend nie zaakceptuje żadnej
      wiadomości od frontendu, dopóki nie
      będzie zawierała dodatkowego parametru \texttt{session\_key} z treścią
      klucza sesji. Celem tego środka jest uniknięcie sytuacji zwielokrotnienia
      klientów na tym samym kanale w razie zresetowania lub wielokrotnego
      nawiązywania połączenia z tego samego urządzenia.
		\strut}\\

	\hline
		\parbox[t]{4cm}{\textbf{Kryteria akceptacji}} & \parbox[t]{11.5cm}{\strut
			\begin{enumreq}
				\item W momencie autoryzacji, backend wygeneruje 32-znakowy,
        alfanumeryczny, losowy klucz.
        \item Backend odeśle klucz z informacją o autoryzacji do frontendu.
        \item Frontent zapisze i utrzyma otrzymany klucz sesji
        \item Frontent będzie automatycznie dodawać parametr \texttt{session\_key},
        otrzymany podczas autoryzacji, do każdej kolejnej, dodawanej wiadomości.
			\end{enumreq}
			\strut}
		\\

    \hline
  		\parbox[t]{4cm}{\textbf{Nakład godzinowy (planowany / włożony)}} &
      \parbox[t]{11.5cm}{\strut
  			2h / 1h
  		\strut}\\

        \hline
          \parbox[t]{4cm}{\textbf{Ukończono?}} &
          \parbox[t]{11.5cm}{\strut
            Tak.
          \strut}\\

          \hline
      \end{tabular}

      \vspace{1em}
