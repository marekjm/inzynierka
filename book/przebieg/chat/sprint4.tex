\subsection{Sprint 4}

\textbf{Termin realizacji:} 7-27 kwietnia 2019 r.

\subsubsection{Cel sprintu}
Sprint ma na celu wykonanie mechanizmu prywatnych wiadomości. Wlicza się do
niego lista użytkowników obecnych na czacie, z którymi ma być prowadzona
rozmowa, a także nawiązywanie rozmowy w pokoju PW oraz wysyłanie i odbiór
wiadomości w pokoju PW.

\subsubsection{Zadania oparte o pierwotne wymagania}

\leavevmode\hbox{}

\begin{tabular}{ | l | l | }
	\hline
		\textbf{Identyfikator} &
		WF-08
		\\

	\hline
		\textbf{Treść} & \parbox[t]{11.5cm}{\strut
			Jako użytkownik serwera czatu, chcę zobaczyć
			okno wiadomości prywatnych, aby odczytać wiadomości,
			które wysłano specjalnie do mnie.
		\strut}\\


	\hline
		\parbox[t]{4cm}{\textbf{Kryteria akceptacji}} & \parbox[t]{11.5cm}{\strut
			\begin{enumreq}
				\item Po kliknięciu w link ,,PW'', użytkownik
				zobaczy okno prywatnych wiadomości
				\item W oknie wiadomości prywatnych, użytkownik
				zobaczy listę użytkowników, od których otrzymał
				wiadomości prywatne.
				\item Po kliknięciu w link z nazwą użytkownika,
				użytkownik zobaczy prywatne wiadomości, których
				nadawcą i odbiorcą jest wskazana osoba.
			\end{enumreq}
			\strut}
		\\
    \hline
      \parbox[t]{4cm}{\textbf{Nakład godzinowy (planowany / włożony)}} &
      \parbox[t]{11.5cm}{\strut
        6h / 5h
      \strut}\\

      \hline
        \parbox[t]{4cm}{\textbf{Ukończono?}} &
        \parbox[t]{11.5cm}{\strut
          Tak.
        \strut}\\
	\hline
\end{tabular}

\vspace{1em}

\begin{tabular}{ | l | l | }
	\hline
		\textbf{Identyfikator} &
		WF-09
		\\

	\hline
		\textbf{Treść} & \parbox[t]{11.5cm}{\strut
			Jako użytkownik serwera czatu, chcę wysłać
			wiadomość prywatną do jednego użytkownika, aby
			prowadzić z nim ciągłą konwersację.
		\strut}\\

	\hline
		\parbox[t]{4cm}{\textbf{Kryteria akceptacji}} & \parbox[t]{11.5cm}{\strut
			\begin{enumreq}
				\item Użytkownik wpisze tekst wiadomości w polu
				tekstowym u dołu okna wiadomości prywatnych
				\item Wiadomość wpisana w polu tekstowym zostanie
				wysłana po wciśnięciu klawisza ,,Enter'', gdy
				aktywne
				będzie pole tekstowe
				\item Wiadomość wpisana w polu tekstowym zostanie
				wysłana po naciśnięciu przycisku ,,Wyślij'',
				widocznego obok pola tekstowego
				\item Po wysłaniu wiadomości, pole tekstowe zostanie
				wyczyszczone (niezależnie od tego czy wiadomość
				zostanie doręczona)
				\item Wiadomość wysłana w oknie zostanie pokazana
				tylko użytkownikowi, z którym trwa otwarta
				konwersacja
				\item Nowa wiadomość jest pokazywana wraz z nazwą
				użytkownika wysyłającego u dołu konwersacji
			\end{enumreq}
			\strut}
		\\

    \hline
      \parbox[t]{4cm}{\textbf{Nakład godzinowy (planowany / włożony)}} &
      \parbox[t]{11.5cm}{\strut
        2h / 1h
      \strut}\\

      \hline
        \parbox[t]{4cm}{\textbf{Ukończono?}} &
        \parbox[t]{11.5cm}{\strut
          Tak.
        \strut}\\
	\hline
\end{tabular}

\vspace{1em}

\begin{tabular}{ | l | l | }
	\hline
		\textbf{Identyfikator} &
		WF-11
		\\

	\hline
		\textbf{Treść} & \parbox[t]{11.5cm}{\strut
			Jako użytkownik serwera czatu, chcę wysłać wiadomość
			prywatną do innego użytkownika, z którym wcześniej nie
			wymieniałem takich wiadomości, aby rozpocząć z nim
			prywatną konwersację.
		\strut}\\

	\hline
		\parbox[t]{4cm}{\textbf{Kryteria akceptacji}} & \parbox[t]{11.5cm}{\strut
			\begin{enumreq}
				\item Użytkownik kliknie w oknie wiadomości
				prywatnych w przyciski ,,Nowy''.
				\item Użytkownik zobaczy monit o podanie nazwy
				użytkownika, z którym chce rozpocząć rozmowę
				\item Jeżeli użytkownik jest aktywny, wówczas
				\item Wiadomość wpisana w polu tekstowym zostanie
				wysłana po wciśnięciu klawisza ,,Enter'', gdy
				aktywne
				będzie pole tekstowe
				\item Wiadomość wpisana w polu tekstowym zostanie
				wysłana po naciśnięciu przycisku ,,Wyślij'',
				widocznego obok pola tekstowego
				\item Po wysłaniu wiadomości, pole tekstowe zostanie
				wyczyszczone (niezależnie od tego czy wiadomość
				zostanie doręczona)
				\item Wiadomość wysłana w oknie zostanie pokazana
				tylko użytkownikowi, z którym trwa otwarta
				konwersacja
				\item Nowa wiadomość jest pokazywana wraz z nazwą
				użytkownika wysyłającego u dołu konwersacji
			\end{enumreq}
			\strut}
		\\
    \hline
      \parbox[t]{4cm}{\textbf{Nakład godzinowy (planowany / włożony)}} &
      \parbox[t]{11.5cm}{\strut
        11h / 5h
      \strut}\\

      \hline
        \parbox[t]{4cm}{\textbf{Ukończono?}} &
        \parbox[t]{11.5cm}{\strut
          Nie -- przeniesiono do kolejnego sprintu.
        \strut}\\

	\hline
\end{tabular}

\vspace{1em}

\begin{tabular}{ | l | l | }
	\hline
		\textbf{Identyfikator} &
		HN-05
		\\

	\hline
		\textbf{Treść} & \parbox[t]{11.5cm}{\strut
			Wiadomości prywatne są czyszczone niezwłocznie po rozłączeniu się przez dowolnego z rozmówców.
		\strut}\\

	\hline
		\parbox[t]{4cm}{\textbf{Powiązane zasady biznesowe}} & \parbox[t]{11.5cm}{\strut
			ZW-10 Wiadomości prywatne są utrzymywane dopóki nadawca i odbiorca mają aktywną sesję na serwerze.
		\strut}\\

	\hline
		\parbox[t]{4cm}{\textbf{Kryteria akceptacji}} & \parbox[t]{11.5cm}{\strut
			\begin{enumreq}
				\item Po zamknięciu sesji użytkownika, wiadomości prywatne których był nadawcą lub odbiorcą ulegają
        usunięciu.
			\end{enumreq}
			\strut}
		\\

	\hline
  \parbox[t]{4cm}{\textbf{Nakład godzinowy (planowany / włożony)}} &
  \parbox[t]{11.5cm}{\strut
   --
  \strut}\\

  \hline
    \parbox[t]{4cm}{\textbf{Ukończono?}} &
    \parbox[t]{11.5cm}{\strut
      Nie -- zadanie przeniesiono do kolejnego sprintu.
    \strut}\\
\hline
\end{tabular}

\vspace{1em}

\begin{tabular}{ | l | l | }
	\hline
		\textbf{Identyfikator} &
		HN-06
		\\

	\hline
		\textbf{Treść} & \parbox[t]{11.5cm}{\strut
			Bufor pokoju wiadomości prywatnych zawiera do 100 wiadomości.
		\strut}\\

	\hline
		\parbox[t]{4cm}{\textbf{Powiązane zasady biznesowe}} & \parbox[t]{11.5cm}{\strut
			ZW-11 Dla każdej pary użytkowników, na serwerze jest gromadzone co najwyżej 100 wiadomości prywatnych.
		\strut}\\

	\hline
		\parbox[t]{4cm}{\textbf{Kryteria akceptacji}} & \parbox[t]{11.5cm}{\strut
			\begin{enumreq}
				\item Po przekroczeniu liczby 100 wiadomości prywatnych w pokoju, bufor
        ulega ,,zawinięciu'', usuwając najstarsze wiadomości.
			\end{enumreq}
			\strut}
		\\

	\hline
  \parbox[t]{4cm}{\textbf{Nakład godzinowy (planowany / włożony)}} &
  \parbox[t]{11.5cm}{\strut
    --
  \strut}\\

  \hline
    \parbox[t]{4cm}{\textbf{Ukończono?}} &
    \parbox[t]{11.5cm}{\strut
      Nie -- zadanie przeniesiono do kolejnego sprintu.
    \strut}\\
\hline
\end{tabular}

\subsubsection{Zadania wykraczające poza pierwotne wymagania}

Brak.
