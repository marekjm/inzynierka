\section{Implementacja kompilatora}

Śledzenie prac nad rozwojem kompilatora było wspierane przez dwa narzędzia:
system kontroli wersji Git\footnote{\url{https://git-scm.com/}} i narzędzie do
śledzenia zadań Issue\footnote{\url{https://github.com/marekjm/issue}}.

\subsection{Zakończone zadania}

Poniżej przedstawiona jest lista zakończonych zadań zarejestrowanych do
wykonania podczas projektu.
Na dzień 2019-05-19 wykonane zostało $64.44\%$ zaplanowanych zadań. Tak niska
wartość wiąże się z tym, że część zadań zostało otwartych z założeniem, że mogą
nie zostać wykonane (na zasadzie ,,jeśli wystarczy czasu''). Mediana czasów
całkowitego czasu (od zgłoszenia do zamknięcia) każdego z ukończonych zadań to
nieco ponad sześć dni (6 dni i 11 godzin).

Repozytorium programu Issue jest dostępne wraz z repozytorium Git dołączonym
jako załącznik do projektu.

\subsubsection{Add an assembly driver tool for executables}

Identyfikator zadania: \texttt{5b8f77f4714be2c16d1f87a02b03692c96b983c1}
\newline
Rozpoczęte 2018-12-18 23:20:11, zakończone 2018-12-20 16:00:54.

\subsubsection{Basic CLI chat}

Identyfikator zadania: \texttt{b8cea1dc8ac19f63ad579ac28196c82f6a9eb7aa}
\newline
Rozpoczęte 2019-01-19 22:15:28, porzucone 2019-03-11 19:04:01.
\newline

Zadanie zostało porzucone ponieważ powstał czat oparty o protokół WebSocket.

\subsubsection{Create a module system}

Identyfikator zadania: \texttt{509ee8af1c2f503e402b34c18224a84caf4c9252}
\newline
Rozpoczęte 2018-11-22 19:26:21, zakończone 2018-11-29 06:54:16.
\newline

Język \ViuAct\phantom{} posiada działający system modułów opisany w rozdziale
\ref{viuact_spec_module_definition}~\nameref{viuact_spec_module_definition}.

\subsubsection{Create an EBNF notation to describe Viuact syntax}

Identyfikator zadania: \texttt{3295295f3d3b08cd0a3ae46c1adebe40bb3ebad4}
\newline
Rozpoczęte 2018-12-19 00:04:24, zakończone 2019-03-26 20:00:00.
\newline

Zadanie zostało zakończone po umieszczeniu w pracy specyfikacji języka \ViuAct,
która opisuje składnię w notacji EBNF. Specyfikacja znajduje się w rozdziale~\ref{viuact_spec}
na stronie \pageref{viuact_spec}.

\subsubsection{Dump intermediate representations}

Identifykator zadania: \texttt{902ffbd31e61d8446375620f8511042ca88f31e6}
\newline
Rozpoczęte 2019-01-07 18:42:28, zakończone 2019-01-07 19:43:09.
\newline

Zadanie polegało na umożliwieniu ,,podejrzenia'' tego na czym tak naprawdę
pracuje kompilator. W jego wyniku powstał mechanizm, który wymusza na
kompilatorze zapisanie do plików zserializowanej postaci strumienia tokenów,
AST, lub obu tych rzeczy.

Do zasygnalizowania kompilatorowi tej potrzeby wykorzystywana jest zmienna
środowiskowa. Przykłady użycia:

\begin{enumerate}
    \item \texttt{VIUAC\_DUMP\_INTERMEDIATE=tokens} -- zapis strumienia tokenów
    \item \texttt{VIUAC\_DUMP\_INTERMEDIATE=exprs} -- zapis AST
    \item \texttt{VIUAC\_DUMP\_INTERMEDIATE=tokens,exprs} -- zapis strumienia
        tokenów i AST
\end{enumerate}

\subsubsection{Emit dependency files for modules and executables}

Identifykator zadania: \texttt{f841ba29230081e4e6e8eba87857532ae98aff0a}
\newline
Rozpoczęte 2018-12-18 23:28:01, zakończone 2018-12-20 16:00:13.
\newline

Aby poprawnie połączyć moduły \texttt{viuact-opt} potrzebuje informacji o tym,
jakie moduły są importowane przez dany moduł lub plik wykonywalny. Te zależności
są zapisywane w plikach \texttt{.d} opisanych dokładnie w rozdziale
\ref{pliki_zaleznosci_modulow} na stronie \pageref{pliki_zaleznosci_modulow}.

\subsubsection{Emit interface files for modules}

Identifykator zadania: \texttt{86fd1a408cf62ecbebb7459b7d08298ce0a851bc}
\newline
Rozpoczęte 2018-12-18 23:13:55, zakończone 2019-03-27 18:09:37.
\newline

Kompilator podczas przetwarzania wyrażeń importowania modułów (rozdział
\ref{viuact_spec_module_import} na stronie \pageref{viuact_spec_module_import})
potrzebuje informacji o tym jakie funkcje i wyliczenia dany moduł udostępnia. Te
informacje zapisane są w plikach z interfejsami opisanych w rozdziale
\ref{pliki_interfejsow_modulow} na stronie \pageref{pliki_interfejsow_modulow}.

\subsubsection{Expose pointers}

Identifykator zadania: \texttt{3208ed0629487b2c02e14cfe5173a333bc364121}
\newline
Rozpoczęte 2019-02-02 14:18:22, zakończone 2019-02-02 19:55:44.
\newline

Kompilator języka \ViuAct\ potrafi śledzić to czy dana wartość jest wskaźnikiem
i automatycznie wstawić dereferencję w odpowiednim momencie jedynie dla
ograniczonego zakresu przypadków. W związku z tym potrzebne było udostępnienie
programistom narzędzi do bezpośredniej obsługi wskaźników. Tym narzędziami są
operator dereferencji (\texttt{\^}) i funkcja \texttt{Std.Pointer.take()}.

Nie było by to potrzebne gdyby kompilator języka \ViuAct\ wymagał weryfikacji
typów wartości na etapie kompilacji -- wtedy informacja o tym czy dana wartość
jest wskaźnikiem czy nie musiałaby być dostępna, a więc nie było by potrzeby
nadzorować tego ręcznie. Wskaźniki to jedno z miejsc gdzie brak czasu negatywnie
wpłynął na specyfikację i implementację języka.

\subsubsection{Expose the mechanism to set a process watchdog}

Identifykator zadania: \texttt{cc03c40b74d5ce2a7e9fa398ae0912c9a107b02d}
\newline
Rozpoczęte 2019-03-25 16:23:52, zakończone 2019-04-01 21:40:18.
\newline

\emph{Watchdog} to jeden z najważniejszych mechanizmów jakie język \ViuAct\ 
,,odziedziczył'' po Viua~VM. Jest to funkcja, która jest automatycznie
wywyoływana w momencie awarii procesu, aby
\begin{enumerate*}[label=(\arabic*)]
    \item zarejestrować wystąpienie awarii
    \item postawać się naprawić awarię restartując proces
    \item poinformować proces nadrzędny o awarii,
\end{enumerate*}
lub w jakikolwiek inny sposób zareagować na awarię, której główna funkcja
procesu nie dała rady obsłużyć.

Dokładny opis funkcji \emph{watchdog} znajduje się w rozdziale
\ref{viuact_spec_watchdog_call} na stronie \pageref{viuact_spec_watchdog_call}.

\subsubsection{Implement an exception catching mechanism}

Identifykator zadania: \texttt{d4bddf8f6b83c9957556465c8e9e112bddf747a1}
\newline
Rozpoczęte 2019-01-14 06:42:52, zakończone 2019-03-26 20:00:07.
\newline

Wyjątki to podstawowy sposób obsługi błędów w języku \ViuAct, a ich obsłużenie
jest pierwszą linią obrony przed awariami procesów.

\subsubsection{Implement an impressive example program}

Identifykator zadania: \texttt{b28ab36f969b4852a735850af1d1f509647b655c}
\newline
Rozpoczęte 2019-01-13 12:24:14, zakończone 2019-03-11 19:08:00.
\newline

W ramach tego zadania powstał prototypowy czat wykorzystujący protokół
WebSocket, który prezentowaliśmy m.in. na seminarium.

\subsubsection{Implement boolean values}

Identifykator zadania: \texttt{0050c352061a18221e21438e697c11d61c11a5a7}
\newline
Rozpoczęte 2019-01-07 18:23:12, zakończone 2019-01-07 18:23:37.
\newline

Wartości typu boolowskiego są istotnym elementem programów i mogą określać np.
czy dana funkcjonalność jest włączona lub wyłączona, czy autoryzacja się
powiodła lub nie, itd.

\subsubsection{Implement compound expressions}

Identifykator zadania: \texttt{d452c8ee6fcc4b245e7021fec2ceb0958aad1f84}
\newline
Rozpoczęte 2018-12-18 23:12:42, zakończone 2019-01-12 12:20:08.
\newline

Wyrażenia złożone (opisane w rozdziale \ref{language_expressions_compound} na
stronie \pageref{language_expressions_compound}) są jednym z ważniejszych
elementów języka. Pozwalają na pisanie funkcji wykonujących kilka operacji
dzięki temu, że umożliwiają zgrupowanie kilku prostszych wyrażeń w jedno.

\subsubsection{Implement proper imports inside Viua VM}

Identifykator zadania: \texttt{6f4f7baff22706628e7003303ff2aa0973460432}
\newline
Rozpoczęte 2018-12-22 16:08:28, zakończone 2019-01-14 06:41:41.
\newline

Mechanizm importowania modułów w Viua~VM był zbyt prymitywny żeby poradzić sobie
w elegancki sposób z importowaniem modułów jakie potrzebne było w języku
\ViuAct. Jedną z najbardziej oczywistych rzeczy, które musiały być poprawione w
samej maszynie wirtualnej było importowanie modułów obcych, które nie mogły być
linkowane statycznie do plików wykonywalnych lub modułów własnych Viua~VM i
musiały być linkowane dynamicznie.

Linkowanie dynamiczne musiało być wykonywane jawnie -- wykonywany program musiał
sam wykonać instrukcje odpowiadające za linkowanie modułu obcego. Po zakończeniu
tego zadania (i odpowiadającego mu zadania
\texttt{0f8c916c2499aab93cf8c55cc1fd6bbf354bd34d} w repozytorium Viua~VM) moduły
obce mogły być importowane automatycznie przez jądro Viua~VM dzięki dodaniu do
plików z bytecode sekcji informującej o modułach, które muszą być dolinkowane
dynamicznie. W pliku z kodem źródłowym w języku assemblera Viua~VM można
zarządać takiego dołączenia za pomocą następującego polecenia:
\begin{lstlisting}
.import: [[dynamic]] some::module
\end{lstlisting}

\subsubsection{Implement structs}

Identifykator zadania: \texttt{1f45e908dcde2f12b761da53183e2475619f64ef}
\newline
Rozpoczęte 2019-01-02 21:13:48, zakończone 2019-01-06 12:03:04.
\newline

Struktury w języku \ViuAct\ są sposobem na tworzenie nowych typów przez
programistę. Bez nich, tworzenie nietrywialnego oprogramowania byłoby mocno
utrudnione. Struktury są opisane w rozdziale \ref{viuact_spec_datatypes_aggregate_struct}
na stronie \pageref{viuact_spec_datatypes_aggregate_struct}.

\subsubsection{Implement tail calls}

Identifykator zadania: \texttt{b680202a36a7358006d87cf371686fc2d42aecdc}
\newline
Rozpoczęte 2019-01-06 22:02:12, zakończone 2019-01-06 22:04:33.
\newline

Wywołania w pozycji ogonowej są przydatne w wielu miejscach: przy obsłudze
błędów (gdzie pozwalają zastąpić funkcję \emph{watchdog} nową funkcją,
restartując proces bez zmiany jego PID), zastępują pętle, oraz pozwalają
implementować wzorzec ,,inicjalizacja-implementacja'' (rozdział
\ref{viuact_spec_tail_call_use_case} na stronie
\pageref{viuact_spec_tail_call_use_case}).

\subsubsection{Implement vectors}

Identyfikator zadania: \texttt{c499}

\subsubsection{Integrate FFI imports}

Identyfikator zadania: \texttt{713e}

\subsubsection{Make function bodies a single expression}

Identyfikator zadania: \texttt{0d91}

\subsubsection{Make functions first class values}

Identyfikator zadania: \texttt{04dd}

\subsubsection{Make the compiler installable}

Identyfikator zadania: \texttt{6b5d}

\subsubsection{Remove the parentheses around function args}

Identyfikator zadania: \texttt{bb64}

\subsubsection{Rework README for the Issue project}

Identyfikator zadania: \texttt{7348}

\subsubsection{Specify tail calls}

Identyfikator zadania: \texttt{9774}

\subsubsection{Specify the module system}

Identyfikator zadania: \texttt{242e}

\subsubsection{Suppress unused register errors thrown by Viua VM assembler}

Identyfikator zadania: \texttt{cd43}

\subsubsection{Test the compiler and assembly driver}

Identyfikator zadania: \texttt{561c}

\subsubsection{Write some tests}

Identyfikator zadania: \texttt{2925}

\subsubsection{Write websockets FFI module}

Identyfikator zadania: \texttt{631b}
