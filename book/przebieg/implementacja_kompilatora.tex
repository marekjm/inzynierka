\section{Implementacja kompilatora}

Śledzenie prac nad rozwojem kompilatora było wspierane przez dwa narzędzia:
system kontroli wersji Git\footnote{\url{https://git-scm.com/}} i narzędzie do
śledzenia zadań Issue\footnote{\url{https://github.com/marekjm/issue}}.

\subsection{Zakończone zadania}

Poniżej przedstawiona jest lista zakończonych zadań zarejestrowanych do
wykonania podczas projektu.
Na dzień 2019-05-19 wykonane zostało $64.44\%$ zaplanowanych zadań. Tak niska
wartość wiąże się z tym, że część zadań zostało otwartych z założeniem, że mogą
nie zostać wykonane (na zasadzie ,,jeśli wystarczy czasu''). Mediana czasów
całkowitego czasu (od zgłoszenia do zamknięcia) każdego z ukończonych zadań to
nieco ponad sześć dni (6 dni i 11 godzin).

Repozytorium programu Issue jest dostępne wraz z repozytorium Git dołączonym
jako załącznik do projektu.

\subsubsection{Add an assembly driver tool for executables}

Identyfikator zadania: \texttt{5b8f77f4714be2c16d1f87a02b03692c96b983c1}
\newline
Rozpoczęte 2018-12-18 23:20:11, zakończone 2018-12-20 16:00:54.

\subsubsection{Basic CLI chat}

Identyfikator zadania: \texttt{b8cea1dc8ac19f63ad579ac28196c82f6a9eb7aa}
\newline
Rozpoczęte 2019-01-19 22:15:28, porzucone 2019-03-11 19:04:01.
\newline

Zadanie zostało porzucone ponieważ powstał czat oparty o protokół WebSocket.

\subsubsection{Create a module system}

Identyfikator zadania: \texttt{509ee8af1c2f503e402b34c18224a84caf4c9252}
\newline
Rozpoczęte 2018-11-22 19:26:21, zakończone 2018-11-29 06:54:16.
\newline

Język \ViuAct\phantom{} posiada działający system modułów opisany w rozdziale
\ref{viuact_spec_module_definition}~\nameref{viuact_spec_module_definition}.

\subsubsection{Create an EBNF notation to describe Viuact syntax}

Identyfikator zadania: \texttt{3295295f3d3b08cd0a3ae46c1adebe40bb3ebad4}
\newline
Rozpoczęte 2018-12-19 00:04:24, zakończone 2019-03-26 20:00:00.
\newline

Zadanie zostało zakończone po umieszczeniu w pracy specyfikacji języka \ViuAct,
która opisuje składnię w notacji EBNF. Specyfikacja znajduje się w rozdziale~\ref{viuact_spec}
na stronie \pageref{viuact_spec}.

\subsubsection{Dump intermediate representations}

Identifykator zadania: \texttt{902ffbd31e61d8446375620f8511042ca88f31e6}
\newline
Rozpoczęte 2019-01-07 18:42:28, zakończone 2019-01-07 19:43:09.
\newline

Zadanie polegało na umożliwieniu ,,podejrzenia'' tego na czym tak naprawdę
pracuje kompilator. W jego wyniku powstał mechanizm, który wymusza na
kompilatorze zapisanie do plików zserializowanej postaci strumienia tokenów,
AST, lub obu tych rzeczy.

Do zasygnalizowania kompilatorowi tej potrzeby wykorzystywana jest zmienna
środowiskowa. Przykłady użycia:

\begin{enumerate}
    \item \texttt{VIUAC\_DUMP\_INTERMEDIATE=tokens} -- zapis strumienia tokenów
    \item \texttt{VIUAC\_DUMP\_INTERMEDIATE=exprs} -- zapis AST
    \item \texttt{VIUAC\_DUMP\_INTERMEDIATE=tokens,exprs} -- zapis strumienia
        tokenów i AST
\end{enumerate}

\subsubsection{Emit dependency files for modules and executables}

Identyfikator zadania: \texttt{f841}

\subsubsection{Emit interface files for modules}

Identyfikator zadania: \texttt{86fd}

\subsubsection{Expose pointers}

Identyfikator zadania: \texttt{3208}

\subsubsection{Expose the mechanism to set a process watchdog}

Identyfikator zadania: \texttt{cc03}

\subsubsection{Implement an exception catching mechanism}

Identyfikator zadania: \texttt{d4bd}

\subsubsection{Implement an impressive example program}

Identyfikator zadania: \texttt{b28a}

\subsubsection{Implement boolean values}

Identyfikator zadania: \texttt{0050}

\subsubsection{Implement compound expressions}

Identyfikator zadania: \texttt{d452}

\subsubsection{Implement proper imports inside Viua VM}

Identyfikator zadania: \texttt{6f4f}

\subsubsection{Implement structs}

Identyfikator zadania: \texttt{1f45}

\subsubsection{Implement tail calls}

Identyfikator zadania: \texttt{b680}

\subsubsection{Implement vectors}

Identyfikator zadania: \texttt{c499}

\subsubsection{Integrate FFI imports}

Identyfikator zadania: \texttt{713e}

\subsubsection{Make function bodies a single expression}

Identyfikator zadania: \texttt{0d91}

\subsubsection{Make functions first class values}

Identyfikator zadania: \texttt{04dd}

\subsubsection{Make the compiler installable}

Identyfikator zadania: \texttt{6b5d}

\subsubsection{Remove the parentheses around function args}

Identyfikator zadania: \texttt{bb64}

\subsubsection{Rework README for the Issue project}

Identyfikator zadania: \texttt{7348}

\subsubsection{Specify tail calls}

Identyfikator zadania: \texttt{9774}

\subsubsection{Specify the module system}

Identyfikator zadania: \texttt{242e}

\subsubsection{Suppress unused register errors thrown by Viua VM assembler}

Identyfikator zadania: \texttt{cd43}

\subsubsection{Test the compiler and assembly driver}

Identyfikator zadania: \texttt{561c}

\subsubsection{Write some tests}

Identyfikator zadania: \texttt{2925}

\subsubsection{Write websockets FFI module}

Identyfikator zadania: \texttt{631b}
