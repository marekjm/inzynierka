\documentclass[11pt,twoside,a4paper,titlepage,onecolumn]{book}

\usepackage[utf8]{inputenc}
\usepackage{textcomp}
\usepackage[official]{eurosym}
\usepackage[polish]{babel}
\usepackage{amsthm}
\usepackage{graphicx}
\usepackage[T1]{fontenc}
\usepackage{scrextend}
\usepackage{hyperref}
\usepackage{xcolor}
\usepackage[inline]{enumitem}
% \usepackage{nameref}
% \usepackage{showlabels}
% \usepackage{titlesec}
\usepackage{geometry}
\usepackage{rotating}
\geometry{a4paper, portrait, margin=2cm}
\graphicspath{ {./fig/} }
\usepackage{listings}
\usepackage{tabularx}
\usepackage{longtable}

\newenvironment{enumreq}
{ \begin{enumerate}[topsep=0pt,itemsep=-1ex,partopsep=1ex,parsep=1ex] }
{ \end{enumerate}                  }

\newlist{labreq}{description}{2}
\setlist[labreq]{topsep=0pt,itemsep=0ex,partopsep=0ex,parsep=0ex,leftmargin=4em}

\newcommand{\inzmaintitlePL}{Viua VM w akcji}
\newcommand{\ViuAct}{ViuAct}
\newcommand*{\fullref}[1]{\hyperref[{#1}]{\ref*{#1} \nameref*{#1}}}

\newcommand{\Dyplomant}{Marek Marecki}
\newcommand{\NumerAlbumu}{14288}

\renewcommand*{\lstlistlistingname}{Spis listingów}

\setcounter{secnumdepth}{4}

%% Author and title
% \author{Marek Marecki \and Gr.52c \and Krzysztof Franek}
\author{Zespół: Marek Marecki i Krzysztof Franek\\Promotor: dr hab. Marek A. Bednarczyk, prof. PJWSTK}
\title{%
    \inzmaintitlePL \\
    \large
    Implementacja języka wysokiego poziomu i \\
    prostej aplikacji\\
    ~\\
    Viua VM in action.\\
    Implementation of a high-level programming language and\\ a simple application}

\definecolor{light-gray}{gray}{0.9}

\begin{document}

\lstset{basicstyle=\ttfamily,
		columns=fixed,
		escapeinside={\%*}{*)},
		inputencoding=utf8,
		extendedchars=true,
		xleftmargin=.4in,
		xrightmargin=.4in,
		backgroundcolor=\color{light-gray},
		numbers=right,
		numberstyle=\tiny
		}

\begin{titlepage}
    \includegraphics[width=\textwidth]{pjwstk_logo}
    \begin{center}
        {\huge\texttt{Polsko-Japońska Akademia Technik Komputerowych}}
        {\huge\texttt{Zamiejscowy Wydział Informatyki w Gdańsku}}
    \end{center}
    \vspace{1cm}
    {\Large\textbf{Imię i nazwisko dyplomanta:} \Dyplomant}
    \vspace{0.5cm}
    \newline
    {\Large\textbf{Nr albumu:} \NumerAlbumu}
    \vspace{0.5cm}
    \newline
    {\Large\textbf{Kierunek studiów:} Informatyka \hfill \textbf{Rodzaj studiów:} niestacjonarne}

    \vspace{1.5cm}
    \begin{center}
        {\huge\textsc{Praca dyplomowa}}
    \end{center}
    \vspace{1cm}

    ~
    \newline
    {\Large\textbf{Temat pracy:} \inzmaintitlePL\\
        \phantom{\textbf{Temat pracy:}} Implementacja języka wysokiego poziomu i
        prostej aplikacji}
    \vspace{1cm}
    \newline
    {\Large\textbf{Temat w języku angielskim:} Viua VM in action\\
        \phantom{\textbf{Temat w języku angielskim:}} Implementation of a high-level programming\\
        \phantom{\textbf{Temat w języku angielskim:}} language and a simple application}
    \vspace{1cm}
    \newline
    {\Large\textbf{Opiekun pracy:} dr hab. Marek. A. Bednarczyk, prof. PJWSTK}
    \vspace{0.5cm}
    \newline
    {\Large\textbf{Wykonawcy:} Marek~Marecki i Krzysztof~Franek}
    \vspace{1.5cm}
    \newline
    {\Large\textbf{Streszczenie:} Głównym pytaniem, na które odpowiadamy w pracy
        jest ,,Czy Viua~VM
        umożliwia tworzenie niezawodnego, wysoce współbieżnego, nietrywialnego
        oprogramowania?''\\ Odpowiadamy na nie projektując język
        programowania wysokiego poziomu (\ViuAct) i implementując jego
        kompilator, oraz pisząc czat (ViuaChat) w tymże języku.}

    \vspace*{\fill}
    \begin{center}
    Gdańsk, 2019 rok
    \end{center}
\end{titlepage}
% \maketitle

\frontmatter
\tableofcontents
% \listoftables
\listoffigures
\lstlistoflistings
\vspace*{\fill}
Praca została złożona za pomocą systemu \LaTeX.

\newpage

\mainmatter
\part{Cel, strategia wykonania, i przebieg prac}

\section{Język assemblera Viua VM}
\label{appendix_viua_vm_assembly_language}

\subsection{Składnia języka assemblera}

\subsubsection{Ogólna składnia instrukcji}
\subsubsection{Definicje funkcji i bloków}
\subsubsection{Deklaracje funkcji i bloków}
\subsubsection{Import modułów}
\subsubsection{Markery}
\subsubsection{Nazywanie rejestrów}

\subsection{Instrukcje Viua VM}

\subsubsection{\texttt{nop}}

\subsubsection{\texttt{izero}}
\subsubsection{\texttt{integer}}
\subsubsection{\texttt{iinc}}
\subsubsection{\texttt{idec}}

\subsubsection{\texttt{float}}

\subsubsection{\texttt{itof}}
\subsubsection{\texttt{ftoi}}
\subsubsection{\texttt{stoi}}
\subsubsection{\texttt{stof}}

\subsubsection{\texttt{add}}
\subsubsection{\texttt{sub}}
\subsubsection{\texttt{mul}}
\subsubsection{\texttt{div}}
\subsubsection{\texttt{lt}}
\subsubsection{\texttt{lte}}
\subsubsection{\texttt{gt}}
\subsubsection{\texttt{gte}}
\subsubsection{\texttt{eq}}

\subsubsection{\texttt{string}}
\subsubsection{\texttt{streq}}

\subsubsection{\texttt{text}}
\subsubsection{\texttt{texteq}}
\subsubsection{\texttt{textat}}
\subsubsection{\texttt{textsub}}
\subsubsection{\texttt{textlength}}
\subsubsection{\texttt{textcommonprefix}}
\subsubsection{\texttt{textcommonsuffix}}
\subsubsection{\texttt{textconcat}}

\subsubsection{\texttt{vector}}
\subsubsection{\texttt{vinsert}}
\subsubsection{\texttt{vpush}}
\subsubsection{\texttt{vpop}}
\subsubsection{\texttt{vat}}
\subsubsection{\texttt{vlen}}

\subsubsection{\texttt{not}}

Negacja boolowska.

\subsubsection{\texttt{and}}

Iloczyn boolowski.

\subsubsection{\texttt{or}}

Suma boolowska.

\subsubsection{\texttt{bits\_of\_integer}}

Konstruktor bitów z liczby całkowitej.

\subsubsection{\texttt{integer\_of\_bits}}

Konstruktor liczby całkowitej z bitów.

\subsubsection{\texttt{bits}}

Konstruktor bitów.

\subsubsection{\texttt{bitand}}

Bitowa operacja \emph{\texttt{and}}.

\subsubsection{\texttt{bitor}}

Bitowa operacja \emph{\texttt{or}}.

\subsubsection{\texttt{bitnot}}

Bitowa operacja \emph{\texttt{not}}.

\subsubsection{\texttt{bitxor}}

Bitowa operacja \emph{\texttt{xor}}.

\subsubsection{\texttt{bitat}}

Sprawdzenie wartości pojedynczego bitu.

\subsubsection{\texttt{bitset}}

Ustawienie wartości pojedynczego bitu.

\subsubsection{\texttt{shl}}

Przesunięcie bitowe w lewo.

\subsubsection{\texttt{shr}}

Przesunięcie bitowe w prawo.

\subsubsection{\texttt{ashl}}

Arytmetyczne (z zachowaniem znaku) przesunięcie bitowe w lewo.

\subsubsection{\texttt{ashr}}

Arytmetyczne przesunięcie bitowe w prawo.

\subsubsection{\texttt{rol}}

Rotacja bitowa w lewo.

\subsubsection{\texttt{ror}}

Rotacja bitowa w prawo.

\subsubsection{\texttt{wrapincrement}}

Modulo inkrementacja ze znakiem.

\subsubsection{\texttt{wrapdecrement}}

Modulo dekrementacja ze znakiem.

\subsubsection{\texttt{wrapadd}}

Modulo dodawanie ze znakiem.

\subsubsection{\texttt{wrapsub}}

Modulo odejmowanie ze znakiem.

\subsubsection{\texttt{wrapmul}}

Modulo mnożenie ze znakiem.

\subsubsection{\texttt{wrapdiv}}

Modulo dzielenie ze znakiem.

\subsubsection{\texttt{checkedsincrement}}

Sprawdzana inkrementacja ze znakiem.

\subsubsection{\texttt{checkedsdecrement}}
\subsubsection{\texttt{checkedsadd}}
\subsubsection{\texttt{checkedssub}}
\subsubsection{\texttt{checkedsmul}}
\subsubsection{\texttt{checkedsdiv}}

\subsubsection{\texttt{saturatingsincrement}}

Nasycająca inkrementacja ze znakiem.

\subsubsection{\texttt{saturatingsdecrement}}
\subsubsection{\texttt{saturatingsadd}}
\subsubsection{\texttt{saturatingssub}}
\subsubsection{\texttt{saturatingsmul}}
\subsubsection{\texttt{saturatingsdiv}}

\subsubsection{\texttt{move}}

Przesunięcie wartości między rejestrami.

\subsubsection{\texttt{copy}}

Skopiowanie wartości między rejestrami.

\subsubsection{\texttt{ptr}}

Konstruktor wskaźnika do wartości.

\subsubsection{\texttt{ptrlive}}

Sprawdzenie poprawności wskaźnika.

\subsubsection{\texttt{swap}}

Zamiana wartości w rejestrach.

\subsubsection{\texttt{delete}}

Wywołanie destruktora wartości.

\subsubsection{\texttt{isnull}}

\subsubsection{\texttt{print}}
\subsubsection{\texttt{echo}}

\subsubsection{\texttt{capture}}
\subsubsection{\texttt{capturecopy}}
\subsubsection{\texttt{capturemove}}

\subsubsection{\texttt{closure}}
\subsubsection{\texttt{function}}

\subsubsection{\texttt{frame}}
\subsubsection{\texttt{param}}

Instrukcja prywatna.

\subsubsection{\texttt{pamv}}

Instrukcja prywatna.

\subsubsection{\texttt{call}}
\subsubsection{\texttt{tailcall}}
\subsubsection{\texttt{defer}}
\subsubsection{\texttt{arg}}

Instrukcja prywatna.

\subsubsection{\texttt{allocate\_registers}}

\subsubsection{\texttt{process}}
\subsubsection{\texttt{self}}
\subsubsection{\texttt{pideq}}
\subsubsection{\texttt{join}}
\subsubsection{\texttt{send}}
\subsubsection{\texttt{receive}}

\subsubsection{\texttt{watchdog}}

\subsubsection{\texttt{jump}}
\subsubsection{\texttt{if}}

\subsubsection{\texttt{throw}}

Rzucenie wartości jako wyjątku.

\subsubsection{\texttt{catch}}
\subsubsection{\texttt{draw}}
\subsubsection{\texttt{try}}
\subsubsection{\texttt{enter}}
\subsubsection{\texttt{leave}}

\subsubsection{\texttt{import}}

Import modułu.

\subsubsection{\texttt{atom}}

Konstruktor atomu.

\subsubsection{\texttt{atomeq}}

\subsubsection{\texttt{struct}}

Konstruktor struktury.

\subsubsection{\texttt{structinsert}}
\subsubsection{\texttt{structremove}}
\subsubsection{\texttt{structat}}
\subsubsection{\texttt{structkeys}}

\subsubsection{\texttt{return}}

Wyjście z funkcji.

\subsubsection{\texttt{halt}}

Zakończenie działania i wyłączenie VM.


\section{Język assemblera Viua VM}
\label{appendix_viua_vm_assembly_language}

\subsection{Składnia języka assemblera}

\subsubsection{Ogólna składnia instrukcji}
\subsubsection{Definicje funkcji i bloków}
\subsubsection{Deklaracje funkcji i bloków}
\subsubsection{Import modułów}
\subsubsection{Markery}
\subsubsection{Nazywanie rejestrów}

\subsection{Instrukcje Viua VM}

\subsubsection{\texttt{nop}}

\subsubsection{\texttt{izero}}
\subsubsection{\texttt{integer}}
\subsubsection{\texttt{iinc}}
\subsubsection{\texttt{idec}}

\subsubsection{\texttt{float}}

\subsubsection{\texttt{itof}}
\subsubsection{\texttt{ftoi}}
\subsubsection{\texttt{stoi}}
\subsubsection{\texttt{stof}}

\subsubsection{\texttt{add}}
\subsubsection{\texttt{sub}}
\subsubsection{\texttt{mul}}
\subsubsection{\texttt{div}}
\subsubsection{\texttt{lt}}
\subsubsection{\texttt{lte}}
\subsubsection{\texttt{gt}}
\subsubsection{\texttt{gte}}
\subsubsection{\texttt{eq}}

\subsubsection{\texttt{string}}
\subsubsection{\texttt{streq}}

\subsubsection{\texttt{text}}
\subsubsection{\texttt{texteq}}
\subsubsection{\texttt{textat}}
\subsubsection{\texttt{textsub}}
\subsubsection{\texttt{textlength}}
\subsubsection{\texttt{textcommonprefix}}
\subsubsection{\texttt{textcommonsuffix}}
\subsubsection{\texttt{textconcat}}

\subsubsection{\texttt{vector}}
\subsubsection{\texttt{vinsert}}
\subsubsection{\texttt{vpush}}
\subsubsection{\texttt{vpop}}
\subsubsection{\texttt{vat}}
\subsubsection{\texttt{vlen}}

\subsubsection{\texttt{not}}

Negacja boolowska.

\subsubsection{\texttt{and}}

Iloczyn boolowski.

\subsubsection{\texttt{or}}

Suma boolowska.

\subsubsection{\texttt{bits\_of\_integer}}

Konstruktor bitów z liczby całkowitej.

\subsubsection{\texttt{integer\_of\_bits}}

Konstruktor liczby całkowitej z bitów.

\subsubsection{\texttt{bits}}

Konstruktor bitów.

\subsubsection{\texttt{bitand}}

Bitowa operacja \emph{\texttt{and}}.

\subsubsection{\texttt{bitor}}

Bitowa operacja \emph{\texttt{or}}.

\subsubsection{\texttt{bitnot}}

Bitowa operacja \emph{\texttt{not}}.

\subsubsection{\texttt{bitxor}}

Bitowa operacja \emph{\texttt{xor}}.

\subsubsection{\texttt{bitat}}

Sprawdzenie wartości pojedynczego bitu.

\subsubsection{\texttt{bitset}}

Ustawienie wartości pojedynczego bitu.

\subsubsection{\texttt{shl}}

Przesunięcie bitowe w lewo.

\subsubsection{\texttt{shr}}

Przesunięcie bitowe w prawo.

\subsubsection{\texttt{ashl}}

Arytmetyczne (z zachowaniem znaku) przesunięcie bitowe w lewo.

\subsubsection{\texttt{ashr}}

Arytmetyczne przesunięcie bitowe w prawo.

\subsubsection{\texttt{rol}}

Rotacja bitowa w lewo.

\subsubsection{\texttt{ror}}

Rotacja bitowa w prawo.

\subsubsection{\texttt{wrapincrement}}

Modulo inkrementacja ze znakiem.

\subsubsection{\texttt{wrapdecrement}}

Modulo dekrementacja ze znakiem.

\subsubsection{\texttt{wrapadd}}

Modulo dodawanie ze znakiem.

\subsubsection{\texttt{wrapsub}}

Modulo odejmowanie ze znakiem.

\subsubsection{\texttt{wrapmul}}

Modulo mnożenie ze znakiem.

\subsubsection{\texttt{wrapdiv}}

Modulo dzielenie ze znakiem.

\subsubsection{\texttt{checkedsincrement}}

Sprawdzana inkrementacja ze znakiem.

\subsubsection{\texttt{checkedsdecrement}}
\subsubsection{\texttt{checkedsadd}}
\subsubsection{\texttt{checkedssub}}
\subsubsection{\texttt{checkedsmul}}
\subsubsection{\texttt{checkedsdiv}}

\subsubsection{\texttt{saturatingsincrement}}

Nasycająca inkrementacja ze znakiem.

\subsubsection{\texttt{saturatingsdecrement}}
\subsubsection{\texttt{saturatingsadd}}
\subsubsection{\texttt{saturatingssub}}
\subsubsection{\texttt{saturatingsmul}}
\subsubsection{\texttt{saturatingsdiv}}

\subsubsection{\texttt{move}}

Przesunięcie wartości między rejestrami.

\subsubsection{\texttt{copy}}

Skopiowanie wartości między rejestrami.

\subsubsection{\texttt{ptr}}

Konstruktor wskaźnika do wartości.

\subsubsection{\texttt{ptrlive}}

Sprawdzenie poprawności wskaźnika.

\subsubsection{\texttt{swap}}

Zamiana wartości w rejestrach.

\subsubsection{\texttt{delete}}

Wywołanie destruktora wartości.

\subsubsection{\texttt{isnull}}

\subsubsection{\texttt{print}}
\subsubsection{\texttt{echo}}

\subsubsection{\texttt{capture}}
\subsubsection{\texttt{capturecopy}}
\subsubsection{\texttt{capturemove}}

\subsubsection{\texttt{closure}}
\subsubsection{\texttt{function}}

\subsubsection{\texttt{frame}}
\subsubsection{\texttt{param}}

Instrukcja prywatna.

\subsubsection{\texttt{pamv}}

Instrukcja prywatna.

\subsubsection{\texttt{call}}
\subsubsection{\texttt{tailcall}}
\subsubsection{\texttt{defer}}
\subsubsection{\texttt{arg}}

Instrukcja prywatna.

\subsubsection{\texttt{allocate\_registers}}

\subsubsection{\texttt{process}}
\subsubsection{\texttt{self}}
\subsubsection{\texttt{pideq}}
\subsubsection{\texttt{join}}
\subsubsection{\texttt{send}}
\subsubsection{\texttt{receive}}

\subsubsection{\texttt{watchdog}}

\subsubsection{\texttt{jump}}
\subsubsection{\texttt{if}}

\subsubsection{\texttt{throw}}

Rzucenie wartości jako wyjątku.

\subsubsection{\texttt{catch}}
\subsubsection{\texttt{draw}}
\subsubsection{\texttt{try}}
\subsubsection{\texttt{enter}}
\subsubsection{\texttt{leave}}

\subsubsection{\texttt{import}}

Import modułu.

\subsubsection{\texttt{atom}}

Konstruktor atomu.

\subsubsection{\texttt{atomeq}}

\subsubsection{\texttt{struct}}

Konstruktor struktury.

\subsubsection{\texttt{structinsert}}
\subsubsection{\texttt{structremove}}
\subsubsection{\texttt{structat}}
\subsubsection{\texttt{structkeys}}

\subsubsection{\texttt{return}}

Wyjście z funkcji.

\subsubsection{\texttt{halt}}

Zakończenie działania i wyłączenie VM.


\section{Język assemblera Viua VM}
\label{appendix_viua_vm_assembly_language}

\subsection{Składnia języka assemblera}

\subsubsection{Ogólna składnia instrukcji}
\subsubsection{Definicje funkcji i bloków}
\subsubsection{Deklaracje funkcji i bloków}
\subsubsection{Import modułów}
\subsubsection{Markery}
\subsubsection{Nazywanie rejestrów}

\subsection{Instrukcje Viua VM}

\subsubsection{\texttt{nop}}

\subsubsection{\texttt{izero}}
\subsubsection{\texttt{integer}}
\subsubsection{\texttt{iinc}}
\subsubsection{\texttt{idec}}

\subsubsection{\texttt{float}}

\subsubsection{\texttt{itof}}
\subsubsection{\texttt{ftoi}}
\subsubsection{\texttt{stoi}}
\subsubsection{\texttt{stof}}

\subsubsection{\texttt{add}}
\subsubsection{\texttt{sub}}
\subsubsection{\texttt{mul}}
\subsubsection{\texttt{div}}
\subsubsection{\texttt{lt}}
\subsubsection{\texttt{lte}}
\subsubsection{\texttt{gt}}
\subsubsection{\texttt{gte}}
\subsubsection{\texttt{eq}}

\subsubsection{\texttt{string}}
\subsubsection{\texttt{streq}}

\subsubsection{\texttt{text}}
\subsubsection{\texttt{texteq}}
\subsubsection{\texttt{textat}}
\subsubsection{\texttt{textsub}}
\subsubsection{\texttt{textlength}}
\subsubsection{\texttt{textcommonprefix}}
\subsubsection{\texttt{textcommonsuffix}}
\subsubsection{\texttt{textconcat}}

\subsubsection{\texttt{vector}}
\subsubsection{\texttt{vinsert}}
\subsubsection{\texttt{vpush}}
\subsubsection{\texttt{vpop}}
\subsubsection{\texttt{vat}}
\subsubsection{\texttt{vlen}}

\subsubsection{\texttt{not}}

Negacja boolowska.

\subsubsection{\texttt{and}}

Iloczyn boolowski.

\subsubsection{\texttt{or}}

Suma boolowska.

\subsubsection{\texttt{bits\_of\_integer}}

Konstruktor bitów z liczby całkowitej.

\subsubsection{\texttt{integer\_of\_bits}}

Konstruktor liczby całkowitej z bitów.

\subsubsection{\texttt{bits}}

Konstruktor bitów.

\subsubsection{\texttt{bitand}}

Bitowa operacja \emph{\texttt{and}}.

\subsubsection{\texttt{bitor}}

Bitowa operacja \emph{\texttt{or}}.

\subsubsection{\texttt{bitnot}}

Bitowa operacja \emph{\texttt{not}}.

\subsubsection{\texttt{bitxor}}

Bitowa operacja \emph{\texttt{xor}}.

\subsubsection{\texttt{bitat}}

Sprawdzenie wartości pojedynczego bitu.

\subsubsection{\texttt{bitset}}

Ustawienie wartości pojedynczego bitu.

\subsubsection{\texttt{shl}}

Przesunięcie bitowe w lewo.

\subsubsection{\texttt{shr}}

Przesunięcie bitowe w prawo.

\subsubsection{\texttt{ashl}}

Arytmetyczne (z zachowaniem znaku) przesunięcie bitowe w lewo.

\subsubsection{\texttt{ashr}}

Arytmetyczne przesunięcie bitowe w prawo.

\subsubsection{\texttt{rol}}

Rotacja bitowa w lewo.

\subsubsection{\texttt{ror}}

Rotacja bitowa w prawo.

\subsubsection{\texttt{wrapincrement}}

Modulo inkrementacja ze znakiem.

\subsubsection{\texttt{wrapdecrement}}

Modulo dekrementacja ze znakiem.

\subsubsection{\texttt{wrapadd}}

Modulo dodawanie ze znakiem.

\subsubsection{\texttt{wrapsub}}

Modulo odejmowanie ze znakiem.

\subsubsection{\texttt{wrapmul}}

Modulo mnożenie ze znakiem.

\subsubsection{\texttt{wrapdiv}}

Modulo dzielenie ze znakiem.

\subsubsection{\texttt{checkedsincrement}}

Sprawdzana inkrementacja ze znakiem.

\subsubsection{\texttt{checkedsdecrement}}
\subsubsection{\texttt{checkedsadd}}
\subsubsection{\texttt{checkedssub}}
\subsubsection{\texttt{checkedsmul}}
\subsubsection{\texttt{checkedsdiv}}

\subsubsection{\texttt{saturatingsincrement}}

Nasycająca inkrementacja ze znakiem.

\subsubsection{\texttt{saturatingsdecrement}}
\subsubsection{\texttt{saturatingsadd}}
\subsubsection{\texttt{saturatingssub}}
\subsubsection{\texttt{saturatingsmul}}
\subsubsection{\texttt{saturatingsdiv}}

\subsubsection{\texttt{move}}

Przesunięcie wartości między rejestrami.

\subsubsection{\texttt{copy}}

Skopiowanie wartości między rejestrami.

\subsubsection{\texttt{ptr}}

Konstruktor wskaźnika do wartości.

\subsubsection{\texttt{ptrlive}}

Sprawdzenie poprawności wskaźnika.

\subsubsection{\texttt{swap}}

Zamiana wartości w rejestrach.

\subsubsection{\texttt{delete}}

Wywołanie destruktora wartości.

\subsubsection{\texttt{isnull}}

\subsubsection{\texttt{print}}
\subsubsection{\texttt{echo}}

\subsubsection{\texttt{capture}}
\subsubsection{\texttt{capturecopy}}
\subsubsection{\texttt{capturemove}}

\subsubsection{\texttt{closure}}
\subsubsection{\texttt{function}}

\subsubsection{\texttt{frame}}
\subsubsection{\texttt{param}}

Instrukcja prywatna.

\subsubsection{\texttt{pamv}}

Instrukcja prywatna.

\subsubsection{\texttt{call}}
\subsubsection{\texttt{tailcall}}
\subsubsection{\texttt{defer}}
\subsubsection{\texttt{arg}}

Instrukcja prywatna.

\subsubsection{\texttt{allocate\_registers}}

\subsubsection{\texttt{process}}
\subsubsection{\texttt{self}}
\subsubsection{\texttt{pideq}}
\subsubsection{\texttt{join}}
\subsubsection{\texttt{send}}
\subsubsection{\texttt{receive}}

\subsubsection{\texttt{watchdog}}

\subsubsection{\texttt{jump}}
\subsubsection{\texttt{if}}

\subsubsection{\texttt{throw}}

Rzucenie wartości jako wyjątku.

\subsubsection{\texttt{catch}}
\subsubsection{\texttt{draw}}
\subsubsection{\texttt{try}}
\subsubsection{\texttt{enter}}
\subsubsection{\texttt{leave}}

\subsubsection{\texttt{import}}

Import modułu.

\subsubsection{\texttt{atom}}

Konstruktor atomu.

\subsubsection{\texttt{atomeq}}

\subsubsection{\texttt{struct}}

Konstruktor struktury.

\subsubsection{\texttt{structinsert}}
\subsubsection{\texttt{structremove}}
\subsubsection{\texttt{structat}}
\subsubsection{\texttt{structkeys}}

\subsubsection{\texttt{return}}

Wyjście z funkcji.

\subsubsection{\texttt{halt}}

Zakończenie działania i wyłączenie VM.


\part{Język \ViuAct\ i jego kompilator}

\section{Język assemblera Viua VM}
\label{appendix_viua_vm_assembly_language}

\subsection{Składnia języka assemblera}

\subsubsection{Ogólna składnia instrukcji}
\subsubsection{Definicje funkcji i bloków}
\subsubsection{Deklaracje funkcji i bloków}
\subsubsection{Import modułów}
\subsubsection{Markery}
\subsubsection{Nazywanie rejestrów}

\subsection{Instrukcje Viua VM}

\subsubsection{\texttt{nop}}

\subsubsection{\texttt{izero}}
\subsubsection{\texttt{integer}}
\subsubsection{\texttt{iinc}}
\subsubsection{\texttt{idec}}

\subsubsection{\texttt{float}}

\subsubsection{\texttt{itof}}
\subsubsection{\texttt{ftoi}}
\subsubsection{\texttt{stoi}}
\subsubsection{\texttt{stof}}

\subsubsection{\texttt{add}}
\subsubsection{\texttt{sub}}
\subsubsection{\texttt{mul}}
\subsubsection{\texttt{div}}
\subsubsection{\texttt{lt}}
\subsubsection{\texttt{lte}}
\subsubsection{\texttt{gt}}
\subsubsection{\texttt{gte}}
\subsubsection{\texttt{eq}}

\subsubsection{\texttt{string}}
\subsubsection{\texttt{streq}}

\subsubsection{\texttt{text}}
\subsubsection{\texttt{texteq}}
\subsubsection{\texttt{textat}}
\subsubsection{\texttt{textsub}}
\subsubsection{\texttt{textlength}}
\subsubsection{\texttt{textcommonprefix}}
\subsubsection{\texttt{textcommonsuffix}}
\subsubsection{\texttt{textconcat}}

\subsubsection{\texttt{vector}}
\subsubsection{\texttt{vinsert}}
\subsubsection{\texttt{vpush}}
\subsubsection{\texttt{vpop}}
\subsubsection{\texttt{vat}}
\subsubsection{\texttt{vlen}}

\subsubsection{\texttt{not}}

Negacja boolowska.

\subsubsection{\texttt{and}}

Iloczyn boolowski.

\subsubsection{\texttt{or}}

Suma boolowska.

\subsubsection{\texttt{bits\_of\_integer}}

Konstruktor bitów z liczby całkowitej.

\subsubsection{\texttt{integer\_of\_bits}}

Konstruktor liczby całkowitej z bitów.

\subsubsection{\texttt{bits}}

Konstruktor bitów.

\subsubsection{\texttt{bitand}}

Bitowa operacja \emph{\texttt{and}}.

\subsubsection{\texttt{bitor}}

Bitowa operacja \emph{\texttt{or}}.

\subsubsection{\texttt{bitnot}}

Bitowa operacja \emph{\texttt{not}}.

\subsubsection{\texttt{bitxor}}

Bitowa operacja \emph{\texttt{xor}}.

\subsubsection{\texttt{bitat}}

Sprawdzenie wartości pojedynczego bitu.

\subsubsection{\texttt{bitset}}

Ustawienie wartości pojedynczego bitu.

\subsubsection{\texttt{shl}}

Przesunięcie bitowe w lewo.

\subsubsection{\texttt{shr}}

Przesunięcie bitowe w prawo.

\subsubsection{\texttt{ashl}}

Arytmetyczne (z zachowaniem znaku) przesunięcie bitowe w lewo.

\subsubsection{\texttt{ashr}}

Arytmetyczne przesunięcie bitowe w prawo.

\subsubsection{\texttt{rol}}

Rotacja bitowa w lewo.

\subsubsection{\texttt{ror}}

Rotacja bitowa w prawo.

\subsubsection{\texttt{wrapincrement}}

Modulo inkrementacja ze znakiem.

\subsubsection{\texttt{wrapdecrement}}

Modulo dekrementacja ze znakiem.

\subsubsection{\texttt{wrapadd}}

Modulo dodawanie ze znakiem.

\subsubsection{\texttt{wrapsub}}

Modulo odejmowanie ze znakiem.

\subsubsection{\texttt{wrapmul}}

Modulo mnożenie ze znakiem.

\subsubsection{\texttt{wrapdiv}}

Modulo dzielenie ze znakiem.

\subsubsection{\texttt{checkedsincrement}}

Sprawdzana inkrementacja ze znakiem.

\subsubsection{\texttt{checkedsdecrement}}
\subsubsection{\texttt{checkedsadd}}
\subsubsection{\texttt{checkedssub}}
\subsubsection{\texttt{checkedsmul}}
\subsubsection{\texttt{checkedsdiv}}

\subsubsection{\texttt{saturatingsincrement}}

Nasycająca inkrementacja ze znakiem.

\subsubsection{\texttt{saturatingsdecrement}}
\subsubsection{\texttt{saturatingsadd}}
\subsubsection{\texttt{saturatingssub}}
\subsubsection{\texttt{saturatingsmul}}
\subsubsection{\texttt{saturatingsdiv}}

\subsubsection{\texttt{move}}

Przesunięcie wartości między rejestrami.

\subsubsection{\texttt{copy}}

Skopiowanie wartości między rejestrami.

\subsubsection{\texttt{ptr}}

Konstruktor wskaźnika do wartości.

\subsubsection{\texttt{ptrlive}}

Sprawdzenie poprawności wskaźnika.

\subsubsection{\texttt{swap}}

Zamiana wartości w rejestrach.

\subsubsection{\texttt{delete}}

Wywołanie destruktora wartości.

\subsubsection{\texttt{isnull}}

\subsubsection{\texttt{print}}
\subsubsection{\texttt{echo}}

\subsubsection{\texttt{capture}}
\subsubsection{\texttt{capturecopy}}
\subsubsection{\texttt{capturemove}}

\subsubsection{\texttt{closure}}
\subsubsection{\texttt{function}}

\subsubsection{\texttt{frame}}
\subsubsection{\texttt{param}}

Instrukcja prywatna.

\subsubsection{\texttt{pamv}}

Instrukcja prywatna.

\subsubsection{\texttt{call}}
\subsubsection{\texttt{tailcall}}
\subsubsection{\texttt{defer}}
\subsubsection{\texttt{arg}}

Instrukcja prywatna.

\subsubsection{\texttt{allocate\_registers}}

\subsubsection{\texttt{process}}
\subsubsection{\texttt{self}}
\subsubsection{\texttt{pideq}}
\subsubsection{\texttt{join}}
\subsubsection{\texttt{send}}
\subsubsection{\texttt{receive}}

\subsubsection{\texttt{watchdog}}

\subsubsection{\texttt{jump}}
\subsubsection{\texttt{if}}

\subsubsection{\texttt{throw}}

Rzucenie wartości jako wyjątku.

\subsubsection{\texttt{catch}}
\subsubsection{\texttt{draw}}
\subsubsection{\texttt{try}}
\subsubsection{\texttt{enter}}
\subsubsection{\texttt{leave}}

\subsubsection{\texttt{import}}

Import modułu.

\subsubsection{\texttt{atom}}

Konstruktor atomu.

\subsubsection{\texttt{atomeq}}

\subsubsection{\texttt{struct}}

Konstruktor struktury.

\subsubsection{\texttt{structinsert}}
\subsubsection{\texttt{structremove}}
\subsubsection{\texttt{structat}}
\subsubsection{\texttt{structkeys}}

\subsubsection{\texttt{return}}

Wyjście z funkcji.

\subsubsection{\texttt{halt}}

Zakończenie działania i wyłączenie VM.


\section{Język assemblera Viua VM}
\label{appendix_viua_vm_assembly_language}

\subsection{Składnia języka assemblera}

\subsubsection{Ogólna składnia instrukcji}
\subsubsection{Definicje funkcji i bloków}
\subsubsection{Deklaracje funkcji i bloków}
\subsubsection{Import modułów}
\subsubsection{Markery}
\subsubsection{Nazywanie rejestrów}

\subsection{Instrukcje Viua VM}

\subsubsection{\texttt{nop}}

\subsubsection{\texttt{izero}}
\subsubsection{\texttt{integer}}
\subsubsection{\texttt{iinc}}
\subsubsection{\texttt{idec}}

\subsubsection{\texttt{float}}

\subsubsection{\texttt{itof}}
\subsubsection{\texttt{ftoi}}
\subsubsection{\texttt{stoi}}
\subsubsection{\texttt{stof}}

\subsubsection{\texttt{add}}
\subsubsection{\texttt{sub}}
\subsubsection{\texttt{mul}}
\subsubsection{\texttt{div}}
\subsubsection{\texttt{lt}}
\subsubsection{\texttt{lte}}
\subsubsection{\texttt{gt}}
\subsubsection{\texttt{gte}}
\subsubsection{\texttt{eq}}

\subsubsection{\texttt{string}}
\subsubsection{\texttt{streq}}

\subsubsection{\texttt{text}}
\subsubsection{\texttt{texteq}}
\subsubsection{\texttt{textat}}
\subsubsection{\texttt{textsub}}
\subsubsection{\texttt{textlength}}
\subsubsection{\texttt{textcommonprefix}}
\subsubsection{\texttt{textcommonsuffix}}
\subsubsection{\texttt{textconcat}}

\subsubsection{\texttt{vector}}
\subsubsection{\texttt{vinsert}}
\subsubsection{\texttt{vpush}}
\subsubsection{\texttt{vpop}}
\subsubsection{\texttt{vat}}
\subsubsection{\texttt{vlen}}

\subsubsection{\texttt{not}}

Negacja boolowska.

\subsubsection{\texttt{and}}

Iloczyn boolowski.

\subsubsection{\texttt{or}}

Suma boolowska.

\subsubsection{\texttt{bits\_of\_integer}}

Konstruktor bitów z liczby całkowitej.

\subsubsection{\texttt{integer\_of\_bits}}

Konstruktor liczby całkowitej z bitów.

\subsubsection{\texttt{bits}}

Konstruktor bitów.

\subsubsection{\texttt{bitand}}

Bitowa operacja \emph{\texttt{and}}.

\subsubsection{\texttt{bitor}}

Bitowa operacja \emph{\texttt{or}}.

\subsubsection{\texttt{bitnot}}

Bitowa operacja \emph{\texttt{not}}.

\subsubsection{\texttt{bitxor}}

Bitowa operacja \emph{\texttt{xor}}.

\subsubsection{\texttt{bitat}}

Sprawdzenie wartości pojedynczego bitu.

\subsubsection{\texttt{bitset}}

Ustawienie wartości pojedynczego bitu.

\subsubsection{\texttt{shl}}

Przesunięcie bitowe w lewo.

\subsubsection{\texttt{shr}}

Przesunięcie bitowe w prawo.

\subsubsection{\texttt{ashl}}

Arytmetyczne (z zachowaniem znaku) przesunięcie bitowe w lewo.

\subsubsection{\texttt{ashr}}

Arytmetyczne przesunięcie bitowe w prawo.

\subsubsection{\texttt{rol}}

Rotacja bitowa w lewo.

\subsubsection{\texttt{ror}}

Rotacja bitowa w prawo.

\subsubsection{\texttt{wrapincrement}}

Modulo inkrementacja ze znakiem.

\subsubsection{\texttt{wrapdecrement}}

Modulo dekrementacja ze znakiem.

\subsubsection{\texttt{wrapadd}}

Modulo dodawanie ze znakiem.

\subsubsection{\texttt{wrapsub}}

Modulo odejmowanie ze znakiem.

\subsubsection{\texttt{wrapmul}}

Modulo mnożenie ze znakiem.

\subsubsection{\texttt{wrapdiv}}

Modulo dzielenie ze znakiem.

\subsubsection{\texttt{checkedsincrement}}

Sprawdzana inkrementacja ze znakiem.

\subsubsection{\texttt{checkedsdecrement}}
\subsubsection{\texttt{checkedsadd}}
\subsubsection{\texttt{checkedssub}}
\subsubsection{\texttt{checkedsmul}}
\subsubsection{\texttt{checkedsdiv}}

\subsubsection{\texttt{saturatingsincrement}}

Nasycająca inkrementacja ze znakiem.

\subsubsection{\texttt{saturatingsdecrement}}
\subsubsection{\texttt{saturatingsadd}}
\subsubsection{\texttt{saturatingssub}}
\subsubsection{\texttt{saturatingsmul}}
\subsubsection{\texttt{saturatingsdiv}}

\subsubsection{\texttt{move}}

Przesunięcie wartości między rejestrami.

\subsubsection{\texttt{copy}}

Skopiowanie wartości między rejestrami.

\subsubsection{\texttt{ptr}}

Konstruktor wskaźnika do wartości.

\subsubsection{\texttt{ptrlive}}

Sprawdzenie poprawności wskaźnika.

\subsubsection{\texttt{swap}}

Zamiana wartości w rejestrach.

\subsubsection{\texttt{delete}}

Wywołanie destruktora wartości.

\subsubsection{\texttt{isnull}}

\subsubsection{\texttt{print}}
\subsubsection{\texttt{echo}}

\subsubsection{\texttt{capture}}
\subsubsection{\texttt{capturecopy}}
\subsubsection{\texttt{capturemove}}

\subsubsection{\texttt{closure}}
\subsubsection{\texttt{function}}

\subsubsection{\texttt{frame}}
\subsubsection{\texttt{param}}

Instrukcja prywatna.

\subsubsection{\texttt{pamv}}

Instrukcja prywatna.

\subsubsection{\texttt{call}}
\subsubsection{\texttt{tailcall}}
\subsubsection{\texttt{defer}}
\subsubsection{\texttt{arg}}

Instrukcja prywatna.

\subsubsection{\texttt{allocate\_registers}}

\subsubsection{\texttt{process}}
\subsubsection{\texttt{self}}
\subsubsection{\texttt{pideq}}
\subsubsection{\texttt{join}}
\subsubsection{\texttt{send}}
\subsubsection{\texttt{receive}}

\subsubsection{\texttt{watchdog}}

\subsubsection{\texttt{jump}}
\subsubsection{\texttt{if}}

\subsubsection{\texttt{throw}}

Rzucenie wartości jako wyjątku.

\subsubsection{\texttt{catch}}
\subsubsection{\texttt{draw}}
\subsubsection{\texttt{try}}
\subsubsection{\texttt{enter}}
\subsubsection{\texttt{leave}}

\subsubsection{\texttt{import}}

Import modułu.

\subsubsection{\texttt{atom}}

Konstruktor atomu.

\subsubsection{\texttt{atomeq}}

\subsubsection{\texttt{struct}}

Konstruktor struktury.

\subsubsection{\texttt{structinsert}}
\subsubsection{\texttt{structremove}}
\subsubsection{\texttt{structat}}
\subsubsection{\texttt{structkeys}}

\subsubsection{\texttt{return}}

Wyjście z funkcji.

\subsubsection{\texttt{halt}}

Zakończenie działania i wyłączenie VM.


\section{Język assemblera Viua VM}
\label{appendix_viua_vm_assembly_language}

\subsection{Składnia języka assemblera}

\subsubsection{Ogólna składnia instrukcji}
\subsubsection{Definicje funkcji i bloków}
\subsubsection{Deklaracje funkcji i bloków}
\subsubsection{Import modułów}
\subsubsection{Markery}
\subsubsection{Nazywanie rejestrów}

\subsection{Instrukcje Viua VM}

\subsubsection{\texttt{nop}}

\subsubsection{\texttt{izero}}
\subsubsection{\texttt{integer}}
\subsubsection{\texttt{iinc}}
\subsubsection{\texttt{idec}}

\subsubsection{\texttt{float}}

\subsubsection{\texttt{itof}}
\subsubsection{\texttt{ftoi}}
\subsubsection{\texttt{stoi}}
\subsubsection{\texttt{stof}}

\subsubsection{\texttt{add}}
\subsubsection{\texttt{sub}}
\subsubsection{\texttt{mul}}
\subsubsection{\texttt{div}}
\subsubsection{\texttt{lt}}
\subsubsection{\texttt{lte}}
\subsubsection{\texttt{gt}}
\subsubsection{\texttt{gte}}
\subsubsection{\texttt{eq}}

\subsubsection{\texttt{string}}
\subsubsection{\texttt{streq}}

\subsubsection{\texttt{text}}
\subsubsection{\texttt{texteq}}
\subsubsection{\texttt{textat}}
\subsubsection{\texttt{textsub}}
\subsubsection{\texttt{textlength}}
\subsubsection{\texttt{textcommonprefix}}
\subsubsection{\texttt{textcommonsuffix}}
\subsubsection{\texttt{textconcat}}

\subsubsection{\texttt{vector}}
\subsubsection{\texttt{vinsert}}
\subsubsection{\texttt{vpush}}
\subsubsection{\texttt{vpop}}
\subsubsection{\texttt{vat}}
\subsubsection{\texttt{vlen}}

\subsubsection{\texttt{not}}

Negacja boolowska.

\subsubsection{\texttt{and}}

Iloczyn boolowski.

\subsubsection{\texttt{or}}

Suma boolowska.

\subsubsection{\texttt{bits\_of\_integer}}

Konstruktor bitów z liczby całkowitej.

\subsubsection{\texttt{integer\_of\_bits}}

Konstruktor liczby całkowitej z bitów.

\subsubsection{\texttt{bits}}

Konstruktor bitów.

\subsubsection{\texttt{bitand}}

Bitowa operacja \emph{\texttt{and}}.

\subsubsection{\texttt{bitor}}

Bitowa operacja \emph{\texttt{or}}.

\subsubsection{\texttt{bitnot}}

Bitowa operacja \emph{\texttt{not}}.

\subsubsection{\texttt{bitxor}}

Bitowa operacja \emph{\texttt{xor}}.

\subsubsection{\texttt{bitat}}

Sprawdzenie wartości pojedynczego bitu.

\subsubsection{\texttt{bitset}}

Ustawienie wartości pojedynczego bitu.

\subsubsection{\texttt{shl}}

Przesunięcie bitowe w lewo.

\subsubsection{\texttt{shr}}

Przesunięcie bitowe w prawo.

\subsubsection{\texttt{ashl}}

Arytmetyczne (z zachowaniem znaku) przesunięcie bitowe w lewo.

\subsubsection{\texttt{ashr}}

Arytmetyczne przesunięcie bitowe w prawo.

\subsubsection{\texttt{rol}}

Rotacja bitowa w lewo.

\subsubsection{\texttt{ror}}

Rotacja bitowa w prawo.

\subsubsection{\texttt{wrapincrement}}

Modulo inkrementacja ze znakiem.

\subsubsection{\texttt{wrapdecrement}}

Modulo dekrementacja ze znakiem.

\subsubsection{\texttt{wrapadd}}

Modulo dodawanie ze znakiem.

\subsubsection{\texttt{wrapsub}}

Modulo odejmowanie ze znakiem.

\subsubsection{\texttt{wrapmul}}

Modulo mnożenie ze znakiem.

\subsubsection{\texttt{wrapdiv}}

Modulo dzielenie ze znakiem.

\subsubsection{\texttt{checkedsincrement}}

Sprawdzana inkrementacja ze znakiem.

\subsubsection{\texttt{checkedsdecrement}}
\subsubsection{\texttt{checkedsadd}}
\subsubsection{\texttt{checkedssub}}
\subsubsection{\texttt{checkedsmul}}
\subsubsection{\texttt{checkedsdiv}}

\subsubsection{\texttt{saturatingsincrement}}

Nasycająca inkrementacja ze znakiem.

\subsubsection{\texttt{saturatingsdecrement}}
\subsubsection{\texttt{saturatingsadd}}
\subsubsection{\texttt{saturatingssub}}
\subsubsection{\texttt{saturatingsmul}}
\subsubsection{\texttt{saturatingsdiv}}

\subsubsection{\texttt{move}}

Przesunięcie wartości między rejestrami.

\subsubsection{\texttt{copy}}

Skopiowanie wartości między rejestrami.

\subsubsection{\texttt{ptr}}

Konstruktor wskaźnika do wartości.

\subsubsection{\texttt{ptrlive}}

Sprawdzenie poprawności wskaźnika.

\subsubsection{\texttt{swap}}

Zamiana wartości w rejestrach.

\subsubsection{\texttt{delete}}

Wywołanie destruktora wartości.

\subsubsection{\texttt{isnull}}

\subsubsection{\texttt{print}}
\subsubsection{\texttt{echo}}

\subsubsection{\texttt{capture}}
\subsubsection{\texttt{capturecopy}}
\subsubsection{\texttt{capturemove}}

\subsubsection{\texttt{closure}}
\subsubsection{\texttt{function}}

\subsubsection{\texttt{frame}}
\subsubsection{\texttt{param}}

Instrukcja prywatna.

\subsubsection{\texttt{pamv}}

Instrukcja prywatna.

\subsubsection{\texttt{call}}
\subsubsection{\texttt{tailcall}}
\subsubsection{\texttt{defer}}
\subsubsection{\texttt{arg}}

Instrukcja prywatna.

\subsubsection{\texttt{allocate\_registers}}

\subsubsection{\texttt{process}}
\subsubsection{\texttt{self}}
\subsubsection{\texttt{pideq}}
\subsubsection{\texttt{join}}
\subsubsection{\texttt{send}}
\subsubsection{\texttt{receive}}

\subsubsection{\texttt{watchdog}}

\subsubsection{\texttt{jump}}
\subsubsection{\texttt{if}}

\subsubsection{\texttt{throw}}

Rzucenie wartości jako wyjątku.

\subsubsection{\texttt{catch}}
\subsubsection{\texttt{draw}}
\subsubsection{\texttt{try}}
\subsubsection{\texttt{enter}}
\subsubsection{\texttt{leave}}

\subsubsection{\texttt{import}}

Import modułu.

\subsubsection{\texttt{atom}}

Konstruktor atomu.

\subsubsection{\texttt{atomeq}}

\subsubsection{\texttt{struct}}

Konstruktor struktury.

\subsubsection{\texttt{structinsert}}
\subsubsection{\texttt{structremove}}
\subsubsection{\texttt{structat}}
\subsubsection{\texttt{structkeys}}

\subsubsection{\texttt{return}}

Wyjście z funkcji.

\subsubsection{\texttt{halt}}

Zakończenie działania i wyłączenie VM.


\part{Program ViuaChat}

\section{Język assemblera Viua VM}
\label{appendix_viua_vm_assembly_language}

\subsection{Składnia języka assemblera}

\subsubsection{Ogólna składnia instrukcji}
\subsubsection{Definicje funkcji i bloków}
\subsubsection{Deklaracje funkcji i bloków}
\subsubsection{Import modułów}
\subsubsection{Markery}
\subsubsection{Nazywanie rejestrów}

\subsection{Instrukcje Viua VM}

\subsubsection{\texttt{nop}}

\subsubsection{\texttt{izero}}
\subsubsection{\texttt{integer}}
\subsubsection{\texttt{iinc}}
\subsubsection{\texttt{idec}}

\subsubsection{\texttt{float}}

\subsubsection{\texttt{itof}}
\subsubsection{\texttt{ftoi}}
\subsubsection{\texttt{stoi}}
\subsubsection{\texttt{stof}}

\subsubsection{\texttt{add}}
\subsubsection{\texttt{sub}}
\subsubsection{\texttt{mul}}
\subsubsection{\texttt{div}}
\subsubsection{\texttt{lt}}
\subsubsection{\texttt{lte}}
\subsubsection{\texttt{gt}}
\subsubsection{\texttt{gte}}
\subsubsection{\texttt{eq}}

\subsubsection{\texttt{string}}
\subsubsection{\texttt{streq}}

\subsubsection{\texttt{text}}
\subsubsection{\texttt{texteq}}
\subsubsection{\texttt{textat}}
\subsubsection{\texttt{textsub}}
\subsubsection{\texttt{textlength}}
\subsubsection{\texttt{textcommonprefix}}
\subsubsection{\texttt{textcommonsuffix}}
\subsubsection{\texttt{textconcat}}

\subsubsection{\texttt{vector}}
\subsubsection{\texttt{vinsert}}
\subsubsection{\texttt{vpush}}
\subsubsection{\texttt{vpop}}
\subsubsection{\texttt{vat}}
\subsubsection{\texttt{vlen}}

\subsubsection{\texttt{not}}

Negacja boolowska.

\subsubsection{\texttt{and}}

Iloczyn boolowski.

\subsubsection{\texttt{or}}

Suma boolowska.

\subsubsection{\texttt{bits\_of\_integer}}

Konstruktor bitów z liczby całkowitej.

\subsubsection{\texttt{integer\_of\_bits}}

Konstruktor liczby całkowitej z bitów.

\subsubsection{\texttt{bits}}

Konstruktor bitów.

\subsubsection{\texttt{bitand}}

Bitowa operacja \emph{\texttt{and}}.

\subsubsection{\texttt{bitor}}

Bitowa operacja \emph{\texttt{or}}.

\subsubsection{\texttt{bitnot}}

Bitowa operacja \emph{\texttt{not}}.

\subsubsection{\texttt{bitxor}}

Bitowa operacja \emph{\texttt{xor}}.

\subsubsection{\texttt{bitat}}

Sprawdzenie wartości pojedynczego bitu.

\subsubsection{\texttt{bitset}}

Ustawienie wartości pojedynczego bitu.

\subsubsection{\texttt{shl}}

Przesunięcie bitowe w lewo.

\subsubsection{\texttt{shr}}

Przesunięcie bitowe w prawo.

\subsubsection{\texttt{ashl}}

Arytmetyczne (z zachowaniem znaku) przesunięcie bitowe w lewo.

\subsubsection{\texttt{ashr}}

Arytmetyczne przesunięcie bitowe w prawo.

\subsubsection{\texttt{rol}}

Rotacja bitowa w lewo.

\subsubsection{\texttt{ror}}

Rotacja bitowa w prawo.

\subsubsection{\texttt{wrapincrement}}

Modulo inkrementacja ze znakiem.

\subsubsection{\texttt{wrapdecrement}}

Modulo dekrementacja ze znakiem.

\subsubsection{\texttt{wrapadd}}

Modulo dodawanie ze znakiem.

\subsubsection{\texttt{wrapsub}}

Modulo odejmowanie ze znakiem.

\subsubsection{\texttt{wrapmul}}

Modulo mnożenie ze znakiem.

\subsubsection{\texttt{wrapdiv}}

Modulo dzielenie ze znakiem.

\subsubsection{\texttt{checkedsincrement}}

Sprawdzana inkrementacja ze znakiem.

\subsubsection{\texttt{checkedsdecrement}}
\subsubsection{\texttt{checkedsadd}}
\subsubsection{\texttt{checkedssub}}
\subsubsection{\texttt{checkedsmul}}
\subsubsection{\texttt{checkedsdiv}}

\subsubsection{\texttt{saturatingsincrement}}

Nasycająca inkrementacja ze znakiem.

\subsubsection{\texttt{saturatingsdecrement}}
\subsubsection{\texttt{saturatingsadd}}
\subsubsection{\texttt{saturatingssub}}
\subsubsection{\texttt{saturatingsmul}}
\subsubsection{\texttt{saturatingsdiv}}

\subsubsection{\texttt{move}}

Przesunięcie wartości między rejestrami.

\subsubsection{\texttt{copy}}

Skopiowanie wartości między rejestrami.

\subsubsection{\texttt{ptr}}

Konstruktor wskaźnika do wartości.

\subsubsection{\texttt{ptrlive}}

Sprawdzenie poprawności wskaźnika.

\subsubsection{\texttt{swap}}

Zamiana wartości w rejestrach.

\subsubsection{\texttt{delete}}

Wywołanie destruktora wartości.

\subsubsection{\texttt{isnull}}

\subsubsection{\texttt{print}}
\subsubsection{\texttt{echo}}

\subsubsection{\texttt{capture}}
\subsubsection{\texttt{capturecopy}}
\subsubsection{\texttt{capturemove}}

\subsubsection{\texttt{closure}}
\subsubsection{\texttt{function}}

\subsubsection{\texttt{frame}}
\subsubsection{\texttt{param}}

Instrukcja prywatna.

\subsubsection{\texttt{pamv}}

Instrukcja prywatna.

\subsubsection{\texttt{call}}
\subsubsection{\texttt{tailcall}}
\subsubsection{\texttt{defer}}
\subsubsection{\texttt{arg}}

Instrukcja prywatna.

\subsubsection{\texttt{allocate\_registers}}

\subsubsection{\texttt{process}}
\subsubsection{\texttt{self}}
\subsubsection{\texttt{pideq}}
\subsubsection{\texttt{join}}
\subsubsection{\texttt{send}}
\subsubsection{\texttt{receive}}

\subsubsection{\texttt{watchdog}}

\subsubsection{\texttt{jump}}
\subsubsection{\texttt{if}}

\subsubsection{\texttt{throw}}

Rzucenie wartości jako wyjątku.

\subsubsection{\texttt{catch}}
\subsubsection{\texttt{draw}}
\subsubsection{\texttt{try}}
\subsubsection{\texttt{enter}}
\subsubsection{\texttt{leave}}

\subsubsection{\texttt{import}}

Import modułu.

\subsubsection{\texttt{atom}}

Konstruktor atomu.

\subsubsection{\texttt{atomeq}}

\subsubsection{\texttt{struct}}

Konstruktor struktury.

\subsubsection{\texttt{structinsert}}
\subsubsection{\texttt{structremove}}
\subsubsection{\texttt{structat}}
\subsubsection{\texttt{structkeys}}

\subsubsection{\texttt{return}}

Wyjście z funkcji.

\subsubsection{\texttt{halt}}

Zakończenie działania i wyłączenie VM.


\section{Język assemblera Viua VM}
\label{appendix_viua_vm_assembly_language}

\subsection{Składnia języka assemblera}

\subsubsection{Ogólna składnia instrukcji}
\subsubsection{Definicje funkcji i bloków}
\subsubsection{Deklaracje funkcji i bloków}
\subsubsection{Import modułów}
\subsubsection{Markery}
\subsubsection{Nazywanie rejestrów}

\subsection{Instrukcje Viua VM}

\subsubsection{\texttt{nop}}

\subsubsection{\texttt{izero}}
\subsubsection{\texttt{integer}}
\subsubsection{\texttt{iinc}}
\subsubsection{\texttt{idec}}

\subsubsection{\texttt{float}}

\subsubsection{\texttt{itof}}
\subsubsection{\texttt{ftoi}}
\subsubsection{\texttt{stoi}}
\subsubsection{\texttt{stof}}

\subsubsection{\texttt{add}}
\subsubsection{\texttt{sub}}
\subsubsection{\texttt{mul}}
\subsubsection{\texttt{div}}
\subsubsection{\texttt{lt}}
\subsubsection{\texttt{lte}}
\subsubsection{\texttt{gt}}
\subsubsection{\texttt{gte}}
\subsubsection{\texttt{eq}}

\subsubsection{\texttt{string}}
\subsubsection{\texttt{streq}}

\subsubsection{\texttt{text}}
\subsubsection{\texttt{texteq}}
\subsubsection{\texttt{textat}}
\subsubsection{\texttt{textsub}}
\subsubsection{\texttt{textlength}}
\subsubsection{\texttt{textcommonprefix}}
\subsubsection{\texttt{textcommonsuffix}}
\subsubsection{\texttt{textconcat}}

\subsubsection{\texttt{vector}}
\subsubsection{\texttt{vinsert}}
\subsubsection{\texttt{vpush}}
\subsubsection{\texttt{vpop}}
\subsubsection{\texttt{vat}}
\subsubsection{\texttt{vlen}}

\subsubsection{\texttt{not}}

Negacja boolowska.

\subsubsection{\texttt{and}}

Iloczyn boolowski.

\subsubsection{\texttt{or}}

Suma boolowska.

\subsubsection{\texttt{bits\_of\_integer}}

Konstruktor bitów z liczby całkowitej.

\subsubsection{\texttt{integer\_of\_bits}}

Konstruktor liczby całkowitej z bitów.

\subsubsection{\texttt{bits}}

Konstruktor bitów.

\subsubsection{\texttt{bitand}}

Bitowa operacja \emph{\texttt{and}}.

\subsubsection{\texttt{bitor}}

Bitowa operacja \emph{\texttt{or}}.

\subsubsection{\texttt{bitnot}}

Bitowa operacja \emph{\texttt{not}}.

\subsubsection{\texttt{bitxor}}

Bitowa operacja \emph{\texttt{xor}}.

\subsubsection{\texttt{bitat}}

Sprawdzenie wartości pojedynczego bitu.

\subsubsection{\texttt{bitset}}

Ustawienie wartości pojedynczego bitu.

\subsubsection{\texttt{shl}}

Przesunięcie bitowe w lewo.

\subsubsection{\texttt{shr}}

Przesunięcie bitowe w prawo.

\subsubsection{\texttt{ashl}}

Arytmetyczne (z zachowaniem znaku) przesunięcie bitowe w lewo.

\subsubsection{\texttt{ashr}}

Arytmetyczne przesunięcie bitowe w prawo.

\subsubsection{\texttt{rol}}

Rotacja bitowa w lewo.

\subsubsection{\texttt{ror}}

Rotacja bitowa w prawo.

\subsubsection{\texttt{wrapincrement}}

Modulo inkrementacja ze znakiem.

\subsubsection{\texttt{wrapdecrement}}

Modulo dekrementacja ze znakiem.

\subsubsection{\texttt{wrapadd}}

Modulo dodawanie ze znakiem.

\subsubsection{\texttt{wrapsub}}

Modulo odejmowanie ze znakiem.

\subsubsection{\texttt{wrapmul}}

Modulo mnożenie ze znakiem.

\subsubsection{\texttt{wrapdiv}}

Modulo dzielenie ze znakiem.

\subsubsection{\texttt{checkedsincrement}}

Sprawdzana inkrementacja ze znakiem.

\subsubsection{\texttt{checkedsdecrement}}
\subsubsection{\texttt{checkedsadd}}
\subsubsection{\texttt{checkedssub}}
\subsubsection{\texttt{checkedsmul}}
\subsubsection{\texttt{checkedsdiv}}

\subsubsection{\texttt{saturatingsincrement}}

Nasycająca inkrementacja ze znakiem.

\subsubsection{\texttt{saturatingsdecrement}}
\subsubsection{\texttt{saturatingsadd}}
\subsubsection{\texttt{saturatingssub}}
\subsubsection{\texttt{saturatingsmul}}
\subsubsection{\texttt{saturatingsdiv}}

\subsubsection{\texttt{move}}

Przesunięcie wartości między rejestrami.

\subsubsection{\texttt{copy}}

Skopiowanie wartości między rejestrami.

\subsubsection{\texttt{ptr}}

Konstruktor wskaźnika do wartości.

\subsubsection{\texttt{ptrlive}}

Sprawdzenie poprawności wskaźnika.

\subsubsection{\texttt{swap}}

Zamiana wartości w rejestrach.

\subsubsection{\texttt{delete}}

Wywołanie destruktora wartości.

\subsubsection{\texttt{isnull}}

\subsubsection{\texttt{print}}
\subsubsection{\texttt{echo}}

\subsubsection{\texttt{capture}}
\subsubsection{\texttt{capturecopy}}
\subsubsection{\texttt{capturemove}}

\subsubsection{\texttt{closure}}
\subsubsection{\texttt{function}}

\subsubsection{\texttt{frame}}
\subsubsection{\texttt{param}}

Instrukcja prywatna.

\subsubsection{\texttt{pamv}}

Instrukcja prywatna.

\subsubsection{\texttt{call}}
\subsubsection{\texttt{tailcall}}
\subsubsection{\texttt{defer}}
\subsubsection{\texttt{arg}}

Instrukcja prywatna.

\subsubsection{\texttt{allocate\_registers}}

\subsubsection{\texttt{process}}
\subsubsection{\texttt{self}}
\subsubsection{\texttt{pideq}}
\subsubsection{\texttt{join}}
\subsubsection{\texttt{send}}
\subsubsection{\texttt{receive}}

\subsubsection{\texttt{watchdog}}

\subsubsection{\texttt{jump}}
\subsubsection{\texttt{if}}

\subsubsection{\texttt{throw}}

Rzucenie wartości jako wyjątku.

\subsubsection{\texttt{catch}}
\subsubsection{\texttt{draw}}
\subsubsection{\texttt{try}}
\subsubsection{\texttt{enter}}
\subsubsection{\texttt{leave}}

\subsubsection{\texttt{import}}

Import modułu.

\subsubsection{\texttt{atom}}

Konstruktor atomu.

\subsubsection{\texttt{atomeq}}

\subsubsection{\texttt{struct}}

Konstruktor struktury.

\subsubsection{\texttt{structinsert}}
\subsubsection{\texttt{structremove}}
\subsubsection{\texttt{structat}}
\subsubsection{\texttt{structkeys}}

\subsubsection{\texttt{return}}

Wyjście z funkcji.

\subsubsection{\texttt{halt}}

Zakończenie działania i wyłączenie VM.


\part{Podsumowanie}

\section{Język assemblera Viua VM}
\label{appendix_viua_vm_assembly_language}

\subsection{Składnia języka assemblera}

\subsubsection{Ogólna składnia instrukcji}
\subsubsection{Definicje funkcji i bloków}
\subsubsection{Deklaracje funkcji i bloków}
\subsubsection{Import modułów}
\subsubsection{Markery}
\subsubsection{Nazywanie rejestrów}

\subsection{Instrukcje Viua VM}

\subsubsection{\texttt{nop}}

\subsubsection{\texttt{izero}}
\subsubsection{\texttt{integer}}
\subsubsection{\texttt{iinc}}
\subsubsection{\texttt{idec}}

\subsubsection{\texttt{float}}

\subsubsection{\texttt{itof}}
\subsubsection{\texttt{ftoi}}
\subsubsection{\texttt{stoi}}
\subsubsection{\texttt{stof}}

\subsubsection{\texttt{add}}
\subsubsection{\texttt{sub}}
\subsubsection{\texttt{mul}}
\subsubsection{\texttt{div}}
\subsubsection{\texttt{lt}}
\subsubsection{\texttt{lte}}
\subsubsection{\texttt{gt}}
\subsubsection{\texttt{gte}}
\subsubsection{\texttt{eq}}

\subsubsection{\texttt{string}}
\subsubsection{\texttt{streq}}

\subsubsection{\texttt{text}}
\subsubsection{\texttt{texteq}}
\subsubsection{\texttt{textat}}
\subsubsection{\texttt{textsub}}
\subsubsection{\texttt{textlength}}
\subsubsection{\texttt{textcommonprefix}}
\subsubsection{\texttt{textcommonsuffix}}
\subsubsection{\texttt{textconcat}}

\subsubsection{\texttt{vector}}
\subsubsection{\texttt{vinsert}}
\subsubsection{\texttt{vpush}}
\subsubsection{\texttt{vpop}}
\subsubsection{\texttt{vat}}
\subsubsection{\texttt{vlen}}

\subsubsection{\texttt{not}}

Negacja boolowska.

\subsubsection{\texttt{and}}

Iloczyn boolowski.

\subsubsection{\texttt{or}}

Suma boolowska.

\subsubsection{\texttt{bits\_of\_integer}}

Konstruktor bitów z liczby całkowitej.

\subsubsection{\texttt{integer\_of\_bits}}

Konstruktor liczby całkowitej z bitów.

\subsubsection{\texttt{bits}}

Konstruktor bitów.

\subsubsection{\texttt{bitand}}

Bitowa operacja \emph{\texttt{and}}.

\subsubsection{\texttt{bitor}}

Bitowa operacja \emph{\texttt{or}}.

\subsubsection{\texttt{bitnot}}

Bitowa operacja \emph{\texttt{not}}.

\subsubsection{\texttt{bitxor}}

Bitowa operacja \emph{\texttt{xor}}.

\subsubsection{\texttt{bitat}}

Sprawdzenie wartości pojedynczego bitu.

\subsubsection{\texttt{bitset}}

Ustawienie wartości pojedynczego bitu.

\subsubsection{\texttt{shl}}

Przesunięcie bitowe w lewo.

\subsubsection{\texttt{shr}}

Przesunięcie bitowe w prawo.

\subsubsection{\texttt{ashl}}

Arytmetyczne (z zachowaniem znaku) przesunięcie bitowe w lewo.

\subsubsection{\texttt{ashr}}

Arytmetyczne przesunięcie bitowe w prawo.

\subsubsection{\texttt{rol}}

Rotacja bitowa w lewo.

\subsubsection{\texttt{ror}}

Rotacja bitowa w prawo.

\subsubsection{\texttt{wrapincrement}}

Modulo inkrementacja ze znakiem.

\subsubsection{\texttt{wrapdecrement}}

Modulo dekrementacja ze znakiem.

\subsubsection{\texttt{wrapadd}}

Modulo dodawanie ze znakiem.

\subsubsection{\texttt{wrapsub}}

Modulo odejmowanie ze znakiem.

\subsubsection{\texttt{wrapmul}}

Modulo mnożenie ze znakiem.

\subsubsection{\texttt{wrapdiv}}

Modulo dzielenie ze znakiem.

\subsubsection{\texttt{checkedsincrement}}

Sprawdzana inkrementacja ze znakiem.

\subsubsection{\texttt{checkedsdecrement}}
\subsubsection{\texttt{checkedsadd}}
\subsubsection{\texttt{checkedssub}}
\subsubsection{\texttt{checkedsmul}}
\subsubsection{\texttt{checkedsdiv}}

\subsubsection{\texttt{saturatingsincrement}}

Nasycająca inkrementacja ze znakiem.

\subsubsection{\texttt{saturatingsdecrement}}
\subsubsection{\texttt{saturatingsadd}}
\subsubsection{\texttt{saturatingssub}}
\subsubsection{\texttt{saturatingsmul}}
\subsubsection{\texttt{saturatingsdiv}}

\subsubsection{\texttt{move}}

Przesunięcie wartości między rejestrami.

\subsubsection{\texttt{copy}}

Skopiowanie wartości między rejestrami.

\subsubsection{\texttt{ptr}}

Konstruktor wskaźnika do wartości.

\subsubsection{\texttt{ptrlive}}

Sprawdzenie poprawności wskaźnika.

\subsubsection{\texttt{swap}}

Zamiana wartości w rejestrach.

\subsubsection{\texttt{delete}}

Wywołanie destruktora wartości.

\subsubsection{\texttt{isnull}}

\subsubsection{\texttt{print}}
\subsubsection{\texttt{echo}}

\subsubsection{\texttt{capture}}
\subsubsection{\texttt{capturecopy}}
\subsubsection{\texttt{capturemove}}

\subsubsection{\texttt{closure}}
\subsubsection{\texttt{function}}

\subsubsection{\texttt{frame}}
\subsubsection{\texttt{param}}

Instrukcja prywatna.

\subsubsection{\texttt{pamv}}

Instrukcja prywatna.

\subsubsection{\texttt{call}}
\subsubsection{\texttt{tailcall}}
\subsubsection{\texttt{defer}}
\subsubsection{\texttt{arg}}

Instrukcja prywatna.

\subsubsection{\texttt{allocate\_registers}}

\subsubsection{\texttt{process}}
\subsubsection{\texttt{self}}
\subsubsection{\texttt{pideq}}
\subsubsection{\texttt{join}}
\subsubsection{\texttt{send}}
\subsubsection{\texttt{receive}}

\subsubsection{\texttt{watchdog}}

\subsubsection{\texttt{jump}}
\subsubsection{\texttt{if}}

\subsubsection{\texttt{throw}}

Rzucenie wartości jako wyjątku.

\subsubsection{\texttt{catch}}
\subsubsection{\texttt{draw}}
\subsubsection{\texttt{try}}
\subsubsection{\texttt{enter}}
\subsubsection{\texttt{leave}}

\subsubsection{\texttt{import}}

Import modułu.

\subsubsection{\texttt{atom}}

Konstruktor atomu.

\subsubsection{\texttt{atomeq}}

\subsubsection{\texttt{struct}}

Konstruktor struktury.

\subsubsection{\texttt{structinsert}}
\subsubsection{\texttt{structremove}}
\subsubsection{\texttt{structat}}
\subsubsection{\texttt{structkeys}}

\subsubsection{\texttt{return}}

Wyjście z funkcji.

\subsubsection{\texttt{halt}}

Zakończenie działania i wyłączenie VM.


\section{Język assemblera Viua VM}
\label{appendix_viua_vm_assembly_language}

\subsection{Składnia języka assemblera}

\subsubsection{Ogólna składnia instrukcji}
\subsubsection{Definicje funkcji i bloków}
\subsubsection{Deklaracje funkcji i bloków}
\subsubsection{Import modułów}
\subsubsection{Markery}
\subsubsection{Nazywanie rejestrów}

\subsection{Instrukcje Viua VM}

\subsubsection{\texttt{nop}}

\subsubsection{\texttt{izero}}
\subsubsection{\texttt{integer}}
\subsubsection{\texttt{iinc}}
\subsubsection{\texttt{idec}}

\subsubsection{\texttt{float}}

\subsubsection{\texttt{itof}}
\subsubsection{\texttt{ftoi}}
\subsubsection{\texttt{stoi}}
\subsubsection{\texttt{stof}}

\subsubsection{\texttt{add}}
\subsubsection{\texttt{sub}}
\subsubsection{\texttt{mul}}
\subsubsection{\texttt{div}}
\subsubsection{\texttt{lt}}
\subsubsection{\texttt{lte}}
\subsubsection{\texttt{gt}}
\subsubsection{\texttt{gte}}
\subsubsection{\texttt{eq}}

\subsubsection{\texttt{string}}
\subsubsection{\texttt{streq}}

\subsubsection{\texttt{text}}
\subsubsection{\texttt{texteq}}
\subsubsection{\texttt{textat}}
\subsubsection{\texttt{textsub}}
\subsubsection{\texttt{textlength}}
\subsubsection{\texttt{textcommonprefix}}
\subsubsection{\texttt{textcommonsuffix}}
\subsubsection{\texttt{textconcat}}

\subsubsection{\texttt{vector}}
\subsubsection{\texttt{vinsert}}
\subsubsection{\texttt{vpush}}
\subsubsection{\texttt{vpop}}
\subsubsection{\texttt{vat}}
\subsubsection{\texttt{vlen}}

\subsubsection{\texttt{not}}

Negacja boolowska.

\subsubsection{\texttt{and}}

Iloczyn boolowski.

\subsubsection{\texttt{or}}

Suma boolowska.

\subsubsection{\texttt{bits\_of\_integer}}

Konstruktor bitów z liczby całkowitej.

\subsubsection{\texttt{integer\_of\_bits}}

Konstruktor liczby całkowitej z bitów.

\subsubsection{\texttt{bits}}

Konstruktor bitów.

\subsubsection{\texttt{bitand}}

Bitowa operacja \emph{\texttt{and}}.

\subsubsection{\texttt{bitor}}

Bitowa operacja \emph{\texttt{or}}.

\subsubsection{\texttt{bitnot}}

Bitowa operacja \emph{\texttt{not}}.

\subsubsection{\texttt{bitxor}}

Bitowa operacja \emph{\texttt{xor}}.

\subsubsection{\texttt{bitat}}

Sprawdzenie wartości pojedynczego bitu.

\subsubsection{\texttt{bitset}}

Ustawienie wartości pojedynczego bitu.

\subsubsection{\texttt{shl}}

Przesunięcie bitowe w lewo.

\subsubsection{\texttt{shr}}

Przesunięcie bitowe w prawo.

\subsubsection{\texttt{ashl}}

Arytmetyczne (z zachowaniem znaku) przesunięcie bitowe w lewo.

\subsubsection{\texttt{ashr}}

Arytmetyczne przesunięcie bitowe w prawo.

\subsubsection{\texttt{rol}}

Rotacja bitowa w lewo.

\subsubsection{\texttt{ror}}

Rotacja bitowa w prawo.

\subsubsection{\texttt{wrapincrement}}

Modulo inkrementacja ze znakiem.

\subsubsection{\texttt{wrapdecrement}}

Modulo dekrementacja ze znakiem.

\subsubsection{\texttt{wrapadd}}

Modulo dodawanie ze znakiem.

\subsubsection{\texttt{wrapsub}}

Modulo odejmowanie ze znakiem.

\subsubsection{\texttt{wrapmul}}

Modulo mnożenie ze znakiem.

\subsubsection{\texttt{wrapdiv}}

Modulo dzielenie ze znakiem.

\subsubsection{\texttt{checkedsincrement}}

Sprawdzana inkrementacja ze znakiem.

\subsubsection{\texttt{checkedsdecrement}}
\subsubsection{\texttt{checkedsadd}}
\subsubsection{\texttt{checkedssub}}
\subsubsection{\texttt{checkedsmul}}
\subsubsection{\texttt{checkedsdiv}}

\subsubsection{\texttt{saturatingsincrement}}

Nasycająca inkrementacja ze znakiem.

\subsubsection{\texttt{saturatingsdecrement}}
\subsubsection{\texttt{saturatingsadd}}
\subsubsection{\texttt{saturatingssub}}
\subsubsection{\texttt{saturatingsmul}}
\subsubsection{\texttt{saturatingsdiv}}

\subsubsection{\texttt{move}}

Przesunięcie wartości między rejestrami.

\subsubsection{\texttt{copy}}

Skopiowanie wartości między rejestrami.

\subsubsection{\texttt{ptr}}

Konstruktor wskaźnika do wartości.

\subsubsection{\texttt{ptrlive}}

Sprawdzenie poprawności wskaźnika.

\subsubsection{\texttt{swap}}

Zamiana wartości w rejestrach.

\subsubsection{\texttt{delete}}

Wywołanie destruktora wartości.

\subsubsection{\texttt{isnull}}

\subsubsection{\texttt{print}}
\subsubsection{\texttt{echo}}

\subsubsection{\texttt{capture}}
\subsubsection{\texttt{capturecopy}}
\subsubsection{\texttt{capturemove}}

\subsubsection{\texttt{closure}}
\subsubsection{\texttt{function}}

\subsubsection{\texttt{frame}}
\subsubsection{\texttt{param}}

Instrukcja prywatna.

\subsubsection{\texttt{pamv}}

Instrukcja prywatna.

\subsubsection{\texttt{call}}
\subsubsection{\texttt{tailcall}}
\subsubsection{\texttt{defer}}
\subsubsection{\texttt{arg}}

Instrukcja prywatna.

\subsubsection{\texttt{allocate\_registers}}

\subsubsection{\texttt{process}}
\subsubsection{\texttt{self}}
\subsubsection{\texttt{pideq}}
\subsubsection{\texttt{join}}
\subsubsection{\texttt{send}}
\subsubsection{\texttt{receive}}

\subsubsection{\texttt{watchdog}}

\subsubsection{\texttt{jump}}
\subsubsection{\texttt{if}}

\subsubsection{\texttt{throw}}

Rzucenie wartości jako wyjątku.

\subsubsection{\texttt{catch}}
\subsubsection{\texttt{draw}}
\subsubsection{\texttt{try}}
\subsubsection{\texttt{enter}}
\subsubsection{\texttt{leave}}

\subsubsection{\texttt{import}}

Import modułu.

\subsubsection{\texttt{atom}}

Konstruktor atomu.

\subsubsection{\texttt{atomeq}}

\subsubsection{\texttt{struct}}

Konstruktor struktury.

\subsubsection{\texttt{structinsert}}
\subsubsection{\texttt{structremove}}
\subsubsection{\texttt{structat}}
\subsubsection{\texttt{structkeys}}

\subsubsection{\texttt{return}}

Wyjście z funkcji.

\subsubsection{\texttt{halt}}

Zakończenie działania i wyłączenie VM.


\bibliographystyle{ieeetr}
% \bibliographystyle{apalike}
% \bibliographystyle{acm}
% \bibliographystyle{alpha}
\bibliography{bibliografia}

\part{Załączniki}

\appendix
\section{Język assemblera Viua VM}
\label{appendix_viua_vm_assembly_language}

\subsection{Składnia języka assemblera}

\subsubsection{Ogólna składnia instrukcji}
\subsubsection{Definicje funkcji i bloków}
\subsubsection{Deklaracje funkcji i bloków}
\subsubsection{Import modułów}
\subsubsection{Markery}
\subsubsection{Nazywanie rejestrów}

\subsection{Instrukcje Viua VM}

\subsubsection{\texttt{nop}}

\subsubsection{\texttt{izero}}
\subsubsection{\texttt{integer}}
\subsubsection{\texttt{iinc}}
\subsubsection{\texttt{idec}}

\subsubsection{\texttt{float}}

\subsubsection{\texttt{itof}}
\subsubsection{\texttt{ftoi}}
\subsubsection{\texttt{stoi}}
\subsubsection{\texttt{stof}}

\subsubsection{\texttt{add}}
\subsubsection{\texttt{sub}}
\subsubsection{\texttt{mul}}
\subsubsection{\texttt{div}}
\subsubsection{\texttt{lt}}
\subsubsection{\texttt{lte}}
\subsubsection{\texttt{gt}}
\subsubsection{\texttt{gte}}
\subsubsection{\texttt{eq}}

\subsubsection{\texttt{string}}
\subsubsection{\texttt{streq}}

\subsubsection{\texttt{text}}
\subsubsection{\texttt{texteq}}
\subsubsection{\texttt{textat}}
\subsubsection{\texttt{textsub}}
\subsubsection{\texttt{textlength}}
\subsubsection{\texttt{textcommonprefix}}
\subsubsection{\texttt{textcommonsuffix}}
\subsubsection{\texttt{textconcat}}

\subsubsection{\texttt{vector}}
\subsubsection{\texttt{vinsert}}
\subsubsection{\texttt{vpush}}
\subsubsection{\texttt{vpop}}
\subsubsection{\texttt{vat}}
\subsubsection{\texttt{vlen}}

\subsubsection{\texttt{not}}

Negacja boolowska.

\subsubsection{\texttt{and}}

Iloczyn boolowski.

\subsubsection{\texttt{or}}

Suma boolowska.

\subsubsection{\texttt{bits\_of\_integer}}

Konstruktor bitów z liczby całkowitej.

\subsubsection{\texttt{integer\_of\_bits}}

Konstruktor liczby całkowitej z bitów.

\subsubsection{\texttt{bits}}

Konstruktor bitów.

\subsubsection{\texttt{bitand}}

Bitowa operacja \emph{\texttt{and}}.

\subsubsection{\texttt{bitor}}

Bitowa operacja \emph{\texttt{or}}.

\subsubsection{\texttt{bitnot}}

Bitowa operacja \emph{\texttt{not}}.

\subsubsection{\texttt{bitxor}}

Bitowa operacja \emph{\texttt{xor}}.

\subsubsection{\texttt{bitat}}

Sprawdzenie wartości pojedynczego bitu.

\subsubsection{\texttt{bitset}}

Ustawienie wartości pojedynczego bitu.

\subsubsection{\texttt{shl}}

Przesunięcie bitowe w lewo.

\subsubsection{\texttt{shr}}

Przesunięcie bitowe w prawo.

\subsubsection{\texttt{ashl}}

Arytmetyczne (z zachowaniem znaku) przesunięcie bitowe w lewo.

\subsubsection{\texttt{ashr}}

Arytmetyczne przesunięcie bitowe w prawo.

\subsubsection{\texttt{rol}}

Rotacja bitowa w lewo.

\subsubsection{\texttt{ror}}

Rotacja bitowa w prawo.

\subsubsection{\texttt{wrapincrement}}

Modulo inkrementacja ze znakiem.

\subsubsection{\texttt{wrapdecrement}}

Modulo dekrementacja ze znakiem.

\subsubsection{\texttt{wrapadd}}

Modulo dodawanie ze znakiem.

\subsubsection{\texttt{wrapsub}}

Modulo odejmowanie ze znakiem.

\subsubsection{\texttt{wrapmul}}

Modulo mnożenie ze znakiem.

\subsubsection{\texttt{wrapdiv}}

Modulo dzielenie ze znakiem.

\subsubsection{\texttt{checkedsincrement}}

Sprawdzana inkrementacja ze znakiem.

\subsubsection{\texttt{checkedsdecrement}}
\subsubsection{\texttt{checkedsadd}}
\subsubsection{\texttt{checkedssub}}
\subsubsection{\texttt{checkedsmul}}
\subsubsection{\texttt{checkedsdiv}}

\subsubsection{\texttt{saturatingsincrement}}

Nasycająca inkrementacja ze znakiem.

\subsubsection{\texttt{saturatingsdecrement}}
\subsubsection{\texttt{saturatingsadd}}
\subsubsection{\texttt{saturatingssub}}
\subsubsection{\texttt{saturatingsmul}}
\subsubsection{\texttt{saturatingsdiv}}

\subsubsection{\texttt{move}}

Przesunięcie wartości między rejestrami.

\subsubsection{\texttt{copy}}

Skopiowanie wartości między rejestrami.

\subsubsection{\texttt{ptr}}

Konstruktor wskaźnika do wartości.

\subsubsection{\texttt{ptrlive}}

Sprawdzenie poprawności wskaźnika.

\subsubsection{\texttt{swap}}

Zamiana wartości w rejestrach.

\subsubsection{\texttt{delete}}

Wywołanie destruktora wartości.

\subsubsection{\texttt{isnull}}

\subsubsection{\texttt{print}}
\subsubsection{\texttt{echo}}

\subsubsection{\texttt{capture}}
\subsubsection{\texttt{capturecopy}}
\subsubsection{\texttt{capturemove}}

\subsubsection{\texttt{closure}}
\subsubsection{\texttt{function}}

\subsubsection{\texttt{frame}}
\subsubsection{\texttt{param}}

Instrukcja prywatna.

\subsubsection{\texttt{pamv}}

Instrukcja prywatna.

\subsubsection{\texttt{call}}
\subsubsection{\texttt{tailcall}}
\subsubsection{\texttt{defer}}
\subsubsection{\texttt{arg}}

Instrukcja prywatna.

\subsubsection{\texttt{allocate\_registers}}

\subsubsection{\texttt{process}}
\subsubsection{\texttt{self}}
\subsubsection{\texttt{pideq}}
\subsubsection{\texttt{join}}
\subsubsection{\texttt{send}}
\subsubsection{\texttt{receive}}

\subsubsection{\texttt{watchdog}}

\subsubsection{\texttt{jump}}
\subsubsection{\texttt{if}}

\subsubsection{\texttt{throw}}

Rzucenie wartości jako wyjątku.

\subsubsection{\texttt{catch}}
\subsubsection{\texttt{draw}}
\subsubsection{\texttt{try}}
\subsubsection{\texttt{enter}}
\subsubsection{\texttt{leave}}

\subsubsection{\texttt{import}}

Import modułu.

\subsubsection{\texttt{atom}}

Konstruktor atomu.

\subsubsection{\texttt{atomeq}}

\subsubsection{\texttt{struct}}

Konstruktor struktury.

\subsubsection{\texttt{structinsert}}
\subsubsection{\texttt{structremove}}
\subsubsection{\texttt{structat}}
\subsubsection{\texttt{structkeys}}

\subsubsection{\texttt{return}}

Wyjście z funkcji.

\subsubsection{\texttt{halt}}

Zakończenie działania i wyłączenie VM.

\section{Język assemblera Viua VM}
\label{appendix_viua_vm_assembly_language}

\subsection{Składnia języka assemblera}

\subsubsection{Ogólna składnia instrukcji}
\subsubsection{Definicje funkcji i bloków}
\subsubsection{Deklaracje funkcji i bloków}
\subsubsection{Import modułów}
\subsubsection{Markery}
\subsubsection{Nazywanie rejestrów}

\subsection{Instrukcje Viua VM}

\subsubsection{\texttt{nop}}

\subsubsection{\texttt{izero}}
\subsubsection{\texttt{integer}}
\subsubsection{\texttt{iinc}}
\subsubsection{\texttt{idec}}

\subsubsection{\texttt{float}}

\subsubsection{\texttt{itof}}
\subsubsection{\texttt{ftoi}}
\subsubsection{\texttt{stoi}}
\subsubsection{\texttt{stof}}

\subsubsection{\texttt{add}}
\subsubsection{\texttt{sub}}
\subsubsection{\texttt{mul}}
\subsubsection{\texttt{div}}
\subsubsection{\texttt{lt}}
\subsubsection{\texttt{lte}}
\subsubsection{\texttt{gt}}
\subsubsection{\texttt{gte}}
\subsubsection{\texttt{eq}}

\subsubsection{\texttt{string}}
\subsubsection{\texttt{streq}}

\subsubsection{\texttt{text}}
\subsubsection{\texttt{texteq}}
\subsubsection{\texttt{textat}}
\subsubsection{\texttt{textsub}}
\subsubsection{\texttt{textlength}}
\subsubsection{\texttt{textcommonprefix}}
\subsubsection{\texttt{textcommonsuffix}}
\subsubsection{\texttt{textconcat}}

\subsubsection{\texttt{vector}}
\subsubsection{\texttt{vinsert}}
\subsubsection{\texttt{vpush}}
\subsubsection{\texttt{vpop}}
\subsubsection{\texttt{vat}}
\subsubsection{\texttt{vlen}}

\subsubsection{\texttt{not}}

Negacja boolowska.

\subsubsection{\texttt{and}}

Iloczyn boolowski.

\subsubsection{\texttt{or}}

Suma boolowska.

\subsubsection{\texttt{bits\_of\_integer}}

Konstruktor bitów z liczby całkowitej.

\subsubsection{\texttt{integer\_of\_bits}}

Konstruktor liczby całkowitej z bitów.

\subsubsection{\texttt{bits}}

Konstruktor bitów.

\subsubsection{\texttt{bitand}}

Bitowa operacja \emph{\texttt{and}}.

\subsubsection{\texttt{bitor}}

Bitowa operacja \emph{\texttt{or}}.

\subsubsection{\texttt{bitnot}}

Bitowa operacja \emph{\texttt{not}}.

\subsubsection{\texttt{bitxor}}

Bitowa operacja \emph{\texttt{xor}}.

\subsubsection{\texttt{bitat}}

Sprawdzenie wartości pojedynczego bitu.

\subsubsection{\texttt{bitset}}

Ustawienie wartości pojedynczego bitu.

\subsubsection{\texttt{shl}}

Przesunięcie bitowe w lewo.

\subsubsection{\texttt{shr}}

Przesunięcie bitowe w prawo.

\subsubsection{\texttt{ashl}}

Arytmetyczne (z zachowaniem znaku) przesunięcie bitowe w lewo.

\subsubsection{\texttt{ashr}}

Arytmetyczne przesunięcie bitowe w prawo.

\subsubsection{\texttt{rol}}

Rotacja bitowa w lewo.

\subsubsection{\texttt{ror}}

Rotacja bitowa w prawo.

\subsubsection{\texttt{wrapincrement}}

Modulo inkrementacja ze znakiem.

\subsubsection{\texttt{wrapdecrement}}

Modulo dekrementacja ze znakiem.

\subsubsection{\texttt{wrapadd}}

Modulo dodawanie ze znakiem.

\subsubsection{\texttt{wrapsub}}

Modulo odejmowanie ze znakiem.

\subsubsection{\texttt{wrapmul}}

Modulo mnożenie ze znakiem.

\subsubsection{\texttt{wrapdiv}}

Modulo dzielenie ze znakiem.

\subsubsection{\texttt{checkedsincrement}}

Sprawdzana inkrementacja ze znakiem.

\subsubsection{\texttt{checkedsdecrement}}
\subsubsection{\texttt{checkedsadd}}
\subsubsection{\texttt{checkedssub}}
\subsubsection{\texttt{checkedsmul}}
\subsubsection{\texttt{checkedsdiv}}

\subsubsection{\texttt{saturatingsincrement}}

Nasycająca inkrementacja ze znakiem.

\subsubsection{\texttt{saturatingsdecrement}}
\subsubsection{\texttt{saturatingsadd}}
\subsubsection{\texttt{saturatingssub}}
\subsubsection{\texttt{saturatingsmul}}
\subsubsection{\texttt{saturatingsdiv}}

\subsubsection{\texttt{move}}

Przesunięcie wartości między rejestrami.

\subsubsection{\texttt{copy}}

Skopiowanie wartości między rejestrami.

\subsubsection{\texttt{ptr}}

Konstruktor wskaźnika do wartości.

\subsubsection{\texttt{ptrlive}}

Sprawdzenie poprawności wskaźnika.

\subsubsection{\texttt{swap}}

Zamiana wartości w rejestrach.

\subsubsection{\texttt{delete}}

Wywołanie destruktora wartości.

\subsubsection{\texttt{isnull}}

\subsubsection{\texttt{print}}
\subsubsection{\texttt{echo}}

\subsubsection{\texttt{capture}}
\subsubsection{\texttt{capturecopy}}
\subsubsection{\texttt{capturemove}}

\subsubsection{\texttt{closure}}
\subsubsection{\texttt{function}}

\subsubsection{\texttt{frame}}
\subsubsection{\texttt{param}}

Instrukcja prywatna.

\subsubsection{\texttt{pamv}}

Instrukcja prywatna.

\subsubsection{\texttt{call}}
\subsubsection{\texttt{tailcall}}
\subsubsection{\texttt{defer}}
\subsubsection{\texttt{arg}}

Instrukcja prywatna.

\subsubsection{\texttt{allocate\_registers}}

\subsubsection{\texttt{process}}
\subsubsection{\texttt{self}}
\subsubsection{\texttt{pideq}}
\subsubsection{\texttt{join}}
\subsubsection{\texttt{send}}
\subsubsection{\texttt{receive}}

\subsubsection{\texttt{watchdog}}

\subsubsection{\texttt{jump}}
\subsubsection{\texttt{if}}

\subsubsection{\texttt{throw}}

Rzucenie wartości jako wyjątku.

\subsubsection{\texttt{catch}}
\subsubsection{\texttt{draw}}
\subsubsection{\texttt{try}}
\subsubsection{\texttt{enter}}
\subsubsection{\texttt{leave}}

\subsubsection{\texttt{import}}

Import modułu.

\subsubsection{\texttt{atom}}

Konstruktor atomu.

\subsubsection{\texttt{atomeq}}

\subsubsection{\texttt{struct}}

Konstruktor struktury.

\subsubsection{\texttt{structinsert}}
\subsubsection{\texttt{structremove}}
\subsubsection{\texttt{structat}}
\subsubsection{\texttt{structkeys}}

\subsubsection{\texttt{return}}

Wyjście z funkcji.

\subsubsection{\texttt{halt}}

Zakończenie działania i wyłączenie VM.

\section{Język assemblera Viua VM}
\label{appendix_viua_vm_assembly_language}

\subsection{Składnia języka assemblera}

\subsubsection{Ogólna składnia instrukcji}
\subsubsection{Definicje funkcji i bloków}
\subsubsection{Deklaracje funkcji i bloków}
\subsubsection{Import modułów}
\subsubsection{Markery}
\subsubsection{Nazywanie rejestrów}

\subsection{Instrukcje Viua VM}

\subsubsection{\texttt{nop}}

\subsubsection{\texttt{izero}}
\subsubsection{\texttt{integer}}
\subsubsection{\texttt{iinc}}
\subsubsection{\texttt{idec}}

\subsubsection{\texttt{float}}

\subsubsection{\texttt{itof}}
\subsubsection{\texttt{ftoi}}
\subsubsection{\texttt{stoi}}
\subsubsection{\texttt{stof}}

\subsubsection{\texttt{add}}
\subsubsection{\texttt{sub}}
\subsubsection{\texttt{mul}}
\subsubsection{\texttt{div}}
\subsubsection{\texttt{lt}}
\subsubsection{\texttt{lte}}
\subsubsection{\texttt{gt}}
\subsubsection{\texttt{gte}}
\subsubsection{\texttt{eq}}

\subsubsection{\texttt{string}}
\subsubsection{\texttt{streq}}

\subsubsection{\texttt{text}}
\subsubsection{\texttt{texteq}}
\subsubsection{\texttt{textat}}
\subsubsection{\texttt{textsub}}
\subsubsection{\texttt{textlength}}
\subsubsection{\texttt{textcommonprefix}}
\subsubsection{\texttt{textcommonsuffix}}
\subsubsection{\texttt{textconcat}}

\subsubsection{\texttt{vector}}
\subsubsection{\texttt{vinsert}}
\subsubsection{\texttt{vpush}}
\subsubsection{\texttt{vpop}}
\subsubsection{\texttt{vat}}
\subsubsection{\texttt{vlen}}

\subsubsection{\texttt{not}}

Negacja boolowska.

\subsubsection{\texttt{and}}

Iloczyn boolowski.

\subsubsection{\texttt{or}}

Suma boolowska.

\subsubsection{\texttt{bits\_of\_integer}}

Konstruktor bitów z liczby całkowitej.

\subsubsection{\texttt{integer\_of\_bits}}

Konstruktor liczby całkowitej z bitów.

\subsubsection{\texttt{bits}}

Konstruktor bitów.

\subsubsection{\texttt{bitand}}

Bitowa operacja \emph{\texttt{and}}.

\subsubsection{\texttt{bitor}}

Bitowa operacja \emph{\texttt{or}}.

\subsubsection{\texttt{bitnot}}

Bitowa operacja \emph{\texttt{not}}.

\subsubsection{\texttt{bitxor}}

Bitowa operacja \emph{\texttt{xor}}.

\subsubsection{\texttt{bitat}}

Sprawdzenie wartości pojedynczego bitu.

\subsubsection{\texttt{bitset}}

Ustawienie wartości pojedynczego bitu.

\subsubsection{\texttt{shl}}

Przesunięcie bitowe w lewo.

\subsubsection{\texttt{shr}}

Przesunięcie bitowe w prawo.

\subsubsection{\texttt{ashl}}

Arytmetyczne (z zachowaniem znaku) przesunięcie bitowe w lewo.

\subsubsection{\texttt{ashr}}

Arytmetyczne przesunięcie bitowe w prawo.

\subsubsection{\texttt{rol}}

Rotacja bitowa w lewo.

\subsubsection{\texttt{ror}}

Rotacja bitowa w prawo.

\subsubsection{\texttt{wrapincrement}}

Modulo inkrementacja ze znakiem.

\subsubsection{\texttt{wrapdecrement}}

Modulo dekrementacja ze znakiem.

\subsubsection{\texttt{wrapadd}}

Modulo dodawanie ze znakiem.

\subsubsection{\texttt{wrapsub}}

Modulo odejmowanie ze znakiem.

\subsubsection{\texttt{wrapmul}}

Modulo mnożenie ze znakiem.

\subsubsection{\texttt{wrapdiv}}

Modulo dzielenie ze znakiem.

\subsubsection{\texttt{checkedsincrement}}

Sprawdzana inkrementacja ze znakiem.

\subsubsection{\texttt{checkedsdecrement}}
\subsubsection{\texttt{checkedsadd}}
\subsubsection{\texttt{checkedssub}}
\subsubsection{\texttt{checkedsmul}}
\subsubsection{\texttt{checkedsdiv}}

\subsubsection{\texttt{saturatingsincrement}}

Nasycająca inkrementacja ze znakiem.

\subsubsection{\texttt{saturatingsdecrement}}
\subsubsection{\texttt{saturatingsadd}}
\subsubsection{\texttt{saturatingssub}}
\subsubsection{\texttt{saturatingsmul}}
\subsubsection{\texttt{saturatingsdiv}}

\subsubsection{\texttt{move}}

Przesunięcie wartości między rejestrami.

\subsubsection{\texttt{copy}}

Skopiowanie wartości między rejestrami.

\subsubsection{\texttt{ptr}}

Konstruktor wskaźnika do wartości.

\subsubsection{\texttt{ptrlive}}

Sprawdzenie poprawności wskaźnika.

\subsubsection{\texttt{swap}}

Zamiana wartości w rejestrach.

\subsubsection{\texttt{delete}}

Wywołanie destruktora wartości.

\subsubsection{\texttt{isnull}}

\subsubsection{\texttt{print}}
\subsubsection{\texttt{echo}}

\subsubsection{\texttt{capture}}
\subsubsection{\texttt{capturecopy}}
\subsubsection{\texttt{capturemove}}

\subsubsection{\texttt{closure}}
\subsubsection{\texttt{function}}

\subsubsection{\texttt{frame}}
\subsubsection{\texttt{param}}

Instrukcja prywatna.

\subsubsection{\texttt{pamv}}

Instrukcja prywatna.

\subsubsection{\texttt{call}}
\subsubsection{\texttt{tailcall}}
\subsubsection{\texttt{defer}}
\subsubsection{\texttt{arg}}

Instrukcja prywatna.

\subsubsection{\texttt{allocate\_registers}}

\subsubsection{\texttt{process}}
\subsubsection{\texttt{self}}
\subsubsection{\texttt{pideq}}
\subsubsection{\texttt{join}}
\subsubsection{\texttt{send}}
\subsubsection{\texttt{receive}}

\subsubsection{\texttt{watchdog}}

\subsubsection{\texttt{jump}}
\subsubsection{\texttt{if}}

\subsubsection{\texttt{throw}}

Rzucenie wartości jako wyjątku.

\subsubsection{\texttt{catch}}
\subsubsection{\texttt{draw}}
\subsubsection{\texttt{try}}
\subsubsection{\texttt{enter}}
\subsubsection{\texttt{leave}}

\subsubsection{\texttt{import}}

Import modułu.

\subsubsection{\texttt{atom}}

Konstruktor atomu.

\subsubsection{\texttt{atomeq}}

\subsubsection{\texttt{struct}}

Konstruktor struktury.

\subsubsection{\texttt{structinsert}}
\subsubsection{\texttt{structremove}}
\subsubsection{\texttt{structat}}
\subsubsection{\texttt{structkeys}}

\subsubsection{\texttt{return}}

Wyjście z funkcji.

\subsubsection{\texttt{halt}}

Zakończenie działania i wyłączenie VM.

\section{Język assemblera Viua VM}
\label{appendix_viua_vm_assembly_language}

\subsection{Składnia języka assemblera}

\subsubsection{Ogólna składnia instrukcji}
\subsubsection{Definicje funkcji i bloków}
\subsubsection{Deklaracje funkcji i bloków}
\subsubsection{Import modułów}
\subsubsection{Markery}
\subsubsection{Nazywanie rejestrów}

\subsection{Instrukcje Viua VM}

\subsubsection{\texttt{nop}}

\subsubsection{\texttt{izero}}
\subsubsection{\texttt{integer}}
\subsubsection{\texttt{iinc}}
\subsubsection{\texttt{idec}}

\subsubsection{\texttt{float}}

\subsubsection{\texttt{itof}}
\subsubsection{\texttt{ftoi}}
\subsubsection{\texttt{stoi}}
\subsubsection{\texttt{stof}}

\subsubsection{\texttt{add}}
\subsubsection{\texttt{sub}}
\subsubsection{\texttt{mul}}
\subsubsection{\texttt{div}}
\subsubsection{\texttt{lt}}
\subsubsection{\texttt{lte}}
\subsubsection{\texttt{gt}}
\subsubsection{\texttt{gte}}
\subsubsection{\texttt{eq}}

\subsubsection{\texttt{string}}
\subsubsection{\texttt{streq}}

\subsubsection{\texttt{text}}
\subsubsection{\texttt{texteq}}
\subsubsection{\texttt{textat}}
\subsubsection{\texttt{textsub}}
\subsubsection{\texttt{textlength}}
\subsubsection{\texttt{textcommonprefix}}
\subsubsection{\texttt{textcommonsuffix}}
\subsubsection{\texttt{textconcat}}

\subsubsection{\texttt{vector}}
\subsubsection{\texttt{vinsert}}
\subsubsection{\texttt{vpush}}
\subsubsection{\texttt{vpop}}
\subsubsection{\texttt{vat}}
\subsubsection{\texttt{vlen}}

\subsubsection{\texttt{not}}

Negacja boolowska.

\subsubsection{\texttt{and}}

Iloczyn boolowski.

\subsubsection{\texttt{or}}

Suma boolowska.

\subsubsection{\texttt{bits\_of\_integer}}

Konstruktor bitów z liczby całkowitej.

\subsubsection{\texttt{integer\_of\_bits}}

Konstruktor liczby całkowitej z bitów.

\subsubsection{\texttt{bits}}

Konstruktor bitów.

\subsubsection{\texttt{bitand}}

Bitowa operacja \emph{\texttt{and}}.

\subsubsection{\texttt{bitor}}

Bitowa operacja \emph{\texttt{or}}.

\subsubsection{\texttt{bitnot}}

Bitowa operacja \emph{\texttt{not}}.

\subsubsection{\texttt{bitxor}}

Bitowa operacja \emph{\texttt{xor}}.

\subsubsection{\texttt{bitat}}

Sprawdzenie wartości pojedynczego bitu.

\subsubsection{\texttt{bitset}}

Ustawienie wartości pojedynczego bitu.

\subsubsection{\texttt{shl}}

Przesunięcie bitowe w lewo.

\subsubsection{\texttt{shr}}

Przesunięcie bitowe w prawo.

\subsubsection{\texttt{ashl}}

Arytmetyczne (z zachowaniem znaku) przesunięcie bitowe w lewo.

\subsubsection{\texttt{ashr}}

Arytmetyczne przesunięcie bitowe w prawo.

\subsubsection{\texttt{rol}}

Rotacja bitowa w lewo.

\subsubsection{\texttt{ror}}

Rotacja bitowa w prawo.

\subsubsection{\texttt{wrapincrement}}

Modulo inkrementacja ze znakiem.

\subsubsection{\texttt{wrapdecrement}}

Modulo dekrementacja ze znakiem.

\subsubsection{\texttt{wrapadd}}

Modulo dodawanie ze znakiem.

\subsubsection{\texttt{wrapsub}}

Modulo odejmowanie ze znakiem.

\subsubsection{\texttt{wrapmul}}

Modulo mnożenie ze znakiem.

\subsubsection{\texttt{wrapdiv}}

Modulo dzielenie ze znakiem.

\subsubsection{\texttt{checkedsincrement}}

Sprawdzana inkrementacja ze znakiem.

\subsubsection{\texttt{checkedsdecrement}}
\subsubsection{\texttt{checkedsadd}}
\subsubsection{\texttt{checkedssub}}
\subsubsection{\texttt{checkedsmul}}
\subsubsection{\texttt{checkedsdiv}}

\subsubsection{\texttt{saturatingsincrement}}

Nasycająca inkrementacja ze znakiem.

\subsubsection{\texttt{saturatingsdecrement}}
\subsubsection{\texttt{saturatingsadd}}
\subsubsection{\texttt{saturatingssub}}
\subsubsection{\texttt{saturatingsmul}}
\subsubsection{\texttt{saturatingsdiv}}

\subsubsection{\texttt{move}}

Przesunięcie wartości między rejestrami.

\subsubsection{\texttt{copy}}

Skopiowanie wartości między rejestrami.

\subsubsection{\texttt{ptr}}

Konstruktor wskaźnika do wartości.

\subsubsection{\texttt{ptrlive}}

Sprawdzenie poprawności wskaźnika.

\subsubsection{\texttt{swap}}

Zamiana wartości w rejestrach.

\subsubsection{\texttt{delete}}

Wywołanie destruktora wartości.

\subsubsection{\texttt{isnull}}

\subsubsection{\texttt{print}}
\subsubsection{\texttt{echo}}

\subsubsection{\texttt{capture}}
\subsubsection{\texttt{capturecopy}}
\subsubsection{\texttt{capturemove}}

\subsubsection{\texttt{closure}}
\subsubsection{\texttt{function}}

\subsubsection{\texttt{frame}}
\subsubsection{\texttt{param}}

Instrukcja prywatna.

\subsubsection{\texttt{pamv}}

Instrukcja prywatna.

\subsubsection{\texttt{call}}
\subsubsection{\texttt{tailcall}}
\subsubsection{\texttt{defer}}
\subsubsection{\texttt{arg}}

Instrukcja prywatna.

\subsubsection{\texttt{allocate\_registers}}

\subsubsection{\texttt{process}}
\subsubsection{\texttt{self}}
\subsubsection{\texttt{pideq}}
\subsubsection{\texttt{join}}
\subsubsection{\texttt{send}}
\subsubsection{\texttt{receive}}

\subsubsection{\texttt{watchdog}}

\subsubsection{\texttt{jump}}
\subsubsection{\texttt{if}}

\subsubsection{\texttt{throw}}

Rzucenie wartości jako wyjątku.

\subsubsection{\texttt{catch}}
\subsubsection{\texttt{draw}}
\subsubsection{\texttt{try}}
\subsubsection{\texttt{enter}}
\subsubsection{\texttt{leave}}

\subsubsection{\texttt{import}}

Import modułu.

\subsubsection{\texttt{atom}}

Konstruktor atomu.

\subsubsection{\texttt{atomeq}}

\subsubsection{\texttt{struct}}

Konstruktor struktury.

\subsubsection{\texttt{structinsert}}
\subsubsection{\texttt{structremove}}
\subsubsection{\texttt{structat}}
\subsubsection{\texttt{structkeys}}

\subsubsection{\texttt{return}}

Wyjście z funkcji.

\subsubsection{\texttt{halt}}

Zakończenie działania i wyłączenie VM.


\end{document}
