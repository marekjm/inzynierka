\documentclass[11pt,oneside,a4paper,titlepage,onecolumn]{book}

\usepackage[utf8]{inputenc}
\usepackage{textcomp}
\usepackage[official]{eurosym}
\usepackage[polish]{babel}
\usepackage{amsthm}
\usepackage{graphicx}
\usepackage[T1]{fontenc}
\usepackage{scrextend}
\usepackage{hyperref}
\usepackage{xcolor}
\usepackage{enumitem}
% \usepackage{nameref}
% \usepackage{showlabels}
% \usepackage{titlesec}
\usepackage{geometry}
\geometry{a4paper, portrait, margin=2cm}
\graphicspath{ {./fig/} }
\usepackage{listings}

\newenvironment{enumreq}
{ \begin{enumerate}[topsep=0pt,itemsep=-1ex,partopsep=1ex,parsep=1ex] }
{ \end{enumerate}                  } 

\newcommand{\inzmaintitlePL}{Viua VM w akcji}

\setcounter{secnumdepth}{4}

%% Author and title
% \author{Marek Marecki \and Gr.52c \and Krzysztof Franek}
\author{Zespół: Marek Marecki i Krzysztof Franek\\Promotor: dr hab. Marek A. Bednarczyk, prof. PJWSTK}
\title{%
    \inzmaintitlePL \\
    \large
    Implementacja języka wysokiego poziomu i \\
    realizacja prostej aplikacji\\
    ~\\
    Viua VM in action. Implementation of a high-level programming language and\\ a simple application}

\begin{document}

\lstset{basicstyle=\ttfamily,
columns=fixed,
escapeinside={\%*}{*)},
inputencoding=utf8,
extendedchars=true}

\maketitle

\tableofcontents
% \listoftables
\listoffigures

Praca została złożona za pomocą systemu \LaTeX.

\newpage

\chapter{Wprowadzenie}

We wprowadzeniu chcielibyśmy poruszyć ogólne tematy związane z naszą pracą, oraz przedstawić Czytelnikowi
spójny wstęp zapewniający solidne podstawy do dalszej lektury. Zakładamy, że Czytelnik ma doświadczenie z
językami programowania i umie biegle posługiwać się co najmniej jednym ''\emph{mainstreamowym}'' językiem, np.
C++, Java, czy Python.

Wprowadzenie jest dla nas miejscem, w którym przedstawimy problemy dręczące popularne obecnie języki
programowania oraz powiemy skąd się te problemy biorą i dlaczego w miarę postępu technologii są one coraz
trudniejsze do zignorowania.

Po krótce przedstawiona zostanie również maszyna wirtualna Viua, która jest platformą, na której nasza praca
się opiera.

\section{Układ pracy inżynierskiej}

W rozdziale \ref{jezyk_viuact_i_jego_kompilator}.~\nameref{jezyk_viuact_i_jego_kompilator} (na stronie
\pageref{jezyk_viuact_i_jego_kompilator}) przedstawiamy język Viuact oraz jego implementację - kompilator.
Są to narzędzia wykorzystywane do stworzenia aplikacji użytkowej przedstawionej w rozdziale
\ref{program_viuachat}.~\nameref{program_viuachat} (na stronie \pageref{program_viuachat}).

Wkład własny członków zespołu przedstawiony jest w rozdziale \ref{wklad_wlasny_czlonkow_zespolu} na stronie
\pageref{wklad_wlasny_czlonkow_zespolu}.

W rozdziale \ref{slownik_pojec}.~\nameref{slownik_pojec} na stronie \pageref{slownik_pojec} prezentujemy listę
pojęć i terminów używanych w pracy, które mogą wymagać doprecyzowania lub definicji. Dla wygody Czytelnika są
one umieszczone w jednym miejscu, oraz podzielone na sekcje związane z językiem i kompilatorem
(rozdział~\ref{slownik_pojec_jezyka}) i związane z chatem (rozdział~\ref{slownik_pojec_chatu}).

\section{Przedstawienie problemu}

Tytułem pracy jest ,,\inzmaintitlePL''. Problemem, który poruszamy jest sprawdzenie czy Viua VM w stanie
,,zastanym'' (tj. w takim w jakim znajdowała się jej implementacja na początku prac nad projektem
inżynierskim) umożliwia pisanie niezawodnego, wysoce współbieżnego, nietrywialnego oprogramowania.

W naszej pracy prezentujemy język programowania, który z założenia ma pozwalać na tworzenie oprogramowania
niezawodnego i wykorzystującego potencjał współbieżności w stopniu wyższym niż powszechnie używane,
,,mainstreamowe'' języki programowania.

Aby udowodnić, że w wytworzonym języku możliwe jest tworzenie nietrywialnego oprogramowania prezenetujemy
aplikację użytkową -- czat -- napisany w tym języku. Czat umożliwi komunikację (zorganizowaną kanałach) wielu
użytkownikom naraz. Wybór rodzaju aplikacji (czat) jest warunkowany tym, że oprogramowanie komunikacyjne
powinno charakteryzować się tymi cechami, które chcemy uzyskać:

\begin{enumerate}
    \item niezawodnością -- jeśli jedno połączenie ulegnie awarii to pozostałe powinny działać dalej
    \item izolacją procesów -- każde połączenie powinno być izolowane od wszystkich innych
    \item współbieżnością -- wiele połączeń musi być obsługiwanych w tym samym czasie
\end{enumerate}

\subsection{Ogląd sytuacji}

Współczesny \emph{hardware} zmierza coraz bardziej w stronę współbieżności oraz przetwarzania równoległego.
Firma AMD w roku 2018 wprowadziła na rynek konsumencki procesory wielordzeniowe z serii
Threadripper\footnote{https://www.amd.com/en/products/ryzen-threadripper} prezentujące nawet 32 rdzenie logiczne.

Współczesny \emph{software} stoi w miejscu. Poza oprogramowaniem specjalistycznym (np.
Blender\footnote{https://www.blender.org/}) mało który program jest w stanie wykorzystać więcej niż kilka wątków.
Współbieżności szuka się nieco ''na siłę''; np. przeglądarki internetowe uruchamiają współbieżnie obsługę
wielu kart.

Mainstreamowe języki w większości korzystają z wątków (np. Java, C++) lub są stricte jednowątkowe i
niedostosowane do przetwarzania współbieżnego i równoległego (np. Python).

Dla takich języków tworzone są biblioteki ułatwiające wykorzystanie mechanizmów systemu operacyjnego
(np.~wieloprocesowość dla języka Python) lub cała infrastruktura umożliwiająca rozproszenie pracy,
np.~sewery pracy (np. Gearman, Celery) czy kolejki wiadomości (np. RabbitMQ, ZeroMQ).
To wszystko są jednak jedynie ''łatki'' mające za zadanie dodać do istniejących języków mechanizmy, z którymi
pierwotnie nie były one projektowane. Istnieje dla takiego działania angielski termin bardzo dobrze oddający
jego postać: \emph{retrofitting}.

Istnieją środowiska i języki zaprojektowane od zera z myślą o współbieżności i przetwarzaniu równoległym, a
nawet rozproszonym.
Najbardziej znanym przykładem takiego środowiska jest BEAM -- maszyna wirtualna
języka Erlang\footnote{http://www.erlang.org/}. To środowisko wraz z językiem jest z powodzeniem
wykorzystywane w sprzęcie telekomunikacyjnym firmy Ericsson, oraz do tworzenia aplikacji użytkowych, których
działanie obejmuje niemal z definicji działanie rozproszone i współbieżne, na przykład w serwerach
komunikatora Discord\footnote{https://discordapp.com/}.

\section{Cel}

Ta praca inżynierska motywowana jest chęcią stworzenia bazy programistycznej (środowiska uruchomieniowego i
języka programowania) umożliwiającej programistom pisanie oprogramowania od początku uwzględniającego
przetwarzanie współbieżne na pierwszym miejscu, oraz charakteryzującego się wysokim poziomem niezawodności i
stabliności działania.

Osiągniemy to dzięki zbudowaniu języka wyposażonego w łatwo dostępne konstrukcje umożliwiające wprowadzenie
współbieżności do programu, oraz uruchamianiu programów wynikowych w środowisku zdolnym do rozłożenia pracy na
całość dostępnych zasobów sprzętowych.

\subsection{Problemy i ryzyko}

Postawienie współbieżności na pierwszym miejscu powoduje, że problemy związane z pisaniem poprawnych programów
są bardziej liczne niż w innych (niewspółbieżnych) językach -- oprócz zapewnienia poprawności pojedynczego
procesu, programista musi zadbać o poprawność interakcji między procesami, oraz o stabilność działania
oprogramowania w momencie awarii któregoś z procesów składowych programu.

Zminimalizujemy ryzyko płynące z wprowadzenia współbieżności do warsztatu programistów udostępniając im
mechanizmy umożliwiające opanowanie awarii, propagowanie informacji o błędach, oraz izolację poszczególnych
procesów składowych.

\chapter{Strategia prowadzenia prac}

W tym rozdziale opisujemy metodologię prac nad poszczególnymi częściami projektu. Każda z nich była rozwijana
innym sposobem z dużych rozbieżności specyfiki konkretnych problemu -- zupełnie inaczej pracuje się nad
semi-formalną specyfikacją języka programowania niż nad aplikacją użytkową, więc podjęliśmy decyzję o
rozdzieleniu tych części i poprowadzeniu ich jako osobne, mniejsze projekty wewnątrz pracy inżynierskiej.

\section{Język ViuAct}

% Specyfikacja. Analiza wymagań. Formalizm. Kaskada.

Tworzenie specyfikacji języka odbywa się w modelu kaskadowo-prototypowym i przeplata się z implementacją
prototypu kompilatora.

\subsection{Określenie wymagań}

Przed rozpoczęciem specyfikowania poszczególnych konstrukcji językowych należało podjąć decyzję co w języku
powinno się znaleźć, a co nie powinno zostać włączone do specyfikacji.

\subsection{Specyfikacja konkretnej konstrukcji języka}

Każda konstrukcja językowa jest najpierw projektowana, następnie dokumentowana i omawiana (żeby wychwycić
pewne oczywiste błędy ,,grube'' na wczesnym etapie prac). Potem proces specyfikacji zostaje zawieszony do
momentu wytworzenia w kompilatorze prototypu danej konstrukcji, lub odrzucenia jej jako niewykonalnej w
zakładanym czasie lub zakresie funkcjonalności.

Wynik prac prototypowych (sukces bądź porażka) decyduje o tym czy prace specyfikacyjne są prowadzone dalej.
Jeśli konstrukcja jest możliwa do zaimplementowania jej specyfikacja jest formalizowana oraz przygotowana dla
niej zostaje dokumentacja (obejmująca m.in. przykłady użycia).

\section{Kompilator języka ViuAct}

Issue tracking, open-source. Prototypowanie. Iteracja.
Prace przeplatają się z pracami nad specyfikacją celem walidacji co jest możliwe.

\subsection{Testowanie}

Zestaw prostych programów testowych. Test sprawdza czy program się kompiluje i czy po uruchomieniu daje
spodziewane wyniki.

\section{Program ViuaChat}

Mini-SCRUM.

\subsection{Testowanie}

Jak wygląda testowanie czatu.

\chapter{Dokumentacja prac}

\section{Specyfikacja języka ViuAct}

Tworzenie formalnej specyfikacji języka ViuAct było, paradoksalnie, procesem mało sformalizowanym. Polegało
na określeniu głównych założeń języka (opisanych w rozdziale
\ref{specyfikacja_jezyka_viuact_model_programowania}
\nameref{specyfikacja_jezyka_viuact_model_programowania}).

\section{Implementacja kompilatora}

Issue tracking. Zapis issue, daty rozpoczęcia i zakończenia, branche z Git. Trochę słowno-muzycznego opisu co
się działo podczas prac.

\section{Implementacja ViuaChat}

Opis prac nad implementacją ViuaChat.

\chapter{Język ViuAct i jego kompilator -- formalności}

W tym rozdziale prezentujemy wszelkie ''formalności'' związane z pierwszą częścią naszej pracy, czyli
dokumenty takie jak DZW\footnote{Dokument Założeń Wstępnych} czy SWS\footnote{Spis Wymagań Systemowych}.
Opis projektu języka i jego implementacji prezentujemy w rozdziale \nameref{jezyk_viuact_i_jego_kompilator} na
stronie~\pageref{jezyk_viuact_i_jego_kompilator}.

\section{Wprowadzenie}

W poniższym rozdziale zdefiniowano wymagania dla czatu ViuaChat. Ich opracowanie nastąpiło na podstawie analizy otoczenia aplikacji oraz analizy potrzeb projektu w stosunku do niej. W ramach tego procesu nastąpiły:
\begin{itemize}
    \item analiza otoczenia, wraz z z klientami;
    \item wskazanie kontekstu biznesowego systemu;
    \item określenie udziałowców;
	\item wyszczególnienie i uporządkowanie zasad biznesowych, jakie zostały założone w stosunku do aplikacji;
	\item opracowanie historyjek na podstawie ustalonych zasad biznesowych.
\end{itemize}

\textbf{Uwaga:} Niniejszy rozdział nie dotyczy języka ViuAct ani jego kompilatora.

\subsection{Odbiorcy}
Rozdział został pierwotnie napisany przede wszystkim dla członków zespołu, aby ułatwić im współpracę - w szczególności wówczas, gdy funkcjonalności czatu mogły pociągać za sobą modyfikację zestawu bibliotek Viua VM bądź struktury składni projektowanego języka ViuAct.

\section{Czat w kontekście}

\subsection{Kontekst biznesowy}

Niniejszy czat stanowi część szerszego kontekstu, jakim jest potrzeba zademonstrowania działania języka ViuAct oraz całego środowiska wytwórczego powiązanego z maszyną wirtualną ViuaVM.

\begin{figure}[h]
	\centering
	\includegraphics[width=\textwidth]{chat/fig/viuavm-env}
	\caption{Ilustracja środowiska wytwórczego wraz zasięgiem, którym są objęte prace przewidziane projektem inżynierskim}
\end{figure}

Cel demonstracyjny był pierwszym i najważniejszym, jaki przyświecał budowie
czatu. Ponadto, proces wytwórczy pozwolił przetestować wydajność całego
środowiska w jego praktycznym wymiarze. Tym samym, możliwe było poprawienie
konstrukcji kompilatora lub zastosowanych konstrukcji językowych ViuAct,
podnosząc tym samym jego użyteczność.

Wszelcy odbiorcy dla aplikacji czatu zostaną, podobnie jak sama aplikacja,
skonstruowani na cele demonstracyjne. Nie powinni oni odbiegać od modelowych
odbiorców podobnych komunikatorów, tak, aby potencjalny, poczatkujący
użytkownik środowiska ViuaVM mógł zrozumieć intencje stojące za rozwiązaniami
zastosowanymi w ViuaChat oraz przenieść je do swoich pierwszych programów.

\subsection{Udziałowcy}

Poniżej wyszczególniono udziałowców, mających wpływ na rozwój czatu.

\begin{tabular}{ | l | l | }

	\hline
	\multicolumn{2}{ | l | }{\textbf{Karta udziałowca}}  \\

	\hline
    \parbox[t]{3cm}{
    	\textbf{Identyfikator}
    } & UN-01 \\

    \hline
    \parbox[t]{3cm}{
    	\textbf{Nazwa}
    } & ViuaVM \\

    \hline
    \parbox[t]{3cm}{
    	\textbf{Opis}
    } & \parbox[t]{12cm}{
    	Maszyna wirtualna, oparta o przechowywanie danych w rejestrach zamiast
      \textit{płaskich} tablic pamięci. Stanowi ona platformę, na której musi
      zostać uruchomiony serwer czatu. Ponieważ jej największym atutem jest
      zorientowanie na kod wykonywany współbieżnie, sam serwer czatu powinien
      tę cechę wykorzystywać w maksymalnym stopniu.
    	} \\

    \hline
    \parbox[t]{3cm}{
    	\textbf{Typ}
    } & Nieożywiony, bezpośredni \\

    \hline
    \parbox[t]{3cm}{
    	\textbf{Punkt widzenia}
    } & \parbox[t]{12cm}{
    	ViuaVM jest absolutnie nieodzownym elementem projektu, a serwer czatu
      stanowi przede wszystkim dowód jej użyteczności. O ile jądro maszyny nie
      ma być poddawane już żadnym zmianom i być wykorzystane takie, jakie było
      na inicjalnym etapie pracy inżynierskiej, o tyle dopuszcza się
      poszerzanie funkcjonalności o dodatkowe biblioteki zewnętrzne.
    	} \\

    \hline
    \parbox[t]{3cm}{
    	\textbf{Ograniczenia}
    } & \parbox[t]{12cm}{
    	Maszyna wirtualna, jakkolwiek stanowi istotny czynnik dla decyzji w
      zakresie architektury czy konstrukcji oprogramowania, nie powinna mieć
      wpływu na wymagania stricte biznesowe, jest bowiem jedynie środowiskiem
      do uruchamiania współbieżnych programów, \textit{przezroczystym} dla
      końcowego użytkownika czy zleceniodawcy zrealizowanego oprogramowania.
    	} \\

    \hline
    \parbox[t]{3cm}{
    	\textbf{Wymagania}
    } & \colorbox{yellow}{...} \\

    \hline
\end{tabular}

\vspace{2em}

\begin{tabular}{ | l | l | }

	\hline
	\multicolumn{2}{ | l | }{\textbf{Karta udziałowca}}  \\

	\hline
    \parbox[t]{3cm}{
    	\textbf{Identyfikator}
    } & UO-01 \\

    \hline
    \parbox[t]{3cm}{
    	\textbf{Nazwa}
    } & Opiekun pracy inżynierskiej \\

    \hline
    \parbox[t]{3cm}{
    	\textbf{Opis}
    } & \parbox[t]{12cm}{
    	Pracownik uczelni, wyznaczony do opieki nad całym projektem inżynierskim
      - nadzorowania jego postępów, wskazywania problemów oraz sugerowania
      decyzji podwyższających walor pracy oraz szanse na jej skuteczne
      obronienie. Ma również zasadniczy wpływ na decyzję o dopuszczeniu pracy
      do recenzji.
    } \\

    \hline
    \parbox[t]{3cm}{
    	\textbf{Typ}
    } & Ożywiony, bezpośredni \\

    \hline
    \parbox[t]{3cm}{
    	\textbf{Punkt widzenia}
    } & \parbox[t]{12cm}{
    	Opiekun pracy patrzy na czat przede wszystkim przez pryzmat jego
      użyteczności jako efektownego przykładu implementacji modelu
    	aktora w praktycznym, programistycznym ujęciu. Stąd, jego uwaga skupia
      się przede wszystkim na konstrukcjach językowych, strukturach oraz
      rozwiązaniach od strony kodu źródłowego. Czat stanowi jedynie pretekst do
      przeniesienia teoretycznych, akademickich rozważań na praktyczny grunt.
    	} \\

    \hline
    \parbox[t]{3cm}{
    	\textbf{Ograniczenia}
    } & \parbox[t]{12cm}{
    	Opiekun pracy, pomimo bycia jej nadzorcą i posiadania istotnych uprawnień
      decyzyjnych w stosunku do jej dalszego rozwoju, nie ma możliwości
      bieżącego śledzenia prac oraz podejmowania decyzji w przypadku
      konkretnych problemów. Powinien zachować dystans, pozwalający na
      samodzielną realizację projektu przez zespół. Stąd, jego faktyczny udział
      ogranicza się do udzielania porad w przypadku strategicznych kierunków, w
      jakich będzie podążała grupa, a także doraźnego recenzowania ograniczonej
      puli zagadnień, wyłapanych w trakcie wspólnych spotkań.
    	} \\

    \hline
    \parbox[t]{3cm}{
    	\textbf{Wymagania}
    } & \colorbox{yellow}{...} \\

    \hline
\end{tabular}

\vspace{2em}

\begin{tabular}{ | l | l | }

	\hline
	\multicolumn{2}{ | l | }{\textbf{Karta udziałowca}}  \\

	\hline
    \parbox[t]{3cm}{
    	\textbf{Identyfikator}
    } & UO-02 \\

    \hline
    \parbox[t]{3cm}{
    	\textbf{Nazwa}
    } & \parbox[t]{12cm}{
    Członek zespołu ds. ViuAct
    } \\

    \hline
    \parbox[t]{3cm}{
    	\textbf{Opis}
    } & \parbox[t]{12cm}{
    	Student i członek zespołu, skupiający się w pierwszej kolejności nad rozwojem języka programowania ViuAct, jego kompilatora oraz
    	ewentualnego rozbudowania maszyny ViuaVM o kolejne, zewnętrzne biblioteki.
    } \\

    \hline
    \parbox[t]{3cm}{
    	\textbf{Typ}
    } & Ożywiony, bezpośredni \\

    \hline
    \parbox[t]{3cm}{
    	\textbf{Punkt widzenia}
    } & \parbox[t]{12cm}{
    	Przede wszystkim, postrzega czat jako produkt, realizowany na końcowej platformie. Stąd, musi brać udział w formułowaniu
    	wymagań związanych z ViuaVM oraz językiem ViuAct. Jego zadaniem jest doprowadzenia do zaprojektowania czatu w sposób,
    	który ukaże możliwości ViuAct jako solidnego, kompletnego rozwiązania. Przy tym, musi trzymać rękę na pulsie i reagować,
    	gdyby pojawiały się przeszkody w zaprogramowaniu czatu, wynikające z niedoskonałości środowiska wytwórczego.

    	Podczas współudziału w definiowaniu wymagań, istotny jest dla niego zakres pracy, wiążący się z
    	urzeczywistnianiem poszczególnych, proponowanych wymagań. Zbyt rozbudowany czat może opóźnić prace nad całym projektem,
    	a w efekcie - zniweczyć trud włożony w rozwój języka programowania i dedykowanego mu kompilatora.
    	} \\

    \hline
    \parbox[t]{3cm}{
    	\textbf{Ograniczenia}
    } & \parbox[t]{12cm}{
    	Jego udział w pracach nad czatem jest z gruntu nieograniczony. Jednakże, decydując się na podział odpowiedzialności
    	podyktowany zespołowym charakterem projektu oraz własnymi ograniczeniami czasowymi, zrezygnował z decydowania o biznesowej
    	części wymagań, faktycznie pozostając w roli konsultanta.

    	} \\

    \hline
    \parbox[t]{3cm}{
    	\textbf{Wymagania}
    } & \colorbox{yellow}{...} \\

    \hline
\end{tabular}

\vspace{2em}

\begin{tabular}{ | l | l | }

	\hline
	\multicolumn{2}{ | l | }{\textbf{Karta udziałowca}}  \\

	\hline
    \parbox[t]{3cm}{
    	\textbf{Identyfikator}
    } & UO-03 \\

    \hline
    \parbox[t]{3cm}{
    	\textbf{Nazwa}
    } & \parbox[t]{12cm}{
    Członek zespołu ds. Czatu
    } \\

    \hline
    \parbox[t]{3cm}{
    	\textbf{Opis}
    } & \parbox[t]{12cm}{
    	Student i członek zespołu, odpowiedzialny za prace nad czatem
    } \\

    \hline
    \parbox[t]{3cm}{
    	\textbf{Typ}
    } & Ożywiony, bezpośredni \\

    \hline
    \parbox[t]{3cm}{
    	\textbf{Punkt widzenia}
    } & \parbox[t]{12cm}{
    	Czat stanowi dla niego, obok dokumentacji, najistotniejszą część przedsięwzięcia. Musi z jednej strony nauczyć się poruszać
    	w nowym, dynamicznie zmieniającym się środowisku programistycznym, a z drugiej strony - zrealizować przy jego użyciu serwer
    	czatu, który pokaże jego możliwości i zastosowania innym nowicjuszom.

    	Podczas współudziału w definiowaniu wymagań, istotny jest dla niego zakres końcowych funkcjonalności czatu. Nie może być zbyt
    	wąski. Z drugiej strony, konstrukcja programu powinna pozostać prosta i przejrzysta. Przykładowy kod nie powinien odstraszać
    	potencjalnego programisty, dla którego cała koncepcja ViuaVM oraz modelu aktorów może wydawać się na pierwszy rzut oka nieco egzotyczna.
    	} \\

    \hline
    \parbox[t]{3cm}{
    	\textbf{Ograniczenia}
    } & \parbox[t]{12cm}{
    	Nie ma w zasadzie organizacyjnych czy kompetencyjnych ograniczeń dla formułowania wymagań. Nie oznacza to jednak, że może
    	definiować wymagań w oderwaniu od pozostałych udziałowców (ich role i punkty widzenia opisano wcześniej).
    	} \\

    \hline
    \parbox[t]{3cm}{
    	\textbf{Wymagania}
    } & \colorbox{yellow}{...} \\

    \hline
\end{tabular}

\subsection{Charakterystyka użytkowników}

Na etapie analizy kontekstu, w którym ma zostać zaprojektowany i zrealizowany czat, zadecydowano o zaprojektowaniu następujących,
modelowych użytkowników docelowego oprogramowania:

\begin{enumerate}

	\item \textbf{Użytkownik tymczasowy.} Typ użytkownika, którego konto jest tworzone podczas połączenia z serwerem czatu oraz
		niszczone po jego zakończeniu. Podczas łączenia z czatem, nie będzie musiał się autoryzować przy użyciu hasła, a deklarować
		tylko unikalną nazwę, nie powtarzającą się z nazwą innego użytkownika, posiadającego konto na danym serwerze czatu.

	\item \textbf{Użytkownik stały.} Typ użytkownika, którego W
		zamierzeniu, adresatami takiego rozwiązania mają być stali bywalcy serwera, którzy chcą mieć zarezerwowaną określoną nazwę
		dla siebie i uniknąć ewentualnego podszywania się. Stąd

	\item \textbf{Administrator.} To użytkownik, który jest dodatkowo wyróżniony i posiada uprawienia do szeroko pojętego
		zarządzania serwerem (w tym - pozostałymi użytkownikami). Konto administratora jest utrzymywane przez serwer pomiędzy połączeniami do czatu. Każdorazowo, przed rozpoczęciem sesji połączenia z serwerem, muszą się dodatkowo autoryzować przy użyciu hasła. Równocześnie, ich nazwa jest zarezerwowana wyłącznie do jego użytku oraz niedostępna
		dla użytkowników tymczasowych.Nie wyróżnia się wśród administratorów żadnych dodatkowych, szczególnych
		ról (np. superadministrator, właściciel).

\end{enumerate}

Poza wspomnianymi różnicami, wszyscy użytkownicy po rozpoczęciu sesji połączenia mają prawo do dołączania do pokojów oraz wysyłania sobie
nawzajem wiadomości prywatnych. Łącznie, pula użytkowników przebywających na serwerze czatu w jednym momencie nie powinna przekraczać 320,
zaś w jednym pokoju - nie więcej niż 32. W związku z tym można przyjąć, że czat jest przeznaczony dla niewielkich społeczności, np.
szkolnych, uczelnianych czy hobbystycznych.

\subsection{Istniejąca infrastruktura}

\begin{itemize}
	\item \textbf{Komputer A}
	\begin{itemize}
		\item komputer przenośny z procesorem Intel Core i5 oraz systemem operacyjnym Windows 10
		\item XAMPP 7.2.7, obejmujący serwer Apache 2.4 oraz interpreter języka PHP w wersji 7.2.7.
    \item oprogramowanie VirtualBox z uruchomioną maszyną wirtualną z systemem
    operacyjnym Linux Mint 19 ,,Tara''.
	\end{itemize}

	\item \textbf{Komputer B}
	\begin{itemize}
		\item komputer przenośny, na którym zainstalowano system operacyjny Linux Mint 19 ,,Tara''
		\item GNU Compiler Collection 8.2
		\item wirtualna maszyna Viua VM w wersji 0.9.0
		\item \textit{należy doinstalować serwer Nginx, odpowiedzialny za wysłanie frontendu do
		użytkownika łączącego się z czatem oraz za handshake Websocketu}
	\end{itemize}

	\item \textit{\textbf{Do uzupełnienia}
	\begin{itemize}
		\item Kolejne urządzenie końcowe (trzecie), dzięki któremu będzie można symulować połączenie kolejnej
		osoby do usługi czatu
	\end{itemize}}

\end{itemize}

\section{Zasady biznesowe}

Zidentyfikowane zasady pogrupowano w 3 kategorie, biorąc pod uwagę podstawowe bloki funkcjonalności. Przydzielenie
do kategorii jest sygnalizowanie literą alfabetu, będącą prefiksem identyfikatora danej zasady. Numeracja identyfikatorów może być nieciągła, gdyż
część z wymagań została usunięta w trakcie prac nad dokumentacją.

Dokonano
również priorytetyzacji zasad biznesowych według klasycznej skali ,,MoSCoW":

\begin{itemize}
	\item \textbf{,,M''} (z ang. \textit{must}) - zasady, których spełnienie jest niezbędne dla realizacji systemu
	\item \textbf{,,S''} (z ang. \textit{should}) - są to zasady o wysokim priorytecie, które powinny;
	zostać spełnione, o ile tylko jest to możliwe;
	\item \textbf{,,C''} (z ang. \textit{could}) - dobrze byłoby zrealizować takie wymagania, ale zależy to od czasu
	i zasobów, jakie pozostaną do dyspozycji po ukończeniu zadań ,,M" i ,,C";
	\item \textbf{,,W''} (z ang. \textit{won't}) - takie wymagania, po dyskusji, zostały wycofane dalszej realizacji.
\end{itemize}



\subsection{System użytkowników [ZU]}
  \begin{tabular}{ | l | l | l | }
	\hline
    \textbf{ID} & \parbox[t]{14cm}{
    	\textbf{Zasada biznesowa}
    } & \textbf{Priorytet} \\

    \hline
    ZU-01 & \parbox[t]{14cm}{
      Podczas wejścia na czat, użytkownikowi pokazuje się monit z polem do wpisania nazwy użytkownika.
    } & M \\

    \hline
    ZU-02 & \parbox[t]{14cm}{
      Użytkownicy bez stałego konta podczas logowania podają tylko nazwę użytkownika, pole hasła pozostaje puste.
    } & M \\

    \hline
    ZU-03 & \parbox[t]{14cm}{
      Nazwa użytkownika to ciąg od 3 do 32 alfanumerycznych znaków.
    } & M \\

    \hline
    ZU-04 & \parbox[t]{14cm}{
      Można rozpocząć sesję jako użytkownik, pod warunkiem, że zadeklarowana nazwa nie będzie powtarzać się z nazwami już zalogowanych użytkowników.
    } & M \\

    \hline
    ZU-05 & \parbox[t]{14cm}{
      Monit podczas wejścia na czat jest wyposażony w pole do wpisania hasła (nieobowiązkowe).
    } & S \\

    \hline
    ZU-06 & \parbox[t]{14cm}{
      Administrator podczas logowania podają nazwę i odpowiadające mu hasło.
    } & S \\

    \hline
    ZU-07 & \parbox[t]{14cm}{
      Konta administratorów są utrzymywane na serwerze w postaci par wartości: nazwa użytkownika i hasło.
    } & S \\

    \hline
    ZU-08 & \parbox[t]{14cm}{
      Nie można rozpocząć sesji użytkownika o nazwie, która pasuje do istniejącego konta, jeżeli nie zostanie podane prawidłowe hasło (nie można podszywać się pod nazwy administratorów).
    } & S \\

    \hline
    ZU-09 & \parbox[t]{14cm}{
      Można rozpocząć sesję jako użytkownik bez podawania hasła, pod warunkiem, że zadeklarowana nazwa nie będzie powtarzać się z nazwami kont administratorów i już zalogowanych użytkowników tymczasowych.
    } & S \\

    \hline
    ZU-10 & \parbox[t]{14cm}{
      W okienkach czatu, loginy administratoró są pogrubione i pokolorowane na czerwono.
      } & S \\

    \hline
    ZU-11 & \parbox[t]{14cm}{
      Administratorzy mają prawo przeglądać nazwy pokojów na serwerze.
    } & M \\

    \hline
    ZU-12 & \parbox[t]{14cm}{
      Administratorzy mają prawo tworzyć i usuwać pokoje.
    } & S \\

    \hline
    ZU-14 & \parbox[t]{14cm}{
      Administratorzy mają prawo wyrzucać użytkowników z pokojów.
    } & C \\

    \hline
    ZU-15 & \parbox[t]{14cm}{
      Administratorzy mają prawo wyrzucać użytkowników z serwera.
    } & C \\

    \hline
    ZU-16 & \parbox[t]{14cm}{
      Administratorzy mają prawo przeglądać nazwy i poziomy uprawnień użytkowników.
    } & M \\

    \hline
    ZU-18 & \parbox[t]{14cm}{
      Administratorzy mają prawo zmieniać swoje hasła użytkowników.
    } & C \\

    \hline
  \end{tabular}

\subsection{Pokoje [ZP]}
  \begin{tabular}{ | l | l | l | }
	\hline
    \textbf{ID} & \parbox[t]{14cm}{
    	\textbf{Zasada biznesowa}
    } & \textbf{Priorytet} \\

    \hline
    ZP-01 & \parbox[t]{14cm}{
      Pokoje to właściwe czaty – tam użytkownicy mogą wejść i pisać do siebie nazwajem
    } & M \\

    \hline
    ZP-02 & \parbox[t]{14cm}{
      Każdy pokój ma unikalną nazwę będącą ciągiem alfanumerycznym od 3 do 32 znaków
    } & M \\

    \hline
    ZP-03 & \parbox[t]{14cm}{
      Lista pokojów jest widoczna dla każdego użytkownika po zalogowaniu się do serwera czatu
    } & M \\

    \hline
    ZP-04 & \parbox[t]{14cm}{
      Użytkownik może być równocześnie wpięty do jednego pokoju
    } & M \\

    \hline
    ZP-05 & \parbox[t]{14cm}{
      Wiadomość wysłana w pokoju jest widoczna w oknie pokoju dla wszystkich użytkowników podpiętych do tego pokoju
    } & M \\

    \hline
    ZP-06 & \parbox[t]{14cm}{
      Użytkownik może się samodzielnie wypiąć z pokoju, do którego jest wpięty
    } & S \\

    \hline
    ZP-07 & \parbox[t]{14cm}{
      Pokój może mieć ustanowione hasło, które użytkownik musi wpisać przed podpięciem się do niego
    } & C \\

    \hline
    ZP-08 & \parbox[t]{14cm}{
      Nowo wpięty użytkownik widzi 10 najnowszych wiadomości,
      które zostały wysłane do pokoju tuż przed wpięciem
    } & S \\

    \hline
    ZP-09 & \parbox[t]{14cm}{
      Serwer czatu automatycznie wysyła do pokoju wiadomości,
      zawierające powiadomienia o wydarzeniach związanych z
      pokojem, tzw. wiadomości systemowe
    } & S \\

    \hline
    ZP-10 & \parbox[t]{14cm}{
      Wiadomości systemowe są niepodpisane przez
      żadnego użytkownika i zapisane kursywą
   	} & C \\

   	\hline
    ZP-11 & \parbox[t]{14cm}{
      Wiadomość systemowa zostaje wysłana podczas wpięcia się
      nowego użytkownika do pokoju
   	} & S \\

   	\hline
    ZP-12 & \parbox[t]{14cm}{
      Wiadomość systemowa zostaje wysłana podczas wypięcia
      użytkownika z pokoju
   	} & S \\

   	\hline
    ZP-13 & \parbox[t]{14cm}{
      Wiadomość systemowa zostaje wysłana, gdy użytkownik
      wpięty do pokoju traci połączenie z serwerem czatu
   	} & S \\

   	\hline
    ZP-14 & \parbox[t]{14cm}{
      Wiadomość systemowa zostaje wysłana, gdy użytkownik
      zostaje wyrzucony z pokoju
   	} & S \\

    \hline

  \end{tabular}

\subsection{Prywatne wiadomości [ZW]}
  \begin{tabular}{ | l | l | l | }
  	\hline
    \textbf{ID} & \parbox[t]{14cm}{
    	\textbf{Zasada biznesowa}
    } & \textbf{Priorytet} \\

    \hline
    ZW-01 & \parbox[t]{14cm}{
      Wiadomości prywatne to wiadomości, które są wysyłane do
      konkretnego odbiorcy, innego niż nadawca. Są one widoczne
      wyłącznie dla nadawcy i odbiorcy takiej wiadomości
    } & M \\

    \hline
    ZW-06 & \parbox[t]{14cm}{
      Użytkownik dysponuje dedykowanym oknem, w którym widzi wiadomości prywatne.
    } & M \\

    \hline
    ZW-07 & \parbox[t]{14cm}{
      Z okna wiadomości prywatnych można odbierać i wysyłać wyłącznie
      wiadomości prywatne, których nadawcą/odbiorcą jest wybrany
      użytkownik
    } & M \\

    \hline
    ZW-08 & \parbox[t]{14cm}{
      W oknie wiadomości prywatnych można przeglądać wiadomości wysłane do i odebrane od jednego, wybranego użytkownika.

    } & S \\

    \hline
    ZW-09 & \parbox[t]{14cm}{
      W oknie wiadomości prywatnych można wysyłać wiadomości
      wyłącznie do nadawcy, którego wiadomości są w danym momencie
      pokazywane.
    } & S \\

    \hline
    ZW-10 & \parbox[t]{14cm}{
      Wiadomości prywatne są utrzymywane dopóki nadawca i odbiorca mają aktywną
      sesję na serwerze.
    } & S \\

    \hline
    ZW-11 & \parbox[t]{14cm}{
      Dla każdej pary użytkowników, na serwerze jest
      gromadzone co najwyżej 100 wiadomości prywatnych.
    } & S \\
    \hline
  \end{tabular}

\section{Wymagania}
\label{program_viuachat_wymagania}

\subsection{Wymagania funkcjonalne}

Ponieważ obraną metodologią wytwarzania aplikacji jest \textit{mini-Scrum}, należący do kategorii metodyk zwinnych, wymagania funkcjonalne ujęto w formie historyjek (\textit{user stories}).

\vspace{2em}

\begin{tabular}{ | l | l | }
	\hline
		\textbf{Identyfikator} &
		WF-01
		\\

	\hline
		\textbf{Treść} & \parbox[t]{11cm}{
			Jako użytkownik serwera czatu, chcę się do niego zalogować, aby zobaczyć listę pokojów dyskusyjnych.
		}\\

	\hline
		\parbox[t]{4cm}{\textbf{Powiązane zasady biznesowe}} & \parbox[t]{11cm}{
			ZU-01 Podczas wejścia na czat, użytkownikowi pokazuje się monit z polem do wpisania nazwy użytkownika. \\
			ZP-03 Lista pokojów jest widoczna dla każdego użytkownika
			po zalogowaniu się do serwera czatu
		}\\

	\hline
		\parbox[t]{4cm}{\textbf{Kryteria akceptacji}} & \parbox[t]{11cm}{
			\begin{enumreq}
				\item Po wejściu na czat bez rozpoczętej sesji, pokazuje się monit o podanie nazwy użytkownika.
				\item Po wpisaniu nazwy użytkownika i zatwierdzeniu, użytkownik rozpocznie sesję na serwerze czatu.
				\item Tuż po rozpoczęciu sesji czatu, użytkownik zobaczy listę pokojów.
			\end{enumreq}
			}
		\\

	\hline
\end{tabular}

\vspace{2em}

\begin{tabular}{ | l | l | }
	\hline
		\textbf{Identyfikator} &
		WF-02
		\\

	\hline
		\textbf{Treść} & \parbox[t]{11cm}{
			Jako użytkownik serwera czatu, chcę wpiąć się do pokoju,
			aby wziąć udział w dyskusji.
		}\\

	\hline
		\parbox[t]{4cm}{\textbf{Powiązane zasady biznesowe}} & \parbox[t]{11cm}{
			ZP-01 Pokoje to właściwe czaty - tam użytkownicy mogą
			wejść i pisać do siebie nawzajem
		}\\

	\hline
		\parbox[t]{4cm}{\textbf{Kryteria akceptacji}} & \parbox[t]{11cm}{
			\begin{enumreq}
				\item Użytkownik, który ma otwartą sesję
				połączenia z serwerem czatu i nie jest wpięty
				do żadnego pokoju, zobaczy listę pokojów.
				\item Użytkownik, po kilknięciu w liście pokojów
				na nazwę pokoju, zostanie do niego podpięty
				\item Użytkownik po wpięciu się do pokoju zobaczy
				okno pokoju
				\item Użytkownik, który ma otwartą sesję
				połączenia z serwerem i jest wpięty do pokoju,
				po odświeżeniu przeglądarki zobaczy okno pokoju,
				do którego jest wpięty
			\end{enumreq}
			}
		\\

	\hline
\end{tabular}

\vspace{2em}

\begin{tabular}{ | l | l | }
	\hline
		\textbf{Identyfikator} &
		WF-03
		\\

	\hline
		\textbf{Treść} & \parbox[t]{11cm}{
			Jako użytkownik serwera czatu, chcę po wpięciu
			do pokoju zobaczyć ostatnie wiadomości wysłane
			przed moim dołączeniem, aby dowiedzieć się, co
			tam się obecnie dzieje.
		}\\

	\hline
		\parbox[t]{4cm}{\textbf{Powiązane zasady biznesowe}} & \parbox[t]{11cm}{
			ZP-08 Nowo wpięty użytkownik widzi 10 najnowszych
			wiadomości, które zostały wysłane do pokoju tuż
			przed wpięciem
		}\\

	\hline
		\parbox[t]{4cm}{\textbf{Kryteria akceptacji}} & \parbox[t]{11cm}{
			\begin{enumreq}
				\item Użytkownik po wpięciu się do pokoju zobaczy
				10 najnowszych wiadomości wysłanych do pokoju
				przed jego dołączeniem (lub mniej, jeżeli
				dotychczas nie wysłano do pokoju co najmniej
				10 wiadomości)
			\end{enumreq}
			}
		\\

	\hline
\end{tabular}

\vspace{2em}

\begin{tabular}{ | l | l | }
	\hline
		\textbf{Identyfikator} &
		WF-04
		\\

	\hline
		\textbf{Treść} & \parbox[t]{11cm}{
			Jako użytkownik serwera czatu, chcę chcę wysłać
			wiadomość do pokoju w który jestem wpięty, aby
			zobaczyli ją inni uczestnicy dyskusji.
		}\\

	\hline
		\parbox[t]{4cm}{\textbf{Powiązane zasady biznesowe}} & \parbox[t]{11cm}{
			ZP-01 Pokoje to właściwe czaty - tam użytkownicy mogą
			wejść i pisać do siebie nawzajem
		}\\

	\hline
		\parbox[t]{4cm}{\textbf{Kryteria akceptacji}} & \parbox[t]{11cm}{
			\begin{enumreq}
				\item Użytkownik wpisze tekst wiadomości w polu
				tekstowym u dołu czatu
				\item Wiadomość wpisana w polu tekstowym zostanie
				wysłana po wciśnięciu klawisza ,,Enter'', gdy aktywne
				będzie pole tekstowe
				\item Wiadomość wpisana w polu tekstowym zostanie
				wysłana po naciśnięciu przycisku ,,Wyślij'',
				widocznego obok pola tekstowego
				\item Po wysłaniu wiadomości, pole tekstowe zostanie
				wyczyszczone (niezależnie od tego czy wiadomość
				zostanie doręczona)
				\item Wiadomość wysłana do pokoju jest pokazywana
				wszystkim użytkownikom podpiętym do czatu u dołu
				strony
				\item Nowa wiadomość jest pokazywana wraz z nazwą
				użytkownika wysyłającego u dołu konwersacji
			\end{enumreq}
			}
		\\

	\hline
\end{tabular}

\vspace{2em}

\begin{tabular}{ | l | l | }
	\hline
		\textbf{Identyfikator} &
		WF-05
		\\

	\hline
		\textbf{Treść} & \parbox[t]{11cm}{
			Jako użytkownik serwera czatu, chcę chcę zobaczyć
			powiadomienie o wpięciu się nowego użytkownika do
			pokoju w którym sam jestem obecnie wpięty, aby powitać
			nowego dyskutanta
		}\\

	\hline
		\parbox[t]{4cm}{\textbf{Powiązane zasady biznesowe}} & \parbox[t]{11cm}{
			ZP-09 Serwer czatu automatycznie wysyła do pokoju
			wiadomości, zawierające powiadomienia o wydarzeniach
			związanych z pokojem, tzw. wiadomości systemowe \\
			ZP-11 Wiadomość systemowa zostaje wysłana podczas
			wpięcia się nowego użytkownika do pokoju
		}\\

	\hline
		\parbox[t]{4cm}{\textbf{Kryteria akceptacji}} & \parbox[t]{11cm}{
			\begin{enumreq}
				\item Niezwłocznie po wpięciu się użytkownika do
				pokoju, serwer wyśle wiadomość systemową o treści
				,,Użytkownik ... dołączył do pokoju'', widoczną
				dla wszystkich użytkowników wpiętych do tego pokoju
			\end{enumreq}
			}
		\\

	\hline
\end{tabular}

\vspace{2em}

\begin{tabular}{ | l | l | }
	\hline
		\textbf{Identyfikator} &
		WF-06
		\\

	\hline
		\textbf{Treść} & \parbox[t]{11cm}{
			Jako użytkownik serwera czatu, chcę zobaczyć
			powiadomienie o opuszczeniu pokoju przez użytkownika,
			aby łatwo zorientować się, że nie bierze już udziału
			w dyskusji.
		}\\

	\hline
		\parbox[t]{4cm}{\textbf{Powiązane zasady biznesowe}} & \parbox[t]{11cm}{
			ZP-09 Serwer czatu automatycznie wysyła do pokoju
			wiadomości, zawierające powiadomienia o wydarzeniach
			związanych z pokojem, tzw. wiadomości systemowe \\
			ZP-12 Wiadomość systemowa zostaje wysłana podczas
			wpięcia wypięcia użytkownika z pokoju \\
			ZP-13 Wiadomość systemowa zostaje wysłana, gdy użytkownik
			wpięty do pokoju traci połączenie z serwerem \\
			ZP-14 Wiadomość systemowa zostaje wysłana, gdy użytkownik
			zostaje wyrzucony z pokoju

		}\\

	\hline
		\parbox[t]{4cm}{\textbf{Kryteria akceptacji}} & \parbox[t]{11cm}{
			\begin{enumreq}
				\item Niezwłocznie po wypięciu się użytkownika z
				pokoju, serwer wyśle wiadomość systemową, widoczną
				dla wszystkich użytkowników wpiętych do tego pokoju,
				o treści:
				\begin{enumerate}
					\item ,,Użytkownik ... opuścił pokój'', gdy
					użytkownik samodzielnie wypiął się z pokoju
					\item ,,Użytkownik ... stracił połączenie'',
					gdy użytkownik został wypięty z pokoju na skutek
					przerwania sesji z uwagi na zerwanie połączenia
					\item ,,Użytkownik ... został wyrzucony'', gdy
					użytkownik został wypięty wskutek interwencji
					administratora
				\end{enumerate}
			\end{enumreq}
			}
		\\

	\hline
\end{tabular}

\vspace{2em}

\begin{tabular}{ | l | l | }
	\hline
		\textbf{Identyfikator} &
		WF-07
		\\

	\hline
		\textbf{Treść} & \parbox[t]{11cm}{
			Jako użytkownik serwera czatu, chcę odpiąć się od pokoju,
			aby wpiąć się do innego pokoju.
		}\\

	\hline
		\parbox[t]{4cm}{\textbf{Powiązane zasady biznesowe}} & \parbox[t]{11cm}{
			ZP-06 Użytkownik może się samodzielnie wypiąć z pokoju,
			do którego jest wpięty

		}\\

	\hline
		\parbox[t]{4cm}{\textbf{Kryteria akceptacji}} & \parbox[t]{11cm}{
			\begin{enumreq}
				\item W oknie pokoju użytkownik zobaczy przycisk
				lub link ,,Opuść pokój''.
				\item Po kliknięciu w ,,Opuść pokój'', użytkownik
				zobaczy listę pokojów.
			\end{enumreq}
			}
		\\

	\hline
\end{tabular}

\vspace{2em}

\begin{tabular}{ | l | l | }
	\hline
		\textbf{Identyfikator} &
		WF-08
		\\

	\hline
		\textbf{Treść} & \parbox[t]{11cm}{
			Jako użytkownik serwera czatu, chcę zobaczyć
			okno wiadomości prywatnych, aby odczytać wiadomości,
			które wysłano specjalnie do mnie.
		}\\

	\hline
		\parbox[t]{4cm}{\textbf{Powiązane zasady biznesowe}} & \parbox[t]{11cm}{
			ZW-01 Wiadomości prywatne to wiadomości, które są
			wysyłane do konkretnego odbiorcy, innego niż
			nadawca... \\
			ZW-06 Użytkownik dysponuje dodatkowym oknem, na którym
			widzi wiadomości prywatne. \\
			ZW-07 Z okna wiadomości prywatnych można odbierać i
			wysyłać wyłącznie wiadomości prywatne, których
			nadawcą / odbiorcą jest wybrany użytkownik. \\
		}\\

	\hline
		\parbox[t]{4cm}{\textbf{Kryteria akceptacji}} & \parbox[t]{11cm}{
			\begin{enumreq}
				\item Po kliknięciu w link ,,PW'', użytkownik
				zobaczy okno prywatnych wiadomości
				\item W oknie wiadomości prywatnych, użytkownik
				zobaczy listę użytkowników, od których otrzymał
				wiadomości prywatne.
				\item Po kliknięciu w link z nazwą użytkownika,
				użytkownik zobaczy prywatne wiadomości, których
				nadawcą i odbiorcą jest wskazana osoba.
			\end{enumreq}
			}
		\\

	\hline
\end{tabular}

\vspace{2em}

\begin{tabular}{ | l | l | }
	\hline
		\textbf{Identyfikator} &
		WF-09
		\\

	\hline
		\textbf{Treść} & \parbox[t]{11cm}{
			Jako użytkownik serwera czatu, chcę wysłać
			wiadomość prywatną do jednego użytkownika, aby
			prowadzić z nim ciągłą konwersację.
		}\\

	\hline
		\parbox[t]{4cm}{\textbf{Powiązane zasady biznesowe}} & \parbox[t]{11cm}{


		}\\

	\hline
		\parbox[t]{4cm}{\textbf{Kryteria akceptacji}} & \parbox[t]{11cm}{
			\begin{enumreq}
				\item Użytkownik wpisze tekst wiadomości w polu
				tekstowym u dołu okna wiadomości prywatnych
				\item Wiadomość wpisana w polu tekstowym zostanie
				wysłana po wciśnięciu klawisza ,,Enter'', gdy
				aktywne
				będzie pole tekstowe
				\item Wiadomość wpisana w polu tekstowym zostanie
				wysłana po naciśnięciu przycisku ,,Wyślij'',
				widocznego obok pola tekstowego
				\item Po wysłaniu wiadomości, pole tekstowe zostanie
				wyczyszczone (niezależnie od tego czy wiadomość
				zostanie doręczona)
				\item Wiadomość wysłana w oknie zostanie pokazana
				tylko użytkownikowi, z którym trwa otwarta
				konwersacja
				\item Nowa wiadomość jest pokazywana wraz z nazwą
				użytkownika wysyłającego u dołu konwersacji
			\end{enumreq}
			}
		\\

	\hline
\end{tabular}

\vspace{2em}

\begin{tabular}{ | l | l | }
	\hline
		\textbf{Identyfikator} &
		WF-11
		\\

	\hline
		\textbf{Treść} & \parbox[t]{11cm}{
			Jako użytkownik serwera czatu, chcę wysłać wiadomość
			prywatną do innego użytkownika, z którym wcześniej nie
			wymieniałem takich wiadomości, aby rozpocząć z nim
			prywatną konwersację.
		}\\

	\hline
		\parbox[t]{4cm}{\textbf{Powiązane zasady biznesowe}} & \parbox[t]{11cm}{


		}\\

	\hline
		\parbox[t]{4cm}{\textbf{Kryteria akceptacji}} & \parbox[t]{11cm}{
			\begin{enumreq}
				\item Użytkownik kliknie w oknie wiadomości
				prywatnych w przyciski ,,Nowy''.
				\item Użytkownik zobaczy monit o podanie nazwy
				użytkownika, z którym chce rozpocząć rozmowę
				\item Jeżeli użytkownik jest aktywny, wówczas
				\item Wiadomość wpisana w polu tekstowym zostanie
				wysłana po wciśnięciu klawisza ,,Enter'', gdy
				aktywne
				będzie pole tekstowe
				\item Wiadomość wpisana w polu tekstowym zostanie
				wysłana po naciśnięciu przycisku ,,Wyślij'',
				widocznego obok pola tekstowego
				\item Po wysłaniu wiadomości, pole tekstowe zostanie
				wyczyszczone (niezależnie od tego czy wiadomość
				zostanie doręczona)
				\item Wiadomość wysłana w oknie zostanie pokazana
				tylko użytkownikowi, z którym trwa otwarta
				konwersacja
				\item Nowa wiadomość jest pokazywana wraz z nazwą
				użytkownika wysyłającego u dołu konwersacji
			\end{enumreq}
			}
		\\

	\hline
\end{tabular}

\vspace{2em}

\begin{tabular}{ | l | l | }
	\hline
		\textbf{Identyfikator} &
		WF-12
		\\

	\hline
		\textbf{Treść} & \parbox[t]{11cm}{
			Jako administrator, chcę utworzyć nowy pokój, aby umożliwić użytkownikom konwersację w węższym gronie.
		}\\

	\hline
		\parbox[t]{4cm}{\textbf{Powiązane zasady biznesowe}} & \parbox[t]{11cm}{
    ZU-12 Administratorzy mają prawo tworzyć i usuwać pokoje.
		}\\

	\hline
		\parbox[t]{4cm}{\textbf{Kryteria akceptacji}} & \parbox[t]{11cm}{
			\begin{enumreq}
				\item Administrator kliknie w oknie z listą pokojów w przycisk ,,Nowy''.
				\item Administrator zobaczy monit o podanie nazwy
				nowego pokoju.
				\item Administrator po podaniu nazwy i zaakceptowaniu,
        zostanie przeniesiony do listy pokojów, na której
        będzie widoczna nazwa dodanego pokoju.
				\item Administrator i inni użytkownicy mogą wpiąć się do nowoutworzonego pokoju.
			\end{enumreq}
			}
		\\

	\hline
\end{tabular}

\vspace{2em}

\begin{tabular}{ | l | l | }
	\hline
		\textbf{Identyfikator} &
		WF-13
		\\

	\hline
		\textbf{Treść} & \parbox[t]{11cm}{
			Jako administrator, chcę usunąć zbędny pokój, aby utrzymać porządek na swoim serwerze czatu.
		}\\

	\hline
		\parbox[t]{4cm}{\textbf{Powiązane zasady biznesowe}} & \parbox[t]{11cm}{
    ZU-12 Administratorzy mają prawo tworzyć i usuwać pokoje.
		}\\

	\hline
		\parbox[t]{4cm}{\textbf{Kryteria akceptacji}} & \parbox[t]{11cm}{
			\begin{enumreq}
				\item Administrator wejdzie do pokoju, który chce
        usunąć.
        \item Administrator zobaczy obok tytułu z nazwą pokoju
        przycisk ,,Usuń''.
				\item Administrator po kliknięciu przycisku zobaczy
        monit z potwierdzeniem działania.
        \item Administrator potwierdzi decyzję w monicie.
        \item Po potwierdzeniu decyzji o usunięciu, administrator
        zostanie przeniesiony do listy pokojów, na której nie
         będzie już widniała nazwa usuniętego pokoju.
				\item Pozostali użytkownicy w usuniętym pokoju zostaną niezwłocznie od niego odpięci i zobaczą monit systemowy informujący o usunięciu pokoju.
			\end{enumreq}
			}
		\\

	\hline
\end{tabular}

\vspace{2em}

\begin{tabular}{ | l | l | }
	\hline
		\textbf{Identyfikator} &
		WF-14
		\\

	\hline
		\textbf{Treść} & \parbox[t]{11cm}{
			Jako administrator, chcę usunąć użytkownika z pokoju,
      aby utrzymać należyty poziom konwersacji.
		}\\

	\hline
		\parbox[t]{4cm}{\textbf{Powiązane zasady biznesowe}} & \parbox[t]{11cm}{
    ZU-14 Administratorzy mają prawo wyrzucać użytkowników z pokojów.
		}\\

	\hline
		\parbox[t]{4cm}{\textbf{Kryteria akceptacji}} & \parbox[t]{11cm}{
			\begin{enumreq}
				\item Administrator wejdzie do pokoju.
        \item Administrator najedzie na nazwę użytkownika którego chce usunąć z pokoju i kliknie na przycisk z
        nazwą ,,Usuń z pokoju''.
				\item Administrator po kliknięciu przycisku zobaczy
        monit z potwierdzeniem działania.
        \item Administrator potwierdzi decyzję w monicie.
        \item Po potwierdzeniu decyzji o usunięciu, administrator
        (tak samo jak każdy inny użytkownik podpięty do pokoju) zobaczy wiadomość systemową o usunięciu z konwersacji.
				\item Usunięty użytkownik zostanie niezwłocznie wypięty
        z pokoju, a także zobaczy monit o przyczynie wypięcia.
			\end{enumreq}
			}
		\\

	\hline
\end{tabular}

\vspace{2em}

\begin{tabular}{ | l | l | }
	\hline
		\textbf{Identyfikator} &
		WF-15
		\\

	\hline
		\textbf{Treść} & \parbox[t]{11cm}{
			Jako administrator, chcę usunąć użytkownika z serwera,
      aby ukarać go za łamanie zasad netykiety.
		}\\

	\hline
		\parbox[t]{4cm}{\textbf{Powiązane zasady biznesowe}} & \parbox[t]{11cm}{
    ZU-15 Administratorzy mają prawo wyrzucać użytkowników z serwera.
		}\\

	\hline
		\parbox[t]{4cm}{\textbf{Kryteria akceptacji}} & \parbox[t]{11cm}{
			\begin{enumreq}
				\item Administrator wejdzie do pokoju.
        \item Administrator najedzie na nazwę użytkownika którego chce usunąć z pokoju i kliknie na przycisk z
        nazwą ,,Usuń z serwera''.
				\item Administrator po kliknięciu przycisku zobaczy
        monit z potwierdzeniem działania.
        \item Administrator potwierdzi decyzję w monicie.
        \item Po potwierdzeniu decyzji o usunięciu, administrator
        (tak samo jak każdy inny użytkownik podpięty do pokoju) zobaczy wiadomość systemową o usunięciu z serwera.
				\item Usunięty użytkownik zostanie niezwłocznie wypięty
        z pokoju i jego sesja zostanie zakończona, a także pokazany zostanie monit o przyczynie tych zdarzeń (usunięcie z serwera czatu).
			\end{enumreq}
			}
		\\

	\hline
\end{tabular}

\vspace{2em}


\begin{tabular}{ | l | l | }
	\hline
		\textbf{Identyfikator} &
		WF-16
		\\

	\hline
		\textbf{Treść} & \parbox[t]{11cm}{
			Jako administrator, chcę zmienić swoje hasło, aby zabezpieczyć swoje hasło w razie ujawnienia go osobie niepowołanej, bez zmiany plików konfiguracyjnych i restartowania całego serwera.
		}\\

	\hline
		\parbox[t]{4cm}{\textbf{Powiązane zasady biznesowe}} & \parbox[t]{11cm}{
    ZU-18 Administratorzy mają prawo zmieniać swoje hasła
    użytkowników.
		}\\

	\hline
		\parbox[t]{4cm}{\textbf{Kryteria akceptacji}} & \parbox[t]{11cm}{
			\begin{enumreq}
				\item Administrator wejdzie na kartę ,,Moje konto''.
        \item Administrator kliknie na przycisk ,,Zmień hasło'',
        widoczny pod nazwą użytkownika.
				\item Administrator zobaczy monit zmiany hasła,
        zawierający jedno pole tekstowe na stare hasło i dwa na
        nowe hasło (wszystkie trzy ukryte przed podglądaniem
        treści podczas ich wprowadzania).
        \item Administrator potwierdzi decyzję o zmianie hasła w monicie.
        \item Po potwierdzeniu decyzji, administrator zobaczy wiadomość systemową o zmianie hasła.
        \item Administrator rozłączy się z serwerem.
        \item Administrator spróbuje rozpocząć nową sesję z
        serwerem, autoryzując się nowym hasłem.
        \item Nowe hasło zostanie zaakceptowane przez serwer,
        sesja zostanie rozpoczęta prawidłowo.
			\end{enumreq}
			}
		\\

	\hline
\end{tabular}

\subsection{Wymagania niefunkcjonalne}

\phantom{}

\begin{tabular}{ | l | l | }
	\hline
		\textbf{Identyfikator} &
		HN-01
		\\

	\hline
		\textbf{Treść} & \parbox[t]{13cm}{
			Długość nazwy użytkownika jest ograniczona od 2 do 32 znaków alfanumerycznych, w celu uniknięcia problemów z identyfikacją użytkownika na serwerze.
		}\\

	\hline
		\parbox[t]{4cm}{\textbf{Powiązane zasady biznesowe}} & \parbox[t]{13cm}{
			ZU-03 Nazwa użytkownika to ciąg od 3 do 32 alfanumerycznych znaków.
		}\\

	\hline
		\parbox[t]{4cm}{\textbf{Kryteria akceptacji}} & \parbox[t]{13cm}{
			\begin{enumreq}
				\item Po wpisaniu do pola użytkownika nazwy krótszej niż 2 znaki, dłużej niż 32 znaki lub zawierającej inne znaki niż alfanumeryczne, zwracany jest błąd.
			\end{enumreq}
			}
		\\

	\hline
\end{tabular}

\vspace{2em}

\begin{tabular}{ | l | l | }
	\hline
		\textbf{Identyfikator} &
		HN-02
		\\

	\hline
		\textbf{Treść} & \parbox[t]{13cm}{
			Loginy i hasła administratorów są gromadzone w plikach
      konfiguracyjnych, a po uruchomieniu serwera - w jego
      pamięci operacyjnej.
		}\\

	\hline
		\parbox[t]{4cm}{\textbf{Powiązane zasady biznesowe}} & \parbox[t]{13cm}{
			ZU-07 Konta administratorów są utrzymywane na serwerze w postaci par wartości: nazwa użytkownika i hasło.
		}\\

	\hline
		\parbox[t]{4cm}{\textbf{Kryteria akceptacji}} & \parbox[t]{13cm}{
			\begin{enumreq}
				\item Serwer jest wyposażony w pliki konfiguracyjne
        \item Po załadowaniu serwera, z plików konfiguracyjnych
        są odczytywane dane kont administracyjnych
        \item Serwer po uruchomieniu jest wyposażony w konta o
        nazwach i hasłach zgodnych z wpisami w plikach konfiguracyjnych.
			\end{enumreq}
			}
		\\

	\hline
\end{tabular}

\vspace{2em}

\begin{tabular}{ | l | l | }
	\hline
		\textbf{Identyfikator} &
		HN-03
		\\

	\hline
		\textbf{Treść} & \parbox[t]{13cm}{
			Loginy administratorów w oknach czatu są pogrubione i pokolorowane na czerwono.
		}\\

	\hline
		\parbox[t]{4cm}{\textbf{Powiązane zasady biznesowe}} & \parbox[t]{13cm}{
			ZU-10 W okienkach czatu, loginy administratorów są pogrubione i pokolorowane na czerwono.
		}\\

	\hline
		\parbox[t]{4cm}{\textbf{Kryteria akceptacji}} & \parbox[t]{13cm}{
			\begin{enumreq}
				\item Nazwy administratorów w oknach czatu są pogrubione
        i pokolorowane na czerowono.
			\end{enumreq}
			}
		\\

	\hline
\end{tabular}

\vspace{2em}

\begin{tabular}{ | l | l | }
	\hline
		\textbf{Identyfikator} &
		HN-04
		\\

	\hline
		\textbf{Treść} & \parbox[t]{13cm}{
			Nazwy pokojów mają od 3 do 32 znaków alfanumerycznych długości, bez
		}\\

	\hline
		\parbox[t]{4cm}{\textbf{Powiązane zasady biznesowe}} & \parbox[t]{13cm}{
			ZP-02 Każdy pokój ma unikalną nazwę będącą ciągiem
      alfanumerycznym od 3 do 32 znaków.
		}\\

	\hline
		\parbox[t]{4cm}{\textbf{Kryteria akceptacji}} & \parbox[t]{13cm}{
			\begin{enumreq}
				\item Nie jest możlwe utworzenie pokoju o nazwie, która
        już wcześniej się pojawiała
        \item Nie jest możliwe utworzenie pokoju o nazwie krótszej niż 3 znaki i dłuższej niż 32 znaki.
        \item Nie jest możliwe utworzenie pokoju o nazwie zawierającej znaki inne niż litery alfabetu łacińskiego, cyfry i znak podkreślenia.
			\end{enumreq}
			}
		\\

	\hline
\end{tabular}

\vspace{2em}

\begin{tabular}{ | l | l | }
	\hline
		\textbf{Identyfikator} &
		HN-05
		\\

	\hline
		\textbf{Treść} & \parbox[t]{13cm}{
			Wiadomości prywatne są czyszczone niezwłocznie po rozłączeniu się przez dowolnego z rozmówców.
		}\\

	\hline
		\parbox[t]{4cm}{\textbf{Powiązane zasady biznesowe}} & \parbox[t]{13cm}{
			ZW-10 Wiadomości prywatne są utrzymywane dopóki nadawca i odbiorca mają aktywną sesję na serwerze.
		}\\

	\hline
		\parbox[t]{4cm}{\textbf{Kryteria akceptacji}} & \parbox[t]{13cm}{
			\begin{enumreq}
				\item Po zamknięciu sesji użytkownika, wiadomości prywatne których był nadawcą lub odbiorcą ulegają
        usunięciu.
			\end{enumreq}
			}
		\\

	\hline
\end{tabular}

\vspace{2em}

\begin{tabular}{ | l | l | }
	\hline
		\textbf{Identyfikator} &
		HN-06
		\\

	\hline
		\textbf{Treść} & \parbox[t]{13cm}{
			Bufor pokoju wiadomości prywatnych zawiera do 100 wiadomości.
		}\\

	\hline
		\parbox[t]{4cm}{\textbf{Powiązane zasady biznesowe}} & \parbox[t]{13cm}{
			ZW-11 Dla każdej pary użytkowników, na serwerze jest gromadzone co najwyżej 100 wiadomości prywatnych.
		}\\

	\hline
		\parbox[t]{4cm}{\textbf{Kryteria akceptacji}} & \parbox[t]{13cm}{
			\begin{enumreq}
				\item Po przekroczeniu liczby 100 wiadomości prywatnych w pokoju, bufor ulega ,,zawinięciu'', usuwając najstarsze 100 wiadomości.
			\end{enumreq}
			}
		\\

	\hline
\end{tabular}

\subsection{Wymagania na środowisko docelowe}

\begin{tabular}{ | l | l | }
	\hline
		\textbf{Identyfikator} &
	WS-01
		\\

	\hline
		\textbf{Treść} & \parbox[t]{13cm}{
			W rozwiązaniu należy wykorzystać środowisko Viua VM i
			język ViuAct
		}\\

	\hline
\end{tabular}

\begin{tabular}{ | l | l | }
	\hline
		\textbf{Identyfikator} &
	WS-02
		\\

	\hline
		\textbf{Treść} & \parbox[t]{13cm}{
			Czat będzie użytkowany jako aplikacja webowa typu
			\textit{single page application}
		}\\

	\hline
\end{tabular}

\begin{tabular}{ | l | l | }
	\hline
		\textbf{Identyfikator} &
	WS-03
		\\

	\hline
		\textbf{Treść} & \parbox[t]{13cm}{
			Aplikacja czatu będzie dostosowana przede wszystkim
			do obsługi z wykorzystaniem urządzeń mobilnych.
		}\\

	\hline
\end{tabular}

\subsection{Wymagania dotyczące procesu wytwarzania}

\begin{tabular}{ | l | l | }
	\hline
		\textbf{Identyfikator} &
	WW-01
		\\

	\hline
		\textbf{Treść} & \parbox[t]{13cm}{
			W procesie wytwórczym należy korzystać z metodologii
			mini-Scrum
		}\\

	\hline
\end{tabular}

\begin{tabular}{ | l | l | }
	\hline
		\textbf{Identyfikator} &
	WW-02
		\\

	\hline
		\textbf{Treść} & \parbox[t]{13cm}{
			W procesie wytwórczym należy korzystać uprzednio przygotować specyfikację przypadków użycia - co wynika z wymagań uczelni.
		}\\

	\hline
\end{tabular}


\include{viuact/language_specification}

\chapter{Język ViuAct i jego kompilator}
\label{jezyk_viuact_i_jego_kompilator}

Pierwszą częścią naszej pracy jest zaprojektowanie wysokopoziomowego języka programowania i opracowanie jego
implementacji. Z uwagi na to, że platforma uruchomieniowa, którą wykorzystujemy (czyli Viua VM) jest maszyną
wirtualną wykonującą programy w postaci \emph{bytecode} wybranym sposobem implementacji języka jest
kompilator - program tłumaczący kod źródłowy w jednym języku na kod źródłowy w innym języku przy jednoczesnym
zachowaniu zachowania programu. W przypadku naszej pracy językiem źródłowym jest język Viuact (dokładniej
opisany w rozdziale \ref{specyfikacja_jezyka_viuact}.~\nameref{specyfikacja_jezyka_viuact} na stronie
\pageref{specyfikacja_jezyka_viuact}), a językiem docelowym język assemblera Viua VM.

\section{Architektura}

\subsection{Użyte wzorce projektowe -- Sposób konstrukcji kompilatora}

\begin{figure}[!htp]
    \centering
    \includegraphics[width=5cm]{basic-compiler-flow}
    \caption{Podstawowy schemat budowy kompilatora}
    \label{basic_compiler_flow}
\end{figure}

Na rysunku \ref{basic_compiler_flow} przedstawiony jest uproszczony schemat budowy kompilatora.
W kompilatorach ''produkcyjnych'' (np. GCC, Clang, czy ICC) tych faz jest więcej -- przede wszystkim etap
emisji kodu jest dużo bardziej rozbudowany, oraz dochodzą etapy analizy semantycznej (czy program ma sens) czy
optymalizacji (prób takiego przekształcenia kodu programu żeby przy zachowaniu znaczenia działał wydajniej).

Kompilator języka ViuAct dostarczany jako element tej pracy inżynierskiej jest pozbawiony etapów
analizy semantycznej oraz optymalizacji. Analiza semantyczna (oraz weryfikacja typów i wykrywanie błędów na
etapie kompilacji) jest oddelegowana do assemblera dostarczanego przez platformę. Optymalizacja jest
całkowicie pominięta gdyż jest to temat niezwykle rozległy; implementacja i doszlifowanie algorytmów
optymalizujących kod jest sama w sobie materiałem wystarczającym na napisanie osobnej pracy inżynierskiej.

Architektura kompilatora języka ViuAct jest dokładniej opisana w rozdziale
\ref{architektura_kompilatora_viuact} (\nameref{architektura_kompilatora_viuact}) na
stronie \pageref{architektura_kompilatora_viuact}.
Sposób działania kompilatora jest opisany w rozdziale \ref{opis_etapow_kompilacji}
(\nameref{opis_etapow_kompilacji}) na stronie \pageref{opis_etapow_kompilacji}.
Omówienie interakcji kompilatora języka ViuAct z narzędziami dostarczanymi przez platformę Viua VM znajduje
się w rozdziale \ref{lang_architektura_systemu} (\nameref{lang_architektura_systemu}) na stronie
\pageref{lang_architektura_systemu}.

Oprócz kompilatora (rozdział \ref{opis_kompilatora} na stronie \pageref{opis_kompilatora}) dostarczany jest
również ''program łączący'' (rozdział \ref{opis_linkera} na stronie \pageref{opis_linkera}) dokonujący
automatycznego połączenia wymaganych modułów w gotowy plik wykonywalny.

\subsection{Architektura systemu}
\label{lang_architektura_systemu}

Rysunek \ref{schemat_interakcji_viuact_z_viuavm} (''\nameref{schemat_interakcji_viuact_z_viuavm}'') prezentuje
schemat interakcji jakie zachodzą w całym systemie od momentu wczytania pliku źródłowego przez kompilator do
uruchomienia programu przez jądro Viua VM.

Ostatnią fazą jaką zajmuje się kompilator języka ViuAct dostarczany jako element tej pracy inżynierskiej jest
emisja kodu (''Assembly code emission''), której wynikiem jest plik z kodem źródłowym w języku assemblera Viua
VM (''\texttt{hello\_world.asm}'' na rysunku \ref{schemat_interakcji_viuact_z_viuavm}).
Rozdział \ref{architektura_kompilatora_viuact} (\nameref{architektura_kompilatora_viuact}) dokładniej opisuje
działanie samego kompilatora.

\begin{figure}[!htp]
    \centering
    \includegraphics[width=9cm]{viuact-pipeline}
    \caption{Interakcje: od pliku źródłowego do działającego programu}
    \label{schemat_interakcji_viuact_z_viuavm}
\end{figure}

Zakres pracy inżynierskiej obejmuje wygenerowanie pliku zawierającego poprawny kod w języku
assemblera Viua VM oraz plików pomocniczych (zadanie kompilatora), oraz takim pokierowaniu
narzędziami dostarczanymi przez platformę, żeby wyemitowały one plik wykonywalny bądź bibliotekę (zadanie
''programu łączącego''). Zakładamy, że narzędzia dostarczane przez platformę działają poprawnie.

Pliki pomocnicze są wymagane przez ''program łączący'' (opisany w rozdziale \ref{opis_linkera} na stronie
\pageref{opis_linkera}). Ich dokładniejsze opisy znajdują się w rozdziałach
''\nameref{pliki_interfejsow_modulow}'' na stronie \pageref{pliki_interfejsow_modulow} i
''\nameref{pliki_zaleznosci_modulow}'' na stronie \pageref{pliki_zaleznosci_modulow}

\subsection{Dekompozycja systemu na podsystemy}
\label{architektura_kompilatora_viuact}

Język ViuAct jest implementowany przez dwa programy:

\begin{enumerate}
    \item \textbf{kompilator} - który przetwarza kod źródłowy w języku ViuAct na kod źródłowy w języku
        assemblera Viua VM
    \item \textbf{linker} - który na podstawie wyników pracy kompilatora tworzy pliki wykonywalne, które
        mogą być uruchomione na jądrze Viua VM
\end{enumerate}

Większość pracy w tym układzie wykonuje kompilator, opisany w rozdziale \ref{opis_kompilatora} na stronie
\pageref{opis_kompilatora}. Generuje on pliki zawierające kod źródłowy w języku assemblera gotowe do
przetworzenia przez assembler Viua VM na formę binarną, oraz pliki pomocnicze.

Z plików pomocniczych korzysta zarówno sam kompilator (do określenia interfejsów modułów importowanych przez
aktualnie kompilowany moduł), ale też linker -- do określenia jakie moduły powinny być dołączone do aktualnie
emitowanego pliku wykonywalnego, aby zapewnić dostępność wszystkich wymaganych funkcji. Linker jest dokładniej
opisany w rozdziale \ref{opis_linkera} na stronie \pageref{opis_linkera}.

\subsubsection{Kompilator -- \texttt{viuact-cc}}
\label{opis_kompilatora}

Kompilator składa się z kilku podsystemów, zgodnie z
przedstawieniem na rysunku \ref{ogolny_schemat_kompilatora_viuact}.

\begin{figure}[!htp]
    \centering
    \includegraphics[width=10cm]{viuact-ogolny-schemat-kompilatora}
    \caption{Podział kompilatora na podsystemy}
    \label{ogolny_schemat_kompilatora_viuact}
\end{figure}

Każdy podsystem implementuje jedną z faz kompilacji:

\begin{enumerate}
    \item \textbf{lexer} dokonuje analizy leksykalnej wczytanego pliku źródłowego, dzieląc go na listę tokenów
    \item \textbf{parser} dokonuje analizy składniowej łącząc tokeny w grupy reprezentujące większe
        konstrukcje językowe
    \item \textbf{lowerer} mapuje grupy wyprodukowane przez \emph{parser} do odpowiednich funkcji
        \emph{emittera}; jest to dość banalny etap, ale upraszcza budowę kompilatora
    \item \textbf{emitter} tłumaczy konstrukcje językowe ViuAct na równoznaczne konstrukcje w języku
        assemblera Viua VM
\end{enumerate}

Różnica między podsystemami \emph{lowerer} i \emph{emitter} może być niejasna. Oba biorą udział w emisji
kodu wynikowego, ale \emph{lowerer} bezpośrednio zajmuje się tylko modułami i funkcjami, natomiast
\emph{emitter} implementuje emisję pojedynczych wyrażeń języka ViuAct -- przypisań \texttt{let}, konstrukcji
warunkowych \texttt{if}, wywołań funkcji, itd.

Proces kompilacji dokłaniej opisany jest w rozdziale \ref{opis_etapow_kompilacji}
(\nameref{opis_etapow_kompilacji}) na stronie \pageref{opis_etapow_kompilacji}.

\subsubsection{Program łączący -- \texttt{viuact-opt}}
\label{opis_linkera}

Program łączący (tzw. ''\emph{linker}'') zajmuje się finalną fazę ''kompilacji''.
Jest to stwierdzenie o tyle trafne, co niepoprawne. Zazwyczaj jednak nie ma to znaczenia, ponieważ zarówno
linker jak i kompilator jest ukrywany przed programistą. Popularne ''kompilatory'' jak np \emph{\texttt{g++}}
z GCC to tak naprawdę ''drivery''; wywołanie polecenia \emph{\texttt{g++}} powoduje wywołanie zarówno
kompilatora (\emph{\texttt{cc1}}), assemblera (\emph{\texttt{as}}), jak i linkera (\emph{\texttt{ld}}) w taki
sposób aby na wyjściu uzyskać oczekiwany wynik, czyli na przykład plik wykonywalny w formacie
ELF \emph{\texttt{a.out}}.

W przypadku kompilatora ViuAct proces ten wygląda podobnie, ale nie jest aż tak zautomatyzowany.
Rolę ''drivera'' pełni programista, który jest odpowiedzialny za wywołanie zarówno kompilatora jak i linkera.
Przykładowo:

\begin{lstlisting}
viuact-cc --mode exec ./hello_world.lisp
viuact-opt ./build/_default/hello_world.asm
\end{lstlisting}

Program łączący przeprowadzi proces asemblacji pliku podanego na wejściu, oraz dołączy do niego wszelkie
wymagane moduły. Zarówno asemblacja jak i łączenie będa przeprowadzone przez narzędzie dostarczane przez
platformę Viua VM -- \texttt{viuact-opt} zajmuje się jedynie wygenerowaniem odpowiednich poleceń dla tego
narzędzia.

Informacja o tym jakie moduły muszą zostać dołączone jest tworzona w oparciu o pliki zależności (opisane w
rozdziale \nameref{pliki_zaleznosci_modulow} na stronie \pageref{pliki_zaleznosci_modulow}).
Dla uproszczenia projektu program łączący nie zbiera informacji o zależnościach rekurencyjnie.

Po zebraniu informacji o zależnościach program łączący dokonuje asemblacji wszystkich modułów, od których
zależy kompilowany moduł główny. Następnie asembluje moduł główny i łączy wszystkie wyemitowane modułu
bytecode'u w gotowy plik wykonywalny.

\subsection{Przebieg procesu kompilacji}
\label{opis_etapow_kompilacji}

Ten rozdział zawiera dokładny opis procesu kompilacji, od momentu wczytania pliku z kodem źródłowym w języku
ViuAct do momentu wyemitowania kodu wynikowego w języku assemblera Viua VM. Kompilator jest wywoływany
poleceniem \texttt{viuact-cc} z opcją \texttt{-}\texttt{-mode} określającą czy kompilowany jest moduł wykonywalny
(\texttt{exec}) czy moduł biblioteki (\texttt{module}):

\begin{lstlisting}
viuact-cc --mode ( 'exec' | 'module' ) file.lisp
\end{lstlisting}

\subsubsection{Wczytanie pliku źródłowego}

Kompilator wczytuje do pamięci cały plik źródłowy jako pojedynczy string.

\subsubsection{Analiza leksykalna}

Lexer patrzy na wczytany kod źródłowy i do pierwszego nieprzeanalizowanego znaku (czyli na początku analizy do
znaku na indeksie 0) próbuje przypasować wzorzec określający jaki token znajduje się na tej pozycji. Po udanym
przypasowaniu pozycja, która będzie rozpatrywana przez lexer jest przesuwana o tyle znaków ile wynosi długość
wygenerowanego tokenu i lexer rozpoczyna pracę od nowa. Ten proces trwa do momentu aż cały string wejściowy
nie zostanie przeanalizowany, albo odrzucony jako nieprawidłowy.

Algorytm przypasowania jest banalny. Lexer dysponuje listą wzorców (określonych przez wyrażenia regularne),
które określają jak wygląda każdy możliwy token w języku. Lexer po kolei próbuje przypasować każdy wzorzec z
listy i kończy na pierwszym trafieniu. Jeśli żaden wzorzec nie może zostać przypisany lexer odrzuca kod
źródłowy jako nieprawidłowy.

Wzorce są uszeregowane w taki sposób żeby nie była możliwa pomyłka i
na przykład przypasowanie początku nazwy zmiennej \texttt{letter} jako słowa kluczowego \texttt{let}.

\subsubsection{Analiza składniowa}
\label{opis_etapow_kompilacji_analiza_skladniowa}

Składnia języka została zaprojektowana w taki sposób aby analiza składniowa mogła być uproszczona do maksimum
i prosta w implementaji.

\paragraph{Grupowanie nawiasów}

W pierwszej fazie analizy składniowej tokeny grupowane są wegług nawiasów okrągłych, przy czym grupowanie jest
rekurencyjne (jeśli jakaś grupa zawiera podgrupę w nawiasach to zagnieżdżona grupa będzie widoczna jako
pojedynczy element w grupie zewnętrznej).
Dla przykładu:

\begin{lstlisting}
(let x (frobnicate 42))
\end{lstlisting}

będzie zgrupowane w następujący sposób:

\begin{lstlisting}
[ "let"; "x"; [ "frobnicate"; "42" ] ]
\end{lstlisting}

\paragraph{Grupowanie id}

Kolejnym etapem jest grupowanie id. Id jest nazwą składającą się z kilku członów, na przykład
\texttt{Std.Posix.Network.socket} składa się z 7 tokenów: trzech \emph{nazw modułów} (\texttt{Str},
\texttt{Posix}, i \texttt{Network}), trzech \emph{operatorów dostępu} (kropek), i jednej \emph{nazwy}
(\texttt{socket}).
Taka grupa zostanie na tym etapie zredukowana do pojedynczego elementu.

\paragraph{Oznczanie wyrażeń złożonych}

Wyrażenia złożone składają się z kilku wyrażeń (prostych bądź złożonych). Z uwagi na fakt, że formą pośrednią
wykorzystywaną na etapie grupowania są listy tokenów takie wyrażenie byłoby nieodróżnialne od listy
reprezentującej wywołanie funkcji. Dlatego na etapie analizy składniowej do list reprezentujących wyrażenia
złożone dodawany jest specjalny token-fantom. Dzięki temu zostaje zachowana właściwość umożliwiająca szybkie
klasyfikowanie grup. Dla przykładu:

\begin{lstlisting}
(let x { ... })
\end{lstlisting}

będzie zgrupowane w następujący sposób:

\begin{lstlisting}
[ "let"; "x"; [ Compound_expression_marker; ... ] ]
\end{lstlisting}

\paragraph{Klasyfikacja grup}

Ostatnim etapem analizy składniowej jest klasyfikacja grup. W większości przypadków do klasyfikacji listy
tokenów do grupy reprezentującej konkretną konstrukcję językową wystarczy spojrzeć na pierwszy token na
liście. W niektórych przypadkach algorytm musi się posiłkować długością listy.

Dla przykładu:

\begin{lstlisting}
[ "let"; "x";          ... ]        -> let-binding
[ "let"; "x"; [ ... ]; ... ]        -> function-definition
[ ... ]                             -> function-call
[ "actor"; ... ]                    -> actor-call
[ Compound_expression_marker; ... ] -> compound-call
\end{lstlisting}

Różnicą między definicją zmiennej (\texttt{let-binding}), a definicją funkcji (\texttt{function-definition})
jest długość listy - definicja zmiennej zawiera trzy elementy (słowo kluczowe \texttt{let}, nazwę, i
wyrażenie), a definicja funkcji cztery elementy (słowo kluczowe \texttt{let}, nazwę, listę parametrów
formalnych, i wyrażenie).

\subsubsection{Emisja modułów}

W następnej kolejności emitowane są wszystkie moduły zagnieżdżone w aktualnie kompilowanym module, przy czym
ten etap postępuje rekurencyjnie. Moduły zagnieżdżone musżą być wyemitowane przed modułem głównym, aby
kompilator miał dostęp do ich plików interfejsów i umożliwić ich importowanie.

\subsubsection{Analiza importu modułów}

Kolejnym krokiem jest analiza modułów importowanych przez aktualnie kompilowany moduł i wczytanie ich
interfejsów. Kompilator ładuje listy sygnatur funkcji i wyliczenia z każdego zaimportowanego modułu.

Jeśli kompilator nie może znaleźć pliku interfejsu danego modułu to kończy kompilację informując o błędzie.
Kompilator szuka plików interfejsów i modułów w ścieżkach podanych w zmiennej środowiskowej
\texttt{VIUAC\_LIBRARY\_PATH} (opisanej na stronie \pageref{viuact_manual_env_viuac_library_path}).

\subsubsection{Emisja kodu wynikowego}
\label{opis_etapow_kompilacji_emisja_kodu_wynikowego}

Na końcu następuje emisja kodu wynikowego w języku assemblera Viua VM. Ten etap jest wykonywany osobno dla
każdej funkcji zdefiniowanej w kompilowanym module.

\paragraph{Redukcja poziomu wyrażeń}

Najpierw następuje redukcja poziomu wyrażeń. Tym etapem zajmuje się \emph{lowerer}. Jest to mechaniczny proces
mapujący sklasyfikowane grupy reprezentujące konretne konstrukcje językowe do funkcji udostępnianych przez
\emph{emitter}, opakowanie wyniku w sposób jakiego wymagają zasady języka assemblera Viua VM, oraz
serializacja wyników do stringów.

\paragraph{Emisja instrukcji języka assemblera}

Emisja instrukcji języka assemblera jest wykonywana per-wyrażenie. Ten etap przeplata się z redukcją poziomu
wyrażeń i jest implementowany przez \emph{emitter}. \emph{Emitter} emituje sekwencje instrukcji języka
assemblera Viua VM odpowiadające zadanym konstrukcjom językowym ViuAct.

Dla przykładu \texttt{(let x 42)} zostanie wyemitowane jako pojedyncza instrukcja: \texttt{integer \%x local
42}.  Natomiast \texttt{(Some\_module.frobnicate 42)} zostanie wyemitowane jako sekwencja instrukcji:

\begin{lstlisting}
integer %3 local 42
frame %1 arguments
copy %0 arguments %3 local
call void Some_module::frobnicate/1
\end{lstlisting}

\subsubsection{Zapis pliku \texttt{.asm}}

Dla każdego wyemitowanego modułu kompilator zapisze plik \texttt{\emph{nazwa\_modulu}.asm} zawierający kod
wynikowy w języku assemblera Viua VM.

\subsubsection{Zapis pliku \texttt{.i}}

Dla każdego wyemitowanego modułu biblioteki kompilator zapisze plik \texttt{\emph{nazwa\_modulu}.i}
zawierający definicję interfejsu tego modułu. Pliki interfejsów są opisane w rozdziale
\nameref{pliki_interfejsow_modulow} na stronie \pageref{pliki_interfejsow_modulow}.

\subsubsection{Zapis pliku \texttt{.d}}

Dla każdego wyemitowanego modułu kompilator zapisze plik \texttt{\emph{nazwa\_modulu}.d}
zawierający definicję zależności tego modułu. Pliki zależności są opisane w rozdziale
\nameref{pliki_zaleznosci_modulow} na stronie \pageref{pliki_zaleznosci_modulow}.


\section{Architektura systemu}
\label{chat_architektura_systemu}

Architektura systemu opiera się na klasycznym modelu klient-serwer.

Klienci będą uzyskiwali dostęp do usługi czatu za pośrednictwem przeglądarki
internetowej. Łącząc się z podanym adresem (IP lub domeny), na którym połączeń
nasłuchuje serwer, w pierwszej kolejności przeglądarka będzie próbować połączyć
się z nim przy użyciu protokołu http i standardowego portu 80, wysyłając do
niego żądanie metodą GET. Wówczas, serwer będzie zawsze odpowiadał statycznym
plikiem HTML, zawierającym odwołania do skryptów w języku JavaScript (JS) oraz
pozostałej, statycznej treści (np. grafiki czy arkusze stylów CSS). Serwer
będzie odsyłał te pliki do przeglądarki w odpowiedzi na kolejne żądania HTTP
GET, wysyłane w miarę dalszego renderowania pliku HTML. W ten sposób, po stronie
klienta zostanie pobrana i uruchomiona aplikacja internetowa typu Single Page
Application, której interfejs będzie reagował z użytkownikiem oraz ulegał
zmianom wskutek działania skryptów JS, załadowanych na pierwszym etapie
uruchomienia. Po stronie serwera, dostarczaniem treści statycznych będzie
zajmował się daemon HTTP - Nginx.

Gdy tylko skrypty JS wykryją pobranie wszystkich plików składowych aplikacji
z serwera, podjęta zostanie próba nawiązania połączenia z tym serwerem przy
użyciu protokołu WebSocket. Będzie on od tego momentu podstawowym kanałem
komunikacji pomiędzy klientem a serwerem.

Zgodnie ze standardem WebSocketu, zanim zostanie nawiązane właściwe połączenie,
powinno dojść do „uścisku dłoni” (ang. \textit{handshake}) pomiędzy klientem a
serwerem. W związku z tym, pierwsza próba połączenia również zostanie podjęta
przy użyciu protokołu http, jednakże tym razem pod innym, dedykowanym portem
(w naszym przypadku będzie to port 8000), a także zawierać nagłówki wskazujące
na żądanie zmiany używanego protokołu na WebSocket, jego wersję oraz klucz
(„Sec-WebSocketKey”). Serwer udzieli wówczas odpowiedzi ze swoim własnym
kluczem, informując o zmianie stosowanego protokołu na WebSocket.

W chwili prawidłowego rozpoczęcia połączenia WebSocket, aplikacja po stronie
klienta wyświetli użytkownikowi okno autoryzacyjne. Wpisane tam dane zostaną
następnie przesłane do serwera. Po jego stronie, komunikat zostanie zdekodowany
przez aktora \texttt{WSConnector} i przekazany powiązanemu aktorowi
\texttt{Authorizer}. Jego zadaniem będzie weryfikacja przedstawionych informacji
oraz podjęcie decyzji o autoryzacji lub jej odmowie. Decyzja ta jest odsyłana do
\texttt{WSConnectora} i następnie przekazywana do aplikacji po stronie klienta.

Jezeli autentykacja przebiegnie pomyślnie, aktor \texttt{Authorizer} uruchamia
aktora \texttt{UserSession}, spina go z aktorem \texttt{WSConnector} używanym
wcześniej do komunikacji z frontendem, oraz ulega autodestrukcji.

\newpage

\subsection{Dekompozycja systemu na podsystemy}
\label{architektura_chatu}

\subsubsection{Strona serwera (,,backend'')}
Na serwer czatu składa się grupa współdziałających, ale zupełnie odrębnych od
siebie aktorów, ukazanych na rysunku \ref{diag-komp}

\nameref{diag-komp}.
\begin{figure}[!htp]
	\centering
	\includegraphics[width=\textwidth]{chat/fig/pck-diag}
	\caption{Diagram komponentów serwera ViuaChat}
	\label{diag-komp}
\end{figure}

\begin{labeling}{UsernameLessor}

  \item[\texttt{Architect}] Uruchamiany jako pierwszy wraz z całym
  serwerem, a następnie inicjalizuje i nadzoruje aktorów \texttt{WSListener},
  \texttt{UsernameLessor} oraz \texttt{Landlord}. W razie nieprawidłwego
  działania lub wyłączenia któregokolwiek z tych trzech głównych aktorów,
  \texttt{Architect} automatycznie zainicjuje przeładowanie całego serwera.
  Ponadto, \texttt{Architect} potrafi w momencie uruchamiania serwera odczytać
  jego pliki konfiguracyjne i na tej podstawie należycie skonfigurować pokoje
  oraz administratorów. Zawsze występuje w jednym egzemplarzu.

  \item[\texttt{WSListener}] Odpowiada za nasłuchiwanie na porcie 8000 po
  stronie serwera, a przy każdej próbie połączenia będzie tworzyć kolejnego,
  niezależnego aktora \texttt{WSInitializer}. Występuje zawsze w jednym
  egzemplarzu.

  \item[\texttt{WSInitializer}] Odpowiada za  realizację ,,uścisku dłoni'' i
  formułowanie odpowiedzi zwrotnej po stronie serwera w odniesieniu do
  połączenia na konkretnym gnieździe. W razie prawidłowego nawiązania
  połączenia, aktor ten utworzy kolejną parę aktorów, \texttt{WSConnector} oraz
  \texttt{Authorizer}, zaś WSInitializer ulegnie samozniszczeniu.

  \item[\texttt{WSConnector}] Odpowiada za dalszą, bezpośrednią obsługę
  przydzielonego gniazda. Występuje ich tylu, ile jest otwartych połącznień.
  Jego rolą będzie również kodowanie i dekodowanie wiadomości (ang.
  ,,messages''), czyli podstawowych logicznych jednostek informacji, które są
  używane przy połączeniach z użyciem protokołu WebSocket. Pilnuje również, czy
  połączenie nie zostało zerwane oraz inicjuje zamykanie sesji użytkownika.

  \item[\texttt{Authorizer}] Wymienia wiadomości od \texttt{WSConnectora}, który
  został uruchomiony wraz z nim i odpowiada za należytą autentykację i/lub
  autoryzację użytkownika w usłudze czatu. Egzemplarz aktora tego typu jest
  powoływany dla każdego otwartego połączenia bez nawiązanej sesji. Aby dokonać
  autoryzacji, aktor \texttt{Authorizer} kontaktuje się \texttt{UsernameLessor}.
  Jeżeli autentykacja przebiegnie pomyślnie, aktor \texttt{Authorizer} uruchamia
  aktora \texttt{UserSession}, spina go z aktorem \texttt{WSConnector} używanym
  wcześniej do komunikacji z frontendem, oraz ulega autodestrukcji.

  \item[\texttt{UsernameLessor}] Jego zadaniem jest zarządzanie informacjami na
  temat tymczasowych nazw użytkowników, należących do	użytkowników bez stałych
  kont (ich gromadzenie, udzielanie, weryfikacja, dbanie o unikalność), a także
  weryfikacja tożsamości kont administratorów z dodatkowym użyciem hasła.
  Występuje w jednym egzemplarzu, przez cały czas istnienia serwera. Ponadto,
  nadzoruje działanie aktorów \texttt{Authorizer} oraz \texttt{UserSession}.

  \item[\texttt{UserSession}] Przejmuje komunikację z użyciem
  \texttt{WSConnector}, pozwalając na zwyczajne użytkowanie czatu. Aktor
  ,,UserSession'' gromadzi informacje na temat nazwy oraz poziomu uprawnień
  użytkownika, a także tego, z jakim pokojem jest obecnie spięty.

  \item[\texttt{Landlord}] Jego zadaniem jest współudział w podpinaniu
  użytkowników do pokoju, tworzeniem nowych i usuwaniem istniejących pokojów, a
  także utrzymywanie i udostępnianie kompletnej listy aktywnych pokojów.

  \item[\texttt{Room}] Działa jak router wiadomości i przechowuje listę użytkowników którzy są do niego wpięci. Istnieje w tylu egzemplarzach, ile jest aktywnych pokojów.

  \item[\texttt{MessageCache}] Przechowuje i odtwarza 10 najnowszych wiadomości wysłanych do pokoju. Występuje po jednym egzemplarzu dla każdego aktywnego aktora \texttt{Room}.

\end{labeling}

\newpage

\subsubsection{Warstwa interfejsu użytkownika (,,frontend'')}
Podczas pracy nad wartwą frontendu, zastosowano framework webowy Vue.js. Jedną
z przyczyn dla tej decyzji jest możliwość zdekomponowania projektowanej aplikacji na mniejsze części, nazywane modułami (ang. \textit{modules}). Są one zorganizowane hierarchiczne. Każdy z modułów zawiera własny skrypt JavaScript, a także kod HTML i arkusz CSS. Powoduje to, że każdy z modułów jest niezależny od pozostałych i może realizować swoje zadania w pełni autonomicznie.

Każdy z modułów udostępnia swojemu rodzicowi pewne określone parametry. Ich zmiana
jest podstawowym sposobem na interakcję pomiędzy nimi, co jest zgodne z
paradygmatem \textit{data driven application} (z ang. ,,aplikacja sterowana
poprzez dane''). W podobny sposób następuje zmiana kodu HTML modułów. Zamiast
zmieniać węzły DOM w sposób jawny poprzez skrypt, programista wskazuje w
szablonie HTML te miejsca, które ulegają określonym przemianom wraz ze
zmian wewnętrznych parametrów modułu.

Podstawowa struktura aplikacji frontendowej przewiduje podział na
pięć części:
\begin{itemize}
	\item Część autoryzacyjna -- służy do nawiązania połączenia z
	serwerem i rozpoczęcie sesji użytkownika.

	\item Pokoje (publiczne) -- dostarcza listę ogólnodostępnych pokojów,
	umożliwia podpięcie się wybranych z nich oraz prowadzenie rozmów z
	innymi użytkownikami, którzy się do nich podpięli.

	\item Wiadomości prywatne -- umożliwia nawiązywanie prywatnych
	rozmów pomiędzy użytkownikami oraz wymianę wiadomości prywatnych.

	\item Narzędzia administracyjne -- pozwalają administratorom
	na dodawanie i usuwanie pokojów, wyrzucanie użytkowników z pokojów
	i z serwera.

	\item Profil użytkownika -- umożliwia podejrzenie informacji na
	temat własnego konta, zmianę przez administratora swojego hasła
	użytkownika oraz rozłączenie się z serwerem.

\end{itemize}

W projekcie przewidziano zastosowanie modułów, przedstawionych na rysunku \ref{diag-komp-front}.

\nameref{diag-komp-front}.
\begin{figure}[!htp]
	\centering
	\includegraphics[width=\textwidth]{chat/fig/pck-diag-front}
	\caption{Diagram komponentów aplikacji webowej ViuaChat}
	\label{diag-komp-front}
\end{figure}

\begin{labeling}{\texttt{UserProfile}}

  \item[\texttt{App}] Nadrzędny moduł, istniejący przez cały czas użycia
	instancji aplikacji. Obejmuje najbardziej zewnętrzne struktury HTML,
	przechowuje podstawowy stan aplikacji (stan autoryzacji, nazwę
	użytkownika, poziom uprawnień, gniazdo WebSocket dla połączeń
	z wartwą backendu), a także zapewnia routing do głównych części
	aplikacji - czyli modułu logowania, modułu pokojów (publicznych),
	pokojów prywatnych wiadomości, profilu użytkownika i narzędzi
	administracyjnych.

	\item[\texttt{SignIn}] Moduł odpowiadający za początkową autoryzację
	do serwera (potocznie znane jako ,,logowanie'').

	\item[\texttt{Rooms}] Moduł reprezentuje część aplikacji poświęconą
	ogólnodostępnym pokojom dla wspólnych konwersacji. Podlegają mu:

	\begin{labeling}{\texttt{RoomsChat}}
		\item[\texttt{RoomsChat}] Moduł odpowiedzialny \textit{stricte} za
		okno czatu - wyświetlanie konwersacji oraz wysyłanie wiadomości
		do pokoju.

		\item[\texttt{RoomsList}] Moduł wyświetlający listę pokojów, a także
		umożliwiający podpięcie się do wybranego z nich.

	\end{labeling}

	\item[\texttt{PM}] Moduł odwzorowujący całą część aplikacji poświęconą
	wymianie pomiędzy użytkownikami wiadomości prywatnych. W jego skład
	wchodzą:

	\begin{labeling}{\texttt{PMUserList}}
		\item[\texttt{PMFindUser}] Moduł, którego zadaniem jest odnalezienie
		użytkowika, do którego ma zostać wysłana wiadomość prywatna

		\item[\texttt{PMUserList}] Moduł obsługujący listę użytkowników,
		którzy wcześniej otrzymali lub wysłali wiadomości prywatne.

		\item[\texttt{PMChat}] Moduł odpowiedzialny za obsługę zasadniczego
		okna czatu wiadomości prywatnych, pozwalającego wymieniać wiadomości
		prywatne z jednym, wybranym użytkownikem.
	\end{labeling}

\item[\texttt{UserProfile}] Moduł pozwalający na podstawową kontrolę
własnego profilu, w tym:

\begin{labeling}{\texttt{ChangePassword}}
	\item[\texttt{SignOut}] Moduł odpowiedzialny za zamknięcie sesji
	użytkownika, rozłączenie z serwerem oraz powrót aplikacji do
	stanu sprzed autoryzacji

	\item[\texttt{ChangePassword}] Moduł mający umożliwiać administratorom
	zmianę swoich własnych haseł

\end{labeling}

\item[\texttt{Admin}] Moduł obejmujący pod sobą wszystkie narzędzia
administracyjne. W jego skład wchodzą:

\begin{labeling}{\texttt{BanFromServer}}
	\item[\texttt{AdminRooms}] Moduł do zarządzania ogólnodostępnymi
	pokojami czatu, w tym:

	\begin{labeling}{\texttt{AdminRoomsList}}
		\item[\texttt{AdminRoomsList}] Moduł listy wszystkich ogólnodostępnych
		pokojów z narzędziami do ich edycji

		\item[\texttt{CreateRoom}] Moduł pozwalający na dodawanie pokojów
		do serweru czatu

		\item[\texttt{RemoveRoom}] Moduł pozwalający na usuwanie istniejących
		pokojów z serwera
	\end{labeling}

	\item[\texttt{BanFromRoom}] Moduł służący do wyrzucania użytkowników
	z pokojów, w których obecnie się znajdują.

	\item[\texttt{BanFromServer}] Moduł pozwalający na wyrzucenie
	użytkowników z serwera.
\end{labeling}

\end{labeling}
\section{Decyzje projektowe}

\subsection{Środowisko docelowe}

\subsection{Środowisko implementacji}

\subsection{Priorytety implementacyjne}

\section{Projekt algorytmów i przyjętych protokołów}

\subsection{Protokół frontend-backend}
Komunikacja pomiędzy frontendem a backendem...

\section{Projekt rozwiązań sprzętowych}

\section{Projekt interfejsu}

\subsection{Interfejs użytkownika}

\subsubsection{Założenia konstrukcji interfejsu}

\section{Projekt bazy danych}

\section{Opis implementacji}


\section{Testowanie}

Testy kompilatora.

\subsection{Zestaw przypadków testowych}

Jeden testowy program na każdą funkcjonalność języka.
Kilka większych testowych programów sprawdzających integrację języka, np. wielomodułowych, wykorzystujących
moduły obce.

\subsection{Wykonanie testów}

Opis tego jak wygląda uruchomienie testów, w jaki sposób został zbudowany framework, itp.

\subsection{Trudności w testowaniu}

Niedeterminizm wynikający z równoległego działania aktorów stwarza problemy w testach. Trzeba uciekać się do
"sztuczek", np. sortowanie wyników programu testowego.

\section{Instrukcja użytkownika kompilatora języka Viuact}
\label{viuact_manual}

Tradycja nakazuje, aby pierwszym programem jaki pisze się w nowym języku był program, który wypisze na ekran
napis ,,\emph{Hello World!}''. W ViuAct ten program wygląda następująco:

\begin{lstlisting}
(let main () {
    (print "Hello World!")
    0
})
\end{lstlisting}

Aby skopilować ten program, należy wykonać w konsoli następujące polecenia:

\begin{lstlisting}
$ viuact-cc --mode exec ./hello_world.lisp
$ viuact-opt ./build/_default/hello_world.asm
\end{lstlisting}

Kod wykonywalny (\emph{bytecode} wykonywalny przez Viua VM) będzie umieszczony w pliku
\texttt{hello\_world.bc} w katalogu \texttt{./build/\_default}.
Aby go uruchomić należy użyć jądra Viua VM:

\begin{lstlisting}
$ viua-vm ./build/_default/hello_world.bc
%*\emph{Hello World!}*)
$
\end{lstlisting}

Nazwy plików pośrednich są wywodzone z nazwy pliku źródłowego:

\begin{description}
    \item[\texttt{\emph{example}.lisp}] plik z kodem źródłowym w języku ViuAct
    \item[\texttt{\emph{example}.asm}] plik wynikowy kompilatora, z kodem źródłowym w języku assemblera Viua
        VM
    \item[\texttt{\emph{example}.bc}] plik wynikowy assemblera Viua VM, zawierający wykonywalny bytecode
\end{description}

Pliki \texttt{.asm} i \texttt{.bc} są umieszczane w katalogu \texttt{./build/\_default}.

\subsection{Opcje kompilatora}

Jedyną opcją kompilatora jest \texttt{--mode}, która przyjmuje dwie możliwe wartości:

\begin{description}
    \item[\texttt{exec}] jeśli plik źródłowy definiuje moduł wykonywalny
    \item[\texttt{module}] jeśli plik źródłowy definiuje moduł biblioteczny
\end{description}

\subsection{Zmienne środowiskowe}

Zachowanie kompilatora można częściowo zmodyfikować ustawiając zmienne środowiskowe.

\subsubsection{\texttt{DEFAULT\_OUTPUT\_DIRECTORY}}

Kontroluje katalog, w którym kompilator składuje pliki wynikowe. Domyślnie pliki wynikowe są składowane w
katalogu \texttt{./build/\_default}.

\subsubsection{\texttt{VIUAC\_LIBRARY\_PATH}}

Jak \texttt{LD\_LIBRARY\_PATH}.

\subsubsection{\texttt{VIUA\_ASM\_PATH}}

Kontroluje ścieżkę do assemblera Viua VM.

\subsubsection{\texttt{VIUAC\_VERBOSE}}

Wartość \texttt{true} powoduje wyświetlenie komunikatów podczas kompilacji.

\subsubsection{\texttt{VIUAC\_DEBUGGING}}

Wartość \texttt{true} włącza komunikaty debugowania.

\subsubsection{\texttt{VIUAC\_INFO}}

Wartość \texttt{true} włącza dodatkowe komunikaty informacyjne.

\subsubsection{\texttt{VIUAC\_DUMP\_INTERMEDIATE}}

Wartość \texttt{tokens} powoduje zrzut strumienia tokenów do pliku \texttt{\emph{example}.tokens}.
Wartość \texttt{exprs} powoduje zrzut drzewa składni do pliku \texttt{\emph{example}.expressions}.
Można podać obie wartości, oddzielone przecinkiem.

\subsection{Opcje programu łączącego}


\chapter{Program ViuaChat -- formalności}

\section{Wprowadzenie}

W poniższym rozdziale zdefiniowano wymagania dla czatu ViuaChat. Ich opracowanie nastąpiło na podstawie analizy otoczenia aplikacji oraz analizy potrzeb projektu w stosunku do niej. W ramach tego procesu nastąpiły:
\begin{itemize}
    \item analiza otoczenia, wraz z z klientami;
    \item wskazanie kontekstu biznesowego systemu;
    \item określenie udziałowców;
	\item wyszczególnienie i uporządkowanie zasad biznesowych, jakie zostały założone w stosunku do aplikacji;
	\item opracowanie historyjek na podstawie ustalonych zasad biznesowych.
\end{itemize}

\textbf{Uwaga:} Niniejszy rozdział nie dotyczy języka ViuAct ani jego kompilatora.

\subsection{Odbiorcy}
Rozdział został pierwotnie napisany przede wszystkim dla członków zespołu, aby ułatwić im współpracę - w szczególności wówczas, gdy funkcjonalności czatu mogły pociągać za sobą modyfikację zestawu bibliotek Viua VM bądź struktury składni projektowanego języka ViuAct.

\section{Czat w kontekście}

\subsection{Kontekst biznesowy}

Niniejszy czat stanowi część szerszego kontekstu, jakim jest potrzeba zademonstrowania działania języka ViuAct oraz całego środowiska wytwórczego powiązanego z maszyną wirtualną ViuaVM.

\begin{figure}[h]
	\centering
	\includegraphics[width=\textwidth]{chat/fig/viuavm-env}
	\caption{Ilustracja środowiska wytwórczego wraz zasięgiem, którym są objęte prace przewidziane projektem inżynierskim}
\end{figure}

Cel demonstracyjny był pierwszym i najważniejszym, jaki przyświecał budowie
czatu. Ponadto, proces wytwórczy pozwolił przetestować wydajność całego
środowiska w jego praktycznym wymiarze. Tym samym, możliwe było poprawienie
konstrukcji kompilatora lub zastosowanych konstrukcji językowych ViuAct,
podnosząc tym samym jego użyteczność.

Wszelcy odbiorcy dla aplikacji czatu zostaną, podobnie jak sama aplikacja,
skonstruowani na cele demonstracyjne. Nie powinni oni odbiegać od modelowych
odbiorców podobnych komunikatorów, tak, aby potencjalny, poczatkujący
użytkownik środowiska ViuaVM mógł zrozumieć intencje stojące za rozwiązaniami
zastosowanymi w ViuaChat oraz przenieść je do swoich pierwszych programów.

\subsection{Udziałowcy}

Poniżej wyszczególniono udziałowców, mających wpływ na rozwój czatu.

\begin{tabular}{ | l | l | }

	\hline
	\multicolumn{2}{ | l | }{\textbf{Karta udziałowca}}  \\

	\hline
    \parbox[t]{3cm}{
    	\textbf{Identyfikator}
    } & UN-01 \\

    \hline
    \parbox[t]{3cm}{
    	\textbf{Nazwa}
    } & ViuaVM \\

    \hline
    \parbox[t]{3cm}{
    	\textbf{Opis}
    } & \parbox[t]{12cm}{
    	Maszyna wirtualna, oparta o przechowywanie danych w rejestrach zamiast
      \textit{płaskich} tablic pamięci. Stanowi ona platformę, na której musi
      zostać uruchomiony serwer czatu. Ponieważ jej największym atutem jest
      zorientowanie na kod wykonywany współbieżnie, sam serwer czatu powinien
      tę cechę wykorzystywać w maksymalnym stopniu.
    	} \\

    \hline
    \parbox[t]{3cm}{
    	\textbf{Typ}
    } & Nieożywiony, bezpośredni \\

    \hline
    \parbox[t]{3cm}{
    	\textbf{Punkt widzenia}
    } & \parbox[t]{12cm}{
    	ViuaVM jest absolutnie nieodzownym elementem projektu, a serwer czatu
      stanowi przede wszystkim dowód jej użyteczności. O ile jądro maszyny nie
      ma być poddawane już żadnym zmianom i być wykorzystane takie, jakie było
      na inicjalnym etapie pracy inżynierskiej, o tyle dopuszcza się
      poszerzanie funkcjonalności o dodatkowe biblioteki zewnętrzne.
    	} \\

    \hline
    \parbox[t]{3cm}{
    	\textbf{Ograniczenia}
    } & \parbox[t]{12cm}{
    	Maszyna wirtualna, jakkolwiek stanowi istotny czynnik dla decyzji w
      zakresie architektury czy konstrukcji oprogramowania, nie powinna mieć
      wpływu na wymagania stricte biznesowe, jest bowiem jedynie środowiskiem
      do uruchamiania współbieżnych programów, \textit{przezroczystym} dla
      końcowego użytkownika czy zleceniodawcy zrealizowanego oprogramowania.
    	} \\

    \hline
    \parbox[t]{3cm}{
    	\textbf{Wymagania}
    } & \colorbox{yellow}{...} \\

    \hline
\end{tabular}

\vspace{2em}

\begin{tabular}{ | l | l | }

	\hline
	\multicolumn{2}{ | l | }{\textbf{Karta udziałowca}}  \\

	\hline
    \parbox[t]{3cm}{
    	\textbf{Identyfikator}
    } & UO-01 \\

    \hline
    \parbox[t]{3cm}{
    	\textbf{Nazwa}
    } & Opiekun pracy inżynierskiej \\

    \hline
    \parbox[t]{3cm}{
    	\textbf{Opis}
    } & \parbox[t]{12cm}{
    	Pracownik uczelni, wyznaczony do opieki nad całym projektem inżynierskim
      - nadzorowania jego postępów, wskazywania problemów oraz sugerowania
      decyzji podwyższających walor pracy oraz szanse na jej skuteczne
      obronienie. Ma również zasadniczy wpływ na decyzję o dopuszczeniu pracy
      do recenzji.
    } \\

    \hline
    \parbox[t]{3cm}{
    	\textbf{Typ}
    } & Ożywiony, bezpośredni \\

    \hline
    \parbox[t]{3cm}{
    	\textbf{Punkt widzenia}
    } & \parbox[t]{12cm}{
    	Opiekun pracy patrzy na czat przede wszystkim przez pryzmat jego
      użyteczności jako efektownego przykładu implementacji modelu
    	aktora w praktycznym, programistycznym ujęciu. Stąd, jego uwaga skupia
      się przede wszystkim na konstrukcjach językowych, strukturach oraz
      rozwiązaniach od strony kodu źródłowego. Czat stanowi jedynie pretekst do
      przeniesienia teoretycznych, akademickich rozważań na praktyczny grunt.
    	} \\

    \hline
    \parbox[t]{3cm}{
    	\textbf{Ograniczenia}
    } & \parbox[t]{12cm}{
    	Opiekun pracy, pomimo bycia jej nadzorcą i posiadania istotnych uprawnień
      decyzyjnych w stosunku do jej dalszego rozwoju, nie ma możliwości
      bieżącego śledzenia prac oraz podejmowania decyzji w przypadku
      konkretnych problemów. Powinien zachować dystans, pozwalający na
      samodzielną realizację projektu przez zespół. Stąd, jego faktyczny udział
      ogranicza się do udzielania porad w przypadku strategicznych kierunków, w
      jakich będzie podążała grupa, a także doraźnego recenzowania ograniczonej
      puli zagadnień, wyłapanych w trakcie wspólnych spotkań.
    	} \\

    \hline
    \parbox[t]{3cm}{
    	\textbf{Wymagania}
    } & \colorbox{yellow}{...} \\

    \hline
\end{tabular}

\vspace{2em}

\begin{tabular}{ | l | l | }

	\hline
	\multicolumn{2}{ | l | }{\textbf{Karta udziałowca}}  \\

	\hline
    \parbox[t]{3cm}{
    	\textbf{Identyfikator}
    } & UO-02 \\

    \hline
    \parbox[t]{3cm}{
    	\textbf{Nazwa}
    } & \parbox[t]{12cm}{
    Członek zespołu ds. ViuAct
    } \\

    \hline
    \parbox[t]{3cm}{
    	\textbf{Opis}
    } & \parbox[t]{12cm}{
    	Student i członek zespołu, skupiający się w pierwszej kolejności nad rozwojem języka programowania ViuAct, jego kompilatora oraz
    	ewentualnego rozbudowania maszyny ViuaVM o kolejne, zewnętrzne biblioteki.
    } \\

    \hline
    \parbox[t]{3cm}{
    	\textbf{Typ}
    } & Ożywiony, bezpośredni \\

    \hline
    \parbox[t]{3cm}{
    	\textbf{Punkt widzenia}
    } & \parbox[t]{12cm}{
    	Przede wszystkim, postrzega czat jako produkt, realizowany na końcowej platformie. Stąd, musi brać udział w formułowaniu
    	wymagań związanych z ViuaVM oraz językiem ViuAct. Jego zadaniem jest doprowadzenia do zaprojektowania czatu w sposób,
    	który ukaże możliwości ViuAct jako solidnego, kompletnego rozwiązania. Przy tym, musi trzymać rękę na pulsie i reagować,
    	gdyby pojawiały się przeszkody w zaprogramowaniu czatu, wynikające z niedoskonałości środowiska wytwórczego.

    	Podczas współudziału w definiowaniu wymagań, istotny jest dla niego zakres pracy, wiążący się z
    	urzeczywistnianiem poszczególnych, proponowanych wymagań. Zbyt rozbudowany czat może opóźnić prace nad całym projektem,
    	a w efekcie - zniweczyć trud włożony w rozwój języka programowania i dedykowanego mu kompilatora.
    	} \\

    \hline
    \parbox[t]{3cm}{
    	\textbf{Ograniczenia}
    } & \parbox[t]{12cm}{
    	Jego udział w pracach nad czatem jest z gruntu nieograniczony. Jednakże, decydując się na podział odpowiedzialności
    	podyktowany zespołowym charakterem projektu oraz własnymi ograniczeniami czasowymi, zrezygnował z decydowania o biznesowej
    	części wymagań, faktycznie pozostając w roli konsultanta.

    	} \\

    \hline
    \parbox[t]{3cm}{
    	\textbf{Wymagania}
    } & \colorbox{yellow}{...} \\

    \hline
\end{tabular}

\vspace{2em}

\begin{tabular}{ | l | l | }

	\hline
	\multicolumn{2}{ | l | }{\textbf{Karta udziałowca}}  \\

	\hline
    \parbox[t]{3cm}{
    	\textbf{Identyfikator}
    } & UO-03 \\

    \hline
    \parbox[t]{3cm}{
    	\textbf{Nazwa}
    } & \parbox[t]{12cm}{
    Członek zespołu ds. Czatu
    } \\

    \hline
    \parbox[t]{3cm}{
    	\textbf{Opis}
    } & \parbox[t]{12cm}{
    	Student i członek zespołu, odpowiedzialny za prace nad czatem
    } \\

    \hline
    \parbox[t]{3cm}{
    	\textbf{Typ}
    } & Ożywiony, bezpośredni \\

    \hline
    \parbox[t]{3cm}{
    	\textbf{Punkt widzenia}
    } & \parbox[t]{12cm}{
    	Czat stanowi dla niego, obok dokumentacji, najistotniejszą część przedsięwzięcia. Musi z jednej strony nauczyć się poruszać
    	w nowym, dynamicznie zmieniającym się środowisku programistycznym, a z drugiej strony - zrealizować przy jego użyciu serwer
    	czatu, który pokaże jego możliwości i zastosowania innym nowicjuszom.

    	Podczas współudziału w definiowaniu wymagań, istotny jest dla niego zakres końcowych funkcjonalności czatu. Nie może być zbyt
    	wąski. Z drugiej strony, konstrukcja programu powinna pozostać prosta i przejrzysta. Przykładowy kod nie powinien odstraszać
    	potencjalnego programisty, dla którego cała koncepcja ViuaVM oraz modelu aktorów może wydawać się na pierwszy rzut oka nieco egzotyczna.
    	} \\

    \hline
    \parbox[t]{3cm}{
    	\textbf{Ograniczenia}
    } & \parbox[t]{12cm}{
    	Nie ma w zasadzie organizacyjnych czy kompetencyjnych ograniczeń dla formułowania wymagań. Nie oznacza to jednak, że może
    	definiować wymagań w oderwaniu od pozostałych udziałowców (ich role i punkty widzenia opisano wcześniej).
    	} \\

    \hline
    \parbox[t]{3cm}{
    	\textbf{Wymagania}
    } & \colorbox{yellow}{...} \\

    \hline
\end{tabular}

\subsection{Charakterystyka użytkowników}

Na etapie analizy kontekstu, w którym ma zostać zaprojektowany i zrealizowany czat, zadecydowano o zaprojektowaniu następujących,
modelowych użytkowników docelowego oprogramowania:

\begin{enumerate}

	\item \textbf{Użytkownik tymczasowy.} Typ użytkownika, którego konto jest tworzone podczas połączenia z serwerem czatu oraz
		niszczone po jego zakończeniu. Podczas łączenia z czatem, nie będzie musiał się autoryzować przy użyciu hasła, a deklarować
		tylko unikalną nazwę, nie powtarzającą się z nazwą innego użytkownika, posiadającego konto na danym serwerze czatu.

	\item \textbf{Użytkownik stały.} Typ użytkownika, którego W
		zamierzeniu, adresatami takiego rozwiązania mają być stali bywalcy serwera, którzy chcą mieć zarezerwowaną określoną nazwę
		dla siebie i uniknąć ewentualnego podszywania się. Stąd

	\item \textbf{Administrator.} To użytkownik, który jest dodatkowo wyróżniony i posiada uprawienia do szeroko pojętego
		zarządzania serwerem (w tym - pozostałymi użytkownikami). Konto administratora jest utrzymywane przez serwer pomiędzy połączeniami do czatu. Każdorazowo, przed rozpoczęciem sesji połączenia z serwerem, muszą się dodatkowo autoryzować przy użyciu hasła. Równocześnie, ich nazwa jest zarezerwowana wyłącznie do jego użytku oraz niedostępna
		dla użytkowników tymczasowych.Nie wyróżnia się wśród administratorów żadnych dodatkowych, szczególnych
		ról (np. superadministrator, właściciel).

\end{enumerate}

Poza wspomnianymi różnicami, wszyscy użytkownicy po rozpoczęciu sesji połączenia mają prawo do dołączania do pokojów oraz wysyłania sobie
nawzajem wiadomości prywatnych. Łącznie, pula użytkowników przebywających na serwerze czatu w jednym momencie nie powinna przekraczać 320,
zaś w jednym pokoju - nie więcej niż 32. W związku z tym można przyjąć, że czat jest przeznaczony dla niewielkich społeczności, np.
szkolnych, uczelnianych czy hobbystycznych.

\subsection{Istniejąca infrastruktura}

\begin{itemize}
	\item \textbf{Komputer A}
	\begin{itemize}
		\item komputer przenośny z procesorem Intel Core i5 oraz systemem operacyjnym Windows 10
		\item XAMPP 7.2.7, obejmujący serwer Apache 2.4 oraz interpreter języka PHP w wersji 7.2.7.
    \item oprogramowanie VirtualBox z uruchomioną maszyną wirtualną z systemem
    operacyjnym Linux Mint 19 ,,Tara''.
	\end{itemize}

	\item \textbf{Komputer B}
	\begin{itemize}
		\item komputer przenośny, na którym zainstalowano system operacyjny Linux Mint 19 ,,Tara''
		\item GNU Compiler Collection 8.2
		\item wirtualna maszyna Viua VM w wersji 0.9.0
		\item \textit{należy doinstalować serwer Nginx, odpowiedzialny za wysłanie frontendu do
		użytkownika łączącego się z czatem oraz za handshake Websocketu}
	\end{itemize}

	\item \textit{\textbf{Do uzupełnienia}
	\begin{itemize}
		\item Kolejne urządzenie końcowe (trzecie), dzięki któremu będzie można symulować połączenie kolejnej
		osoby do usługi czatu
	\end{itemize}}

\end{itemize}

\section{Zasady biznesowe}

Zidentyfikowane zasady pogrupowano w 3 kategorie, biorąc pod uwagę podstawowe bloki funkcjonalności. Przydzielenie
do kategorii jest sygnalizowanie literą alfabetu, będącą prefiksem identyfikatora danej zasady. Numeracja identyfikatorów może być nieciągła, gdyż
część z wymagań została usunięta w trakcie prac nad dokumentacją.

Dokonano
również priorytetyzacji zasad biznesowych według klasycznej skali ,,MoSCoW":

\begin{itemize}
	\item \textbf{,,M''} (z ang. \textit{must}) - zasady, których spełnienie jest niezbędne dla realizacji systemu
	\item \textbf{,,S''} (z ang. \textit{should}) - są to zasady o wysokim priorytecie, które powinny;
	zostać spełnione, o ile tylko jest to możliwe;
	\item \textbf{,,C''} (z ang. \textit{could}) - dobrze byłoby zrealizować takie wymagania, ale zależy to od czasu
	i zasobów, jakie pozostaną do dyspozycji po ukończeniu zadań ,,M" i ,,C";
	\item \textbf{,,W''} (z ang. \textit{won't}) - takie wymagania, po dyskusji, zostały wycofane dalszej realizacji.
\end{itemize}



\subsection{System użytkowników [ZU]}
  \begin{tabular}{ | l | l | l | }
	\hline
    \textbf{ID} & \parbox[t]{14cm}{
    	\textbf{Zasada biznesowa}
    } & \textbf{Priorytet} \\

    \hline
    ZU-01 & \parbox[t]{14cm}{
      Podczas wejścia na czat, użytkownikowi pokazuje się monit z polem do wpisania nazwy użytkownika.
    } & M \\

    \hline
    ZU-02 & \parbox[t]{14cm}{
      Użytkownicy bez stałego konta podczas logowania podają tylko nazwę użytkownika, pole hasła pozostaje puste.
    } & M \\

    \hline
    ZU-03 & \parbox[t]{14cm}{
      Nazwa użytkownika to ciąg od 3 do 32 alfanumerycznych znaków.
    } & M \\

    \hline
    ZU-04 & \parbox[t]{14cm}{
      Można rozpocząć sesję jako użytkownik, pod warunkiem, że zadeklarowana nazwa nie będzie powtarzać się z nazwami już zalogowanych użytkowników.
    } & M \\

    \hline
    ZU-05 & \parbox[t]{14cm}{
      Monit podczas wejścia na czat jest wyposażony w pole do wpisania hasła (nieobowiązkowe).
    } & S \\

    \hline
    ZU-06 & \parbox[t]{14cm}{
      Administrator podczas logowania podają nazwę i odpowiadające mu hasło.
    } & S \\

    \hline
    ZU-07 & \parbox[t]{14cm}{
      Konta administratorów są utrzymywane na serwerze w postaci par wartości: nazwa użytkownika i hasło.
    } & S \\

    \hline
    ZU-08 & \parbox[t]{14cm}{
      Nie można rozpocząć sesji użytkownika o nazwie, która pasuje do istniejącego konta, jeżeli nie zostanie podane prawidłowe hasło (nie można podszywać się pod nazwy administratorów).
    } & S \\

    \hline
    ZU-09 & \parbox[t]{14cm}{
      Można rozpocząć sesję jako użytkownik bez podawania hasła, pod warunkiem, że zadeklarowana nazwa nie będzie powtarzać się z nazwami kont administratorów i już zalogowanych użytkowników tymczasowych.
    } & S \\

    \hline
    ZU-10 & \parbox[t]{14cm}{
      W okienkach czatu, loginy administratoró są pogrubione i pokolorowane na czerwono.
      } & S \\

    \hline
    ZU-11 & \parbox[t]{14cm}{
      Administratorzy mają prawo przeglądać nazwy pokojów na serwerze.
    } & M \\

    \hline
    ZU-12 & \parbox[t]{14cm}{
      Administratorzy mają prawo tworzyć i usuwać pokoje.
    } & S \\

    \hline
    ZU-14 & \parbox[t]{14cm}{
      Administratorzy mają prawo wyrzucać użytkowników z pokojów.
    } & C \\

    \hline
    ZU-15 & \parbox[t]{14cm}{
      Administratorzy mają prawo wyrzucać użytkowników z serwera.
    } & C \\

    \hline
    ZU-16 & \parbox[t]{14cm}{
      Administratorzy mają prawo przeglądać nazwy i poziomy uprawnień użytkowników.
    } & M \\

    \hline
    ZU-18 & \parbox[t]{14cm}{
      Administratorzy mają prawo zmieniać swoje hasła użytkowników.
    } & C \\

    \hline
  \end{tabular}

\subsection{Pokoje [ZP]}
  \begin{tabular}{ | l | l | l | }
	\hline
    \textbf{ID} & \parbox[t]{14cm}{
    	\textbf{Zasada biznesowa}
    } & \textbf{Priorytet} \\

    \hline
    ZP-01 & \parbox[t]{14cm}{
      Pokoje to właściwe czaty – tam użytkownicy mogą wejść i pisać do siebie nazwajem
    } & M \\

    \hline
    ZP-02 & \parbox[t]{14cm}{
      Każdy pokój ma unikalną nazwę będącą ciągiem alfanumerycznym od 3 do 32 znaków
    } & M \\

    \hline
    ZP-03 & \parbox[t]{14cm}{
      Lista pokojów jest widoczna dla każdego użytkownika po zalogowaniu się do serwera czatu
    } & M \\

    \hline
    ZP-04 & \parbox[t]{14cm}{
      Użytkownik może być równocześnie wpięty do jednego pokoju
    } & M \\

    \hline
    ZP-05 & \parbox[t]{14cm}{
      Wiadomość wysłana w pokoju jest widoczna w oknie pokoju dla wszystkich użytkowników podpiętych do tego pokoju
    } & M \\

    \hline
    ZP-06 & \parbox[t]{14cm}{
      Użytkownik może się samodzielnie wypiąć z pokoju, do którego jest wpięty
    } & S \\

    \hline
    ZP-07 & \parbox[t]{14cm}{
      Pokój może mieć ustanowione hasło, które użytkownik musi wpisać przed podpięciem się do niego
    } & C \\

    \hline
    ZP-08 & \parbox[t]{14cm}{
      Nowo wpięty użytkownik widzi 10 najnowszych wiadomości,
      które zostały wysłane do pokoju tuż przed wpięciem
    } & S \\

    \hline
    ZP-09 & \parbox[t]{14cm}{
      Serwer czatu automatycznie wysyła do pokoju wiadomości,
      zawierające powiadomienia o wydarzeniach związanych z
      pokojem, tzw. wiadomości systemowe
    } & S \\

    \hline
    ZP-10 & \parbox[t]{14cm}{
      Wiadomości systemowe są niepodpisane przez
      żadnego użytkownika i zapisane kursywą
   	} & C \\

   	\hline
    ZP-11 & \parbox[t]{14cm}{
      Wiadomość systemowa zostaje wysłana podczas wpięcia się
      nowego użytkownika do pokoju
   	} & S \\

   	\hline
    ZP-12 & \parbox[t]{14cm}{
      Wiadomość systemowa zostaje wysłana podczas wypięcia
      użytkownika z pokoju
   	} & S \\

   	\hline
    ZP-13 & \parbox[t]{14cm}{
      Wiadomość systemowa zostaje wysłana, gdy użytkownik
      wpięty do pokoju traci połączenie z serwerem czatu
   	} & S \\

   	\hline
    ZP-14 & \parbox[t]{14cm}{
      Wiadomość systemowa zostaje wysłana, gdy użytkownik
      zostaje wyrzucony z pokoju
   	} & S \\

    \hline

  \end{tabular}

\subsection{Prywatne wiadomości [ZW]}
  \begin{tabular}{ | l | l | l | }
  	\hline
    \textbf{ID} & \parbox[t]{14cm}{
    	\textbf{Zasada biznesowa}
    } & \textbf{Priorytet} \\

    \hline
    ZW-01 & \parbox[t]{14cm}{
      Wiadomości prywatne to wiadomości, które są wysyłane do
      konkretnego odbiorcy, innego niż nadawca. Są one widoczne
      wyłącznie dla nadawcy i odbiorcy takiej wiadomości
    } & M \\

    \hline
    ZW-06 & \parbox[t]{14cm}{
      Użytkownik dysponuje dedykowanym oknem, w którym widzi wiadomości prywatne.
    } & M \\

    \hline
    ZW-07 & \parbox[t]{14cm}{
      Z okna wiadomości prywatnych można odbierać i wysyłać wyłącznie
      wiadomości prywatne, których nadawcą/odbiorcą jest wybrany
      użytkownik
    } & M \\

    \hline
    ZW-08 & \parbox[t]{14cm}{
      W oknie wiadomości prywatnych można przeglądać wiadomości wysłane do i odebrane od jednego, wybranego użytkownika.

    } & S \\

    \hline
    ZW-09 & \parbox[t]{14cm}{
      W oknie wiadomości prywatnych można wysyłać wiadomości
      wyłącznie do nadawcy, którego wiadomości są w danym momencie
      pokazywane.
    } & S \\

    \hline
    ZW-10 & \parbox[t]{14cm}{
      Wiadomości prywatne są utrzymywane dopóki nadawca i odbiorca mają aktywną
      sesję na serwerze.
    } & S \\

    \hline
    ZW-11 & \parbox[t]{14cm}{
      Dla każdej pary użytkowników, na serwerze jest
      gromadzone co najwyżej 100 wiadomości prywatnych.
    } & S \\
    \hline
  \end{tabular}

\section{Wymagania}
\label{program_viuachat_wymagania}

\subsection{Wymagania funkcjonalne}

Ponieważ obraną metodologią wytwarzania aplikacji jest \textit{mini-Scrum}, należący do kategorii metodyk zwinnych, wymagania funkcjonalne ujęto w formie historyjek (\textit{user stories}).

\vspace{2em}

\begin{tabular}{ | l | l | }
	\hline
		\textbf{Identyfikator} &
		WF-01
		\\

	\hline
		\textbf{Treść} & \parbox[t]{11cm}{
			Jako użytkownik serwera czatu, chcę się do niego zalogować, aby zobaczyć listę pokojów dyskusyjnych.
		}\\

	\hline
		\parbox[t]{4cm}{\textbf{Powiązane zasady biznesowe}} & \parbox[t]{11cm}{
			ZU-01 Podczas wejścia na czat, użytkownikowi pokazuje się monit z polem do wpisania nazwy użytkownika. \\
			ZP-03 Lista pokojów jest widoczna dla każdego użytkownika
			po zalogowaniu się do serwera czatu
		}\\

	\hline
		\parbox[t]{4cm}{\textbf{Kryteria akceptacji}} & \parbox[t]{11cm}{
			\begin{enumreq}
				\item Po wejściu na czat bez rozpoczętej sesji, pokazuje się monit o podanie nazwy użytkownika.
				\item Po wpisaniu nazwy użytkownika i zatwierdzeniu, użytkownik rozpocznie sesję na serwerze czatu.
				\item Tuż po rozpoczęciu sesji czatu, użytkownik zobaczy listę pokojów.
			\end{enumreq}
			}
		\\

	\hline
\end{tabular}

\vspace{2em}

\begin{tabular}{ | l | l | }
	\hline
		\textbf{Identyfikator} &
		WF-02
		\\

	\hline
		\textbf{Treść} & \parbox[t]{11cm}{
			Jako użytkownik serwera czatu, chcę wpiąć się do pokoju,
			aby wziąć udział w dyskusji.
		}\\

	\hline
		\parbox[t]{4cm}{\textbf{Powiązane zasady biznesowe}} & \parbox[t]{11cm}{
			ZP-01 Pokoje to właściwe czaty - tam użytkownicy mogą
			wejść i pisać do siebie nawzajem
		}\\

	\hline
		\parbox[t]{4cm}{\textbf{Kryteria akceptacji}} & \parbox[t]{11cm}{
			\begin{enumreq}
				\item Użytkownik, który ma otwartą sesję
				połączenia z serwerem czatu i nie jest wpięty
				do żadnego pokoju, zobaczy listę pokojów.
				\item Użytkownik, po kilknięciu w liście pokojów
				na nazwę pokoju, zostanie do niego podpięty
				\item Użytkownik po wpięciu się do pokoju zobaczy
				okno pokoju
				\item Użytkownik, który ma otwartą sesję
				połączenia z serwerem i jest wpięty do pokoju,
				po odświeżeniu przeglądarki zobaczy okno pokoju,
				do którego jest wpięty
			\end{enumreq}
			}
		\\

	\hline
\end{tabular}

\vspace{2em}

\begin{tabular}{ | l | l | }
	\hline
		\textbf{Identyfikator} &
		WF-03
		\\

	\hline
		\textbf{Treść} & \parbox[t]{11cm}{
			Jako użytkownik serwera czatu, chcę po wpięciu
			do pokoju zobaczyć ostatnie wiadomości wysłane
			przed moim dołączeniem, aby dowiedzieć się, co
			tam się obecnie dzieje.
		}\\

	\hline
		\parbox[t]{4cm}{\textbf{Powiązane zasady biznesowe}} & \parbox[t]{11cm}{
			ZP-08 Nowo wpięty użytkownik widzi 10 najnowszych
			wiadomości, które zostały wysłane do pokoju tuż
			przed wpięciem
		}\\

	\hline
		\parbox[t]{4cm}{\textbf{Kryteria akceptacji}} & \parbox[t]{11cm}{
			\begin{enumreq}
				\item Użytkownik po wpięciu się do pokoju zobaczy
				10 najnowszych wiadomości wysłanych do pokoju
				przed jego dołączeniem (lub mniej, jeżeli
				dotychczas nie wysłano do pokoju co najmniej
				10 wiadomości)
			\end{enumreq}
			}
		\\

	\hline
\end{tabular}

\vspace{2em}

\begin{tabular}{ | l | l | }
	\hline
		\textbf{Identyfikator} &
		WF-04
		\\

	\hline
		\textbf{Treść} & \parbox[t]{11cm}{
			Jako użytkownik serwera czatu, chcę chcę wysłać
			wiadomość do pokoju w który jestem wpięty, aby
			zobaczyli ją inni uczestnicy dyskusji.
		}\\

	\hline
		\parbox[t]{4cm}{\textbf{Powiązane zasady biznesowe}} & \parbox[t]{11cm}{
			ZP-01 Pokoje to właściwe czaty - tam użytkownicy mogą
			wejść i pisać do siebie nawzajem
		}\\

	\hline
		\parbox[t]{4cm}{\textbf{Kryteria akceptacji}} & \parbox[t]{11cm}{
			\begin{enumreq}
				\item Użytkownik wpisze tekst wiadomości w polu
				tekstowym u dołu czatu
				\item Wiadomość wpisana w polu tekstowym zostanie
				wysłana po wciśnięciu klawisza ,,Enter'', gdy aktywne
				będzie pole tekstowe
				\item Wiadomość wpisana w polu tekstowym zostanie
				wysłana po naciśnięciu przycisku ,,Wyślij'',
				widocznego obok pola tekstowego
				\item Po wysłaniu wiadomości, pole tekstowe zostanie
				wyczyszczone (niezależnie od tego czy wiadomość
				zostanie doręczona)
				\item Wiadomość wysłana do pokoju jest pokazywana
				wszystkim użytkownikom podpiętym do czatu u dołu
				strony
				\item Nowa wiadomość jest pokazywana wraz z nazwą
				użytkownika wysyłającego u dołu konwersacji
			\end{enumreq}
			}
		\\

	\hline
\end{tabular}

\vspace{2em}

\begin{tabular}{ | l | l | }
	\hline
		\textbf{Identyfikator} &
		WF-05
		\\

	\hline
		\textbf{Treść} & \parbox[t]{11cm}{
			Jako użytkownik serwera czatu, chcę chcę zobaczyć
			powiadomienie o wpięciu się nowego użytkownika do
			pokoju w którym sam jestem obecnie wpięty, aby powitać
			nowego dyskutanta
		}\\

	\hline
		\parbox[t]{4cm}{\textbf{Powiązane zasady biznesowe}} & \parbox[t]{11cm}{
			ZP-09 Serwer czatu automatycznie wysyła do pokoju
			wiadomości, zawierające powiadomienia o wydarzeniach
			związanych z pokojem, tzw. wiadomości systemowe \\
			ZP-11 Wiadomość systemowa zostaje wysłana podczas
			wpięcia się nowego użytkownika do pokoju
		}\\

	\hline
		\parbox[t]{4cm}{\textbf{Kryteria akceptacji}} & \parbox[t]{11cm}{
			\begin{enumreq}
				\item Niezwłocznie po wpięciu się użytkownika do
				pokoju, serwer wyśle wiadomość systemową o treści
				,,Użytkownik ... dołączył do pokoju'', widoczną
				dla wszystkich użytkowników wpiętych do tego pokoju
			\end{enumreq}
			}
		\\

	\hline
\end{tabular}

\vspace{2em}

\begin{tabular}{ | l | l | }
	\hline
		\textbf{Identyfikator} &
		WF-06
		\\

	\hline
		\textbf{Treść} & \parbox[t]{11cm}{
			Jako użytkownik serwera czatu, chcę zobaczyć
			powiadomienie o opuszczeniu pokoju przez użytkownika,
			aby łatwo zorientować się, że nie bierze już udziału
			w dyskusji.
		}\\

	\hline
		\parbox[t]{4cm}{\textbf{Powiązane zasady biznesowe}} & \parbox[t]{11cm}{
			ZP-09 Serwer czatu automatycznie wysyła do pokoju
			wiadomości, zawierające powiadomienia o wydarzeniach
			związanych z pokojem, tzw. wiadomości systemowe \\
			ZP-12 Wiadomość systemowa zostaje wysłana podczas
			wpięcia wypięcia użytkownika z pokoju \\
			ZP-13 Wiadomość systemowa zostaje wysłana, gdy użytkownik
			wpięty do pokoju traci połączenie z serwerem \\
			ZP-14 Wiadomość systemowa zostaje wysłana, gdy użytkownik
			zostaje wyrzucony z pokoju

		}\\

	\hline
		\parbox[t]{4cm}{\textbf{Kryteria akceptacji}} & \parbox[t]{11cm}{
			\begin{enumreq}
				\item Niezwłocznie po wypięciu się użytkownika z
				pokoju, serwer wyśle wiadomość systemową, widoczną
				dla wszystkich użytkowników wpiętych do tego pokoju,
				o treści:
				\begin{enumerate}
					\item ,,Użytkownik ... opuścił pokój'', gdy
					użytkownik samodzielnie wypiął się z pokoju
					\item ,,Użytkownik ... stracił połączenie'',
					gdy użytkownik został wypięty z pokoju na skutek
					przerwania sesji z uwagi na zerwanie połączenia
					\item ,,Użytkownik ... został wyrzucony'', gdy
					użytkownik został wypięty wskutek interwencji
					administratora
				\end{enumerate}
			\end{enumreq}
			}
		\\

	\hline
\end{tabular}

\vspace{2em}

\begin{tabular}{ | l | l | }
	\hline
		\textbf{Identyfikator} &
		WF-07
		\\

	\hline
		\textbf{Treść} & \parbox[t]{11cm}{
			Jako użytkownik serwera czatu, chcę odpiąć się od pokoju,
			aby wpiąć się do innego pokoju.
		}\\

	\hline
		\parbox[t]{4cm}{\textbf{Powiązane zasady biznesowe}} & \parbox[t]{11cm}{
			ZP-06 Użytkownik może się samodzielnie wypiąć z pokoju,
			do którego jest wpięty

		}\\

	\hline
		\parbox[t]{4cm}{\textbf{Kryteria akceptacji}} & \parbox[t]{11cm}{
			\begin{enumreq}
				\item W oknie pokoju użytkownik zobaczy przycisk
				lub link ,,Opuść pokój''.
				\item Po kliknięciu w ,,Opuść pokój'', użytkownik
				zobaczy listę pokojów.
			\end{enumreq}
			}
		\\

	\hline
\end{tabular}

\vspace{2em}

\begin{tabular}{ | l | l | }
	\hline
		\textbf{Identyfikator} &
		WF-08
		\\

	\hline
		\textbf{Treść} & \parbox[t]{11cm}{
			Jako użytkownik serwera czatu, chcę zobaczyć
			okno wiadomości prywatnych, aby odczytać wiadomości,
			które wysłano specjalnie do mnie.
		}\\

	\hline
		\parbox[t]{4cm}{\textbf{Powiązane zasady biznesowe}} & \parbox[t]{11cm}{
			ZW-01 Wiadomości prywatne to wiadomości, które są
			wysyłane do konkretnego odbiorcy, innego niż
			nadawca... \\
			ZW-06 Użytkownik dysponuje dodatkowym oknem, na którym
			widzi wiadomości prywatne. \\
			ZW-07 Z okna wiadomości prywatnych można odbierać i
			wysyłać wyłącznie wiadomości prywatne, których
			nadawcą / odbiorcą jest wybrany użytkownik. \\
		}\\

	\hline
		\parbox[t]{4cm}{\textbf{Kryteria akceptacji}} & \parbox[t]{11cm}{
			\begin{enumreq}
				\item Po kliknięciu w link ,,PW'', użytkownik
				zobaczy okno prywatnych wiadomości
				\item W oknie wiadomości prywatnych, użytkownik
				zobaczy listę użytkowników, od których otrzymał
				wiadomości prywatne.
				\item Po kliknięciu w link z nazwą użytkownika,
				użytkownik zobaczy prywatne wiadomości, których
				nadawcą i odbiorcą jest wskazana osoba.
			\end{enumreq}
			}
		\\

	\hline
\end{tabular}

\vspace{2em}

\begin{tabular}{ | l | l | }
	\hline
		\textbf{Identyfikator} &
		WF-09
		\\

	\hline
		\textbf{Treść} & \parbox[t]{11cm}{
			Jako użytkownik serwera czatu, chcę wysłać
			wiadomość prywatną do jednego użytkownika, aby
			prowadzić z nim ciągłą konwersację.
		}\\

	\hline
		\parbox[t]{4cm}{\textbf{Powiązane zasady biznesowe}} & \parbox[t]{11cm}{


		}\\

	\hline
		\parbox[t]{4cm}{\textbf{Kryteria akceptacji}} & \parbox[t]{11cm}{
			\begin{enumreq}
				\item Użytkownik wpisze tekst wiadomości w polu
				tekstowym u dołu okna wiadomości prywatnych
				\item Wiadomość wpisana w polu tekstowym zostanie
				wysłana po wciśnięciu klawisza ,,Enter'', gdy
				aktywne
				będzie pole tekstowe
				\item Wiadomość wpisana w polu tekstowym zostanie
				wysłana po naciśnięciu przycisku ,,Wyślij'',
				widocznego obok pola tekstowego
				\item Po wysłaniu wiadomości, pole tekstowe zostanie
				wyczyszczone (niezależnie od tego czy wiadomość
				zostanie doręczona)
				\item Wiadomość wysłana w oknie zostanie pokazana
				tylko użytkownikowi, z którym trwa otwarta
				konwersacja
				\item Nowa wiadomość jest pokazywana wraz z nazwą
				użytkownika wysyłającego u dołu konwersacji
			\end{enumreq}
			}
		\\

	\hline
\end{tabular}

\vspace{2em}

\begin{tabular}{ | l | l | }
	\hline
		\textbf{Identyfikator} &
		WF-11
		\\

	\hline
		\textbf{Treść} & \parbox[t]{11cm}{
			Jako użytkownik serwera czatu, chcę wysłać wiadomość
			prywatną do innego użytkownika, z którym wcześniej nie
			wymieniałem takich wiadomości, aby rozpocząć z nim
			prywatną konwersację.
		}\\

	\hline
		\parbox[t]{4cm}{\textbf{Powiązane zasady biznesowe}} & \parbox[t]{11cm}{


		}\\

	\hline
		\parbox[t]{4cm}{\textbf{Kryteria akceptacji}} & \parbox[t]{11cm}{
			\begin{enumreq}
				\item Użytkownik kliknie w oknie wiadomości
				prywatnych w przyciski ,,Nowy''.
				\item Użytkownik zobaczy monit o podanie nazwy
				użytkownika, z którym chce rozpocząć rozmowę
				\item Jeżeli użytkownik jest aktywny, wówczas
				\item Wiadomość wpisana w polu tekstowym zostanie
				wysłana po wciśnięciu klawisza ,,Enter'', gdy
				aktywne
				będzie pole tekstowe
				\item Wiadomość wpisana w polu tekstowym zostanie
				wysłana po naciśnięciu przycisku ,,Wyślij'',
				widocznego obok pola tekstowego
				\item Po wysłaniu wiadomości, pole tekstowe zostanie
				wyczyszczone (niezależnie od tego czy wiadomość
				zostanie doręczona)
				\item Wiadomość wysłana w oknie zostanie pokazana
				tylko użytkownikowi, z którym trwa otwarta
				konwersacja
				\item Nowa wiadomość jest pokazywana wraz z nazwą
				użytkownika wysyłającego u dołu konwersacji
			\end{enumreq}
			}
		\\

	\hline
\end{tabular}

\vspace{2em}

\begin{tabular}{ | l | l | }
	\hline
		\textbf{Identyfikator} &
		WF-12
		\\

	\hline
		\textbf{Treść} & \parbox[t]{11cm}{
			Jako administrator, chcę utworzyć nowy pokój, aby umożliwić użytkownikom konwersację w węższym gronie.
		}\\

	\hline
		\parbox[t]{4cm}{\textbf{Powiązane zasady biznesowe}} & \parbox[t]{11cm}{
    ZU-12 Administratorzy mają prawo tworzyć i usuwać pokoje.
		}\\

	\hline
		\parbox[t]{4cm}{\textbf{Kryteria akceptacji}} & \parbox[t]{11cm}{
			\begin{enumreq}
				\item Administrator kliknie w oknie z listą pokojów w przycisk ,,Nowy''.
				\item Administrator zobaczy monit o podanie nazwy
				nowego pokoju.
				\item Administrator po podaniu nazwy i zaakceptowaniu,
        zostanie przeniesiony do listy pokojów, na której
        będzie widoczna nazwa dodanego pokoju.
				\item Administrator i inni użytkownicy mogą wpiąć się do nowoutworzonego pokoju.
			\end{enumreq}
			}
		\\

	\hline
\end{tabular}

\vspace{2em}

\begin{tabular}{ | l | l | }
	\hline
		\textbf{Identyfikator} &
		WF-13
		\\

	\hline
		\textbf{Treść} & \parbox[t]{11cm}{
			Jako administrator, chcę usunąć zbędny pokój, aby utrzymać porządek na swoim serwerze czatu.
		}\\

	\hline
		\parbox[t]{4cm}{\textbf{Powiązane zasady biznesowe}} & \parbox[t]{11cm}{
    ZU-12 Administratorzy mają prawo tworzyć i usuwać pokoje.
		}\\

	\hline
		\parbox[t]{4cm}{\textbf{Kryteria akceptacji}} & \parbox[t]{11cm}{
			\begin{enumreq}
				\item Administrator wejdzie do pokoju, który chce
        usunąć.
        \item Administrator zobaczy obok tytułu z nazwą pokoju
        przycisk ,,Usuń''.
				\item Administrator po kliknięciu przycisku zobaczy
        monit z potwierdzeniem działania.
        \item Administrator potwierdzi decyzję w monicie.
        \item Po potwierdzeniu decyzji o usunięciu, administrator
        zostanie przeniesiony do listy pokojów, na której nie
         będzie już widniała nazwa usuniętego pokoju.
				\item Pozostali użytkownicy w usuniętym pokoju zostaną niezwłocznie od niego odpięci i zobaczą monit systemowy informujący o usunięciu pokoju.
			\end{enumreq}
			}
		\\

	\hline
\end{tabular}

\vspace{2em}

\begin{tabular}{ | l | l | }
	\hline
		\textbf{Identyfikator} &
		WF-14
		\\

	\hline
		\textbf{Treść} & \parbox[t]{11cm}{
			Jako administrator, chcę usunąć użytkownika z pokoju,
      aby utrzymać należyty poziom konwersacji.
		}\\

	\hline
		\parbox[t]{4cm}{\textbf{Powiązane zasady biznesowe}} & \parbox[t]{11cm}{
    ZU-14 Administratorzy mają prawo wyrzucać użytkowników z pokojów.
		}\\

	\hline
		\parbox[t]{4cm}{\textbf{Kryteria akceptacji}} & \parbox[t]{11cm}{
			\begin{enumreq}
				\item Administrator wejdzie do pokoju.
        \item Administrator najedzie na nazwę użytkownika którego chce usunąć z pokoju i kliknie na przycisk z
        nazwą ,,Usuń z pokoju''.
				\item Administrator po kliknięciu przycisku zobaczy
        monit z potwierdzeniem działania.
        \item Administrator potwierdzi decyzję w monicie.
        \item Po potwierdzeniu decyzji o usunięciu, administrator
        (tak samo jak każdy inny użytkownik podpięty do pokoju) zobaczy wiadomość systemową o usunięciu z konwersacji.
				\item Usunięty użytkownik zostanie niezwłocznie wypięty
        z pokoju, a także zobaczy monit o przyczynie wypięcia.
			\end{enumreq}
			}
		\\

	\hline
\end{tabular}

\vspace{2em}

\begin{tabular}{ | l | l | }
	\hline
		\textbf{Identyfikator} &
		WF-15
		\\

	\hline
		\textbf{Treść} & \parbox[t]{11cm}{
			Jako administrator, chcę usunąć użytkownika z serwera,
      aby ukarać go za łamanie zasad netykiety.
		}\\

	\hline
		\parbox[t]{4cm}{\textbf{Powiązane zasady biznesowe}} & \parbox[t]{11cm}{
    ZU-15 Administratorzy mają prawo wyrzucać użytkowników z serwera.
		}\\

	\hline
		\parbox[t]{4cm}{\textbf{Kryteria akceptacji}} & \parbox[t]{11cm}{
			\begin{enumreq}
				\item Administrator wejdzie do pokoju.
        \item Administrator najedzie na nazwę użytkownika którego chce usunąć z pokoju i kliknie na przycisk z
        nazwą ,,Usuń z serwera''.
				\item Administrator po kliknięciu przycisku zobaczy
        monit z potwierdzeniem działania.
        \item Administrator potwierdzi decyzję w monicie.
        \item Po potwierdzeniu decyzji o usunięciu, administrator
        (tak samo jak każdy inny użytkownik podpięty do pokoju) zobaczy wiadomość systemową o usunięciu z serwera.
				\item Usunięty użytkownik zostanie niezwłocznie wypięty
        z pokoju i jego sesja zostanie zakończona, a także pokazany zostanie monit o przyczynie tych zdarzeń (usunięcie z serwera czatu).
			\end{enumreq}
			}
		\\

	\hline
\end{tabular}

\vspace{2em}


\begin{tabular}{ | l | l | }
	\hline
		\textbf{Identyfikator} &
		WF-16
		\\

	\hline
		\textbf{Treść} & \parbox[t]{11cm}{
			Jako administrator, chcę zmienić swoje hasło, aby zabezpieczyć swoje hasło w razie ujawnienia go osobie niepowołanej, bez zmiany plików konfiguracyjnych i restartowania całego serwera.
		}\\

	\hline
		\parbox[t]{4cm}{\textbf{Powiązane zasady biznesowe}} & \parbox[t]{11cm}{
    ZU-18 Administratorzy mają prawo zmieniać swoje hasła
    użytkowników.
		}\\

	\hline
		\parbox[t]{4cm}{\textbf{Kryteria akceptacji}} & \parbox[t]{11cm}{
			\begin{enumreq}
				\item Administrator wejdzie na kartę ,,Moje konto''.
        \item Administrator kliknie na przycisk ,,Zmień hasło'',
        widoczny pod nazwą użytkownika.
				\item Administrator zobaczy monit zmiany hasła,
        zawierający jedno pole tekstowe na stare hasło i dwa na
        nowe hasło (wszystkie trzy ukryte przed podglądaniem
        treści podczas ich wprowadzania).
        \item Administrator potwierdzi decyzję o zmianie hasła w monicie.
        \item Po potwierdzeniu decyzji, administrator zobaczy wiadomość systemową o zmianie hasła.
        \item Administrator rozłączy się z serwerem.
        \item Administrator spróbuje rozpocząć nową sesję z
        serwerem, autoryzując się nowym hasłem.
        \item Nowe hasło zostanie zaakceptowane przez serwer,
        sesja zostanie rozpoczęta prawidłowo.
			\end{enumreq}
			}
		\\

	\hline
\end{tabular}

\subsection{Wymagania niefunkcjonalne}

\phantom{}

\begin{tabular}{ | l | l | }
	\hline
		\textbf{Identyfikator} &
		HN-01
		\\

	\hline
		\textbf{Treść} & \parbox[t]{13cm}{
			Długość nazwy użytkownika jest ograniczona od 2 do 32 znaków alfanumerycznych, w celu uniknięcia problemów z identyfikacją użytkownika na serwerze.
		}\\

	\hline
		\parbox[t]{4cm}{\textbf{Powiązane zasady biznesowe}} & \parbox[t]{13cm}{
			ZU-03 Nazwa użytkownika to ciąg od 3 do 32 alfanumerycznych znaków.
		}\\

	\hline
		\parbox[t]{4cm}{\textbf{Kryteria akceptacji}} & \parbox[t]{13cm}{
			\begin{enumreq}
				\item Po wpisaniu do pola użytkownika nazwy krótszej niż 2 znaki, dłużej niż 32 znaki lub zawierającej inne znaki niż alfanumeryczne, zwracany jest błąd.
			\end{enumreq}
			}
		\\

	\hline
\end{tabular}

\vspace{2em}

\begin{tabular}{ | l | l | }
	\hline
		\textbf{Identyfikator} &
		HN-02
		\\

	\hline
		\textbf{Treść} & \parbox[t]{13cm}{
			Loginy i hasła administratorów są gromadzone w plikach
      konfiguracyjnych, a po uruchomieniu serwera - w jego
      pamięci operacyjnej.
		}\\

	\hline
		\parbox[t]{4cm}{\textbf{Powiązane zasady biznesowe}} & \parbox[t]{13cm}{
			ZU-07 Konta administratorów są utrzymywane na serwerze w postaci par wartości: nazwa użytkownika i hasło.
		}\\

	\hline
		\parbox[t]{4cm}{\textbf{Kryteria akceptacji}} & \parbox[t]{13cm}{
			\begin{enumreq}
				\item Serwer jest wyposażony w pliki konfiguracyjne
        \item Po załadowaniu serwera, z plików konfiguracyjnych
        są odczytywane dane kont administracyjnych
        \item Serwer po uruchomieniu jest wyposażony w konta o
        nazwach i hasłach zgodnych z wpisami w plikach konfiguracyjnych.
			\end{enumreq}
			}
		\\

	\hline
\end{tabular}

\vspace{2em}

\begin{tabular}{ | l | l | }
	\hline
		\textbf{Identyfikator} &
		HN-03
		\\

	\hline
		\textbf{Treść} & \parbox[t]{13cm}{
			Loginy administratorów w oknach czatu są pogrubione i pokolorowane na czerwono.
		}\\

	\hline
		\parbox[t]{4cm}{\textbf{Powiązane zasady biznesowe}} & \parbox[t]{13cm}{
			ZU-10 W okienkach czatu, loginy administratorów są pogrubione i pokolorowane na czerwono.
		}\\

	\hline
		\parbox[t]{4cm}{\textbf{Kryteria akceptacji}} & \parbox[t]{13cm}{
			\begin{enumreq}
				\item Nazwy administratorów w oknach czatu są pogrubione
        i pokolorowane na czerowono.
			\end{enumreq}
			}
		\\

	\hline
\end{tabular}

\vspace{2em}

\begin{tabular}{ | l | l | }
	\hline
		\textbf{Identyfikator} &
		HN-04
		\\

	\hline
		\textbf{Treść} & \parbox[t]{13cm}{
			Nazwy pokojów mają od 3 do 32 znaków alfanumerycznych długości, bez
		}\\

	\hline
		\parbox[t]{4cm}{\textbf{Powiązane zasady biznesowe}} & \parbox[t]{13cm}{
			ZP-02 Każdy pokój ma unikalną nazwę będącą ciągiem
      alfanumerycznym od 3 do 32 znaków.
		}\\

	\hline
		\parbox[t]{4cm}{\textbf{Kryteria akceptacji}} & \parbox[t]{13cm}{
			\begin{enumreq}
				\item Nie jest możlwe utworzenie pokoju o nazwie, która
        już wcześniej się pojawiała
        \item Nie jest możliwe utworzenie pokoju o nazwie krótszej niż 3 znaki i dłuższej niż 32 znaki.
        \item Nie jest możliwe utworzenie pokoju o nazwie zawierającej znaki inne niż litery alfabetu łacińskiego, cyfry i znak podkreślenia.
			\end{enumreq}
			}
		\\

	\hline
\end{tabular}

\vspace{2em}

\begin{tabular}{ | l | l | }
	\hline
		\textbf{Identyfikator} &
		HN-05
		\\

	\hline
		\textbf{Treść} & \parbox[t]{13cm}{
			Wiadomości prywatne są czyszczone niezwłocznie po rozłączeniu się przez dowolnego z rozmówców.
		}\\

	\hline
		\parbox[t]{4cm}{\textbf{Powiązane zasady biznesowe}} & \parbox[t]{13cm}{
			ZW-10 Wiadomości prywatne są utrzymywane dopóki nadawca i odbiorca mają aktywną sesję na serwerze.
		}\\

	\hline
		\parbox[t]{4cm}{\textbf{Kryteria akceptacji}} & \parbox[t]{13cm}{
			\begin{enumreq}
				\item Po zamknięciu sesji użytkownika, wiadomości prywatne których był nadawcą lub odbiorcą ulegają
        usunięciu.
			\end{enumreq}
			}
		\\

	\hline
\end{tabular}

\vspace{2em}

\begin{tabular}{ | l | l | }
	\hline
		\textbf{Identyfikator} &
		HN-06
		\\

	\hline
		\textbf{Treść} & \parbox[t]{13cm}{
			Bufor pokoju wiadomości prywatnych zawiera do 100 wiadomości.
		}\\

	\hline
		\parbox[t]{4cm}{\textbf{Powiązane zasady biznesowe}} & \parbox[t]{13cm}{
			ZW-11 Dla każdej pary użytkowników, na serwerze jest gromadzone co najwyżej 100 wiadomości prywatnych.
		}\\

	\hline
		\parbox[t]{4cm}{\textbf{Kryteria akceptacji}} & \parbox[t]{13cm}{
			\begin{enumreq}
				\item Po przekroczeniu liczby 100 wiadomości prywatnych w pokoju, bufor ulega ,,zawinięciu'', usuwając najstarsze 100 wiadomości.
			\end{enumreq}
			}
		\\

	\hline
\end{tabular}

\subsection{Wymagania na środowisko docelowe}

\begin{tabular}{ | l | l | }
	\hline
		\textbf{Identyfikator} &
	WS-01
		\\

	\hline
		\textbf{Treść} & \parbox[t]{13cm}{
			W rozwiązaniu należy wykorzystać środowisko Viua VM i
			język ViuAct
		}\\

	\hline
\end{tabular}

\begin{tabular}{ | l | l | }
	\hline
		\textbf{Identyfikator} &
	WS-02
		\\

	\hline
		\textbf{Treść} & \parbox[t]{13cm}{
			Czat będzie użytkowany jako aplikacja webowa typu
			\textit{single page application}
		}\\

	\hline
\end{tabular}

\begin{tabular}{ | l | l | }
	\hline
		\textbf{Identyfikator} &
	WS-03
		\\

	\hline
		\textbf{Treść} & \parbox[t]{13cm}{
			Aplikacja czatu będzie dostosowana przede wszystkim
			do obsługi z wykorzystaniem urządzeń mobilnych.
		}\\

	\hline
\end{tabular}

\subsection{Wymagania dotyczące procesu wytwarzania}

\begin{tabular}{ | l | l | }
	\hline
		\textbf{Identyfikator} &
	WW-01
		\\

	\hline
		\textbf{Treść} & \parbox[t]{13cm}{
			W procesie wytwórczym należy korzystać z metodologii
			mini-Scrum
		}\\

	\hline
\end{tabular}

\begin{tabular}{ | l | l | }
	\hline
		\textbf{Identyfikator} &
	WW-02
		\\

	\hline
		\textbf{Treść} & \parbox[t]{13cm}{
			W procesie wytwórczym należy korzystać uprzednio przygotować specyfikację przypadków użycia - co wynika z wymagań uczelni.
		}\\

	\hline
\end{tabular}


\chapter{Program ViuaChat}
\label{program_viuachat}

Drugą częścią naszej pracy jest program ViuaChat - chat internetowy o funkcjonalności podobnej do IRC.
Umożliwia użytkownikom logowanie się do sieci, zakładanie pokojów tematycznych do rozmów, oraz rozmowy
''prywatne'' (tj. wyłączne dla dwóch użytkowników).

\section{Architektura systemu}
\label{chat_architektura_systemu}

Architektura systemu opiera się na klasycznym modelu klient-serwer.

Klienci będą uzyskiwali dostęp do usługi czatu za pośrednictwem przeglądarki
internetowej. Łącząc się z podanym adresem (IP lub domeny), na którym połączeń
nasłuchuje serwer, w pierwszej kolejności przeglądarka będzie próbować połączyć
się z nim przy użyciu protokołu http i standardowego portu 80, wysyłając do
niego żądanie metodą GET. Wówczas, serwer będzie zawsze odpowiadał statycznym
plikiem HTML, zawierającym odwołania do skryptów w języku JavaScript (JS) oraz
pozostałej, statycznej treści (np. grafiki czy arkusze stylów CSS). Serwer
będzie odsyłał te pliki do przeglądarki w odpowiedzi na kolejne żądania HTTP
GET, wysyłane w miarę dalszego renderowania pliku HTML. W ten sposób, po stronie
klienta zostanie pobrana i uruchomiona aplikacja internetowa typu Single Page
Application, której interfejs będzie reagował z użytkownikiem oraz ulegał
zmianom wskutek działania skryptów JS, załadowanych na pierwszym etapie
uruchomienia. Po stronie serwera, dostarczaniem treści statycznych będzie
zajmował się daemon HTTP - Nginx.

Gdy tylko skrypty JS wykryją pobranie wszystkich plików składowych aplikacji
z serwera, podjęta zostanie próba nawiązania połączenia z tym serwerem przy
użyciu protokołu WebSocket. Będzie on od tego momentu podstawowym kanałem
komunikacji pomiędzy klientem a serwerem.

Zgodnie ze standardem WebSocketu, zanim zostanie nawiązane właściwe połączenie,
powinno dojść do „uścisku dłoni” (ang. \textit{handshake}) pomiędzy klientem a
serwerem. W związku z tym, pierwsza próba połączenia również zostanie podjęta
przy użyciu protokołu http, jednakże tym razem pod innym, dedykowanym portem
(w naszym przypadku będzie to port 8000), a także zawierać nagłówki wskazujące
na żądanie zmiany używanego protokołu na WebSocket, jego wersję oraz klucz
(„Sec-WebSocketKey”). Serwer udzieli wówczas odpowiedzi ze swoim własnym
kluczem, informując o zmianie stosowanego protokołu na WebSocket.

W chwili prawidłowego rozpoczęcia połączenia WebSocket, aplikacja po stronie
klienta wyświetli użytkownikowi okno autoryzacyjne. Wpisane tam dane zostaną
następnie przesłane do serwera. Po jego stronie, komunikat zostanie zdekodowany
przez aktora \texttt{WSConnector} i przekazany powiązanemu aktorowi
\texttt{Authorizer}. Jego zadaniem będzie weryfikacja przedstawionych informacji
oraz podjęcie decyzji o autoryzacji lub jej odmowie. Decyzja ta jest odsyłana do
\texttt{WSConnectora} i następnie przekazywana do aplikacji po stronie klienta.

Jezeli autentykacja przebiegnie pomyślnie, aktor \texttt{Authorizer} uruchamia
aktora \texttt{UserSession}, spina go z aktorem \texttt{WSConnector} używanym
wcześniej do komunikacji z frontendem, oraz ulega autodestrukcji.

\newpage

\subsection{Dekompozycja systemu na podsystemy}
\label{architektura_chatu}

\subsubsection{Strona serwera (,,backend'')}
Na serwer czatu składa się grupa współdziałających, ale zupełnie odrębnych od
siebie aktorów, ukazanych na rysunku \ref{diag-komp}

\nameref{diag-komp}.
\begin{figure}[!htp]
	\centering
	\includegraphics[width=\textwidth]{chat/fig/pck-diag}
	\caption{Diagram komponentów serwera ViuaChat}
	\label{diag-komp}
\end{figure}

\begin{labeling}{UsernameLessor}

  \item[\texttt{Architect}] Uruchamiany jako pierwszy wraz z całym
  serwerem, a następnie inicjalizuje i nadzoruje aktorów \texttt{WSListener},
  \texttt{UsernameLessor} oraz \texttt{Landlord}. W razie nieprawidłwego
  działania lub wyłączenia któregokolwiek z tych trzech głównych aktorów,
  \texttt{Architect} automatycznie zainicjuje przeładowanie całego serwera.
  Ponadto, \texttt{Architect} potrafi w momencie uruchamiania serwera odczytać
  jego pliki konfiguracyjne i na tej podstawie należycie skonfigurować pokoje
  oraz administratorów. Zawsze występuje w jednym egzemplarzu.

  \item[\texttt{WSListener}] Odpowiada za nasłuchiwanie na porcie 8000 po
  stronie serwera, a przy każdej próbie połączenia będzie tworzyć kolejnego,
  niezależnego aktora \texttt{WSInitializer}. Występuje zawsze w jednym
  egzemplarzu.

  \item[\texttt{WSInitializer}] Odpowiada za  realizację ,,uścisku dłoni'' i
  formułowanie odpowiedzi zwrotnej po stronie serwera w odniesieniu do
  połączenia na konkretnym gnieździe. W razie prawidłowego nawiązania
  połączenia, aktor ten utworzy kolejną parę aktorów, \texttt{WSConnector} oraz
  \texttt{Authorizer}, zaś WSInitializer ulegnie samozniszczeniu.

  \item[\texttt{WSConnector}] Odpowiada za dalszą, bezpośrednią obsługę
  przydzielonego gniazda. Występuje ich tylu, ile jest otwartych połącznień.
  Jego rolą będzie również kodowanie i dekodowanie wiadomości (ang.
  ,,messages''), czyli podstawowych logicznych jednostek informacji, które są
  używane przy połączeniach z użyciem protokołu WebSocket. Pilnuje również, czy
  połączenie nie zostało zerwane oraz inicjuje zamykanie sesji użytkownika.

  \item[\texttt{Authorizer}] Wymienia wiadomości od \texttt{WSConnectora}, który
  został uruchomiony wraz z nim i odpowiada za należytą autentykację i/lub
  autoryzację użytkownika w usłudze czatu. Egzemplarz aktora tego typu jest
  powoływany dla każdego otwartego połączenia bez nawiązanej sesji. Aby dokonać
  autoryzacji, aktor \texttt{Authorizer} kontaktuje się \texttt{UsernameLessor}.
  Jeżeli autentykacja przebiegnie pomyślnie, aktor \texttt{Authorizer} uruchamia
  aktora \texttt{UserSession}, spina go z aktorem \texttt{WSConnector} używanym
  wcześniej do komunikacji z frontendem, oraz ulega autodestrukcji.

  \item[\texttt{UsernameLessor}] Jego zadaniem jest zarządzanie informacjami na
  temat tymczasowych nazw użytkowników, należących do	użytkowników bez stałych
  kont (ich gromadzenie, udzielanie, weryfikacja, dbanie o unikalność), a także
  weryfikacja tożsamości kont administratorów z dodatkowym użyciem hasła.
  Występuje w jednym egzemplarzu, przez cały czas istnienia serwera. Ponadto,
  nadzoruje działanie aktorów \texttt{Authorizer} oraz \texttt{UserSession}.

  \item[\texttt{UserSession}] Przejmuje komunikację z użyciem
  \texttt{WSConnector}, pozwalając na zwyczajne użytkowanie czatu. Aktor
  ,,UserSession'' gromadzi informacje na temat nazwy oraz poziomu uprawnień
  użytkownika, a także tego, z jakim pokojem jest obecnie spięty.

  \item[\texttt{Landlord}] Jego zadaniem jest współudział w podpinaniu
  użytkowników do pokoju, tworzeniem nowych i usuwaniem istniejących pokojów, a
  także utrzymywanie i udostępnianie kompletnej listy aktywnych pokojów.

  \item[\texttt{Room}] Działa jak router wiadomości i przechowuje listę użytkowników którzy są do niego wpięci. Istnieje w tylu egzemplarzach, ile jest aktywnych pokojów.

  \item[\texttt{MessageCache}] Przechowuje i odtwarza 10 najnowszych wiadomości wysłanych do pokoju. Występuje po jednym egzemplarzu dla każdego aktywnego aktora \texttt{Room}.

\end{labeling}

\newpage

\subsubsection{Warstwa interfejsu użytkownika (,,frontend'')}
Podczas pracy nad wartwą frontendu, zastosowano framework webowy Vue.js. Jedną
z przyczyn dla tej decyzji jest możliwość zdekomponowania projektowanej aplikacji na mniejsze części, nazywane modułami (ang. \textit{modules}). Są one zorganizowane hierarchiczne. Każdy z modułów zawiera własny skrypt JavaScript, a także kod HTML i arkusz CSS. Powoduje to, że każdy z modułów jest niezależny od pozostałych i może realizować swoje zadania w pełni autonomicznie.

Każdy z modułów udostępnia swojemu rodzicowi pewne określone parametry. Ich zmiana
jest podstawowym sposobem na interakcję pomiędzy nimi, co jest zgodne z
paradygmatem \textit{data driven application} (z ang. ,,aplikacja sterowana
poprzez dane''). W podobny sposób następuje zmiana kodu HTML modułów. Zamiast
zmieniać węzły DOM w sposób jawny poprzez skrypt, programista wskazuje w
szablonie HTML te miejsca, które ulegają określonym przemianom wraz ze
zmian wewnętrznych parametrów modułu.

Podstawowa struktura aplikacji frontendowej przewiduje podział na
pięć części:
\begin{itemize}
	\item Część autoryzacyjna -- służy do nawiązania połączenia z
	serwerem i rozpoczęcie sesji użytkownika.

	\item Pokoje (publiczne) -- dostarcza listę ogólnodostępnych pokojów,
	umożliwia podpięcie się wybranych z nich oraz prowadzenie rozmów z
	innymi użytkownikami, którzy się do nich podpięli.

	\item Wiadomości prywatne -- umożliwia nawiązywanie prywatnych
	rozmów pomiędzy użytkownikami oraz wymianę wiadomości prywatnych.

	\item Narzędzia administracyjne -- pozwalają administratorom
	na dodawanie i usuwanie pokojów, wyrzucanie użytkowników z pokojów
	i z serwera.

	\item Profil użytkownika -- umożliwia podejrzenie informacji na
	temat własnego konta, zmianę przez administratora swojego hasła
	użytkownika oraz rozłączenie się z serwerem.

\end{itemize}

W projekcie przewidziano zastosowanie modułów, przedstawionych na rysunku \ref{diag-komp-front}.

\nameref{diag-komp-front}.
\begin{figure}[!htp]
	\centering
	\includegraphics[width=\textwidth]{chat/fig/pck-diag-front}
	\caption{Diagram komponentów aplikacji webowej ViuaChat}
	\label{diag-komp-front}
\end{figure}

\begin{labeling}{\texttt{UserProfile}}

  \item[\texttt{App}] Nadrzędny moduł, istniejący przez cały czas użycia
	instancji aplikacji. Obejmuje najbardziej zewnętrzne struktury HTML,
	przechowuje podstawowy stan aplikacji (stan autoryzacji, nazwę
	użytkownika, poziom uprawnień, gniazdo WebSocket dla połączeń
	z wartwą backendu), a także zapewnia routing do głównych części
	aplikacji - czyli modułu logowania, modułu pokojów (publicznych),
	pokojów prywatnych wiadomości, profilu użytkownika i narzędzi
	administracyjnych.

	\item[\texttt{SignIn}] Moduł odpowiadający za początkową autoryzację
	do serwera (potocznie znane jako ,,logowanie'').

	\item[\texttt{Rooms}] Moduł reprezentuje część aplikacji poświęconą
	ogólnodostępnym pokojom dla wspólnych konwersacji. Podlegają mu:

	\begin{labeling}{\texttt{RoomsChat}}
		\item[\texttt{RoomsChat}] Moduł odpowiedzialny \textit{stricte} za
		okno czatu - wyświetlanie konwersacji oraz wysyłanie wiadomości
		do pokoju.

		\item[\texttt{RoomsList}] Moduł wyświetlający listę pokojów, a także
		umożliwiający podpięcie się do wybranego z nich.

	\end{labeling}

	\item[\texttt{PM}] Moduł odwzorowujący całą część aplikacji poświęconą
	wymianie pomiędzy użytkownikami wiadomości prywatnych. W jego skład
	wchodzą:

	\begin{labeling}{\texttt{PMUserList}}
		\item[\texttt{PMFindUser}] Moduł, którego zadaniem jest odnalezienie
		użytkowika, do którego ma zostać wysłana wiadomość prywatna

		\item[\texttt{PMUserList}] Moduł obsługujący listę użytkowników,
		którzy wcześniej otrzymali lub wysłali wiadomości prywatne.

		\item[\texttt{PMChat}] Moduł odpowiedzialny za obsługę zasadniczego
		okna czatu wiadomości prywatnych, pozwalającego wymieniać wiadomości
		prywatne z jednym, wybranym użytkownikem.
	\end{labeling}

\item[\texttt{UserProfile}] Moduł pozwalający na podstawową kontrolę
własnego profilu, w tym:

\begin{labeling}{\texttt{ChangePassword}}
	\item[\texttt{SignOut}] Moduł odpowiedzialny za zamknięcie sesji
	użytkownika, rozłączenie z serwerem oraz powrót aplikacji do
	stanu sprzed autoryzacji

	\item[\texttt{ChangePassword}] Moduł mający umożliwiać administratorom
	zmianę swoich własnych haseł

\end{labeling}

\item[\texttt{Admin}] Moduł obejmujący pod sobą wszystkie narzędzia
administracyjne. W jego skład wchodzą:

\begin{labeling}{\texttt{BanFromServer}}
	\item[\texttt{AdminRooms}] Moduł do zarządzania ogólnodostępnymi
	pokojami czatu, w tym:

	\begin{labeling}{\texttt{AdminRoomsList}}
		\item[\texttt{AdminRoomsList}] Moduł listy wszystkich ogólnodostępnych
		pokojów z narzędziami do ich edycji

		\item[\texttt{CreateRoom}] Moduł pozwalający na dodawanie pokojów
		do serweru czatu

		\item[\texttt{RemoveRoom}] Moduł pozwalający na usuwanie istniejących
		pokojów z serwera
	\end{labeling}

	\item[\texttt{BanFromRoom}] Moduł służący do wyrzucania użytkowników
	z pokojów, w których obecnie się znajdują.

	\item[\texttt{BanFromServer}] Moduł pozwalający na wyrzucenie
	użytkowników z serwera.
\end{labeling}

\end{labeling}
\section{Decyzje projektowe}

\subsection{Środowisko docelowe}

\subsection{Środowisko implementacji}

\subsection{Priorytety implementacyjne}

\section{Projekt algorytmów i przyjętych protokołów}

\subsection{Protokół frontend-backend}
Komunikacja pomiędzy frontendem a backendem...

\section{Projekt rozwiązań sprzętowych}

\section{Projekt interfejsu}

\subsection{Interfejs użytkownika}

\subsubsection{Założenia konstrukcji interfejsu}

\section{Projekt bazy danych}

\section{Opis implementacji}


\section{Diagramy Przypadków Użycia}

\subsection{Diagram Przypadków Użycia}

Diagram przypadków użycia jest pokazany na rysunku \ref{dpu-chat-v2}
\nameref{dpu-chat-v2}.
\begin{figure}[!htp]
	\centering
	\includegraphics[width=\textwidth]{chat/fig/dpu-chat-v2}
	\caption{Diagram przypadków użycia aplikacji ViuaChat}
	\label{dpu-chat-v2}
\end{figure}

\subsection{Opis aktorów}

\textit{Uwaga! W niniejszej sekcji, słowo ,,aktor'' jest używane w rozumieniu
języka UML, a nie języka \ViuAct.}

\begin{enumerate}
	\item \textbf{Użytkownik czatu.} Typ użytkownika, którego konto
	jest tworzone podczas połączenia z serwerem czatu oraz niszczone po jego
	zakończeniu. Podczas łączenia z czatem, nie będzie musiał się autoryzować przy
	użyciu hasła, a deklarować tylko unikalną nazwę, nie powtarzającą się z nazwą
	innego użytkownika, posiadającego konto na danym serwerze czatu. Ten typ konta
	jest przeznaczony dla osób, zainteresowanych dołączeniem do dyskusji na czacie
	bez dodatkowych zobowiązań. Jest aktorem aktywnym i głównym.

	\item \textbf{Administrator.} To użytkownik, który jest dodatkowo
	wyróżniony i posiada uprawienia do szeroko pojętego zarządzania serwerem (w
	tym - pozostałymi użytkownikami). Konto administratora jest
	utrzymywane przez serwer pomiędzy połączeniami do czatu. Każdorazowo, przed
	rozpoczęciem sesji połączenia z serwerem, administratorzy muszą się dodatkowo
	autoryzować przyużyciu hasła. Równocześnie, ich nazwa jest zarezerwowana
	wyłącznie do jego	użytku oraz niedostępna	dla użytkowników tymczasowych. Nie
	wyróżnia się wśród administratorów żadnych dodatkowych, szczególnych ról
	(np. superadministrator, właściciel). Jest aktorem aktywnym i głównym.

\end{enumerate}

\subsection{Opis przypadków użycia}

Poniżej opisano przypadki użycia ujęte na rysunku \ref{dpu-chat-v2}
\nameref{dpu-chat-v2}.

{\footnotesize

\vspace{2em}

\begin{tabularx}{\textwidth}{|l|X|}
	\hline
		\textbf{Identyfikator} &
		UC-01
		\\

	\hline
		\textbf{Nazwa} &
		Logowanie anonimowe
		\\

	\hline
		\textbf{Aktorzy} &
			Użytkownik czatu
		\\

	\hline
		\textbf{Streszczenie} &
			Użytkownik rozpoczyna korzystanie z czatu pod wybraną nazwą
			użytkownika, ale bez konieczności podawania hasła.
		\\

	\hline
		\textbf{Warunek wstępny} &
			Użytkownik nie ma rozpoczętej sesji połączenia z serwerem
		\\

	\hline
		\textbf{Wyjątki} &
			\begin{itemize}
				\item Użytkownik ma już wcześniej rozpoczętą sesję z serwerem
			\end{itemize}
		\\

	\hline
		\textbf{Scenariusz podstawowy} &
			\begin{enumreq}
				\item Użytkownik wprowadza nazwę użytkownika i zatwierdza
				\item System sprawdza czy nazwa użytkownika jest wolna
				\item Gdy nazwa jest wolna, serwer rozpoczyna sesję	z użytkownikiem
			\end{enumreq}
		\\

	\hline
		\textbf{Scenariusze alternatywne} &
			\begin{enumreq}
				\item Gdy nazwa użytkownika jest zajęta (przez zalogowanego
				użytkownika lub stałe konto użytkownika), logowanie nie
				powiedzie się.
			\end{enumreq}
		\\

	\hline
		\textbf{Warunek końcowy} &
			Użytkownik ma rozpoczętą sesję z serwerem.
		\\

	\hline
		\textbf{Komentarz} &
			\textit{Nie zamieszczono}
		\\

	\hline
\end{tabularx}

\vspace{2em}

\begin{tabularx}{\textwidth}{|l|X|}
	\hline
		\textbf{Identyfikator} &
		UC-02
		\\

	\hline
		\textbf{Nazwa} &
		Logowanie z autoryzacją
		\\

	\hline
		\textbf{Aktorzy} &
			Administrator
		\\

	\hline
		\textbf{Streszczenie} &
			Administrator rozpoczyna korzystanie z czatu z wykorzystaniem stałego
			konta użytkownika. Podczas logowania, administrator, oprócz podana samej
			nazwy użytkownika, dodatkowo uwierzytelnia się hasłem, aby zweryfikować
			czy ma prawo do wykorzystywania konta.
		\\

	\hline
		\textbf{Warunek wstępny} &
			\begin{enumerate}
				\item Użytkownik nie ma rozpoczętej żadnej sesji połączenia z serwerem.
				\item Użytkownik posiadał wcześniej skonfigurowane konto administratora
				na serwerze
			\end{enumerate}
		\\

	\hline
		\textbf{Wyjątki} &
			\begin{itemize}
				\item Użytkownik ma już wcześniej rozpoczętą sesję z serwerem
			\end{itemize}
		\\

	\hline
		\textbf{Scenariusz podstawowy} &
			\begin{enumerate}
				\item Użytkownik wprowadza nazwę użytkownika, hasło	i zatwierdza
				\item Gdy istnieje konto administratora o wskazanej nazwie, a podane
				hasło jest z nim zgodne, wówczas serwer rozpoczyna sesję z
				użytkownikiem.
			\end{enumerate}
		\\

	\hline
		\textbf{Scenariusze alternatywne} &
			\begin{enumerate}
				\item Gdy nie istnieje konto administratora o podanej nazwie,
				próba zalogowania z hasłem nie powiedzie się - w przypadku logowania bez
				użycia hasła, patrz UC-01.
				\item Gdy podane hasło nie jest zgodne z hasłem do konta administratora
				o podanej nazwie, logowanie nie powiedzie	się.
			\end{enumerate}
		\\

	\hline
		\textbf{Warunek końcowy} &
			Użytkownik ma rozpoczętą sesję z serwerem.
		\\

	\hline
		\textbf{Komentarz} &
			\textit{Nie zamieszczono}
		\\

	\hline
\end{tabularx}

\vspace{2em}

\begin{tabularx}{\textwidth}{|l|X|}
	\hline
		\textbf{Identyfikator} &
		UC-03
		\\

	\hline
		\textbf{Nazwa} &
		Wyświetlanie listy pokojów
		\\

	\hline
		\textbf{Aktorzy} &
			Użytkownik czatu
		\\

	\hline
		\textbf{Streszczenie} &
			Użytkownik uzyskuje wgląd do listy pokojów z której może wybrać ten, do
			którego chce się podpiąć.
		\\

	\hline
		\textbf{Warunek wstępny} &
			\begin{enumreq}
				\item Użytkownik ma rozpoczętą sesję z serwerem
			\end{enumreq}
		\\

	\hline
		\textbf{Wyjątki} &
			\begin{enumreq}
				\item Użytkownik nie może być wcześniej podpięty do żadnego pokoju.
			\end{enumreq}
			\textit{Brak}
		\\

	\hline
		\textbf{Scenariusz podstawowy} &
			\begin{enumreq}
				\item Użytkownik wybiera z górnego menu opcję ,,Pokoje''.
				\item Pokazywany jest ekran listy pokojów.
				\item Użytkownik wybiera z listy nazwę pokoju, do którego chce
				się wpiąć.
			\end{enumreq}
		\\

	\hline
		\textbf{Scenariusze alternatywne} &
			\begin{enumreq}
				\item Wiadomości prywatne są wysyłane z okna czatu w
				specyficzny sposób (patrz UC-10)
				\item Tuż po zalogowaniu do serwera, scenariusz podstawowy zaczyna się
				od kroku 2.
			\end{enumreq}
		\\

	\hline
		\textbf{Warunek końcowy} &
			Użytkownik wybrał pokój do wpięcia się.
		\\

	\hline
		\textbf{Komentarz} &
			\textit{Nie zamieszczono}
		\\

	\hline
\end{tabularx}

\vspace{2em}

\begin{tabular}{ | l | l | }
	\hline
		\textbf{Identyfikator} &
		UC-04
		\\

	\hline
		\textbf{Nazwa} &
		Podpinanie się pod pokój
		\\

	\hline
		\textbf{Aktorzy} & \parbox[t]{11cm}{
			Użytkownik czatu
		}\\

	\hline
		\textbf{Streszczenie} & \parbox[t]{11cm}{
			Użytkownik czatu, po wybraniu pokoju z listy pokojów, jest
			do niego wpinany.

		}\\

	\hline
		\textbf{Warunek wstępny} & \parbox[t]{11cm}{
			\begin{enumreq}
				\item Użytkownik wybrał pokój z listy pokojów.
			\end{enumreq}

		}
		\\

	\hline
		\textbf{Wyjątki} & \parbox[t]{11cm}{
			\begin{enumreq}
				\item Użytkownik nie może być wcześniej wpięty do żadnego pokoju.
			\end{enumreq}

		}
		\\

	\hline
		\textbf{Scenariusz podstawowy} & \parbox[t]{11cm}{
			\begin{enumreq}
				\item Użytkownik wybiera pokój z listy
				\item Serwer weryfikuje, czy użytkownik nie był już wcześniej wpięty do
				innego pokoju
				\item Jeżeli użytkownik pozostawał wcześniej niepodpięty, serwer
				sprawdza, czy wybrany pokój nadal istnieje,
				\item Jeżeli pokój nadal istnieje, następuje podpięcie użytkownika pod
				wybrany pokój.
				\item Uzytkownikowi, który zostaje nowo podpięty do pokoju,	pokazywane
				jest 10 wiadomości wysłanych tuż przed jego dołączeniem do tego pokoju.
			\end{enumreq}
		}
		\\

	\hline
		\textbf{Scenariusze alternatywne} & \parbox[t]
		{11cm}{
			\begin{enumreq}
				\item Gdy użytkownik był wcześniej wpięty do innego pokoju,	najpierw
				zostaje od niego odpięty (UC-05), a dopiero później zostaje wpięty do
				wybranego pokoju.
			\end{enumreq}
		}
		\\

	\hline
		\textbf{Warunek końcowy} & \parbox[t]{11cm}{
			Użytkownik został podpięty pod pokój.
		}
		\\

	\hline
		\textbf{Komentarz} & \parbox[t]{11cm}{
			\textit{Nie zamieszczono}
		}
		\\

	\hline
\end{tabular}

\vspace{2em}

\begin{tabular}{ | l | l | }
	\hline
		\textbf{Identyfikator} &
		UC-05
		\\

	\hline
		\textbf{Nazwa} &
		Odpinanie się od pokoju
		\\

	\hline
		\textbf{Aktorzy} & \parbox[t]{11cm}{
			Użytkownik czatu
		}\\

	\hline
		\textbf{Streszczenie} & \parbox[t]{11cm}{
			Użytkownik, który był wcześniej wpięty do pokoju, może się od niego
			odpiąć, aby wpiąć się do innego pokoju lub po prostu zrezygnować z dalszej
			konwersacji.
		}\\

	\hline
		\textbf{Warunek wstępny} & \parbox[t]{11cm}{
			\begin{enumreq}
				\item Użytkownik ma rozpoczętą sesję z serwerem
				\item Użytkownik jest podpięty do pokoju
			\end{enumreq}

		}
		\\

	\hline
		\textbf{Wyjątki} & \parbox[t]{11cm}{
			\textit{Brak}
		}
		\\

	\hline
		\textbf{Scenariusz podstawowy} & \parbox[t]{11cm}{
			\begin{enumreq}
				\item Użytkownik wybiera przycisk ,,Opuść pokój''.
				\item Serwer weryfikuje, czy użytkownik nadal jest podpięty pod pokój.
				\item Jeżeli użytkownik jest nadal podpięty, następuje odpięcie.
				\item Użytkownik zostaje przekierowany do listy pokojów (UC-03 od kroku
				2)
			\end{enumreq}
		}
		\\

	\hline
		\textbf{Scenariusze alternatywne} & \parbox[t]
		{11cm}{
			\begin{enumreq}
				\item Gdy użytkownik pozostawał wcześniej niepodpięty, akcja kończy się
				niepowodzeniem.
			\end{enumreq}
		}
		\\

	\hline
		\textbf{Warunek końcowy} & \parbox[t]{11cm}{
			Użytkownik zostaje odpięty od pokoju.
		}
		\\

	\hline
		\textbf{Komentarz} & \parbox[t]{11cm}{
			\textit{Nie zamieszczono}
		}
		\\

	\hline
\end{tabular}

\vspace{2em}

\begin{tabular}{ | l | l | }
	\hline
		\textbf{Identyfikator} &
		UC-06
		\\

	\hline
		\textbf{Nazwa} &
		Pisanie wiadomości w pokoju
		\\

	\hline
		\textbf{Aktorzy} & \parbox[t]{11cm}{
			Użytkownik czatu
		}\\

	\hline
		\textbf{Streszczenie} & \parbox[t]{11cm}{
			Użytkownik może napisać wiadomość w pokoju czatu, którą zobaczą
			inni użytkownicy podpięci do tego pokoju (włącznie z jej nadawcą)
		}\\

	\hline
		\textbf{Warunek wstępny} & \parbox[t]{11cm}{
			\begin{enumreq}
				\item Użytkownik ma rozpoczętą sesję z serwerem
				\item Użytkownik jest podpięty do pokoju
			\end{enumreq}

		}
		\\

	\hline
		\textbf{Wyjątki} & \parbox[t]{11cm}{
			\textit{Brak}
		}
		\\

	\hline
		\textbf{Scenariusz podstawowy} & \parbox[t]{11cm}{
			\begin{enumreq}
				\item Użytkownik pisze wiadomość w polu tekstowym pod wiadomościami
				czatu.
				\item Po zatwierdzeniu wiadomości do wysyłki, pole tekstowe jest
				czyszczone.
				\item Gdy użytkownik jest nadal podpięty do pokoju, wiadomość zostaje
				wpisana do listy wiadomości i rozesłana do wszystkich	użytkowników.
				\item Użytkownik widzi swoją wiadomość wyświetloną u dołu ekranu.
			\end{enumreq}
		}
		\\

	\hline
		\textbf{Scenariusze alternatywne} & \parbox[t]
		{11cm}{
			\begin{enumreq}
				\item Gdy użytkownik nie był wpięty do pokoju w momencie wysyłania
				wiadomości, wysyłka kończy się niepowodzeniem
				\item Gdy wiadomość jest poprzedzona znakiem ..\#'', to jest realizowany
				scenariusz UC-11
			\end{enumreq}
		}
		\\

	\hline
		\textbf{Warunek końcowy} & \parbox[t]{11cm}{
			Wiadomość została zaakceptowania do rozesłania przez serwer.
		}
		\\

	\hline
		\textbf{Komentarz} & \parbox[t]{11cm}{
			\textit{Nie zamieszczono}
		}
		\\

	\hline
\end{tabular}

\vspace{2em}

\begin{tabular}{ | l | l | }
	\hline
		\textbf{Identyfikator} &
		UC-08
		\\

	\hline
		\textbf{Nazwa} &
		Wyświetlanie wiadomości prywatnych w pokoju
		\\

	\hline
		\textbf{Aktorzy} & \parbox[t]{11cm}{
			Użytkownik czatu
		}\\

	\hline
		\textbf{Streszczenie} & \parbox[t]{11cm}{
			Użytkownik widzi wiadomości prywatne, skierowane do niego przez
			innego użytkownika podpiętego do tego samego pokoju. Wiadomości
			tych nie widzi żaden inny użytkownik podpięty do pokoju, oprócz
			jej nadawcy i odbiorcy.

		}\\

	\hline
		\textbf{Warunek wstępny} & \parbox[t]{11cm}{
			\begin{enumreq}
				\item Użytkownik ma rozpoczętą sesję z serwerem
				\item Użytkownik jest podpięty do pokoju
				\item Użytkownik wysłał wiadomość do pokoju (patrz UC-07)
				poprzedzoną znakiem ,,\#''
			\end{enumreq}

		}
		\\

	\hline
		\textbf{Wyjątki} & \parbox[t]{11cm}{
			\textit{Brak}

		}
		\\

	\hline
		\textbf{Scenariusz podstawowy} & \parbox[t]{11cm}{
			\begin{enumreq}
				\item Użytkownik wysyła do pokoju wiadomość poprzedzoną znakiem ,,\#''.
				\item Serwer sprawdza, czy przed treścią wiadomości znajduje się łańcuch
				znaków będący prawidłową nazwą użytkownika.
				\item Jeżeli forma nazwy jest prawidłowa, serwer sprawdza, czy
				użytkownik wskazany na początku wiadomości jest podpięty do pokoju.
				\item Jeżeli użytkownik jest podpięty do pokoju, wiadomość zostaje
				przyjęta przez serwer jako wiadomość prywatna.
			\end{enumreq}
		}
		\\

	\hline
		\textbf{Scenariusze alternatywne} & \parbox[t]
		{11cm}{
			\begin{enumreq}
				\item Gdy nazwa użytkownika odbiorcy ma nieprawidłową formę, wiadomość
				nie zostanie przyjęta przez serwer i wysyłka zakończy się niepowodzeniem
				\item Gdy odbiorca wiadomości nie jest podpięty pod pokój, wiadomość nie
				zostanie przyjęta przez serwer i wysyłka zakończy się niepowodzeniem.
			\end{enumreq}
		}
		\\

	\hline
		\textbf{Warunek końcowy} & \parbox[t]{11cm}{
			Wiadomość prywatna została przyjęta przez serwer.
		}
		\\

	\hline
		\textbf{Komentarz} & \parbox[t]{11cm}{
			\textit{Nie zamieszczono}
		}
		\\

	\hline
\end{tabular}

\vspace{2em}

\begin{tabular}{ | l | l | }
	\hline
		\textbf{Identyfikator} &
		UC-09
		\\

	\hline
		\textbf{Nazwa} &
		Wyświetlanie okna wiadomości prywatnych
		\\

	\hline
		\textbf{Aktorzy} & \parbox[t]{11cm}{
			Użytkownik czatu
		}\\

	\hline
		\textbf{Streszczenie} & \parbox[t]{11cm}{
			Użytkownik może zobaczyć okno z wiadomościami prywatnymi (niezależnie od
			tego czy zostały wysłane z pokoju czy z okna wiadomości prywatnych),
			pogrupowane wg ich nadawców/odbiorców
		}\\

	\hline
		\textbf{Warunek wstępny} & \parbox[t]{11cm}{
			\begin{enumreq}
				\item Użytkownik ma rozpoczętą sesję z serwerem.
			\end{enumreq}

		}
		\\

	\hline
		\textbf{Wyjątki} & \parbox[t]{11cm}{
			\textit{Brak}

		}
		\\

	\hline
		\textbf{Scenariusz podstawowy} & \parbox[t]{11cm}{
			\begin{enumreq}
				\item Użytkownik wybiera z menu opcję ,,PW''
				\item Użytkownikowi zostaje pokazana lista nazw użytkowników,	od których
				otrzymał lub którym wysyłał wiadomości prywatne.
			\end{enumreq}
		}
		\\

	\hline
		\textbf{Scenariusze alternatywne} & \parbox[t]
		{11cm}{
			\begin{enumreq}
				\item Gdy użytkownik nie wysłał ani nie odebrał żadnych wiadomości
				prywatnych, lista nazw użytkowników będzie pusta.
			\end{enumreq}
		}
		\\

	\hline
		\textbf{Warunek końcowy} & \parbox[t]{11cm}{
			Użytkownik zobaczy listę nazw użytkowników, od których otrzymał lub którym
			wysłał wiadomości prywatne.
		}
		\\

	\hline
		\textbf{Komentarz} & \parbox[t]{11cm}{
			\textit{Nie zamieszczono}
		}
		\\

	\hline
\end{tabular}

\vspace{2em}

\begin{tabular}{ | l | l | }
	\hline
		\textbf{Identyfikator} &
		UC-10
		\\

	\hline
		\textbf{Nazwa} &
		Wyświetlanie prywatnych wiadomości od wskazanego użytkownika
		\\

	\hline
		\textbf{Aktorzy} & \parbox[t]{11cm}{
			Użytkownik czatu
		}\\

	\hline
		\textbf{Streszczenie} & \parbox[t]{11cm}{
			Użytkownik musi wybrać konkretnego innego użytkownika, aby zobaczyć jego
			wiadomości (tj. wiadomości prywatne, których jest nadawcą/odbiorcą).

		}\\

	\hline
		\textbf{Warunek wstępny} & \parbox[t]{11cm}{
			\begin{enumreq}
				\item Użytkownik wybrał nazwę użytkownika w oknie wiadomości prywatnych.
			\end{enumreq}

		}
		\\

	\hline
		\textbf{Wyjątki} & \parbox[t]{11cm}{
		\begin{enumreq}
		 \item Użytkownik nie wysłał ani nie odebrał dotychczas żadnych wiadomości
		 prywatnych.
	 	\end{enumreq}

		}
		\\

	\hline
		\textbf{Scenariusz podstawowy} & \parbox[t]{11cm}{
			\begin{enumreq}
				\item Użytkownik wybiera jedną z nazw, którą widzi na liście w oknie
				wiadomości prywatnych
				\item Użytkownikowi pokazywana jest lista wiadomości prywatnych, które
				otrzymał od tego użytkownika lub do których je skierował.
			\end{enumreq}
		}
		\\

	\hline
		\textbf{Scenariusze alternatywne} & \parbox[t]
		{11cm}{
			\begin{enumreq}
				\item Gdy wybrany użytkownik nie istnieje i/lub nie jest połączony z
				serwerem, operacja zakończy się błędem.
			\end{enumreq}
		}
		\\

	\hline
		\textbf{Warunek końcowy} & \parbox[t]{11cm}{
			Użytkownik zobaczył wiadomości prywatne, które odebrał od lub nadał do
			konkretnego użytkownika.
		}
		\\

	\hline
		\textbf{Komentarz} & \parbox[t]{11cm}{
			\textit{Nie zamieszczono}
		}
		\\

	\hline
\end{tabular}

\vspace{2em}

\begin{tabular}{ | l | l | }
	\hline
		\textbf{Identyfikator} &
		UC-11
		\\

	\hline
		\textbf{Nazwa} &
		Pisanie prywatnych wiadomości
		\\

	\hline
		\textbf{Aktorzy} & \parbox[t]{11cm}{
			Użytkownik czatu
		}\\

	\hline
		\textbf{Streszczenie} & \parbox[t]{11cm}{
			Użytkownik w oknie wiadomości prywatnych pisze wiadomości, które są
			domyślnie uznawane za wiadomości prywatne skierowane do użytkownika, który
			został wcześniej wybrany.
		}\\

	\hline
		\textbf{Warunek wstępny} & \parbox[t]{11cm}{
			\begin{enumreq}
				\item Użytkownik jest w oknie wiadomości prywatnych.
				\item Użytkownik wybrał, czyje wiadomości ogląda (patrz UC-10)
			\end{enumreq}

		}
		\\

	\hline
		\textbf{Wyjątki} & \parbox[t]{11cm}{
			\begin{enumreq}
				\item Użytkownik wybrany w oknie czatu rozłączył się z serwerem, zanim
				wiadomość została wysłana.
			\end{enumreq}
		}
		\\

	\hline
		\textbf{Scenariusz podstawowy} & \parbox[t]{11cm}{
			\begin{enumreq}
				\item Użytkownik wpisuje wiadomość do pola tekstowego pod wiadomościami
				w oknie wiadomości prywatnych
				\item Użytkownik decyduje o wysłaniu wiadomości.
				\item Pole tekstowe zostaje wyczyszczone.
				\item Serwer przyjmuje wiadomość jako wiadomość prywatną i rozsyła do
				nadawcy i odbiorcy
				\item Użytkownik widzi wysłaną wiadomość u dołu pola tekstowego.
			\end{enumreq}
		}
		\\

	\hline
		\textbf{Scenariusze alternatywne} & \parbox[t]
		{11cm}{
			\begin{enumreq}
				\item Gdy wybrany użytkownik odłączył się od serwera, wysłanie
				wiadomości kończy się niepowodzeniem.
			\end{enumreq}
		}
		\\

	\hline
		\textbf{Warunek końcowy} & \parbox[t]{11cm}{
			Wiadomość została pokazana nadawcy i odbiorcy.
		}
		\\

	\hline
		\textbf{Komentarz} & \parbox[t]{11cm}{
			Wiadomości wysyłane z okna wiadomości prywatych nie muszą być
			poprzedzane znakiem ,,\#'' i nazwą odbiorcy. Użytkownik docelowy
			jest wnioskowany z tego, czyje wiadomości są obecnie pokazywane w
			oknie wiadomości prywatnych (patrz UC-10)
		}
		\\

	\hline
\end{tabular}

\vspace{2em}

\begin{tabular}{ | l | l | }
	\hline
		\textbf{Identyfikator} &
		UC-12
		\\

	\hline
		\textbf{Nazwa} &
		Wyrzucanie użytkowników z serwera
		\\

	\hline
		\textbf{Aktorzy} & \parbox[t]{11cm}{
			Administrator
		}\\

	\hline
		\textbf{Streszczenie} & \parbox[t]{11cm}{
			Administrator ma prawo w dowolnym momencie przerwać połączenie dowolnego
			użytkownika z serwerem.

		}\\

	\hline
		\textbf{Warunek wstępny} & \parbox[t]{11cm}{
			\begin{enumreq}
				\item Administrator ma rozpoczętą sesję z serwerem.
				\item Wybrany uzytkownik ma rozpoczetą sesję z serwerem i nadal jest z
				nim połączony.
			\end{enumreq}

		}
		\\

	\hline
		\textbf{Wyjątki} & \parbox[t]{11cm}{
			Nie jest możliwe wyrzucenie samego siebie.
		}
		\\

	\hline
		\textbf{Scenariusz podstawowy} & \parbox[t]{11cm}{
			\begin{enumreq}
				\item Administrator klika nazwę użytkownika (niezależnie od miejsca, w
				którym została wyświetlona).
				\item Z menu, administrator wybiera opcję ,,Wyrzuć z serwera''.
				\item Wskazany użytkownik zostaje niezwłocznie odpięty z pokoju (o ile
				był podpięty do któregokolwiek).
				\item Połączenie wybranego użytkownika zostaje zakończone.
			\end{enumreq}
		}
		\\

	\hline
		\textbf{Scenariusze alternatywne} & \parbox[t]
		{11cm}{
			\begin{enumreq}
				\item Gdy wybrany użytkownik nie jest już połączony z serwerem, akcja
				kończy się niepowodzeniem.
			\end{enumreq}
		}
		\\

	\hline
		\textbf{Warunek końcowy} & \parbox[t]{11cm}{
			Wskazany użytkownik został odłączony od serwera.
		}
		\\

	\hline
		\textbf{Komentarz} & \parbox[t]{11cm}{
			\textit{Nie zamieszczono}
		}
		\\

	\hline
\end{tabular}

\vspace{2em}

\begin{tabular}{ | l | l | }
	\hline
		\textbf{Identyfikator} &
		UC-13
		\\

	\hline
		\textbf{Nazwa} &
		Wyrzucanie użytkowników z pokojów
		\\

	\hline
		\textbf{Aktorzy} & \parbox[t]{11cm}{
			Administrator
		}\\

	\hline
		\textbf{Streszczenie} & \parbox[t]{11cm}{
			Administrator może odpiąć wybranego użytkownika od pokoju, do którego jest
			obecnie wpięty.
		}\\

	\hline
		\textbf{Warunek wstępny} & \parbox[t]{11cm}{
			\begin{enumreq}
				\item Administrator ma rozpoczętą sesję z serwerem.
				\item Wybrany użytkownik jest wpięty do jakiegokolwiek pokoju.
			\end{enumreq}
		}
		\\

	\hline
		\textbf{Wyjątki} & \parbox[t]{11cm}{
			\textit{Brak}

		}
		\\

	\hline
		\textbf{Scenariusz podstawowy} & \parbox[t]{11cm}{
			\begin{enumreq}
				\item Administrator klika nazwę użytkownika, przebywając w oknie pokoju.
				\item Z menu, administrator wybiera opcję ,,Wyrzuć z pokoju''.
				\item Wskazany użytkownik zostaje niezwłocznie odpięty z pokoju.
			\end{enumreq}
		}
		\\

	\hline
		\textbf{Scenariusze alternatywne} & \parbox[t]
		{11cm}{
			\begin{enumreq}
				\item Gdy użytkownik, przed podjęciem przez administratora decyzji o
				wyrzuceniu, sam wypiął się z pokoju lub zerwał połączenie z serwerem,
				operacja kończy się niepowodzeniem.
			\end{enumreq}
		}
		\\

	\hline
		\textbf{Warunek końcowy} & \parbox[t]{11cm}{
			Użytkownik został wypięty z pokoju.
		}
		\\

	\hline
		\textbf{Komentarz} & \parbox[t]{11cm}{
			\textit{Nie zamieszczono}
		}
		\\

	\hline
\end{tabular}

\vspace{2em}

\begin{tabular}{ | l | l | }
	\hline
		\textbf{Identyfikator} &
		UC-14
		\\

	\hline
		\textbf{Nazwa} &
		Tworzenie pokojów
		\\

	\hline
		\textbf{Aktorzy} & \parbox[t]{11cm}{
			Administrator
		}\\

	\hline
		\textbf{Streszczenie} & \parbox[t]{11cm}{
			Administrator ma prawo tworzyć i usuwać pokoje

		}\\

	\hline
		\textbf{Warunek wstępny} & \parbox[t]{11cm}{
			\begin{enumreq}
				\item Administrator ma rozpoczętą sesję z serwerem
			\end{enumreq}

		}
		\\

	\hline
		\textbf{Wyjątki} & \parbox[t]{11cm}{
			\textit{Brak}

		}
		\\

	\hline
		\textbf{Scenariusz podstawowy} & \parbox[t]{11cm}{
			\begin{enumreq}
				\item Administrator wchodzi w listę pokojów
				\item Administrator klika w ikonę plusa obok nagłówka listy
				\item Pokazany zostaje okno utworzenia nowego pokoju
				\item W oknie utworzenia nowego pokoju, administrator wpisuje nazwę
				pokoju.
				\item Administrator zatwierdza utworzenie pokoju.
				\item Jeżeli nazwa pokoju jest prawidłowa pod względem wymagań
				biznesowych, tworzony jest nowy pokój.
			\end{enumreq}
		}
		\\

	\hline
		\textbf{Scenariusze alternatywne} & \parbox[t]
		{11cm}{
			\begin{enumreq}
				\item Gdy nazwa pokoju pokrywa się z istniejącym pokojem lub nie spełnia
				wymagań biznesowych, administrator jest poproszony o podanie takiej,
				która jest prawidłowa.
				\item Gdy administrator wyjdzie z okna utworzenia nowego pokoju bez
				jego zatwierdzenia, pokój nie zostaje dodany.
			\end{enumreq}
		}
		\\

	\hline
		\textbf{Warunek końcowy} & \parbox[t]{11cm}{
			Pokój został utworzony.
		}
		\\

	\hline
		\textbf{Komentarz} & \parbox[t]{11cm}{
			\textit{Nie zamieszczono}
		}
		\\

	\hline
\end{tabular}

\vspace{2em}

\begin{tabular}{ | l | l | }
	\hline
		\textbf{Identyfikator} &
		UC-15
		\\

	\hline
		\textbf{Nazwa} &
		Usuwanie pokojów.
		\\

	\hline
		\textbf{Aktorzy} & \parbox[t]{11cm}{
			Administrator
		}\\

	\hline
		\textbf{Streszczenie} & \parbox[t]{11cm}{
			Administrator ma prawo usuwać pokoje z serwera.

		}\\

	\hline
		\textbf{Warunek wstępny} & \parbox[t]{11cm}{
			\begin{enumreq}
				\item Administrator ma rozpoczętą sesję z serwerem
			\end{enumreq}

		}
		\\

	\hline
		\textbf{Wyjątki} & \parbox[t]{11cm}{
			\textit{Brak}

		}
		\\

	\hline
		\textbf{Scenariusz podstawowy} & \parbox[t]{11cm}{
			\begin{enumreq}
				\item Administrator przechodzi do listy pokojów.
				\item Administrator klika w ikonę trzech pionowych kropek obok nazwy
				wybranego pokoju.
				\item W rozwijanym menu zostaje pokazana opcja ,,Usuń''.
				\item Zostaje pokazany monit z prośbą o potwierdzenie decyzji, w którym
				zostaje obowiązkowo wymieniona nazwa pokoju.
				\item Po zatwierdzeniu decyzji, użytkownicy wpięci do usuwanego pokoju
				zostają usunięte, a następnie sam pokój zostaje usunięty.
			\end{enumreq}
		}
		\\

	\hline
		\textbf{Scenariusze alternatywne} & \parbox[t]
		{11cm}{
			\begin{enumreq}
				\item Gdy pokój przestahe istnieć pomiędzy wybraniem a potwierdzeniem
				usunięcia, operacja zostanie zakończona niepowodzeniem.
				\item Gdy administrator zrezygnuje z usunięcia pokoju na etapie okna
				potwierdzającego usunięcie, pokój pozostanie nieusunięty.
			\end{enumreq}
		}
		\\

	\hline
		\textbf{Warunek końcowy} & \parbox[t]{11cm}{
			Pokój zostaje usunięty.
		}
		\\

	\hline
		\textbf{Komentarz} & \parbox[t]{11cm}{
			\textit{Nie zamieszczono}
		}
		\\

	\hline
\end{tabular}
}


\chapter{Wkład własny członków zespołu}
\label{wklad_wlasny_czlonkow_zespolu}

\section{Krzysztof Franek}

\section{Marek Marecki}

\section{Słownik pojęć}
\label{slownik_pojec}

W tym rozdziale prezentujemy słownik pojęć używanych w pracy, a których znaczenie może być niejednoznaczne lub
nieznane czytelnikowi.
Pojęcia są ułożone w kolejności alfabetycznej.

\subsection{Pojęcia ogólne}
\label{slownik_pojec_ogolnych}

\begin{labeling}{Viua VM}
    \item [FFI] (ang. \emph{foreign function interface}) interfejs umożliwiający wywoływanie z jednego języka
        funkcji napisanych w innym języku
    \item [ViuAct] język wysokiego poziomu, oparty o modelu aktorów, kompilowany
        do języka asemblera Viua VM
    \item [Viua VM] maszyna wirtualna, umożliwiająca uruchamianie programów
        wykorzystujących współbieżność
\end{labeling}

\subsection{Pojęcia związane z językiem i kompilatorem}
\label{slownik_pojec_jezyka}

\begin{labeling}{jednostka translacji}
    \item[biblioteka] zbiór modułów
    \item[\emph{compile-time}] ''\emph{czas kompilacji}'' czas, w którym program jest kompilowany
    \item[jądro] podsystem Viua VM odpowiadający za uruchamianie programów (tzw. ''\emph{kernel}'')
    \item[jednostka translacji] w przypadku języka ViuAct jest to pojedynczy moduł
    \item[kompilator] program tłumaczący kod w jednym języku (zazwyczaj wysokiego poziomu) na kod o takim
        samym znaczeniu w innym języku (zazwyczaj niższego poziomu)
    \item[leksem] ciąg znaków odpowiadający wzorcowi określającemu możliwe wartości tokenu
    \item[linker] program łączący wiele modułów w plik wykonywalny
    \item[moc funkcji] ilość parametrów formalnych przyjmowanych przez funkcję (pojęcie ściągnięte z pojęcia
        mocy zbioru)
    \item [model aktorów] model przetwarzania współbieżnego, opierający się na
        podstawowych strukturach, nazywanych „aktorami”, posiadających swój
        własny prywatny stan i porozumiewających się pomiędzy sobą za pomocą
        komunikatów
    \item[moduł] w załeżności od kontekstu: \emph{1/} kod źródłowy modułu w języku ViuAct, lub \emph{2/} plik
        zawierający bytecode w formacie, który może zostać wykorzystany przez linker bądź jądro Viua VM do
        dołączenia
    \item[plik wykonywalny] plik zawierający bytecode w formacie, który może zostać wykonany przez jądro Viua
        VM
    \item[runtime] ''\emph{środowisko uruchomieniowe}'' maszyna wirtualna bądź realna, na której
        wykonywany jest program
    \item[\emph{run-time}] ''\emph{czas wykonywania}'' czas, w którym program jest wykonywany przez VM;
        przeciwieństwo \emph{compile-time}
    \item[token] abstrakcyjna reprezentacja konkretnego elementu leksykalnego, np. słowa kluczowego lub
        identyfikatora, składająca się z nazwy typu tokenu i leksemu, który dana instancja tokenu zawiera
    \item[wzorzec] wyrażenie regularne określające jaką formę mogą przyjąć leksemy danego typu tokenu
\end{labeling}

\subsection{Pojęcia związane z chatem}
\label{slownik_pojec_chatu}

\begin{labeling}{Wiadomości prywatne}
    \item[Pokój] Współdzielony czat, do którego dostęp ma równocześnie wielu uczestników, widzących nawzajem
        wysyłane przez siebie wiadomości
    \item[Wiadomości prywatne] Wiadomości, które są wysyłane konkretnemu użytkownikowi i są widoczne wyłącznie
        dla nadawcy i odbiorcy takiej wiadomości
    \item[Wpięcie użytkownika w pokój] Rodzaj relacji, polegający na tym, że dany użytkownik ma możliwość
        nadawania i odbierania wiadomości w ramach określonego pokoju
\end{labeling}


\chapter{Bibliografia}

Książki. Internetowe materiały. Materiały wykładowe. Opracowania własne.

\end{document}
