\section{Słownik pojęć}
\label{slownik_pojec}

W tym rozdziale prezentujemy słownik pojęć używanych w pracy, a których znaczenie może być niejednoznaczne lub
nieznane czytelnikowi.
Pojęcia są ułożone w kolejności alfabetycznej.

\subsection{Pojęcia ogólne}
\label{slownik_pojec_ogolnych}

\begin{labeling}{interakcje języka z platformą}
    \item [FFI] (ang. \emph{foreign function interface}) interfejs umożliwiający wywoływanie z jednego języka
        funkcji napisanych w innym języku
    \item [\ViuAct] język wysokiego poziomu, oparty o modelu aktorów, kompilowany
        do języka asemblera Viua VM
    \item [Viua VM] maszyna wirtualna, umożliwiająca uruchamianie programów
        wykorzystujących współbieżność
    \item[interakcje języka z platformą] wykorzystanie zasobów sprzętowych, operacje I/O, oraz wszelkie
        efekty uboczne będące wynikiem działania programu
	\item [Lisp] język programowania zaprezentowany w 1958 roku przez John'a McCarthy'ego
	\item [Erlang] język programowania zaprojektowany w 1986 roku na potrzeby firmy Ericsson przez
		zespół, w którego skład wchodzili: Joe Armstrong, Robert Virding, Mike Williams
	\item [BEAM] maszyna wirtualna, na której działa Erlang
	\item [\emph{userspace}] (ang. przestrzeń użytkownika) abstrakcyjna przestrzeń, w której działają programy
		,,użytkowe''; w przypadku Viua VM oznacza przestrzeń, w której działają procesy (aktory)
	\item [\emph{kernelspace}] (ang. przestrzeń jądra) abstrakcyjna przestrzeń, w której działa kod jądra
		systemu operacyjnego; w przypadku Viua VM oznacza przestrzeń, w której działa sama Viua VM
\end{labeling}

\subsection{Pojęcia związane z językiem i kompilatorem}
\label{slownik_pojec_jezyka}

\begin{labeling}{jednostka translacji}
    \item[biblioteka] zbiór modułów
    \item[\emph{compile-time}] ,,czas kompilacji''; czas, w którym program jest kompilowany
    \item[jądro] podsystem Viua VM odpowiadający za uruchamianie programów (tzw. ,,kernel'')
    \item[jednostka translacji] w przypadku języka \ViuAct\ jest to pojedynczy moduł
    \item[kompilator] program tłumaczący kod w jednym języku (zazwyczaj wysokiego poziomu) na kod o takim
        samym znaczeniu w innym języku (zazwyczaj niższego poziomu)
    \item[leksem] ciąg znaków odpowiadający wzorcowi określającemu możliwe wartości tokenu
    \item[linker] program łączący wiele modułów w plik wykonywalny
    \item[moc funkcji] (ang. \emph{arity}) ilość parametrów formalnych przyjmowanych przez funkcję (pojęcie
        zapożyczone z pojęcia mocy zbioru)
    \item [model aktorów] model przetwarzania współbieżnego, opierający się na
        podstawowych strukturach, nazywanych „aktorami”, posiadających swój
        własny prywatny stan i porozumiewających się pomiędzy sobą za pomocą
        komunikatów
    \item[moduł] w załeżności od kontekstu: \emph{1/} kod źródłowy modułu w języku \ViuAct, lub \emph{2/} plik
        zawierający bytecode w formacie, który może zostać wykorzystany przez linker bądź jądro Viua VM do
        dołączenia, lub \emph{3/} zbiór funkcji, wyliczeń, i modułów w języku \ViuAct
    \item[plik wykonywalny] plik zawierający bytecode w formacie, który może zostać wykonany przez jądro Viua
        VM
    \item[runtime] ,,środowisko uruchomieniowe''; maszyna wirtualna bądź realna, na której
        wykonywany jest program
    \item[\emph{run-time}] ,,czas wykonywania''; czas, w którym program jest wykonywany przez VM;
        przeciwieństwo \emph{compile-time}
    \item[token] abstrakcyjna reprezentacja konkretnego elementu leksykalnego, np. słowa kluczowego lub
        identyfikatora, składająca się z nazwy typu tokenu i leksemu, który dana instancja tokenu zawiera
    \item[wzorzec] wyrażenie regularne określające jaką formę mogą przyjąć leksemy danego typu tokenu
	\item[wiadomość] wiadomość jest to dowolna wartość (np. liczba całkowita,
		napis, struktura) wysłana przez jednego aktora do drugiego
\end{labeling}

\subsection{Pojęcia związane z chatem}
\label{slownik_pojec_chatu}

\begin{labeling}{Wpięcie użytkownika w pokój}
    \item[Pokój] Współdzielony czat, do którego dostęp ma równocześnie wielu uczestników, widzących nawzajem
        wysyłane przez siebie wiadomości
    \item[Wiadomości prywatne] Wiadomości, które są wysyłane konkretnemu użytkownikowi i są widoczne wyłącznie
        dla nadawcy i odbiorcy takiej wiadomości
    \item[Wpięcie użytkownika w pokój] Rodzaj relacji, polegający na tym, że dany użytkownik ma możliwość
        nadawania i odbierania wiadomości w ramach określonego pokoju
\end{labeling}
