\chapter{Wprowadzenie}
\label{wprowadzenie}

We wprowadzeniu chcielibyśmy poruszyć ogólne tematy związane z naszą pracą, oraz przedstawić Czytelnikowi
spójny wstęp zapewniający solidne podstawy do dalszej lektury. Zakładamy, że Czytelnik ma doświadczenie z
językami programowania i umie biegle posługiwać się co najmniej jednym ,,\emph{mainstreamowym}'' językiem, np.
C++, Java, czy Python.

Wprowadzenie jest dla nas miejscem, w którym przedstawimy problemy dręczące popularne obecnie języki
programowania oraz powiemy skąd się te problemy biorą i dlaczego w miarę postępu technologii są one coraz
trudniejsze do zignorowania.

Po krótce przedstawiona zostanie również maszyna wirtualna Viua, która jest platformą, na której nasza praca
się opiera.

\section{Przedstawienie problemu}

Tytułem pracy jest ,,\inzmaintitlePL''. Problemem, który poruszamy jest sprawdzenie czy Viua VM w stanie
,,zastanym'' (tj. w takim w jakim znajdowała się jej implementacja na początku prac nad projektem
inżynierskim) umożliwia pisanie niezawodnego, wysoce współbieżnego, nietrywialnego oprogramowania.

W naszej pracy prezentujemy język programowania, który z założenia ma pozwalać na tworzenie oprogramowania
niezawodnego i wykorzystującego potencjał współbieżności w stopniu wyższym niż powszechnie używane,
,,mainstreamowe'' języki programowania.

Aby udowodnić, że w wytworzonym języku możliwe jest tworzenie nietrywialnego oprogramowania prezenetujemy
aplikację użytkową -- czat -- napisany w tym języku. Czat umożliwi komunikację (zorganizowaną kanałach) wielu
użytkownikom naraz. Wybór rodzaju aplikacji (czat) jest warunkowany tym, że oprogramowanie komunikacyjne
powinno charakteryzować się tymi cechami, które chcemy uzyskać:

\begin{enumerate}
    \item niezawodnością -- jeśli jedno połączenie ulegnie awarii to pozostałe powinny działać dalej
    \item izolacją procesów -- każde połączenie powinno być izolowane od wszystkich innych
    \item współbieżnością -- wiele połączeń musi być obsługiwanych w tym samym czasie
\end{enumerate}

\subsection{Ogląd sytuacji}

Współczesny \emph{hardware} zmierza coraz bardziej w stronę współbieżności oraz przetwarzania równoległego.
Firma AMD w roku 2018 wprowadziła na rynek konsumencki procesory wielordzeniowe z serii
Threadripper (\cite{AmdProductThreadripper}) prezentujące nawet 32 rdzenie logiczne.

Współczesny \emph{software} stoi w miejscu. Poza oprogramowaniem specjalistycznym (np.
Blender\footnote{\url{https://www.blender.org/}} mało który program jest w stanie wykorzystać więcej niż
kilka wątków. Współbieżności szuka się nieco ,,na siłę''; np. przeglądarki internetowe uruchamiają
współbieżnie obsługę wielu kart.

Mainstreamowe języki w większości korzystają z wątków (np. Java, C++) lub są stricte jednowątkowe i
niedostosowane do przetwarzania współbieżnego i równoległego (np. Python).

Dla takich języków tworzone są biblioteki ułatwiające wykorzystanie mechanizmów systemu operacyjnego
(np.~wieloprocesowość dla języka Python) lub cała infrastruktura umożliwiająca rozproszenie pracy,
np.~sewery pracy (np. Gearman\footnote{\url{http://gearman.org/}},
Celery\footnote{\url{http://www.celeryproject.org/}}) czy kolejki wiadomości
(np.~RabbitMQ\footnote{\url{https://www.rabbitmq.com/}}, ZeroMQ\footnote{\url{http://zeromq.org/}}).
To wszystko są jednak jedynie ,,łatki'' mające za zadanie dodać do istniejących języków mechanizmy, z którymi
pierwotnie nie były one projektowane. Istnieje dla takiego działania angielski termin bardzo dobrze oddający
jego postać --
\emph{retrofitting}\footnote{\url{https://dictionary.cambridge.org/dictionary/english/retrofit}}:

\begin{quote}
    \textbf{retrofit}
    \newline
    \emph{verb}
    \newline
    to provide a machine with a part, or a place with equipment, that it did not originally have when it was
    built
\end{quote}

Polski odpowiednik tego słowa to (za \cite{PWNAngielskoPolskiRetrofit}) ,,doposażyć''.

Istnieją środowiska i języki zaprojektowane od zera z myślą o współbieżności i programowaniu równoległym, a
nawet rozproszonym.
Najbardziej znanym przykładem takiego środowiska jest BEAM -- maszyna wirtualna języka
Erlang\footnote{\url{http://www.erlang.org/}}. To środowisko wraz z językiem jest z powodzeniem wykorzystywane
w sprzęcie telekomunikacyjnym firmy Ericsson, oraz do tworzenia aplikacji użytkowych, których działanie
obejmuje niemal z definicji działanie rozproszone i współbieżne, na przykład w serwerach komunikatora
Discord\footnote{\url{https://discordapp.com/}}.

\section{Cel}

Ta praca inżynierska motywowana jest chęcią stworzenia bazy programistycznej (środowiska uruchomieniowego i
języka programowania) umożliwiającej programistom pisanie oprogramowania od początku uwzględniającego
przetwarzanie współbieżne na pierwszym miejscu, oraz charakteryzującego się wysokim poziomem niezawodności i
stabliności działania.

Osiągniemy to dzięki zbudowaniu języka wyposażonego w łatwo dostępne konstrukcje umożliwiające wprowadzenie
współbieżności do programu, oraz uruchamianiu programów wynikowych w środowisku zdolnym do rozłożenia pracy na
całość dostępnych zasobów sprzętowych.

\subsection{Problemy i ryzyko}

Postawienie współbieżności na pierwszym miejscu powoduje, że problemy związane z pisaniem poprawnych programów
są bardziej liczne niż w innych (niewspółbieżnych) językach -- oprócz zapewnienia poprawności pojedynczego
procesu, programista musi zadbać o poprawność interakcji między procesami, oraz o stabilność działania
oprogramowania w momencie awarii któregoś z procesów składowych programu.

Zminimalizujemy ryzyko płynące z wprowadzenia współbieżności do warsztatu programistów udostępniając im
mechanizmy umożliwiające opanowanie awarii, propagowanie informacji o błędach, oraz izolację poszczególnych
procesów składowych.

\section{Układ pracy inżynierskiej}

W rozdziale \ref{viuact_zal}.~\nameref{viuact_zal} (na stronie \pageref{viuact_zal}) przedstawiamy język
Viuact, a w rozdziale \ref{viuact_impl}. jego implementację - kompilator. Są to narzędzia wykorzystywane do
stworzenia aplikacji użytkowej przedstawionej w rozdziale
\ref{program_viuachat_zal}.~\nameref{program_viuachat_zal} (na stronie \pageref{program_viuachat_zal}).

Wkład własny członków zespołu przedstawiony jest w rozdziale \ref{wklad_wlasny}. na stronie
\pageref{wklad_wlasny}.

\section{Słownik pojęć}
\label{slownik_pojec}

W tym rozdziale prezentujemy słownik pojęć używanych w pracy, a których znaczenie może być niejednoznaczne lub
nieznane czytelnikowi.
Pojęcia są ułożone w kolejności alfabetycznej.

\subsection{Pojęcia ogólne}
\label{slownik_pojec_ogolnych}

\begin{labeling}{interakcje języka z platformą}
    \item [FFI] (ang. \emph{foreign function interface}) interfejs umożliwiający wywoływanie z jednego języka
        funkcji napisanych w innym języku
    \item [ViuAct] język wysokiego poziomu, oparty o modelu aktorów, kompilowany
        do języka asemblera Viua VM
    \item [Viua VM] maszyna wirtualna, umożliwiająca uruchamianie programów
        wykorzystujących współbieżność
    \item[interakcje języka z platformą] wykorzystanie zasobów sprzętowych, operacje I/O, oraz wszelkie
        efekty uboczne będące wynikiem działania programu
\end{labeling}

\subsection{Pojęcia związane z językiem i kompilatorem}
\label{slownik_pojec_jezyka}

\begin{labeling}{jednostka translacji}
    \item[biblioteka] zbiór modułów
    \item[\emph{compile-time}] ,,\emph{czas kompilacji}''; czas, w którym program jest kompilowany
    \item[jądro] podsystem Viua VM odpowiadający za uruchamianie programów (tzw. ''\emph{kernel}'')
    \item[jednostka translacji] w przypadku języka ViuAct jest to pojedynczy moduł
    \item[kompilator] program tłumaczący kod w jednym języku (zazwyczaj wysokiego poziomu) na kod o takim
        samym znaczeniu w innym języku (zazwyczaj niższego poziomu)
    \item[leksem] ciąg znaków odpowiadający wzorcowi określającemu możliwe wartości tokenu
    \item[linker] program łączący wiele modułów w plik wykonywalny
    \item[moc funkcji] (ang. \emph{arity}) ilość parametrów formalnych przyjmowanych przez funkcję (pojęcie
        zapożyczone z pojęcia mocy zbioru)
    \item [model aktorów] model przetwarzania współbieżnego, opierający się na
        podstawowych strukturach, nazywanych „aktorami”, posiadających swój
        własny prywatny stan i porozumiewających się pomiędzy sobą za pomocą
        komunikatów
    \item[moduł] w załeżności od kontekstu: \emph{1/} kod źródłowy modułu w języku ViuAct, lub \emph{2/} plik
        zawierający bytecode w formacie, który może zostać wykorzystany przez linker bądź jądro Viua VM do
        dołączenia
    \item[plik wykonywalny] plik zawierający bytecode w formacie, który może zostać wykonany przez jądro Viua
        VM
    \item[runtime] ,,\emph{środowisko uruchomieniowe}''; maszyna wirtualna bądź realna, na której
        wykonywany jest program
    \item[\emph{run-time}] ,,\emph{czas wykonywania}''; czas, w którym program jest wykonywany przez VM;
        przeciwieństwo \emph{compile-time}
    \item[token] abstrakcyjna reprezentacja konkretnego elementu leksykalnego, np. słowa kluczowego lub
        identyfikatora, składająca się z nazwy typu tokenu i leksemu, który dana instancja tokenu zawiera
    \item[wzorzec] wyrażenie regularne określające jaką formę mogą przyjąć leksemy danego typu tokenu
\end{labeling}

\subsection{Pojęcia związane z chatem}
\label{slownik_pojec_chatu}

\begin{labeling}{Wpięcie użytkownika w pokój}
    \item[Pokój] Współdzielony czat, do którego dostęp ma równocześnie wielu uczestników, widzących nawzajem
        wysyłane przez siebie wiadomości
    \item[Wiadomości prywatne] Wiadomości, które są wysyłane konkretnemu użytkownikowi i są widoczne wyłącznie
        dla nadawcy i odbiorcy takiej wiadomości
    \item[Wpięcie użytkownika w pokój] Rodzaj relacji, polegający na tym, że dany użytkownik ma możliwość
        nadawania i odbierania wiadomości w ramach określonego pokoju
\end{labeling}

