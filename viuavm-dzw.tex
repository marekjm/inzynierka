\documentclass[11pt,oneside,a4paper,titlepage,onecolumn]{article}

\usepackage[utf8]{inputenc}
\usepackage{textcomp}
\usepackage[official]{eurosym}
\usepackage[polish]{babel}
\usepackage{amsthm}
\usepackage{graphicx}
\usepackage[T1]{fontenc}
\usepackage{scrextend}
\usepackage{hyperref}
% \usepackage{nameref}
% \usepackage{showlabels}
% \usepackage{titlesec}

\setcounter{secnumdepth}{4}

%% Author and title
% \author{Marek Marecki \and Gr.52c \and Kod: 95 \and 2017\slash2018-2019}
\author{Marek Marecki \and John Doe}
\title{%
    Proving viability of Viua VM \\
    \large Implementation of high-level language on Viua VM and\\
    deployment of simple a application \\
    ~\\
    Dokument założeń wstępnych}

\begin{document}

\maketitle

Praca wygenerowana w systemie \LaTeX.
Wykorzystana wersja \LaTeX~pochodzi z pakietu: \emph{TeX 3.14159265 (TeX Live 2018/Arch Linux)}.

\tableofcontents

\newpage

\section{Opis problemu}
\label{opis_problemu}

\subsection{Problem niezawodności}

\begin{center}
    Tworzenie \textbf{niezawodnych} systemów jest \textbf{trudne}.
\end{center}

Zazwyczaj, im bardziej rozbudowany i skomplikowany system tym większa jest
jego ``delikatność'' (ang. \emph{fragility}) i ``kruchość'' (ang. \emph{brittleness}).
Obie te cechy opisują brak odporności systemu na błędy.

Brak odporności na błędy wynika częściowo z poniższych problemów:

\begin{itemize}
\item wykrywanie
\item obsługa
\item izolacja
\end{itemize}

\textbf{Wykrywanie} błędów nie zawsze jest automatyczne. Programowanie defensywne, którym
programista się broni przed błędami, wręcz oczekując ich, potrafi prowadzić do rozwlekłego kodu. \\
Problematyczne jest również to, że w modelu manualnego wykrywania błędów to programista-klient
odpowiada za wykrywanie błędów, a nie twórca biblioteki (który z natury rzeczy będzie miał większą
wiedzę co może pójść nie tak).

\textbf{Obsługa} błędów nie zawsze jest oczywista. Mamy tu do czynienia z podziałem na ``szkoły''
obsługi: wyjątki, kodu błędów, czy typy pozwalające opakować błąd, np. \texttt{Maybe} (Haskell) lub
\texttt{option} (OCaml). \\
Problemy przy obsłudze błędów sprawia również sposób zarządzania zasobami. Jeśli nie jest
automatyczny (np. RAII w C++) to należy się liczyć ze zwiększonym ryzykiem wycieków.

\textbf{Izolacja} poszczególnych części systemu jest zazwyczaj niewystarczająca. Powoduje to
sytuacje, w których awaria jednej części systemu powoduje awarię kolejnej części, itd. \\
Problemem jest to, że języki nie zapewniają wystarczających narzędzi do izolowania podystemów co
zmusza programistów do wychodzenia poza ramy języka i wykorzystywania mechanizmów systemu
operacyjnego (np. izolowanie poprzez wieloprocesowość na systemach POSIX).

\subsection{Problem równoległości}

\begin{center}
    Tworzenie \textbf{równoległych} systemów jest \textbf{trudne}.
\end{center}

Mimo iż w ostatnich czasach współbieżność jest w modzie, równoległość w dalszym ciągu pozostaje
zagadnieniem trudnym. Możliwość pisania niezawodnych programów równoległych (tj. mogących wykonywać
wiele czynności naraz w tej samej jednostce czasu) staje się tym ważniejsza im bardziej powszechne
stają się systemu wieloprocesorowe.
Wprowadzenie przez AMD w 2017 roku na rynek \textbf{konsumencki} procesorów z serii
Threadripper (Threadripper 1950x prezentował 32 logiczne rdzenie) podkreśliło ten fakt.

Tymczasem, wiele systemów nie jest w stanie wykorzystać potencjału programowania równoległego.
Wykorzystanie zmiennych globalnych, brak izolacji zadań w programach, brak sposobów na łatwą realizację
komunikacji pomiędzy tymi zadaniami - to wszystko stoi na drodze do rozwoju oprogramowania\ldots

\subsection{Potencjalne rozwiązania}

\subsubsection{Communicating Sequential Processes}

Model ``communicating sequential
processes''\footnote{\url{https://en.wikipedia.org/wiki/Communicating_sequential_processes}}
w swojej oryginalnej wersji\footnote{\url{https://dl.acm.org/citation.cfm?doid=359576.359585}} z 1978
opisywał prosty język programowania.
Jeśli CSP miałoby być przez nas brane pod uwagę to byłaby to właśnie wersja oryginalna. Ma ona jednak
kilka ``wad projektowych'', które stoją w sprzeczności z wyżej wymienionymi problemami niezawodności i
równoległości:

\begin{itemize}
\item Programy CSP opisują pewną stałą liczbę procesów. Jest to dla nas o tyle niewygodne, że liczba
    procesów w programie jest dynamiczna; jedne ulegają awariom, inne są powoływane do życia w celu
    obsługi zadań jakie pojawiły się w systemie.
\item Komunikaty są wymieniane synchronicznie. Jest to bardzo wygodne jeśli chcemy uzyskać semantykę
    synchronizacji, na zasadzie \emph{rendezvous} pomiędzy procesami wymieniającymi wiadomość.
    Wadą tego podejścia jest jednak fakt, że proces odbiorca, z którym proces nadawca chciałby nawiązać
    \emph{rendezvous} mógł ulec awarii (być może w trakcie trwania komunikacji) co może skutkować
    awarią systemu jeśli proces nadawca nie byłby w stanie takiej sytuacji obsłużyć.
\end{itemize}

Wartościowe idee CSP to wewnętrzna sekwencyjność każdego z procesów i możliwość wymiany informacji
\textbf{jedynie} za pomocą komunikatów. Zapewnia to dobrą izolację stanu każdego procesu, co ułatwia
wnioskowanie o zachowaniu każdego procesu z osobna i pozwala na budowanie systemów ze sprawdzonych
części.

\subsubsection{Calculus of Communicating Systems}

Zaprezentowany przez Robina Millera koło 1980 CCS służy do modelowania komunikacji między dwoma
procesami. Może być też używany do analizy poprawności systemów współbieżnych (np. zakleszczeń).

Nie jest to implementowalny język programowania, a narzędzie do analizy i oceny modeli i systemów.
Z tego powodu nie będzie brany pod uwagę w innym zakresie niż ewentualna dyskusja o przedstawionym, finalnym
rozwiązaniu projektu.

\subsubsection{$\pi$-calculus}

$\pi$-calculus jest kolejnym modelem autorstwa Robina Millera, Joachima Parrowa i Davida Walkera
\footnote{\url{https://www.sciencedirect.com/science/article/pii/0890540192900084}}.
Został zaprezentowany w 1992 roku.
Jest również omawiany w pracy ``Communicating and Mobile Systems: the $\pi$-Calculus''
\footnote{\url{https://assets.cambridge.org/052164/3201/sample/0521643201WS.pdf}}.

Podobnie jak CCS, $\pi$-calculus służy do wnioskowania i opisu właściwości systemu, zamiast być
bazą do stworzenia języka programowania. Mógłby być wykorzystany jedynie jako bazowy model interakcji
pomiędzy procesami, wyrażenie fundamentalnych mechanizmów komunikacji zapewnianych przez omawiany
język.

W związku z tym, podobnie jak CCS, nie będzie brany pod uwagę w innym charakterze niż narzędzie do
dyskusji o przedstawionym rozwiązaniu projektu.

\subsubsection{Actor model}

Prawdopodobnie najbardziej znany (obok CSP) model. Zaprezentowany w 1973 przez Carla Hewitta, omawiany
również przez Gula Aghę w 1985, i wielu innych aż do dziś.

Model aktorów jest najbardziej ``elastyczny'' ze wszystkich tu wymienionych:

\begin{itemize}
\item Ilość aktorów w systemie jest zmienna.
\item Komunikacja między nimi jest asynchroniczna.
\item Reakcje na otrzymane komunikaty dynamiczne (aktor może każdorazowo podjąć decyzję jak zareagować
    na otrzymaną wiadomość).
\end{itemize}

Oprócz solidnych podwalin teoretycznych, model aktorów został również sprawdzony w praktyce; m.in. jest
podstawą języka Erlang\footnote{\url{http://www.erlang.org}}.

Istotną zaletą modelu aktorów jest semantyka przekazywania komunikatów: odbywa się to bezpośrednio pomiędzy
aktorami na podstawie adresów aktorów (inaczej niż w CSP, gdzie wiadomości są kierowane do kanałów; tak jak
jest to realizowane m.in. przez język Go\footnote{\url{https://golang.org/}}), asynchronicznie, i
bez wymogu szeregowania wiadomości (tj. wiadomość $n+1$ może dotrzeć wcześniej niż wiadomość $n$).

Tak jak w CSP, wewnętrzny stan każdego aktora jest prywatny.

\subsubsection{Model prezentowany przez Viua VM}

Viua VM prezentuje model programowania będący syntezą cech CSP i modelu aktorów.

``Aktor'' w żargonie Viua nazywany jest ``procesem''. W dalszej części pracy te terminy mogą być używane
zamiennie, jednak częściej termin ``aktor'' będzie oznaczał rozumienie modelowe, a termin ``proces''
rozumienie implementacyjne.

Procesy w Viua VM są izolowane, wewnętrznie sekwencyjne (tj. wykonanie wewnątrz pojedynczego procesu nie
jest równoległe), ich stan jest prywatny.
Komunikacja odbywa się bezpośrednio między procesami na podstawie adresu, za pomocą asynchronicznej wymiany
wiadomości bez gwarancji kolejności i dostarczenia.

Ten model będzie używany w dalszej części pracy.

% \textbf{Uwaga}: opis {\small\ldots problemu}.

\section{Cele systemu}

Celem jest stworzenie maszyny wirtualnej (środowiska uruchomieniowego) zapewniającej narzędzia do
pisania niezawodnych, równoległych systemów. Maszyna wirtualna i język służący do jej programowania
ma w możliwie dużym stopniu asystować programistę w wykorzystywaniu zasobów systemu, oraz
izolacji zadań i obsłudze błędów.

\section{Kontekst systemu}

Kontekstem maszyny wirtualnej Viua VM jest oprogramowanie serwerowe wysokiej dostępności, np.
telekomunikacyjne, diagnostyczne czy monitorujące, w tym demony systemowe.

\section{Zakres systemu (funkcjonalność)}

Maszyna wirtualna powinna:

\begin{enumerate}
    \item prezentować zestaw instrukcji umożliwiający pisanie nietrywialnych, i niezawodnych programów
    \item udostępniać mechanizmy izolacji zadań, i komunikacji między nimi
    \item udostępniać mechanizmy automatycznego zarządzania zasobami (à la RAII)
    \item automatycznie wykrywać błędy w programach i udostępniać mechanizm ich obsługi w spójny, jednolity
        sposób
    \item prezentować statycznie typowany język programowania umożliwiający wykrywanie błędów na etapie
        kompilacji
\end{enumerate}

\subsection{Model programowania VM}

Model programowania Viua VM odpowiada tzw. ``modelowi aktorów''
\footnote{Carl Hewitt; Peter Bishop; Richard Steiger (1973). ``A Universal Moduler Actor Formalism for
Artificial Intelligence''}
\footnote{Gul Agha (1986). ``Actors: A Model of Concurrent Computation in Distributed Systems''}
opracowanemu w latach 70. i 80. XX wieku.

Model ten został z powodzeniem zaimplementowany i wykorzystany w języku Erlang
\footnote{\url{http://www.erlang.org/}} (w firmie Ericsson).
Viua VM jest nowoczesną implementacją maszyny wirtualnej prezentującej model programowania oparty o model
aktorów.

\section{Wymagania jakościowe i inne}

Implementacja Viua VM musi charakteryzować się wysoką jakością wykonania jeśli ma być wykorzystywana jako
stablina podstawa do tworzenia innych systemów. Nie może w niej być miejsca na podejście typu ``laissez
faire''; implementacja powinna być konserwatywna i zachowawcza, stawiając na pierszym miejscu poprawność i
niezwodność.

\subsection{Wymagania jakościowe}

\begin{enumerate}
    \item brak wycieków zasobów (pamięci, czasu procesora, socketów, deskryptorów plików, itp.)
    \item brak zakleszczeń w kodzie wielowątkowym
    \item uniemożliwienie kontynuacji wykonywania programu po wykryciu błędu bez uprzedniego obsłużenia go
    \item uniemożliwienie zawłaszczenia maszyny przez wykonywany program (wywłaszczanie procesów)
    \item uniemożliwienie naruszenia stabilności maszyny przez wykonywany program (np. doprowadzenie do
        wycieku pamięci, niekontrolowanego przepełnia stosu, niekontrolowanego zamknięcia procesu, itp.)
    \item zestaw testów gwarantujących brak regresji podczas rozwoju VM
\end{enumerate}

Wymagania dotyczące uniemożliwienia naruszenia stabilności i zawłaszczenia maszyny dotyczą jedynie programów
pisanych w języku assemblera Viua VM tj. skompilowanych do bytecodu VM i nie wykorzystujących zewnętrznych
(pochodzących spoza biblioteki standardowej) modułów rozszerzeń.

\subsection{Wymagania wydajnościowe}

\begin{enumerate}
    \item uruchamianie wielu współpracujących, schedulerów procesów wirtualnych w celu zwiększenia
        przepustowości logiki (lepsze wykorzystanie czasu procesora)
    \item uruchomienie wielu schedulerów I/O w celu zwiększenia przepustowości operacji I/O
    \item uruchomienie wielu schedulerów FFI w celu zwiększenia przepustwowośc logiki wykonywanej w modułach
        rozszerzeń (czyli niemożliwych do wywłaszczenia przez scheduler procesów wirtualnych)
\end{enumerate}

Założeniem Viua VM nie jest osiągnięcie najwyższej możliwej wydajności. Głównym celem jest niezawodność (tj.
wydajność może być poświęcona na rzecz poprawności, nigdy na odwrót).

\subsection{Wymagania narzędziowe}

W standardowej dystrybucji Viua VM powinny znajdować się następujące narzędzia:

\begin{labeling}{disassembler}
\item [\texttt{assembler}] program pozwalający na przekształcenie kodu źródłowego w języku assemblera VM
    na bytecode (formę binarną zrozumiałą dla VM)
\item [\texttt{disassembler}] program pozwalający na przekształcenie bytecode'u na kod źródłowy w języku assemblera VM
\item [\texttt{kernel}] program implementujący jądro VM pozwalający na uruchamianie programów
\item [\texttt{debugger}] program pomagający w łataniu ewentualnych błędów
\end{labeling}

\subsection{Wymagania inne}

Dodatkowym wymaganiem jest prezentacja przykładowego programu (``use case'') uruchamianego na VM.
Programem tym jest chat internetowy Viua Chat, działający jako aplikacja internetowa, dostępna poprzez przeglądarkę. Podstawowy zakres funkcjonalności Viua Chat przewiduje:

\begin{enumerate}
    \item możliwość logowania i wylogowania dla użytkowników
    \item prymitywny system uprawnień administrator-nieadministrator
    \item obecność oddzielnych kanałów - pokojów, do których mogą dołączać użytkownicy, a w obrębie których wiadomości są widoczne dla wszystkich użytkowników obecnych w pokoju
    \item możliwość wysyłki prywatnych wiadomości bezpośrednio od użytkownika A do użytkownika B, widocznych wyłącznie dla tych użytkowników
    \item podstawowe narzędzia administracyjne: tworzenie i usuwanie pokojów, wyrzucanie użytkowników z pokojów i z serwera
\end{enumerate}

Wymiana wiadomości pomiędzy interfejsem a aplikacją serwerową chatu będzie odbywać się z użyciem protokołu WebSocket. 

\section{Wizja konstrukcyjna}

Jądro maszyny składa się z kilku podsystemów:

\begin{labeling}{process manager}
\item [scheduler IO] zarządzający operacjami I/O
\item [scheduler VP] zarządzający wirtualnymi procesami (przydzielający czas procesora)
\item [scheduler FFI] zarządzający wywołaniami przez FFI
\item [loader] ładujący moduły bytecode i rozszerzeń
\item [process manager] zarządzający zasobami procesów, przydzielający im PID, mailboxy, przechowujący status,
    itp.
\end{labeling}

\section{Ograniczenia}

Czasowe. Na wykonanie pracy inżynierskiej jest rok.

\section{Słownik pojęć}

\begin{labeling}{współbieżność}
\item [RAII] (ang. \emph{Resource Acquisition Is Initialisation}) model obsługi zasobów wprowadzony w C++
\item [współbieżność] przeplatanie wykonania kilku zadań
\item [równoległość] wykonywanie kilku zadań w tej samej jednostce czasu
\item [VM] (ang. \emph{Virtual Machine}) maszyna wirtualna; software'owa realizacja zestawu instrukcji
\item [I/O] wejście-wyjście; np. operacja zapisu do pliku, bądź odczytu z socket'u
\item [FFI] (ang. \emph{Foreign Function Interface}) interfejs umożliwiający wywoływanie z języka A funkcji
    napisanych w języku B
\item [scheduler] ``zarządca''; zarządza wykonywaniem pewnego rodzaju działań, dba o zapewnienie dostępności
    zasobu dla wielu procesów
\item [aktor] Jednostka przetwarzania logiki w programie działająca niezależnie od innych, komunikująca się z
    innymi aktorami za pomocą wymiany komunikatów.
\item [proces] Implementacja ``aktora'' wewnątrz Viua VM.
\end{labeling}

\end{document}
