\documentclass[11pt,oneside,a4paper,titlepage,onecolumn]{article}

\usepackage[utf8]{inputenc}
\usepackage{textcomp}
\usepackage[official]{eurosym}
\usepackage[polish]{babel}
\usepackage{amsthm}
\usepackage{graphicx}
\usepackage[T1]{fontenc}
\usepackage{scrextend}
\usepackage{hyperref}
\usepackage{xcolor}
\usepackage{enumitem}
% \usepackage{nameref}
% \usepackage{showlabels}
% \usepackage{titlesec}
\usepackage{geometry}
\geometry{a4paper, portrait, margin=2cm}



\setcounter{secnumdepth}{4}

%% Author and title
% \author{Marek Marecki \and Krzysztof Franek}
\author{Marek Marecki \and Krzysztof Franek}
\title{%
    Proving viability of Viua VM \\
    \large Implementation of high-level language on Viua VM\\
    and deployment of simple application \\
    ~\\
    Specyfikacja zasad biznesowych i historyjek\\
    (\textit{user stories}) czatu ViuaChat}

\begin{document}

\maketitle
{\footnotesize
\begin{center}
  \begin{tabular}{ | l | l | l | }
    \hline
    \parbox[t]{6.5cm}{\textbf{Temat pracy i akronim projektu:}\\Proving viablity of Viua VM (VVIA)} & \parbox[t]{4.5cm}{\textbf{Zleceniodawca:}\\\colorbox{yellow}{Nieznany}} & \parbox[t]{4.5cm}{\textbf{Konsultant:}\\\colorbox{yellow}{Nieznany}} \\ \hline
    \parbox[t]{6.5cm}{\textbf{Zespół projektowy:}\\Krzysztof Franek, Marek Marecki} & \parbox[t]{4.5cm}{\textbf{Kierownik projektu:}\\Marek Marecki} & \parbox[t]{4.5cm}{\textbf{Opiekun projektu:}\\dr hab. Marek A. Bednarczyk, prof. PJWSTK} \\ \hline
    \parbox[t]{3.5cm}{\textbf{Kierownik projektu:}\\Marek Marecki} & \multicolumn{2}{|l|}{\parbox[t]{9cm}{\textbf{Odpowiedzialny za dokument:}\\Krzysztof Franek}} \\ 
    \hline
  \end{tabular}
\end{center}
}

\section{Wprowadzenie}

\subsection{Cel dokumentu}
Celem dokumentu jest dostarczenie informacji do utworzenia rejestru zaległości produktu oraz zaplanowania rejestrów
zaległości kolejnych sprintów, co stanowi kolejny etap procesu wytwórczego.

\subsection{Zakres dokumentu}
Niniejszy dokument jest produktem pierwszego etapu procesu wytwórczego czatu ViuaChat, na który składają się:
\begin{itemize}
	\item wyszczególnienie i uporządkowanie zasad biznesowych, jakie zostały założone w stosunku do aplikacji;
	\item opracowanie historyjek na podstawie ustalonych zasad biznesowych.
\end{itemize}

\textbf{Uwaga:} Niniejszy dokument nie dotyczy języka ViuAct ani jego kompilatora.

Praca wygenerowana w systemie \LaTeX.

\section{Czat w kontekście}

\subsection{Kontekst biznesowy}
\textit{Tu zostaną umieszczone grafiki z osobnego PDF}

\section{Wymagania biznesowe}

Zidentyfikowane zasady pogrupowano w 3 kategorie, biorąc pod uwagę podstawowe bloki funkcjonalności. Przydzielenie
do kategorii jest sygnalizowanie literą alfabetu, będącą prefiksem identyfikatora danej zasady. Dokonano
również priorytetyzacji zasad biznesowych według klasycznej skali ,,MoSCoW":

\begin{itemize}
	\item \textbf{,,M''} (z ang. \textit{must}) - zasady, których spełnienie jest niezbędne dla realizacji systemu
	\item \textbf{,,S''} (z ang. \textit{should}) - są to zasady o wysokim priorytecie, które powinny;
	zostać spełnione, o ile tylko jest to możliwe;
	\item \textbf{,,C''} (z ang. \textit{could}) - dobrze byłoby zrealizować takie wymagania, ale zależy to od czasu
	i zasobów, jakie pozostaną do dyspozycji po ukończeniu zadań ,,M" i ,,C";
	\item \textbf{,,W''} (z ang. \textit{won't}) - takie wymagania, po dyskusji, zostały wycofane dalszej realizacji.
\end{itemize}

\subsection{System użytkowników [U]}
  \begin{tabular}{ | l | l | l | }
	\hline
    \textbf{ID} & \parbox[t]{14cm}{
    	\textbf{Zasada biznesowa}
    } & \textbf{Priorytet} \\  
  
    \hline
    U-1 & \parbox[t]{14cm}{
      Podczas wejścia na czat, użytkownikowi pokazuje się monit z polem do wpisania nazwy użytkownika. 
    } & M \\

    \hline
    U-2 & \parbox[t]{14cm}{
      Użytkownicy bez stałego konta podczas logowania podają tylko nazwę użytkownika, pole hasła pozostaje puste .
    } & M \\

    \hline
    U-3 & \parbox[t]{14cm}{
      Nazwa użytkownika to ciąg od 3 do 32 alfanumerycznych znaków.
    } & M \\

    \hline
    U-4 & \parbox[t]{14cm}{
      Można rozpocząć sesję jako użytkownik, pod warunkiem, że zadeklarowana nazwa nie będzie powtarzać się z nazwami już zalogowanych użytkowników. 
    } & M \\

    \hline
    U-5 & \parbox[t]{14cm}{
      Monit podczas wejścia na czat jest wyposażony w pole do wpisania hasła (nieobowiązkowe). 
    } & S \\

    \hline
    U-6 & \parbox[t]{14cm}{
      Użytkownicy czatu ze stałym kontem, podczas logowania podają nazwę i odpowiadające mu hasło.
    } & S \\

    \hline
    U-7 & \parbox[t]{14cm}{
      Stałe konta użytkowników są utrzymywane na serwerze w postaci trójek wartości: nazwa użytkownika, hasło (md5), czy jest administratorem. 
    } & S \\

    \hline
    U-8 & \parbox[t]{14cm}{
      Nie można rozpocząć sesji użytkownika o nazwie, która pasuje do istniejącego konta, jeżeli nie zostanie podane prawidłowe hasło (nie można podszywać się pod nazwy użytkowników ze stałym kontem).
    } & S \\

    \hline
    U-9 & \parbox[t]{14cm}{
      Można rozpocząć sesję jako użytkownik bez podawania hasła, pod warunkiem, że zadeklarowana nazwa nie będzie powtarzać się z nazwami stałych kont użytkowników.
    } & S \\

    \hline
    U-10 & \parbox[t]{14cm}{
      W okienkach czatu, loginy użytkowników ze stałym kontem są pogrubione i pokolorowane: Administratorzy na czerwono,
      Pozostali na zielono. 
      } & S \\

    \hline
    U-11 & \parbox[t]{14cm}{
      Administratorzy mają prawo przeglądać nazwy pokojów na serwerze. 
    } & M \\

    \hline
    U-12 & \parbox[t]{14cm}{
      Administratorzy mają prawo tworzyć i usuwać pokoje. 
    } & S \\

    \hline
    U-13 & \parbox[t]{14cm}{
      Administratorzy mają prawo ustanawiać, zmieniać i usuwać hasła do pokojów. 
    } & C \\

    \hline
    U-14 & \parbox[t]{14cm}{
      Administratorzy mają prawo wyrzucać użytkowników z pokojów. 
    } & C \\

    \hline
    U-15 & \parbox[t]{14cm}{
      Administratorzy mają prawo wyrzucać użytkowników z serwera. 
    } & C \\

    \hline
    U-16 & \parbox[t]{14cm}{
      Administratorzy mają prawo przeglądać nazwy i poziomy uprawnień kont stałych użytkowników. 
    } & M \\

    \hline
    U-17 & \parbox[t]{14cm}{
      Administratorzy mają prawo tworzyć i usuwać użytkowników. 
    } & S \\

    \hline
    U-18 & \parbox[t]{14cm}{
      Administratorzy mają prawo zmieniać hasła użytkowników. 
    } & C \\

    \hline
    U-19 & \parbox[t]{14cm}{
      Administratorzy mają prawo zmieniać uprawnienia stałych kont użytkowników. 
    } & C \\

    \hline
    U-20 & \parbox[t]{14cm}{
      Użytkownicy ze stałymi kontami mogą zmieniać swoje hasło.
    } & W \\

    
    \hline
  \end{tabular}

\subsection{Pokoje [P]}
  \begin{tabular}{ | l | l | l | }
	\hline
    \textbf{ID} & \parbox[t]{14cm}{
    	\textbf{Zasada biznesowa}
    } & \textbf{Priorytet} \\  
    
    \hline
    P-1 & \parbox[t]{14cm}{
      Pokoje to właściwe czaty – tam użytkownicy mogą wejść i pisać do siebie nazwajem
    } & M \\

    \hline
    P-2 & \parbox[t]{14cm}{
      Każdy pokój ma unikalną nazwę będącą ciągiem alfanumerycznym od 3 do 32 znaków
    } & M \\

    \hline
    P-3 & \parbox[t]{14cm}{
      Lista pokojów jest widoczna dla każdego użytkownika po zalogowaniu się do serwera czatu
    } & M \\

    \hline
    P-4 & \parbox[t]{14cm}{
      Użytkownik może być równocześnie wpięty do jednego pokoju
    } & M \\

    \hline
    P-5 & \parbox[t]{14cm}{
      Wiadomość wysłana w pokoju jest widoczna w oknie pokoju dla wszystkich użytkowników podpiętych do tego pokoju
    } & M \\

    \hline
    P-6 & \parbox[t]{14cm}{
      Użytkownik może się samodzielnie wypiąć z pokoju, do którego jest wpięty
    } & S \\

    \hline
    P-7 & \parbox[t]{14cm}{
      Pokój może mieć ustanowione hasło, które użytkownik musi wpisać przed podpięciem się do niego
    } & C \\
    
    \hline
    
  \end{tabular}

\subsection{Prywatne wiadomości [W]}
  \begin{tabular}{ | l | l | l | }
  	\hline
    \textbf{ID} & \parbox[t]{14cm}{
    	\textbf{Zasada biznesowa}
    } & \textbf{Priorytet} \\  
    
    
    \hline
    W-1 & \parbox[t]{14cm}{
      Oprócz okien czatu dla każdego z wpiętych pokojów, użytkownik dysponuje dodatkowym oknem, na którym widzi wiadomości prywatne. 
    } & M \\

    \hline
    W-2 & \parbox[t]{14cm}{
      Aby wysłać wiadomość prywatną, należy poprzedzić ją znakiem \# i nazwą użytkownika, do którego jest kierowana wiadomość, oddzielona spacją od komunikatu, np.: „\#user Tajna wiadomość”. Trafia ona wówczas do okna wiadomości prywatnych. 
    } & M \\

    \hline
    W-3 & \parbox[t]{14cm}{
      Wiadomość prywatna może być wysłana z okna czatu pokoju –wiadomości prywatne nie są pokazywane wszystkim uczestnikom czatu, a jedynie użytkownikowi, do którego jest adresowana. 
    } & C \\

    \hline
    W-4 & \parbox[t]{14cm}{
      W oknie czatu pokoju dopuszczalne jest wysyłanie wiadomości prywatnych do dowolnych użytkowników, nawet tych, którzy nie są w danym momencie podpięci do tego pokoju. 
    } & C \\

    \hline
    W-5 & \parbox[t]{14cm}{
      Z okna wiadomości prywatnych można wysyłać wyłącznie wiadomości prywatne. 
    } & M \\

    \hline
    W-6 & \parbox[t]{14cm}{
      Próba wysłania wiadomości prywatnej powinna zostać odrzucona przez serwer i zwrócić błąd w przypadku gdy:
      \begin{itemize}
      	\item Po znaku „\#” pojawi się od razu znak spacji lub nie będzie żadnych znaków (nie zostanie podana 
		nazwa użytkownika który jest adresatem)
		\item Nazwa adresata jest dłuższa niż 32 znaki
		\item Na serwerze nie istnieje użytkownik z nazwą użytą jako nazwa adresata
		\item Adresat wiadomości ma swoje konto na serwerze, ale nie jest zalogowany 
      \end{itemize}
		
    } & M \\
    \hline
    
  \end{tabular}


\section{Historyjki (\textit{user stories})}

\subsection{Funkcjonalne}

\begin{tabular}{ | l | l | }
	\hline
		\textbf{Identyfikator} & 
		HF-1
		\\
		
	\hline
		\textbf{Treść} & \parbox[t]{13cm}{
			Jako użytkownik serwera czatu, chcę móc się do niego zalogować, aby zobaczyć listę pokojów dyskusyjnych.
		}\\
		 
	\hline
		\parbox[t]{4cm}{\textbf{Powiązane zasady biznesowe}} & \parbox[t]{13cm}{
			U-1 Podczas wejścia na czat, użytkownikowi pokazuje się monit z polem do wpisania nazwy użytkownika.
		}\\
		
	\hline
		\parbox[t]{4cm}{\textbf{Kryteria akceptacji}} & \parbox[t]{13cm}{
			\begin{enumerate}
				\item Po wejściu na czat bez rozpoczętej sesji, pokazuje się monit o podanie nazwy użytkownika.
				\item Po wpisaniu nazwy użytkownika i zatwierdzeniu, użytkownik rozpocznie sesję na serwerze czatu.
				\item Tuż po rozpoczęciu sesji czatu, użytkownik zobaczy listę pokojów.
			\end{enumerate}
			}
		\\

	\hline
\end{tabular}

\textit{Tu muszą zostać przeklejone karty z Trello}

\subsection{Wymagania niefunkcjonalne}

\begin{tabular}{ | l | l | }
	\hline
		\textbf{Identyfikator} & 
		HN-1
		\\
		
	\hline
		\textbf{Treść} & \parbox[t]{13cm}{
			Długość nazwy użytkownika jest ograniczona od 2 do 32 znaków alfanumerycznych, w celu uniknięcia problemów z identyfikacją użytkownika na serwerze.
		}\\
		 
	\hline
		\parbox[t]{4cm}{\textbf{Powiązane zasady biznesowe}} & \parbox[t]{13cm}{
			U-3 Nazwa użytkownika to ciąg od 3 do 32 alfanumerycznych znaków.
		}\\
		
	\hline
		\parbox[t]{4cm}{\textbf{Kryteria akceptacji}} & \parbox[t]{13cm}{
			\begin{enumerate}
				\item Po wpisaniu do pola użytkownika nazwy krótszej niż 2 znaki, dłużej niż 32 znaki lub zawierającej inne znaki niż alfanumeryczne, zwracany jest błąd.
			\end{enumerate}
			}
		\\

	\hline
\end{tabular}

\textit{Tu muszą zostać przeklejone karty z Trello}

\end{document}
